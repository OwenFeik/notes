\documentclass[12pt]{report}
\renewcommand{\familydefault}{\sfdefault}

\usepackage{amssymb}
\usepackage{pgfplots}

\setlength{\parskip}{.2cm}

\newcommand{\dx}{\:\mathrm{d}x}

\newenvironment{plot}[1][
    xmin = -5,
    xmax = 5,
    ymin = -5,
    ymax = 5
]{
    \begin{center}
        \begin{tikzpicture}
            \begin{axis}[
                #1,
                samples = 1000,
                width = 12cm,
                height = 9cm,
                xlabel = \(x\),
                ylabel = \(y\),
                axis lines = middle,
                restrict y to domain = -10:10
            ]
}{
            \end{axis}
        \end{tikzpicture}
    \end {center}
}

\begin{document}
\begin{flushleft}

\section*{Electromagnetism}

When dealing with electrostatics, very small units are often required.
It therefore bears revising some of the smaller SI prefixes.
\begin{itemize}
    \item \(\mu = 10^{-6}\) (micro)
    \item \(n = 10^{-9}\) (nano)
    \item \(p = 10^{-12}\) (pico)
\end{itemize}

\subsection*{History}
Electricity was first observed in ancient Greece, where static electricity was
observed in amber. The name for electricity comes from the Greek word for 
amber, ``elektron''. Magnetism was discovered around the same time, but it took
many years for a connection between the two to be established. Only in 1820 did
H.C. Oersted identify the connection. In the late 19th century, Maxwell's 
equations quantified links between electricity and magnetism, and finally the
two were unified by Einstein in 1905, through the theory of special relativity.
This course only teaches up to slightly before 1905.

\subsection*{Charge}
Electromagnetism is one of the four fundamental forces of nature.
\begin{itemize}
    \item Gravitational force
    \item Electromagnetism
    \item Strong force
    \item Weak force
\end{itemize}
The gravitational force is often modelled through Newton's law.
\[F = \frac{Gm_1m_2}{r^2}\hat{r}\]
Electromagnetism mirrors this with Coulomb's Law, dating to 1798.
\[F = \frac{kq_1q_2}{r^2}\hat{r}\]
Where \(k\) is Coulomb's constant and \(q_1\) and \(q_2\) are the two charged
particles interacting. Here, \(\hat{r}\) is the vectore direction defined by
the two charges. \par
While gravity is only positive, electric charge comes in both positive and
negative forms. This charge is measured in Coulombs.
\[1C = 1A \times 1s\]
A Coulomb is a large quantity of charge. Charge is often denoted with \(q\).
As an example, the charge of an electron is
\[1.6 \times 10^{-19}C\]
Charge is often generated by scraping electrons from surfaces, or by touching
conductive surfaces together. 
Charge is quantised; it comes in integer quantities. It can be transferred, but
it cannot be created or destroyed.

\subsubsection*{Conductors}
A conductor contains electrons no strongly bound to any particular nucleus 
within the conductor. These \textit{free electrons} can move when under the
influence of an electric field, creating an electric current. There are huge
quantities of these free electrons available in an ordinary conductor. For
example, copper contains around \(10^{22}\) free electrons per cubic 
centimetre. These are initially held by attractive forces with the nuclei, but
can be moved along by external forces.

\bigskip
The counterpart to a conductor is an insulator, which have very few free 
electrons.

\bigskip
A process of charging by induction entails placing a charged body near to a
neutral body. This will attract the opposite charges of the neutral body to one
side. If the neutral body is then grounded, it will now be charged.

\subsubsection*{Electric Force}
Electric force is calculated, as touched on above, through Coulomb's law.
\[F = \frac{kq_1q_2}{r^2}\hat{r}\]
Here, \(F\) is the electric force in Newtons between the two charged bodies,
\(q_1\) and \(q_2\) are the charges of the two bodies involved, \(r\) is the
distance between the two, \(\hat{r}\) is a unit vector in the direction
between \(q_1\) and \(q_2\) and 
\(k = 8.988\times10^9\mathrm{Nm}^2\mathrm{C}^{-2}\). If this value \(F\) is
negative, the two are attracting each other, otherwise they are repelling each
other.

\bigskip
\(k\) is defined in terms of the \textit{permittivity of free space}, 
\(\epsilon_0\) according to
\[k = \frac{1}{4\pi\epsilon_0}\]
Here, \(\epsilon_0 = 8.85\times10^{-12}\mathrm{C^2N^{-1}m^{-2}}\). This value
is for the permittivity of free space, i.e. vacuum, and a different value is
required to accurately model force between charges with material lying between
them, known as the permittivity for that material. With this value, the 
behaviour of electric force for any material can be clearly defined.

\bigskip
The \textit{superposition principle} allows us to add electric forces (or any
forces) on a particle to determine the resultant force on that particle. This
can be done in either a single direction or using vector addition in multiple
directions. If two identical particles interact from opposite sides with a
particle, their effects will cancel.

\subsubsection*{Electric Fields}
An application of superposition comes in electric fields, where we can use
integration to apply the concept of superposition across a continuous field.
For a linear charge distribution, along a single straight line in the \(x\) 
direction, this integral looks like  
\[F_{q_0} = \int_{x_i}^{x_f} \frac{kq_0}{r_{0x}^2}\hat{r}_{0x}\lambda(x)\dx\]
Here, \(x_i\) through \(x_f\) is the range of charges along the line, \(k\) is
Coulomb's constant, \(q_0\) is the charge of the particle being considered, 
\(r_{0x}\) is the distance to the particle from each point \(x\), 
\(\vec{r}_{0x}\) is the vector between the particle and the point and
\(\lambda(x)\) is the \textit{linear charge density}, a function defining the
charge of the line at any point \(x\), measured in \(\mathrm{Cm}^{-1}\). \par
For a surface charge distribution, for example of a sphere, the integral
looks like
\[F_{q_0} = \int_{S} \frac{kq_0}{r_{0\mathrm{d}A}^2}\hat{r}_{0\mathrm{d}A}
\sigma(\mathrm{d}A)\:\mathrm{d}A\]
Here, we exchange linear position \(x\) for rate of change of area 
\(\mathrm{d}A\). We also exchange \(\lambda\) for \(\sigma\), the surface
charge density function for the body. \par
Finally, we can consider charge distribution throughout a volume
\[F_{q_0} = \int_{V} \frac{kq_0}{r_{0\mathrm{d}V}^2}\hat{r}_{0\mathrm{d}V}
\rho(\mathrm{d}V)\:\mathrm{d}V\]
In this case, integration across area becomes integration across volume, and
we exchange rate of change of area for rate of change of volume. \(\sigma\)
becomes \(\rho\), the volume charge density function.

\bigskip
We consider electric field lines through the construct of electric field lines.
These are imaginary lines flowing away from positive charges toward negative
charges, terminating at infinity or negative charges. The density of these 
lines is greatest around charges, which is where the electric field is 
strongest. Field lines are always perpendicular to the surface of a conductor.
Using this concept, we can define the electric field \(\vec{E}\) for a point
charge as
\[\vec{E} = \frac{\vec{F}}{q}\]
If we use \(\vec{F} = \frac{kq_1q_2}{r^2}\hat{r}\), we can find the value for
the electric field strength at a distance \(r\) to be
\[\vec{E} = \frac{kq}{r^2}\hat{r}\]
This yields a value in units of \(\mathrm{NC}^{-1}\) or equivalently 
\(\mathrm{Vm}^{-1}\). Here \(q\) is the charge of the particle creating the 
field. For a system of charges, we can sum across the effects of individual
charges on a charge in the resultant field.

\bigskip
For a dipole system, with a positive and negative charge seperated by a 
distance \(s\), as shown below, a special equation can be used to calculate
the field strength at a distance \(r\).
\begin{plot}[
    xmin = -1,
    xmax = 4,
    ymin = -3,
    ymax = 3,
    xticklabels = none,
    yticklabels = none,
]
    \addplot[black, dashed] coordinates {(0, 1)(2, 0)};
    \addplot[black, dashed] coordinates {(0, -1)(2, 0)};
    \addplot[black, dashed] coordinates {(-0.15, 1)(-0.15, -1)}
    node[left, pos = 0.4] {\(s\)};
    \addplot[black, dashed] coordinates {(0.1, 1.15)(2, 1.15)}
    node[above, pos = 0.5] {\(r\)};
    \addplot[ultra thick, black, ->] coordinates {(2, 0)(2.5, -0.25)}
    node[below right] {\(\vec{E}_+\)};
    \addplot[ultra thick, black, ->] coordinates {(2, 0)(1.5, -0.25)}
    node[below left, pos = 0.6, yshift = -0.1cm] {\(\vec{E}_-\)};
    \addplot[ultra thick, black, ->] coordinates {(2, 0)(2, -0.5)}
    node[below] {\(\vec{E}_\mathrm{dipole}\)};

    \fill[blue] (axis cs: 0, -1) circle[radius = 5pt];
    \fill[red] (axis cs: 0, 1) circle[radius = 5pt]
    node[below right, black] {\(\theta\)};
    \fill[red] (axis cs: 2, 0) circle[radius = 5pt];
\end{plot}
Here, \(\vec{E}_\mathrm{dipole}\) can be calculated through
\[\frac{kqs}{r^3}\]

\bigskip
Returning again too a linear line of charge, we can consider a case where
\(\lambda\) is a constant function; that is the charge per unit length of the
body is constant throughout. In this case, for a point \(p\) at some distance
\(r\) from the line, the force from two points an equal distance perpendicular
to \(r\) will cancel in the perpendicular direction, leaving only the force
in the direction of the radius. The resultant force vector on \(p\) will 
therefore be directly away from the linear charge line. The magnitude of this
field can be calculated through
\[\vec{E} = \frac{2k\lambda}{r}\hat{r}\]

\subsubsection*{Gauss' Law}
Coulomb's law works best for point charges, and can be generalised through
integration to a broader range of charges. Gauss' law is another law for force
due to electric charge, and can be applied to all charge distributions. Gauss'
law is one of Maxwell's equations; in a way it is a fundamental law of nature.
\par
Gauss' law requires the relevant problem to have symmetry, perhaps cartesian or
cylindrical. The symmetry may be translational, rotational or reflective. For a
point charge, Gauss' law is the same as Coulomb's law:
\[\vec{E} = \frac{kQ}{r^2}\hat{r} =\frac{1}{4\pi\epsilon}\frac{Q}{r^2}\hat{r}\]
This can be equivalently expressed as
\[\vec{E}4\pi r^2 = \frac{Q}{\epsilon}\]
Which can be intuitively understood as the surface area of a sphere multiplied
by the field strength at any point on the surface of the sphere being equal to
the charge of the particle divided by the permittivity of the material.

\bigskip
Flux is an important concept in electric fields. In fluid mechanics, flux 
describes the volume per unit time flowing through an area. If the area 
described is perpendicular to the flow, flux through it is at maximum. If 
perpendicular, the flux is \(0\).  If we define the angle of the area to the
perpendicular direction as \(\theta\), then we can define flux through that 
area in an electric field as
\[\Phi = EA\cos(\theta)\]
Where \(E\) is the field strength. This can be equivalently expressed if we
define a vector \(\vec{A}\) to be the direction of the area and \(\vec{v}\) to
be the flow of material as
\[\Phi = \vec{v} \cdot \vec{A}\]
With electric field as the flow of material this becomes \(E \cdot \vec{A}\).
Using Gauss' law, this generalises from simple planar areas to more complex
\textit{Guassian surfaces}, which are simply surfaces of any three dimensional 
body. The electric field through these surfaces can then be calculated by 
integration across the surface, considering the electric field at each 
infinetismal point by taking the dot product of the surface vector at that
point and the electric field at that point, yielding
\[\Phi = \int\vec{E}\cdot\mathrm{d}\vec{A}\]
The units of electric flux are \(\mathrm{Nm}^2\mathrm{C}^{-1}\). 
\(\mathrm{d}\vec{A}\) is a vector with direction given by the surface normal 
and magnitude of the area of the surface. In the case of a closed surface 
another simplification applies
\[\Phi = \oint_\mathrm{surface}\vec{E}\cdot\mathrm{d}\vec{A} = 
\frac{q_\mathrm{enclosed}}{\epsilon_0}\]
This states that the electric flux through a closed surface is equal to the
enclosed charge divided by \(\epsilon_0\). Thus we can simply sum the internal
charges to find the charge through the surface.

\bigskip
For an infinite line of charge, with charge density \(\lambda\), the electric
field at a distance \(r\) from the line is given by
\[\vec{E} = \frac{\lambda}{2\pi\epsilon_0}\hat{r}\]
For an insulating sheet of charge, with charge density \(\sigma\), the electric
field strength at a distance \(r\) from the sheet has direction normal to the 
sheet and magnitude given by
\[E = \frac{\sigma}{2\epsilon_0}\]

\bigskip
For a charged conductor, the electric field projected will always be normal to
the surface, and will always be zero internally. This is because the conductor
will inherently balance internal forces until \(0\) net field exists. Near to
the surface of such a conductor, the electric field, normal to the surface as
stated, will have field strength given by
\[E = \frac{\sigma}{\epsilon_0}\]
This is applicable on in areas where the conductor is sheet-like; either on a 
flat area or on a small enough section of a curve to make the curve irrelevant.
The reason for the behaviour of the conductor in this way is that the like 
charges repel each other, forcing an even distribution across the outer surface
of the conductor. \par
This applies equally to a hollow conductor. For a torus shaped conductor, the
inner radius will have no charges upon it. For this reason, lightning poses 
minimal risk to a person inside a conducting vehicle like a car. Here the car
acts as a \textit{Faraday cage}.

\subsubsection*{Electric Potential Energy}
It becomes quite complex to accurately describe the electric field at a 
specific point in space. It can therefore be useful to uniquely define a 
potential energy for a point in space; an analogue to the gravitational 
potential energy at a given height. \par
For example, if a positive charge is placed near another positively charged
object, the charge now has potential to move away from the object. Work will be
required to move it closer, and it will gain kinetic energy as it is repelled.
Electrice force is \textit{conservative}, just as gravity is. It is this 
property that allows potential to be uniquely defined with respect to space.

\bigskip
Work in an electric field is described just as mechanical work is, as force
times distance, though in this case our force is electrical force, \(qE\).
\[W = q\vec{E}\cdot\vec{d}\]
\[W = qEd\]
Work done against the field is negative. When work is done against the field,
the electric potential \(U\) increases, i.e. \(\Delta U > 0\). When potential
energy is lost, kinetic energy is gain, work done is positive. The dot product
in the above equation tells us that force at right angles to the field does
not entail a loss in potential energy, and also allows us to use the equation
in the general case through the use of vector arguments.

\subsubsection*{Electric Potential}

Electric potential energy, \(U\) differs from electric potential \(V\).
\begin{itemize}
    \item While \(U\) is a property of a system of charges and interacting 
        electric fields, \(V\) is independent of interacting charges.
    \item \(U\) at a point is not a property of in space, nor is the
        difference in potential between two points. \(V\) is a property of
        two points in space and the difference in potential between them.
    \item \(U\) is dependent on the charge, \(q\), which \(V\) is made 
        independent by dividing through \(U\) by \(q\).
\end{itemize}
\[V(P) = \frac{U(P)}{q}\]
Whenever a charge moves in a field, the change in \(U\) is proportional to
the charge, while the change in potential per unit charge is independent.
\[\Delta V = V_f - V_i = \frac{\Delta U}{q}\]
i.e. the change in electric potential is equal to initial potential minus final
potential is equal to change in electric potential energy divided by charge. 
The unit of electric potential is the Volt \(\mathrm{V}\), \(1\mathrm{V} = 
1 \mathrm{JC}^{-1}\). Electric fields can also be equivalently measured in 
\(\mathrm{Vm}^{-1}\) rather than \(\mathrm{NC}^{-1}\). \par
Potential itself has no meaning; only differences in potential have meaning.
Therefore, we can define zero according to convenience. Usually this is done
as either a charge at infinite distance or the charge of the earth. \par
In general \(E\) points towards regions of low \(V\) and away from high \(V\).
If one moves perpendicular to a field line through an electric field, one is
traversing an equipotential line, moving about without changing potential. For
a point charge, the electric potential at a distance \(r\) is given by
\[\frac{1}{4\pi\epsilon_0}\frac{Q}{r} = \frac{kQ}{r}\]
It is directly proportional to distance. For an infinite line of charge, the
change in potential when moving from position \(r_1\) to position \(r_2\) is
given by
\[V_2 - V_1 = \Delta V = -\frac{\lambda}{2\pi\epsilon_0}
\log\left(\frac{r_2}{r_2}\right)\]
For a hollow conductor, such as a hollow sphere, the potential inside the 
conductor must be the same as the potential at the surface. \par
To find the electric potential at a point \(P\) affected by multiple charges,
one can simply sum the potential of the individual charges to find the 
resultant potential at \(P\).
\[V(P) = V_1(P) + V_2(P) + \ldots + V_n(P) = \frac{U}{q}\]
This is equivalent to the potential divided by the charge.

\subsubsection*{Pointy Conductors}

Consider a conductor which at one end comes to a point. This can be 
approximated by considering a large sphere in proximity to a smaller sphere,
connected by an conductor between the two. In this case, the potential of the
system must be uniform, and therefore a greater proportion of the system's 
charge must be distributed around the radius of the larger sphere. In fact,
the ratio of charges is equivalent to the ratio of the radii, i.e.
\[\frac{q_1}{q_2} = \frac{r_1}{r_2}\]
If we then consider the electric field around each sphere, we find that for
a larger sphere \(1\) and a smaller sphere \(2\)
\[\frac{E_1}{E_2} = \frac{q_2}{q_2}\frac{r_2^2}{r_1^2} = \frac{r_2}{r_1}\]
\[\therefore r_2 < r_1 \Rightarrow E_2 > E_1\]
To summarise, a smaller radius implies a larger electric field. It is for this
reason that lightning tends to strike sharp points, such as the Eiffel Tower.
For the same reason it is dangerous to be atop a mountain during a 
thunderstorm.

\end{flushleft}
\end{document}
