\documentclass[12pt]{report}
\renewcommand{\familydefault}{\sfdefault}

\usepackage{amssymb}
\usepackage{pgfplots}

\setlength{\parskip}{.2cm}

\newcommand{\dx}{\:\mathrm{d}x}

\newenvironment{plot}[1][
    xmin = -5,
    xmax = 5,
    ymin = -5,
    ymax = 5
]{
    \begin{center}
        \begin{tikzpicture}
            \begin{axis}[
                #1,
                samples = 1000,
                width = 12cm,
                height = 9cm,
                xlabel = \(x\),
                ylabel = \(y\),
                axis lines = middle,
                restrict y to domain = -10:10
            ]
}{
            \end{axis}
        \end{tikzpicture}
    \end {center}
}

\begin{document}
\begin{flushleft}

\section*{Electromagnetism}

When dealing with electrostatics, very small units are often required.
It therefore bears revising some of the smaller SI prefixes.
\begin{itemize}
    \item \(\mu = 10^{-6}\) (micro)
    \item \(n = 10^{-9}\) (nano)
    \item \(p = 10^{-12}\) (pico)
\end{itemize}

\subsection*{History}
Electricity was first observed in ancient Greece, where static electricity was
observed in amber. The name for electricity comes from the Greek word for 
amber, ``elektron''. Magnetism was discovered around the same time, but it took
many years for a connection between the two to be established. Only in 1820 did
H.C. Oersted identify the connection. In the late 19th century, Maxwell's 
equations quantified links between electricity and magnetism, and finally the
two were unified by Einstein in 1905, through the theory of special relativity.
This course only teaches up to slightly before 1905.

\subsection*{Charge}
Electromagnetism is one of the four fundamental forces of nature.
\begin{itemize}
    \item Gravitational force
    \item Electromagnetism
    \item Strong force
    \item Weak force
\end{itemize}
The gravitational force is often modelled through Newton's law.
\[F = \frac{Gm_1m_2}{r^2}\hat{r}\]
Electromagnetism mirrors this with Coulomb's Law, dating to 1798.
\[F = \frac{kq_1q_2}{r^2}\hat{r}\]
Where \(k\) is Coulomb's constant and \(q_1\) and \(q_2\) are the two charged
particles interacting. Here, \(\hat{r}\) is the vectore direction defined by
the two charges. \par
While gravity is only positive, electric charge comes in both positive and
negative forms. This charge is measured in Coulombs.
\[1C = 1A \times 1s\]
A Coulomb is a large quantity of charge. Charge is often denoted with \(q\).
As an example, the charge of an electron is
\[1.6 \times 10^{-19}C\]
Charge is often generated by scraping electrons from surfaces, or by touching
conductive surfaces together. 
Charge is quantised; it comes in integer quantities. It can be transferred, but
it cannot be created or destroyed.

\subsubsection*{Conductors}
A conductor contains electrons no strongly bound to any particular nucleus 
within the conductor. These \textit{free electrons} can move when under the
influence of an electric field, creating an electric current. There are huge
quantities of these free electrons available in an ordinary conductor. For
example, copper contains around \(10^{22}\) free electrons per cubic 
centimetre. These are initially held by attractive forces with the nuclei, but
can be moved along by external forces.

\bigskip
The counterpart to a conductor is an insulator, which have very few free 
electrons.

\bigskip
A process of charging by induction entails placing a charged body near to a
neutral body. This will attract the opposite charges of the neutral body to one
side. If the neutral body is then grounded, it will now be charged.

\subsubsection*{Electric Force}
Electric force is calculated, as touched on above, through Coulomb's law.
\[F = \frac{kq_1q_2}{r^2}\hat{r}\]
Here, \(F\) is the electric force in Newtons between the two charged bodies,
\(q_1\) and \(q_2\) are the charges of the two bodies involved, \(r\) is the
distance between the two, \(\hat{r}\) is a unit vector in the direction
between \(q_1\) and \(q_2\) and 
\(k = 8.988\times10^9\mathrm{Nm}^2\mathrm{C}^{-2}\). If this value \(F\) is
negative, the two are attracting each other, otherwise they are repelling each
other.

\bigskip
\(k\) is defined in terms of the \textit{permittivity of free space}, 
\(\epsilon_0\) according to
\[k = \frac{1}{4\pi\epsilon_0}\]
Here, \(\epsilon_0 = 8.85\times10^{-12}\mathrm{C^2N^{-1}m^{-2}}\). This value
is for the permittivity of free space, i.e. vacuum, and a different value is
required to accurately model force between charges with material lying between
them, known as the permittivity for that material. With this value, the 
behaviour of electric force for any material can be clearly defined.

\bigskip
The \textit{superposition principle} allows us to add electric forces (or any
forces) on a particle to determine the resultant force on that particle. This
can be done in either a single direction or using vector addition in multiple
directions. If two identical particles interact from opposite sides with a
particle, their effects will cancel.

\subsubsection*{Electric Fields}
An application of superposition comes in electric fields, where we can use
integration to apply the concept of superposition across a continuous field.
For a linear charge distribution, along a single straight line in the \(x\) 
direction, this integral looks like  
\[F_{q_0} = \int_{x_i}^{x_f} \frac{kq_0}{r_{0x}^2}\hat{r}_{0x}\lambda(x)\dx\]
Here, \(x_i\) through \(x_f\) is the range of charges along the line, \(k\) is
Coulomb's constant, \(q_0\) is the charge of the particle being considered, 
\(r_{0x}\) is the distance to the particle from each point \(x\), 
\(\vec{r}_{0x}\) is the vector between the particle and the point and
\(\lambda(x)\) is the \textit{linear charge density}, a function defining the
charge of the line at any point \(x\), measured in \(\mathrm{Cm}^{-1}\). \par
For a surface charge distribution, for example of a sphere, the integral
looks like
\[F_{q_0} = \int_{S} \frac{kq_0}{r_{0\mathrm{d}A}^2}\hat{r}_{0\mathrm{d}A}
\sigma(\mathrm{d}A)\:\mathrm{d}A\]
Here, we exchange linear position \(x\) for rate of change of area 
\(\mathrm{d}A\). We also exchange \(\lambda\) for \(\sigma\), the surface
charge density function for the body. \par
Finally, we can consider charge distribution throughout a volume
\[F_{q_0} = \int_{V} \frac{kq_0}{r_{0\mathrm{d}V}^2}\hat{r}_{0\mathrm{d}V}
\rho(\mathrm{d}V)\:\mathrm{d}V\]
In this case, integration across area becomes integration across volume, and
we exchange rate of change of area for rate of change of volume. \(\sigma\)
becomes \(\rho\), the volume charge density function.

\bigskip
We consider electric field lines through the construct of electric field lines.
These are imaginary lines flowing away from positive charges toward negative
charges, terminating at infinity or negative charges. The density of these 
lines is greatest around charges, which is where the electric field is 
strongest. Field lines are always perpendicular to the surface of a conductor.
Using this concept, we can define the electric field \(\vec{E}\) for a point
charge as
\[\vec{E} = \frac{\vec{F}}{q}\]
If we use \(\vec{F} = \frac{kq_1q_2}{r^2}\hat{r}\), we can find the value for
the electric field strength at a distance \(r\) to be
\[\vec{E} = \frac{kq}{r^2}\hat{r}\]
This yields a value in units of \(\mathrm{NC}^{-1}\) or equivalently 
\(\mathrm{Vm}^{-1}\). Here \(q\) is the charge of the particle creating the 
field. For a system of charges, we can sum across the effects of individual
charges on a charge in the resultant field.

\bigskip
For a dipole system, with a positive and negative charge seperated by a 
distance \(s\), as shown below, a special equation can be used to calculate
the field strength at a distance \(r\).
\begin{plot}[
    xmin = -1,
    xmax = 4,
    ymin = -3,
    ymax = 3,
    xticklabels = none,
    yticklabels = none,
]
    \addplot[black, dashed] coordinates {(0, 1)(2, 0)};
    \addplot[black, dashed] coordinates {(0, -1)(2, 0)};
    \addplot[black, dashed] coordinates {(-0.15, 1)(-0.15, -1)}
    node[left, pos = 0.4] {\(s\)};
    \addplot[black, dashed] coordinates {(0.1, 1.15)(2, 1.15)}
    node[above, pos = 0.5] {\(r\)};
    \addplot[ultra thick, black, ->] coordinates {(2, 0)(2.5, -0.25)}
    node[below right] {\(\vec{E}_+\)};
    \addplot[ultra thick, black, ->] coordinates {(2, 0)(1.5, -0.25)}
    node[below left, pos = 0.6, yshift = -0.1cm] {\(\vec{E}_-\)};
    \addplot[ultra thick, black, ->] coordinates {(2, 0)(2, -0.5)}
    node[below] {\(\vec{E}_\mathrm{dipole}\)};

    \fill[blue] (axis cs: 0, -1) circle[radius = 5pt];
    \fill[red] (axis cs: 0, 1) circle[radius = 5pt]
    node[below right, black] {\(\theta\)};
    \fill[red] (axis cs: 2, 0) circle[radius = 5pt];
\end{plot}
Here, \(\vec{E}_\mathrm{dipole}\) can be calculated through
\[\frac{kqs}{r^3}\]

\bigskip
Returning again too a linear line of charge, we can consider a case where
\(\lambda\) is a constant function; that is the charge per unit length of the
body is constant throughout. In this case, for a point \(p\) at some distance
\(r\) from the line, the force from two points an equal distance perpendicular
to \(r\) will cancel in the perpendicular direction, leaving only the force
in the direction of the radius. The resultant force vector on \(p\) will 
therefore be directly away from the linear charge line. The magnitude of this
field can be calculated through
\[\vec{E} = \frac{2k\lambda}{r}\hat{r}\]

\subsubsection*{Gauss' Law}
Coulomb's law works best for point charges, and can be generalised through
integration to a broader range of charges. Gauss' law is another law for force
due to electric charge, and can be applied to all charge distributions. Gauss'
law is one of Maxwell's equations; in a way it is a fundamental law of nature.
\par
Gauss' law requires the relevant problem to have symmetry, perhaps cartesian or
cylindrical. The symmetry may be translational, rotational or reflective. For a
point charge, Gauss' law is the same as Coulomb's law:
\[\vec{E} = \frac{kQ}{r^2}\hat{r} =\frac{1}{4\pi\epsilon}\frac{Q}{r^2}\hat{r}\]
This can be equivalently expressed as
\[\vec{E}4\pi r^2 = \frac{Q}{\epsilon}\]
Which can be intuitively understood as the surface area of a sphere multiplied
by the field strength at any point on the surface of the sphere being equal to
the charge of the particle divided by the permittivity of the material.

\bigskip
Flux is an important concept in electric fields. In fluid mechanics, flux 
describes the volume per unit time flowing through an area. If the area 
described is perpendicular to the flow, flux through it is at maximum. If 
perpendicular, the flux is \(0\).  If we define the angle of the area to the
perpendicular direction as \(\theta\), then we can define flux through that 
area in an electric field as
\[\Phi = EA\cos(\theta)\]
Where \(E\) is the field strength. This can be equivalently expressed if we
define a vector \(\vec{A}\) to be the direction of the area and \(\vec{v}\) to
be the flow of material as
\[\Phi = \vec{v} \cdot \vec{A}\]
With electric field as the flow of material this becomes \(E \cdot \vec{A}\).
Using Gauss' law, this generalises from simple planar areas to more complex
\textit{Guassian surfaces}, which are simply surfaces of any three dimensional 
body. The electric field through these surfaces can then be calculated by 
integration across the surface, considering the electric field at each 
infinetismal point by taking the dot product of the surface vector at that
point and the electric field at that point, yielding
\[\Phi = \int\vec{E}\cdot\mathrm{d}\vec{A}\]
The units of electric flux are \(\mathrm{Nm}^2\mathrm{C}^{-1}\). 
\(\mathrm{d}\vec{A}\) is a vector with direction given by the surface normal 
and magnitude of the area of the surface. In the case of a closed surface 
another simplification applies
\[\Phi = \oint_\mathrm{surface}\vec{E}\cdot\mathrm{d}\vec{A} = 
\frac{q_\mathrm{enclosed}}{\epsilon_0}\]
This states that the electric flux through a closed surface is equal to the
enclosed charge divided by \(\epsilon_0\). Thus we can simply sum the internal
charges to find the charge through the surface.

\bigskip
For an infinite line of charge, with charge density \(\lambda\), the electric
field at a distance \(r\) from the line is given by
\[\vec{E} = \frac{\lambda}{2\pi\epsilon_0}\hat{r}\]
For an insulating sheet of charge, with charge density \(\sigma\), the electric
field strength at a distance \(r\) from the sheet has direction normal to the 
sheet and magnitude given by
\[E = \frac{\sigma}{2\epsilon_0}\]

\bigskip
For a charged conductor, the electric field projected will always be normal to
the surface, and will always be zero internally. This is because the conductor
will inherently balance internal forces until \(0\) net field exists. Near to
the surface of such a conductor, the electric field, normal to the surface as
stated, will have field strength given by
\[E = \frac{\sigma}{\epsilon_0}\]
This is applicable on in areas where the conductor is sheet-like; either on a 
flat area or on a small enough section of a curve to make the curve irrelevant.
The reason for the behaviour of the conductor in this way is that the like 
charges repel each other, forcing an even distribution across the outer surface
of the conductor. \par
This applies equally to a hollow conductor. For a torus shaped conductor, the
inner radius will have no charges upon it. For this reason, lightning poses 
minimal risk to a person inside a conducting vehicle like a car. Here the car
acts as a \textit{Faraday cage}.

\subsubsection*{Electric Potential Energy}
It becomes quite complex to accurately describe the electric field at a 
specific point in space. It can therefore be useful to uniquely define a 
potential energy for a point in space; an analogue to the gravitational 
potential energy at a given height. \par
For example, if a positive charge is placed near another positively charged
object, the charge now has potential to move away from the object. Work will be
required to move it closer, and it will gain kinetic energy as it is repelled.
Electrice force is \textit{conservative}, just as gravity is. It is this 
property that allows potential to be uniquely defined with respect to space.

\bigskip
Work in an electric field is described just as mechanical work is, as force
times distance, though in this case our force is electrical force, \(qE\).
\[W = q\vec{E}\cdot\vec{d}\]
\[W = qEd\]
Work done against the field is negative. When work is done against the field,
the electric potential \(U\) increases, i.e. \(\Delta U > 0\). When potential
energy is lost, kinetic energy is gain, work done is positive. The dot product
in the above equation tells us that force at right angles to the field does
not entail a loss in potential energy, and also allows us to use the equation
in the general case through the use of vector arguments.

\subsubsection*{Electric Potential}

Electric potential energy, \(U\) differs from electric potential \(V\).
\begin{itemize}
    \item While \(U\) is a property of a system of charges and interacting 
        electric fields, \(V\) is independent of interacting charges.
    \item \(U\) at a point is not a property of in space, nor is the
        difference in potential between two points. \(V\) is a property of
        two points in space and the difference in potential between them.
    \item \(U\) is dependent on the charge, \(q\), which \(V\) is made 
        independent by dividing through \(U\) by \(q\).
\end{itemize}
\[V(P) = \frac{U(P)}{q}\]
Whenever a charge moves in a field, the change in \(U\) is proportional to
the charge, while the change in potential per unit charge is independent.
\[\Delta V = V_f - V_i = \frac{\Delta U}{q}\]
i.e. the change in electric potential is equal to initial potential minus final
potential is equal to change in electric potential energy divided by charge. 
The unit of electric potential is the Volt \(\mathrm{V}\), \(1\mathrm{V} = 
1 \mathrm{JC}^{-1}\). Electric fields can also be equivalently measured in 
\(\mathrm{Vm}^{-1}\) rather than \(\mathrm{NC}^{-1}\). \par
Potential itself has no meaning; only differences in potential have meaning.
Therefore, we can define zero according to convenience. Usually this is done
as either a charge at infinite distance or the charge of the earth. \par
In general \(E\) points towards regions of low \(V\) and away from high \(V\).
If one moves perpendicular to a field line through an electric field, one is
traversing an equipotential line, moving about without changing potential. For
a point charge, the electric potential at a distance \(r\) is given by
\[\frac{1}{4\pi\epsilon_0}\frac{Q}{r} = \frac{kQ}{r}\]
It is directly proportional to distance. For an infinite line of charge, the
change in potential when moving from position \(r_1\) to position \(r_2\) is
given by
\[V_2 - V_1 = \Delta V = -\frac{\lambda}{2\pi\epsilon_0}
\log\left(\frac{r_2}{r_2}\right)\]
For a hollow conductor, such as a hollow sphere, the potential inside the 
conductor must be the same as the potential at the surface. \par
To find the electric potential at a point \(P\) affected by multiple charges,
one can simply sum the potential of the individual charges to find the 
resultant potential at \(P\).
\[V(P) = V_1(P) + V_2(P) + \ldots + V_n(P) = \frac{U}{q}\]
This is equivalent to the potential divided by the charge.

\subsubsection*{Pointy Conductors}

Consider a conductor which at one end comes to a point. This can be 
approximated by considering a large sphere in proximity to a smaller sphere,
connected by an conductor between the two. In this case, the potential of the
system must be uniform, and therefore a greater proportion of the system's 
charge must be distributed around the radius of the larger sphere. In fact,
the ratio of charges is equivalent to the ratio of the radii, i.e.
\[\frac{q_1}{q_2} = \frac{r_1}{r_2}\]
If we then consider the electric field around each sphere, we find that for
a larger sphere \(1\) and a smaller sphere \(2\)
\[\frac{E_1}{E_2} = \frac{q_2}{q_2}\frac{r_2^2}{r_1^2} = \frac{r_2}{r_1}\]
\[\therefore r_2 < r_1 \Rightarrow E_2 > E_1\]
To summarise, a smaller radius implies a larger electric field. It is for this
reason that lightning tends to strike sharp points, such as the Eiffel Tower.
For the same reason it is dangerous to be atop a mountain during a 
thunderstorm.

\subsubsection*{Electric Dipoles}

Due to the strength of the electric force, it is rare to encounter free charges
in nature. More common are dipoles, conductors with a positive and negative 
pole. \par
A dipole with two charges of equal magnitude in a uniform electric field will
experience equal force to each pole. If the dipole is not parallel to the
direction of the field, it will experience a torque. The magnitude of this
torque will be
\[\tau = Fd\sin(\theta) = qdE\sin(\theta)\]
Where \(d\) is the distance between the two poles of the dipole, each of which
have charge magnitude \(q\). A dipole is often defined as a vector \(p\) where
\[p = qd\]
and \(p\) points from the negative to positive terminal of the dipole. Using
this vector representation, we find that the torque is given by
\[\vec{\tau} = \vec{p} \times \vec{E}\]
To rotate against this field, work must be done on the dipole. When the dipole
is parallel with the field, \(\theta = 0 \Rightarrow \tau = 0\). For the dipole
to rotate to some other angle, work must be done, and the magnitude of this 
work is given by
\[W = \tau\theta\]
The potential energy of the system at any angle is given by the dot product
\[U = -\vec{p} \cdot \vec{E}\]

\subsection*{Capacitors}

A capacitor stores the energy of an electric field. It does this by separating
positive and negative electric charges within itself. Generally, capacitors use
two sheets of electric foil, often coiled into a cylinder. The total charge 
within the capacitor is also zero, though often with a large positive and a 
large negative charge. Between these two plates, a potential difference given
by
\[V = \frac{qd}{A\epsilon_0}\]
Where \(A\) is the area of the two plates and \(d\) is the distance between 
them. In the case that there is air or another material rather than vacuum 
between the two plates, a different \(\epsilon\) must of course be used. This
expression for the potential difference within the capacitor leads to a 
definition of the \textit{capacitance} of the capacitor.
\[C = \frac{\epsilon_0A}{d}\]
\[\Rightarrow q = CV\]
Here \(C\) is measured in units of \(\mathrm{CV}^{-1}\) or Farads, 
\(\mathrm{F}\). The definition above is for a parallel plate capacitor; other
shapes may have subtly different expressions. To create capacitors with high
capacitance, it is common to use a material with a different \(\epsilon\) 
rather than change the seperation or area. \(d\) can be extremely small; in 
applications like DRAM it is often as little as \(50\mathrm{nm}\).

\subsubsection{Energy Storage in Capacitors}

Charging a capacitor requries energy, as electrons are forced onto the plates.
The internal energy of a capacitor is given by
\[U = \frac{1}{2}\frac{q^2}{C} = \frac{1}{2}CV^2\]
If potential is fixed, say by a battery, the energy stored can be increased
through higher capacitance. If the charge is fixed, increasing the capacitance
decreases the potential energy. Increasing voltage is a very effective way to
increase the energy stored of a capacitor. It is useful to be able to talk 
about energy density of a capacitor, the formula for which is
\[\frac{1}{2}\epsilon_0E^2\]
Assuming a vacuum, otherwise an appropriate \(\epsilon\) must be used. 
Interestingly, this formula can be used for \textit{all} electric fields, with
units of \(\mathrm{Jm}^{-3}\). When changing capacitance by using a non-vacuum
filling material, we use the formula
\[C = \frac{\kappa\epsilon_0A}{d} = \frac{\epsilon A}{d}\]
Where \(\kappa\) is the \textit{dielectric constant}, a coefficient to 
\(\epsilon_0\) which relates the permittivity of the material to that of free 
space. This higher \(\epsilon\) allows for a significantly higher storage of
energy. However, if the field becomes too strong, the dielectric will fail.
We therefore have a concept of dielectric strength, which informs of the 
maximum field which can be created across the material without destruction. 
\par
A defibrillator is an example of a very high capacitance application, where a
material boasting a very high \(\kappa\) is utilised.

\subsubsection*{Capacitance in Circuits}

Capacitors in parallel add their capacitance. Effectively, the surface areas of
the individual capacitors add.
\[C_T = \sum_{i\rightarrow n} C_i\]
For capacitors in series, the capacitance adds similarly to resistors in 
parallel. 
\[\frac{1}{C_T} = \sum_{i\rightarrow n} \frac{1}{C_i}\]
This is because the charge between the negative terminal of one must be 
adjacent to the positive terminal of another, and the charge between the two
must be neutral.

\subsection*{Electric Current}

Around an atom, energy exists in quantised levels. Electrons exist in shells
around a nucleus, and each shell has a different binding energy. The number of
electrons that can fit within each shell is different, beginning with two and
increasing moving toward outer shells. \par
Insulators are more tightly bound than are conductors. In conductors, electrons
can move around more freely and conduct to other particles. This implies that 
there is a relatively large gap between energy bands of the particle. \par
Semiconductors have a smaller gap between bands than conductors, but larger 
than insulators. By changing their conductivity by introducing traces of
conductors or altering the temperature, one can control their conducting 
behaviour. \par
Electrons naturally move around rapidly and largely randomly, but by applying
a force to them with an electric field, this random motion can be skewed toward
one end of the circuit resulting in a flow of electricity. Electric current is
the rate of transport of charge along a conductor.
\[I = \frac{\mathrm{d}q}{\mathrm{d}t}\]
This value \(i\) can be calculated by considering the movement of individual
electrons within a conductor, according to the following formula.
\[\frac{\mathrm{d}q}{\mathrm{d}t} = enAv_d = I\]
Here, \(e\) is the charge of an electron, \(n\) is the density of electrons in
the conducting material, \(A\) is the cross sectional area of the flow
direction in in the conductor and \(v_d\) is the drift speed of an electron.
Current is measured in Amps, \(\mathrm{A}\) equal to \(\mathrm{Cs}^{-1}\).
\par
A material with a lower \(v_d\) generally has more things for a flowing 
electron to ``bump in to'', causing it to slow down, and might be said to have
a higher resistance. This resistance is defined more formally through
\[R = \frac{V}{I}\]
Generally, \(R\) is dependent on temperature, voltage and current. A 
superconductor has \(0\) resistance, implying a free flow of electrons. A
device with a linear relationship between \(I\) and \(V\) is an Ohmic resistor.
\par
Power is defined as the product of current and voltage, i.e.
\[P = VI\]
When electricity passes through a material, electrons are bumping into the 
material, releasing heat into it, causing the material to heat up and wasting
power. For a resistor with resistance \(R\omega\), the power dissipated is
\[P_\mathrm{lost} = I^2R\]
It is for this reason that power lines are run at very high voltages like 
\(500\mathrm{kV}\). Because power loss is related to the square of current,
running at a high voltage allows the same delivery of power with a much lower
power loss.

\subsection*{Circuits}

A battery in a circuit is a source of electromotive force, electric potential.
It drives current around a circuit. The electric potential in the circuit is
dissipated across the components of the circuit according to their resistances.
In the case of the resistor, the potential is converted to heat.
\[\epsilon - IR = 0\]
\[I = \frac{\epsilon}{R}\]
Here, \(\epsilon\) is electromotive, equivalent to \(V\). To understand complex
circuits, we use Kirchhoff's Rules for Circuit Analysis. These rules 
essentially dictate conservation of charge and conservation of energy within
circuits. This states that any current within a circuit must be going 
somewhere; it must traverse from one end of potential back to the same. \par
Mathematically, at any node, any combination of path, the sum of currenst must
be \(0\). This applies along a straight line; if the current all passes through
one point, the outflow equals the inflow so the sum is \(0\).
\[\sum I = 0\]
Around any loop, the following statement must be true.
\[\pm\sum\epsilon\pm\sum IR\pm\sum\frac{q}{C} = 0\]
So if we add up all of the voltage input (say batterys), voltage loss 
(say resistors) and capacitors, the sum must be \(0\), i.e. the circuit is
conservative; any gain in charge is matched by an equivalent loss. This
allows us to, for example figure out how to calculate cumulative resistors in
series as
\[R_T = \sum_{i\rightarrow n}R_i\]
It can also be used to solve for resistors in parallel, yielding the equation
\[\frac{1}{R_T} = \sum_{i\rightarrow n}\frac{1}{R_i}\]
These equations are determined by tracing the paths and considering the 
splitting of current that must occur for the circuit to obey the two 
conservations.

\subsection*{Magnetic Fields}

In electric fields, electric monopoles exert forces on each other. Magnetic
fields behave differently, as all magnetic fields are dipoles. No ``magnet
charge'' exists, instead moving charges create magnetic fields. The simplest
example of a magnet is a bar magnet. A bar magnet is a magnetic dipole, with a 
north and south pole. \par
Field lines are drawn from the north pole to the south pole, and the density of
these field lines is known as magnetic flux. This density is highest at either
of these poles. \par
The idea of all magnetic fields being dipoles can be expressed through the
previously explored idea of a closed surface integral. This is Gauss' Law of
magnetism, and is one of Maxwell's equations.
\[\oint \vec{B}\cdot\mathrm{d}\vec{A} = 0\]
This states thates that the magnetic flux through any given three dimensional
surface must be \(0\), which can be understood as the fact that any field lines
which emerge from the surface must also return through the sphere to the 
opposite pole. \par
An electric charge moving through a magnetic field will tend to move
perpendicular to that field, in an arc. This is because it has a certain force
acting on it which causes in to move in a parabola as it accelerates. The
magnitude and direction of this force is given by
\[\vec{F} = q\vec{v}\times\vec{B}\]
Where \(\vec{F}\) is the force vector on the particle with charge \(q\) moving
with velocity vector \(\vec{v}\) through electric field with strength and
direction defined by \(\vec{B}\). The units of \(\vec{B}\) are \(T\), Tesla.
Interestingly a faster moving charge will have a greater force exerted on it.
The magnitude of this force can be calcualted through
\[F = qvB\sin(\phi)\]
Where \(\phi\) is the angle between \(v\) and \(B\). This force will always be
perpendicular to the plane defined by the velocity and magnetic field
directions. This can be simulated with the right hand rule; if one curls their
fingers from \(v\) to \(B\) and extends the thumb, it will be in the direction
of the force. \par
The Lorentz Force Law tells us how to combine the effects of magnet and
electric fields on a moving charge. Intuitively enough, it essentially just
says "add them you Drongo"
\[\vec{F} = q\vec{E} + q\vec{v}\times\vec{B}\]
Because a charge in a field will tend to move in an arc, it is clear that if
the area the charge is in is large enough, it will eventually trace out a
circle. If we want to find the radius of that circle, perhaps for desigining
a Cyclotron or similar, we can use the equation
\[r = \frac{mv}{qB}\]
This assumes a charge of mass \(m\) with velocity \(v\) moving in a field
at right angles to its velocity plane. The period of this rotation is given
by
\[T = \frac{1}{f} = \frac{2\pi m}{qB}\]
Interestingly enough, independent of \(v\).

\subsubsection*{Lorentz Force}

When working with both an electric field and a magnetic field, it is often
useful to have the two perpendicular, such as in the case of a cathode ray
tube. \par
Another application of these perpendicular fields is an ion velocity filter,
where one uses the fact that magnetic force is proportional to velocity to
filter out charged particles of other velocities. This is done by balancing the
electrical and magnetic fields such that ions of the desired velocity will have
balanced forces from the two, while other velocities will have larger or
smaller forces, causing them to crash into the side of the chamber. To balance
in this way, one simply needs to solve the equality
\[qE = qvB \Rightarrow v = \frac{E}{B}\]
This same process can be used to construct a simple mass spectrometer, a device
which measures the charge to mass ratio of ions. By accelerating charges in an
electric field, they will end up with
\[v \propto \sqrt{\frac{q}{m}}\]
And can then be passed through an ion filter to measure velocity.

\subsubsection*{The Hall Effect}

If a magnetic field is running through a wire with a current, the deflection
caused by the magnetic field will result in a build up of negative charges on
one side, creating a potential difference between the two sides of the wire.
This is known as the Hall Effect. The magnitude of this effect continues to
increase until the force exerted by the created electric field is equivalent to
the external magnetic field within the wire, i.e.
\[F_E = F_B\]
is the condition for the process to end. This can be used to measure the drift
velocity within the material because the force due to the magnetic field is
proportional to the velocity of the electrons. The final voltage across the
wire is known as the Hall Voltage for the material.

\subsubsection*{Origin of a Magnetic Field}

Magnetic fields can be created in two ways. The first of these is by magnetic
materials, and the second is by currents. A current produces a magnetic field
according to the Biot-Savart Law, which takes the form
\[\mathrm{d}\vec{B} 
= \frac{\mu_0}{4\pi}\frac{I\mathrm{d}\vec{s}\times\hat{r}}{r^2}\]
Where \(\mathrm{d}\vec{B}\) is the section of magnetic field at a distance
\(r\) in a direction \(\hat{r}\) from the current carrier, \(\mu_0\) is the
vacuum permeability and \(\mathrm{d}\vec{s}\) is the rate of change of the
current carrying surface at the relevant point. \par
\(\mu_0\), the vacuum permeability is rather like the vacuum permittivity we
use for electric fields. Like permittivity, we can replace \(\mu_0\) with a
determined \(\mu\) for a non-vacuum material. \par
This law gives us the right hand rule for a wire. If one places their thumb
along the direction of current in a wire, and wraps their fingers around, the
fingers will indicate the direction of magnetic field. \par
Much as we try to use Gauss' Law rather than Coulomb's law in electrostatics,
the complexity of the Biot-Savart law means it is often better to use Ampere's
law when dealing with magnetic fields.

\end{flushleft}
\end{document}
