\documentclass[12pt]{report}
\renewcommand{\familydefault}{\sfdefault}

\begin{document}

\section*{Electromagnetism}

When dealing with electrostatics, very small units are often required.
It therefore bears revising some of the smaller SI prefixes.
\begin{itemize}
    \item \(\mu = 10^{-6}\) (micro)
    \item \(n = 10^{-9}\) (nano)
    \item \(p = 10^{-12}\) (pico)
\end{itemize}

\subsection*{History}
Electricity was first observed in ancient Greece, where static electricity was
observed in amber. The name for electricity comes from the Greek word for 
amber, ``elektron''. Magnetism was discovered around the same time, but it took
many years for a connection between the two to be established. Only in 1820 did
H.C. Oersted identify the connection. In the late 19th century, Maxwell's 
equations quantified links between electricity and magnetism, and finally the
two were unified by Einstein in 1905, through the theory of special relativity.
This course only teaches up to slightly before 1905.

\subsection*{Charge}
Electromagnetism is one of the four fundamental forces of nature.
\begin{itemize}
    \item Gravitational force
    \item Electromagnetism
    \item Strong force
    \item Weak force
\end{itemize}
The gravitational force is often modelled through Newton's law.
\[F = \frac{Gm_1m_2}{r^2}\vec{r}\]
Electromagnetism mirrors this with Coulomb's Law, dating to 1798.
\[F = \frac{kq_1q_2}{r^2}\vec{r}\]
Where \(k\) is Coulomb's constant and \(q_1\) and \(q_2\) are the two charged
particles interacting. Here, \(\vec{r}\) is the vectore direction defined by
the two charges. \par
While gravity is only positive, electric charge comes in both positive and
negative forms. This charge is measured in Coulombs.
\[1C = 1A \times 1s\]
A Coulomb is a large quantity of charge. Charge is often denoted with \(q\).
As an example, the charge of an electron is
\[1.6 \times 10^{-19}C\]
Charge is often generated by scraping electrons from surfaces, or by touching
conductive surfaces together. 
Charge is quantised; it comes in integer quantities. It can be transferred, but
it cannot be created or destroyed.

\subsubsection*{Conductors}
A conducter contains electrons no strongly bound to any particular nucleus 
within the conductor. These \textit{free electrons} can move when under the
influence of an electric field, creating an electric current. There are huge
quantities of these free electrons available in an ordinary conducter. For
example, copper contains around \(10^{22}\) free electrons per cubic 
centimetre. These are initially held by attractive forces with the nuclei, but
can be moved along by external forces.

\bigskip
The counterpart to a conducter is an insulator, which have very few free 
electrons.

\bigskip
A process of charging by induction entails placing a charged body near to a
neutral body. This will attract the opposite charges of the neutral body to one
side. If the neutral body is then grounded, it will now be charged.

\subsubsection*{Electric Force}
Electric force is calculated, as touched on above, through Coulomb's law.
\[F = \frac{kq_1q_2}{r^2}\vec{r}\]
Here, \(F\) is the electric force in Newtons between the two charged bodies,
\(q_1\) and \(q_2\) are the charges of the two bodies involved, \(r\) is the
distance between the two, \(\vec{r}\) is a unit vector in the direction
between \(q_1\) and \(q_2\) and 
\(k = 8.988\times10^9\mathrm{Nm}^2\mathrm{C}^{-2}\). If this value \(F\) is
negative, the two are attracting each other, otherwise they are repelling each
other.

\bigskip
\(k\) is defined in terms of the \textit{permittivity of free space}, 
\(\epsilon_0\) according to
\[k = \frac{1}{4\pi\epsilon_0}\]
Here, \(\epsilon_0 = 8.85\times10^{-12}\mathrm{C^2N^{-1}m^{-2}}\). This value
is for the permittivity of free space, i.e. vacuum, and a different value is
required to accurately model force between charges with material lying between
them, known as the permittivity for that material. With this value, the 
behaviour of electric force for any material can be clearly defined.

\bigskip
The \textit{superposition principle} allows us to add electric forces (or any
forces) on a particle to determine the resultant force on that particle. This
can be done in either a single direction or using vector addition in multiple
directions. If two identical particles interact from opposite sides with a
particle, their effects will cancel.

\end{document}
