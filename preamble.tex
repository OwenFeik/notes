\renewcommand{\familydefault}{\sfdefault}

\usepackage{textcomp} % suppress warning on gensymb import

\usepackage{amsmath}
\usepackage{amssymb}
\usepackage{bbold}
\usepackage{commath}
\usepackage{environ}
\usepackage{gensymb}
\usepackage{pgfplots}
\usepackage{placeins}

\setlength{\parskip}{.2cm}

\usetikzlibrary{external}
\tikzexternalize

\NewEnviron{plot}[1][
    xmin = -5,
    xmax = 5,
    ymin = -5,
    ymax = 5
]{
    \begin{center}
        \begin{tikzpicture}
            \begin{axis}[
                samples = 1000,
                width = 12cm,
                height = 9cm,
                xlabel = \(x\),
                ylabel = \(y\),
                axis lines = middle,
                restrict y to domain = -10:10,
                #1
            ]
            \BODY\                
            \end{axis}
        \end{tikzpicture}
    \end{center}
}

\NewEnviron{plot3}[1][
    xmin = -5,
    xmax = 5,
    ymin = -5,
    ymax = 5,
    zmin = -5,
    zmax = 5
]{
    \begin{center}
        \begin{tikzpicture}
            \begin{axis}[
                width = 12cm,
                height = 9cm,
                xlabel = \(x\),
                ylabel = \(y\),
                zlabel = \(z\),
                #1
            ]
            \BODY\                
            \end{axis}
        \end{tikzpicture}
    \end{center}
}

\NewEnviron{smallplot}[1][
    xmin = -5,
    xmax = 5,
    ymin = -5,
    ymax = 5
]{
    \begin{center}
        \begin{tikzpicture}
            \begin{axis}[
                samples = 1000,
                #1,
                width = 8cm,
                height = 6cm,
                xlabel = \(x\),
                ylabel = \(y\),
                axis lines = middle,
                restrict y to domain = -10:10
            ]
            \BODY\
            \end{axis}
        \end{tikzpicture}
    \end{center}
}

\NewEnviron{econplot}[2]{
    \begin{center}
        \begin{tikzpicture}
            \begin{axis}[
                xmin = 0,
                xmax = 6,
                ymin = 0,
                ymax = 6,
                width = 12 cm,
                height = 9 cm,
                xlabel = #1,
                xlabel near ticks,
                xtick style = {draw = none},
                xticklabels = \empty,
                axis x line = bottom,
                ylabel = #2,
                ylabel near ticks,
                ytick style = {draw = none},
                yticklabels = \empty,
                axis y line = left,
                samples = 500
            ]
                \BODY\
            \end{axis}
        \end{tikzpicture}
    \end{center}
}

\newenvironment{formulalist}{
    \renewcommand{\arraystretch}{2}
    \begin{center}    
        \begin{tabular}{||c||}
}{
        \end{tabular}
    \end{center}
    \renewcommand{\arraystretch}{1}
}

\newcommand{\N}{\mathbb{N}}
\newcommand{\Z}{\mathbb{Z}}
\newcommand{\Q}{\mathbb{Q}}
\newcommand{\R}{\mathbb{R}}
\newcommand{\C}{\mathbb{C}}
\newcommand{\dx}{\:\mathrm{d}x}
\newcommand{\dy}{\:\mathrm{d}y}
\newcommand{\dd}[2]{\frac{\mathrm{d}#1}{\mathrm{d}#2}}
\newcommand{\ddel}[2]{\frac{\partial#1}{\partial#2}}
\newcommand{\mand}{\:\mathrm{and}\:}
\newcommand{\mor}{\:\mathrm{or}\:}
\newcommand{\mcom}{, \:\:\:}
\newcommand{\limit}{\lim\limits}
\newcommand{\sumninf}[1][1]{\sum\limits_{n = #1}^\infty}
\newcommand{\derivx}[1]{\frac{\mathrm{d}}{\mathrm{d}x}\left[#1\right]}
\newcommand{\pp}{{\prime\prime}}
\newcommand{\at}[1]{\Bigr\rvert_{#1}}
\newcommand{\ddelat}[3]{\ddel{#1}{#2}\at{#3}}

\DeclareMathOperator{\sech}{sech}
\DeclareMathOperator{\cosech}{cosech}
\DeclareMathOperator{\arcsinh}{arcsinh}
\DeclareMathOperator{\arccosh}{arccosh}
\DeclareMathOperator{\arctanh}{arctanh}

\pgfplotsset{compat = 1.16}
