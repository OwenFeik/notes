\documentclass[12pt]{report}
\renewcommand{\familydefault}{\sfdefault}
\usepackage{xcolor}
\usepackage{listings}

\setlength{\parskip}{.2cm}

\lstset{
    frame = tb,
    language = SQL,
    aboveskip = 3mm,
    belowskip = 3mm,
    showstringspaces = false,
    columns = flexible,
    basicstyle = {\small\ttfamily},
    numbers = none,
    numberstyle = \tiny\color{gray},
    keywordstyle = \color{blue},
    commentstyle = \color{darkgray},
    stringstyle = \color{orange},
    breaklines = true,
    breakatwhitespace = true,
    tabsize = 3
}

\newcommand{\code}[1]{\lstinline{#1}}

\begin{document}
\begin{flushleft}

\section*{Databases}

\subsubsection*{Data and Information}

While data is known, discrete facts that have been stored and recorded,
information is data placed in context and presented. It is much more useful to
humans. SQL is a tool for extracting information from a database.

\subsubsection*{Metadata}

Metadata is \textit{data about data}. For example type, length or description.
It helps us to keep data storage consistent, useful and meaningful.

A database is a large, integrated, structured collection of data, used to model
some real world enterprise as entities and relationships. A Database Management
System or DBMS is used to interface with a database. Databases differ from
simple programs interfacing with files by avoiding redundancy and ensuring
consistency. They also allow better file sharing and can improve development
speed and reduce maintenance. 

\subsection*{Database Development Process}

\begin{itemize}
    \item Database Planning
    \item Systems Definition
        \begin{itemize}
            \item Enterprise data model, where the components and interactions
                of a business is defined. 
            \item Specification of scope and boundaries of the system. 
        \end{itemize}
    \item Requirements Definition and Analysis
        \begin{itemize}
            \item Take in requirements for the system and analyse them to define
                a system that will satisfy them.
        \end{itemize}
    \item Design
        \begin{itemize}
            \item Conceptual Design - construction of model of the data to be
                held in the database, independent of any technical
                considerations. Generally using \textit{entity relationship}
                (ER) diagrams.
            \item Logical Design - technical decisions for the conceptual
                design above. While in this subject this will always be a DBMS,
                it could also be a JSON document or even a spreadsheet.
            \item Physical Design - implementation details of a given logical
                design; relations, data types, configurations, etc.
                Specification of types can help to make a database smaller and
                faster. It's important to consider all the factors that inform
                a datatype.
        \end{itemize}
    \item Application Design
        \begin{itemize}
            \item In parallel with the design phase, design of the application
                continues.
        \end{itemize}
    \item Implementation
    \item Data Conversion and Loading
    \item Testing
    \item Operational Maintenance
\end{itemize}

\end{flushleft}
\end{document}
