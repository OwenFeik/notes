\documentclass[12pt]{report}
\renewcommand{\familydefault}{\sfdefault}
\usepackage{xcolor}
\usepackage{listings}

\setlength{\parskip}{.2cm}

\lstset{
    frame = tb,
    language = SQL,
    aboveskip = 3mm,
    belowskip = 3mm,
    showstringspaces = false,
    columns = flexible,
    basicstyle = {\small\ttfamily},
    numbers = none,
    numberstyle = \tiny\color{gray},
    keywordstyle = \color{blue},
    commentstyle = \color{darkgray},
    stringstyle = \color{orange},
    breaklines = true,
    breakatwhitespace = true,
    tabsize = 3
}

\newcommand{\code}[1]{\lstinline{#1}}

\begin{document}
\begin{flushleft}

\section*{Databases}

\subsubsection*{Data and Information}

While data is known, discrete facts that have been stored and recorded,
information is data placed in context and presented. It is much more useful to
humans. SQL is a tool for extracting information from a database.

\subsubsection*{Metadata}

Metadata is \textit{data about data}. For example type, length or description.
It helps us to keep data storage consistent, useful and meaningful.

A database is a large, integrated, structured collection of data, used to model
some real world enterprise as entities and relationships. A Database Management
System or DBMS is used to interface with a database. Databases differ from
simple programs interfacing with files by avoiding redundancy and ensuring
consistency. They also allow better file sharing and can improve development
speed and reduce maintenance. 

\subsection*{Database Development Process}

\begin{itemize}
    \item Database Planning
    \item Systems Definition
        \begin{itemize}
            \item Enterprise data model, where the components and interactions
                of a business is defined. 
            \item Specification of scope and boundaries of the system. 
        \end{itemize}
    \item Requirements Definition and Analysis
        \begin{itemize}
            \item Take in requirements for the system and analyse them to define
                a system that will satisfy them.
        \end{itemize}
    \item Design
        \begin{itemize}
            \item Conceptual Design - construction of model of the data to be
                held in the database, independent of any technical
                considerations. Generally using \textit{entity relationship}
                (ER) diagrams.
            \item Logical Design - technical decisions for the conceptual
                design above. While in this subject this will always be a DBMS,
                it could also be a JSON document or even a spreadsheet.
            \item Physical Design - implementation details of a given logical
                design; relations, data types, configurations, etc.
                Specification of types can help to make a database smaller and
                faster. It's important to consider all the factors that inform
                a datatype.
        \end{itemize}
    \item Application Design
        \begin{itemize}
            \item In parallel with the design phase, design of the application
                continues.
        \end{itemize}
    \item Implementation
    \item Data Conversion and Loading
    \item Testing
    \item Operational Maintenance
\end{itemize}

\subsection*{Entity-Relationship Models}

An entity-relationship model is a method for modelling data needed for an
organisation. An entity is a real-world object, distinguishable from other
objects. An entity set is a collection of similar entities. For example an
entity might be ``John Smith'', of the entity set person. John might have
attributes like name, date of birth, etc. Each entity has a \textit{key}. This
is a unique piece of data describing the record. \par
In a visual model, an entity is represented as a rectangle with attributes
in ovals connected to it by lines. The name of the key is underlined. \par
A relationship is an association between entities. For example, John might
\textit{live in} Melbourne, a City. In this case, City is another entity set
with its own attributes. The relationship might have an attribute like
``since''. A relationship is depicted as a diamond, connecting to the relevant
rectangles, with attributes as ovals. \par
A relationship might connect to the same entity twice, such as in a ``reports
to'' relationship between two employees.

\subsubsection*{Constraints}

Relationships have constraints on them, describing which entities can be
related by them. On of these is key constraints, which are
\begin{itemize}
    \item Many-to-many. Any number of entities on one side are related to any
        number of entities on the other side. For example a lecturer may teach
        many lectures, which may be lectured by many lecturers.
    \item One-to-many. A given entity may be related to a large number of
        another entity.
    \item One-to-one. A given entity is related to a single other entity. For
        example a person is part of one department.
\end{itemize}
A key constraint indicating ``many'' is simply a plain straight line. In a
``one'' relationship, known as a key relationship, an arrow is drawn from the
side which has only one incidence to the relationship. If every entity in a set
has a given relationship, we bold the line describing that relationship. \par
A weak entity is one which can be uniquely identified only with knowledge of
another owner identity. To indicate this, we use a identifying relationship. All
weak entities must have this relationship to their owning identity. To show this
on a diagram, we bold the weak entity, the line (which will always be arrowed,
as each weak entity has one owner) and the relationship diamond and underline
the key of the weak entity to indicate it doesn't uniquely identify the entity.
\par
Some special attribute types exist. These include
\begin{itemize}
    \item A multi-valued attribute, where multiple different pieces of
        information of the same type are stored. For example, all of the phone
        numbers associated with a person. This is depicted with two concentric
        ovals rather than a single oval.
    \item A composite attribute, which is just an attribute with sub-attributes.
        Shown by given the attributes attributes in the same way one would
        entities.
\end{itemize}

\subsubsection*{Data Modelling}

A data model allows the translation of real world concepts into computational
structures. One form of this is relational modelling, as in a relational data
base. A relationship between tables in a relational database takes the form of
a foreign or primary key. \par
A relational database is defined by a schema, which is a specification of
relations between tables and name and type of each column in a table. \par
The \textit{cardinality} of a table refers to its number of records. The degree
or \textit{arity} of a table refers to its number of rows.

\subsubsection*{ER to Relational Models}

An entity in an entity-relationship model becomes a table in a relational
database. This is the transition from conceptual design to logical design. To
progress from this stage to physical design, we choose datatypes for the
columns. The implementation stage is then undertaken by creating the related
SQL for this physical design. \par
To implement our relationships and constraints we need to use keys. A key is a
form of integrity restraint. A primary key enforces uniqueness in a table. A
foreign key ensures that only existing records can be referenced from other
tables. Some important definitions exist for keys.
\begin{itemize}
    \item A set of fields is a \textit{superkey} if no two distinct tuples can
        have the same values in all key fields.
    \item A set of fields is a \textit{key} if it is a superkey and no subset of
        the set is a superkey.
    \item Of all keys, one is chosen as the \textit{primary key}.
    \item Non-chosen possible primary keys are \textit{candidate keys}.
\end{itemize}
A foreign key must always point to another table's primary key. A database has
referential integrity if all foreign keys are enforced. \par
An integrity constraint is a condition that is applied to a column, such as a
domain constraint for a column. A legal instance of a relation (table) satisfies
all such constraints. \par
For multi-valued attributes, we often need to split them
up into multiple attributes such as a home and work phone for phone numbers.
Likewise, for composite attributes, we simply flatten them and include all
columns in the relevant table. \par
To create relationships we often create new tables where we take a primary key
from each of the involved tables and add any relevant data about the
relationship as additional columns. In instances where we have a ``one''
relation it is generally preferred to include this key in the table with the
singleton relationship; succinctly ``the primary key from the many side becomes
a foreign key on the one side''. If the key is mandatory, we simply mark it as
\code{NOT NULL} in our SQL. \par
For a weak entity which exists only in its relation to another entity, we
implement cascading deletion.

\begin{lstlisting}
    CREATE TABLE dependent (
        name CHAR(20) NOT NULL,
        age INTEGER NULL,
        cost DECIMAL(7,2) NOT NULL,
        employee_id CHAR(11) NOT NULL,
        PRIMARY KEY (employee_id, name),
        FOREIGN KEY (employee_id) REFERENCES employees ON DELETE CASCADE
    );
\end{lstlisting}

In the above example we have a table storing employee insurance dependents,
which are meaningless without their related employees.

\end{flushleft}
\end{document}
