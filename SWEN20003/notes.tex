\documentclass[12pt]{report}
\renewcommand{\familydefault}{\sfdefault}
\usepackage{xcolor}
\usepackage{listings}

\lstset{
    frame = tb,
    language = Java,
    aboveskip = 3mm,
    belowskip = 3mm,
    showstringspaces = false,
    columns = flexible,
    basicstyle = {\small\ttfamily},
    numbers = none,
    numberstyle = \tiny\color{gray},
    keywordstyle = \color{blue},
    commentstyle = \color{darkgray},
    stringstyle = \color{orange},
    breaklines = true,
    breakatwhitespace = true,
    tabsize = 3
}

% \newenvironment{codeblock}{\begin{lstlisting}}{\end{lstlisting}}
\newcommand{\code}[1]{\lstinline{#1}}

\begin{document}
\begin{flushleft}
\section*{Java}    

Java was developed by Sun Microsystems, under James Gosling in 1991.
The goal was to develop a language suitable for applications in embedded 
systems. This use case failed to manifest, and they instead reoriented
Java as a language to develop browser applets.

\bigskip
On arrival Java had the unique feature that it was both compiled and
interpreted. It was also platform-independent and portable. Finally it was
object oriented.

\bigskip
While with a C program, plain text code is compiled by a compiler and then
linked by a linker which produces a different executable for every platform.
For Java, plain text source code is compiled and output as byte code, which
is directly interpreted by a Java interpreter. Therefore, all that is needed
to run Java on a given platform is an interpreter. This gives Java its 
platform independence.

\bigskip
In Java, two types of programs exist.
\begin{itemize}
    \item An \textit{application} is a stand-alone program, which has a main 
        method and can be run directly as a program, for example from the 
        command line.
    \item An \textit{applet} is a program, initially intended for use in 
        webpages which has no main method, and instead has a specific 
        contruction which allows the environment to function. This form is
        less popular, generally considered outdated, and is not covered in
        detail in this subject. 
\end{itemize}

\bigskip
The fundamental constructs in Java, and in all object oriented languages
are classes, objects and methods.

\bigskip
Below lies a ``Hello World'' program written in Java.


\begin{lstlisting}
    import java.lang.*;

    // Display "Hello World"
    public class HelloWorld {
        public static void main(String args[]) {
            System.out.println("Hello World");
            return;
        }
    }
\end{lstlisting}

A variety of interesting things are visible in even this small snippet of
code.
\begin{itemize}
    \item The whole program is encased in a class; here \code{HelloWorld}.
    \item The \code{main} function is prefaced by \code{public static}.
    \item The \code{main} function has return type \code{void}.
    \item Where in C we might \code{#include} here we \code{import}. 
\end{itemize}
Aside these differences, the code is rather similar to a comparable C program.

\bigskip
Three types of comments exist in java.
\begin{lstlisting}
    // This is a single line comment, as in C
    /* This is a multi-line comment, as in C */
    /** This is a documentation comment, to be explored later */
\end{lstlisting}

\bigskip
To run a Java program that has been saved as plain text, a few conditions must
be satisfied.
\begin{itemize}
    \item The filename must be the same as the name of the class. For example,
        our ``Hello World'' program would be \code{HelloWorld.java}.
    \item Java build and runtime environment must be installed on the machine.
        This can be checked with \code{javac -version} or \code{java -version}.
\end{itemize}
Then, one can compile with \code{javac HelloWorld.java}. If successful, an 
output file \code{HelloWorld.class} will be generated. This can be run with
\code{java HelloWorld}.

\bigskip
Arguments passed in at program execution can be accessed in the \code{args[]} 
array. The number of arguments can be found through \code{args.length}.

\bigskip
Key differences between Java and C
\begin{itemize}
    \item While C is a procedural language, Java is object oriented.
    \item Java lacks many of the lower level facilities of C; goto, sizeof,
        pointers, etc. No direct memory management.
    \item Java has no preprocessor, macros, defines, etc are not available.
\end{itemize}

\section*{Classes and Objects}
All programming languages have four fundamental operations
\begin{itemize}
    \item Calculation
    \item Selection
    \item Iteration
    \item Abstraction
\end{itemize}
The key focus of this subject is in abstraction, the process of creating
constructs and defining interactions between them to solve a problem. The focus
of oriented oriented programming is on a specific form of abstraction. In a
procedural language like C, abstraction is provided largely through functions.
In Java, abstraction is implemented through classes, which are a form of
abstract data type. \par
A class might represent a real world object, a real world concept or a problem
space object or concept. It has attributes and methods, which define its 
characteristics and behaviours. This class can then be used as a data type.

\bigskip
An \textit{object} is an instance of a class. Many objects may be instances of
the same class, with differing attributes which define their state. An objects
type is the class of which it is an instance.

\bigskip
When designing an application in an object oriented way, one should focus 
initially on nouns; if a concrete concept or object is required for the 
application, then it should be a strong candidate for a class in the program.
The properties of this entity should then become attributes; colour, radius, 
etc. Things that this entity might do should then become methods; move, open,
save, etc.

\bigskip
Features of designing in this way include
\begin{itemize}
    \item Data Abstraction; through creating data types we can simplify the
        manipulation and storage of information around an entity. 
    \item Encapsulation; by creating classes, we group related properties
        behind a common interface, often simplifying problems.
    \item Information hiding
    \item Delegation
    \item Inheritance
    \item Polymorphism
\end{itemize}

\subsection*{Classes}
\subsubsection*{Defining a Class}

The format for a class definition is as follows. The class must be placed in
the file \code{Name.java}. In general classes are named in camelcase with a
leading capital (\code{ClassName}), while attributes and methods are named in 
conventional camelcase (\code{methodName}).
\begin{lstlisting}
    <visibility> class <Name> {
        <attribute declarations>
            <visibility> <type> <name>
        <method declarations>
            <visibility> <void or type> <name> (<arguments>)
    }
\end{lstlisting}
A simple class example follows.
\begin{lstlisting}
    // in Person.java
    public class Person {
        public String firstName;
        public String lastName;

        public String getFullName () {
            return firstName + lastName;
        }
    }
\end{lstlisting}
Attributes included as part of a class are instance variables, differentiated
from local variables defined in the namespace of a method. These variables are
members of the class.

\subsubsection*{Using a Class}

Once a file with a class has been compiled (with \code{javac Class.java}),
other files in the same directory as the \code{.class} file can use the type
without imports. Creating an object reference can then be done with 
\code{Class obj}. Until this object is \textit{instantiated}, it will simply be
a \code{null} reference.

\begin{lstlisting}
    public class Main {
        public static void main(String args[]) {
            Class obja, objb; // obja is a null reference
            obja = new Class(); // now Class instance
            objb = new Class();
            obja = objb; // obja is now a reference to objb
        }
    }
\end{lstlisting}

In the above example, two instances of \code{Class} are created, and then one
is overwritten with another. Because no reference exists to \code{obja}, it 
will be garbage collected. \par
Members of classes are accessed with the member accessor, \code{.}. For 
example, \code{object.attr} or \code{object.method()}. The attributes of an
object can be assigned in this way, i.e. \code{object.attr = value}.

\bigskip
The main method, while simply a method of a class like any other, is special, 
as it and it alone can be the entry point of a program.

\end{flushleft}
\end{document}
