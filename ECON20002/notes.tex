\documentclass[12pt]{report}

\usepackage{import}
\import{../}{preamble.tex}

\begin{document}
\begin{flushleft}
\subsubsection*{Models}
This subject deals with models; simplifications and idealisations of real world
situations. While these might not be directly useful, they can help us gain an
understanding as to how a more realistic situation might pan out.

\subsection*{Supply and Demand}
\subsubsection*{Obesity}
An example application of supply and demand can be observed in the rise of
obesity among people in Australia in the recent past. This is a social problem
which has implications for the economy and for government policy. A possible
policy to counteract this issue is taxation of junk food. \par
The natural implication of taxation is that an increased price due to the tax
will result in a lower equilibrium consumption quantity. We can consider the a
model of supply in the junk food market to see the impacts. If we take a supply
curve, which is upward sloping as a higher price implies a greater number of 
suppliers with incentive to produce such as \(Q_S = P\) and a demand curve like
\(Q_D = c - P\) (negative, as higher price disincentivises purchase), we can
plot the two to find an equilibrium price and quantity level. 

\begin{econplot}{\(Q\)}{\(P\)}
    \addplot[ultra thick, red, ->, name path = qs] {x}
        node[below right, pos=0.95] {\(Q_S\)};
    \addplot[ultra thick, blue, ->, name path = qd] {5 - x}
        node[above right, pos=0.95] {\(Q_D\)};
    \addplot[ultra thick, red, ->, domain={0:4}] {x + 1.5}
        node[above left, pos=0.95] {\(Q_{ST}\)};
    \addplot[black, dashed] coordinates {(3.3, 3.3)(3.3, 4.8)}
        node[left, pos=0.4] {\(T_D \uparrow\)};
    \addplot[black, dashed] coordinates {(3.3, 4.8)(4.8, 4.8)}
        node[above, pos=0.5] {\(\leftarrow T_S\)};
    \addplot[black, dashed] coordinates {(2.5, 0)(2.5, 3)}
        node[above] {\(Q_0\)};
    \addplot[black, dashed] coordinates {(0, 2.5)(3, 2.5)}
        node[right] {\(P_0\)};
    \addplot[black, dashed] coordinates {(1.75, 0)(1.75, 3.75)}
        node[above] {\(Q_1\)};
    \addplot[black, dashed] coordinates {(0, 3.25)(1.75, 3.25)}
        node[above, pos=0.2] {\(P_1\)};
    \addplot[fill=gray, opacity=0.2] fill between [
        of = qd and qs,
        soft clip = {domain=1.75:2.5}
    ];
\end{econplot}

If we want to desire the consumption level to some defined quantity \(Q_1\), we
need to increase the price by a relevant amount. If we charge a flat tax of some
amount \(T\), we will find that the price increases by exactly \(T\). If this
tax is applied on the supplier side, we will observe a reduction in quantity
from \(Q_S\) to \(Q_ST\), as shown by \(T_S\). If it is applied on the demand
side, we will observe an increase in price presenting an effective supply curve
of \(Q_ST\), as shown by \(T_D\). In either case, the gray shaded region
represents a deadweight loss of welfare and the tax revenue is given by
\(T \times Q\). \par
While conceptually this makes sense to us, in reality studies have found that
such taxes have a smaller effect than predicted on consumption of junk food.
This suggests that junk food is actually highly price inelastic at current
quantities; it is not strongly affected by the price.

\subsubsection*{Price Elasticity of Demand}
Price elasticity of demand is given by change in quantity per unit quantity
divided by change in price per unit price. That is
\[E_P = \DD{Q / Q}{P / P} = \DD{Q}{P} \times \frac{P}{Q}\]
Which is simply the inverse of the slope of the demand curve multiplied by the
ratio of price to quantity. This means that at low quantities, the ratio of
price to quantity will be larger so the elasticity will be higher, as for a
linear curve, the inverse of the slope is a constant. Generally elasticity is
always negative, so we simply use the absolute value. Depending on the value of
price elasticity we use certain terminology
\begin{itemize}
    \item If \(\abs{E_P} > 1\) we say that demand is price elastic, which means
        that a given percentage change in price will lead to a greater
        percentage change in quantity.
    \item If \(\abs{E_P < 1}\) we say that demand is price inelastic, suggesting
        that a percentage change to price will produce a smaller percentage
        change in quantity.
\end{itemize}

\subsubsection*{Individual Choice}
In reality, individuals might maximise welfare through different means,
different quantities of junk food, etc. In microeconomics, agents need to make
decisions between goods. Here, we will consider how agents will find an optimal
basket of two goods. We can consider an example of how many hours one should
study. If the benefit derived from \(h\) study hours is
\[B = \omega h\]
Where \(\omega\) is some constant. Then our marginal benefit of studying is
\[\dd{B}{h} = \omega\]
If this was the only information we had, then it would be optimal to spend all
time studying. However, we must also consider the costs involved in studying
some amount, which we can model as
\[C = \frac{h^2}{4} \Rightarrow MC = \dd{C}{h} = \frac{1}{2}h\]
The net benefit of one hour of study is then
\(B - C = B_N = \omega h - \frac{h^2}{4}\).
We can find the point to maximise benefit by setting the derivative of net
benefit equal to zero.
\[\dd{B_N}{h} = 0 \Rightarrow \omega - \frac{1}{2}h = 0
\Rightarrow h = 2\omega\]
This is equivalent to finding \(B^\prime = C^\prime\) or \(MB = MC\). \par
This can be generally applied; if we want to incentivise an action, we should
lower the marginal cost of this action and increase its marginal benefit.

\subsubsection*{Short-Termism}
In some settings, we find that people consistently fail to make decisions which
maximise their welfare. For example, in Africa governments often heavily
subsidise mosquito nets and yet adoption is still fairly lower, because people
don't consider the future benefit sufficiently. \par
We can model this as a \textit{discount factor}, \(V\) which influences our
evaluation of future benefit from a present investment. For a future benefit of
magnitude \(G\), at a present price of \(F\), an individual will only make the
transaction if for them
\[\frac{G}{V} \geq F\]
Thus, individuals with higher discount factors are likely to be disincentivised
from making decisions that may appear to lead to a large increase in welfare for
them in the future.
\end{flushleft}
\end{document}
