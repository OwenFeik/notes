\documentclass[12pt]{report}

\usepackage{import}
\import{../}{preamble.tex}

\begin{document}
\begin{flushleft}
\subsubsection*{Models}
This subject deals with models; simplifications and idealisations of real world
situations. While these might not be directly useful, they can help us gain an
understanding as to how a more realistic situation might pan out.

\subsection*{Supply and Demand}
\subsubsection*{Obesity}
An example application of supply and demand can be observed in the rise of
obesity among people in Australia in the recent past. This is a social problem
which has implications for the economy and for government policy. A possible
policy to counteract this issue is taxation of junk food. \par
The natural implication of taxation is that an increased price due to the tax
will result in a lower equilibrium consumption quantity. We can consider the a
model of supply in the junk food market to see the impacts. If we take a supply
curve, which is upward sloping as a higher price implies a greater number of 
suppliers with incentive to produce such as \(Q_S = P\) and a demand curve like
\(Q_D = c - P\) (negative, as higher price disincentivises purchase), we can
plot the two to find an equilibrium price and quantity level. 

\begin{econplot}{\(Q\)}{\(P\)}
    \addplot[ultra thick, red, ->, name path = qs] {x}
        node[below right, pos=0.95] {\(Q_S\)};
    \addplot[ultra thick, blue, ->, name path = qd] {5 - x}
        node[above right, pos=0.95] {\(Q_D\)};
    \addplot[ultra thick, red, ->, domain={0:4}] {x + 1.5}
        node[above left, pos=0.95] {\(Q_{ST}\)};
    \addplot[black, dashed] coordinates {(3.3, 3.3)(3.3, 4.8)}
        node[left, pos=0.4] {\(T_D \uparrow\)};
    \addplot[black, dashed] coordinates {(3.3, 4.8)(4.8, 4.8)}
        node[above, pos=0.5] {\(\leftarrow T_S\)};
    \addplot[black, dashed] coordinates {(2.5, 0)(2.5, 3)}
        node[above] {\(Q_0\)};
    \addplot[black, dashed] coordinates {(0, 2.5)(3, 2.5)}
        node[right] {\(P_0\)};
    \addplot[black, dashed] coordinates {(1.75, 0)(1.75, 3.75)}
        node[above] {\(Q_1\)};
    \addplot[black, dashed] coordinates {(0, 3.25)(1.75, 3.25)}
        node[above, pos=0.2] {\(P_1\)};
    \addplot[fill=gray, opacity=0.2] fill between [
        of = qd and qs,
        soft clip = {domain=1.75:2.5}
    ];
\end{econplot}

If we want to desire the consumption level to some defined quantity \(Q_1\), we
need to increase the price by a relevant amount. If we charge a flat tax of some
amount \(T\), we will find that the price increases by exactly \(T\). If this
tax is applied on the supplier side, we will observe a reduction in quantity
from \(Q_S\) to \(Q_ST\), as shown by \(T_S\). If it is applied on the demand
side, we will observe an increase in price presenting an effective supply curve
of \(Q_ST\), as shown by \(T_D\). In either case, the gray shaded region
represents a deadweight loss of welfare and the tax revenue is given by
\(T \times Q\). \par
While conceptually this makes sense to us, in reality studies have found that
such taxes have a smaller effect than predicted on consumption of junk food.
This suggests that junk food is actually highly price inelastic at current
quantities; it is not strongly affected by the price.

\subsubsection*{Price Elasticity of Demand}
Price elasticity of demand is given by change in quantity per unit quantity
divided by change in price per unit price. That is
\[E_P = \DD{Q / Q}{P / P} = \DD{Q}{P} \times \frac{P}{Q}\]
Which is simply the inverse of the slope of the demand curve multiplied by the
ratio of price to quantity. This means that at low quantities, the ratio of
price to quantity will be larger so the elasticity will be higher, as for a
linear curve, the inverse of the slope is a constant. Generally elasticity is
always negative, so we simply use the absolute value. Depending on the value of
price elasticity we use certain terminology
\begin{itemize}
    \item If \(\abs{E_P} > 1\) we say that demand is price elastic, which means
        that a given percentage change in price will lead to a greater
        percentage change in quantity.
    \item If \(\abs{E_P < 1}\) we say that demand is price inelastic, suggesting
        that a percentage change to price will produce a smaller percentage
        change in quantity.
\end{itemize}

\subsubsection*{Individual Choice}
In reality, individuals might maximise welfare through different means,
different quantities of junk food, etc. In microeconomics, agents need to make
decisions between goods. Here, we will consider how agents will find an optimal
basket of two goods. We can consider an example of how many hours one should
study. If the benefit derived from \(h\) study hours is
\[B = \omega h\]
Where \(\omega\) is some constant. Then our marginal benefit of studying is
\[\dd{B}{h} = \omega\]
If this was the only information we had, then it would be optimal to spend all
time studying. However, we must also consider the costs involved in studying
some amount, which we can model as
\[C = \frac{h^2}{4} \Rightarrow MC = \dd{C}{h} = \frac{1}{2}h\]
The net benefit of one hour of study is then
\(B - C = B_N = \omega h - \frac{h^2}{4}\).
We can find the point to maximise benefit by setting the derivative of net
benefit equal to zero.
\[\dd{B_N}{h} = 0 \Rightarrow \omega - \frac{1}{2}h = 0
\Rightarrow h = 2\omega\]
This is equivalent to finding \(B^\prime = C^\prime\) or \(MB = MC\). \par
This can be generally applied; if we want to incentivise an action, we should
lower the marginal cost of this action and increase its marginal benefit.

\subsubsection*{Short-Termism}
In some settings, we find that people consistently fail to make decisions which
maximise their welfare. For example, in Africa governments often heavily
subsidise mosquito nets and yet adoption is still fairly lower, because people
don't consider the future benefit sufficiently. \par
We can model this as a \textit{discount factor}, \(V\) which influences our
evaluation of future benefit from a present investment. For a future benefit of
magnitude \(G\), at a present price of \(F\), an individual will only make the
transaction if for them
\[\frac{G}{V} \geq F\]
Thus, individuals with higher discount factors are likely to be disincentivised
from making decisions that may appear to lead to a large increase in welfare for
them in the future.

\subsection*{Consumer Preferences}
In Australia, we have a flat 10\% GST across most goods and services, with some
exemptions for food, health, education, etc. Some probolems exist with this
system, with parties arguing that it can be complex and distortionary. A
proposed reform is the removal of exemptions, with the suggestion for
compensation of disproportionately affected poor person by transfer payment.
\par
To determine an appropriate magnitude for such a payment, we would need to
realistically estimate the basket of goods consumed by such a recipient. The
problem being solved here is a constrained maximisation exercise. \par
We can consider a consumer trying to find an optimal composition of goods
between food and other goods.

\begin{center}    
    \begin{tabular}{c|c|c}
        Basket & Other & Food \\
        \hline
        \(A\) & \(4\) & \(8\) \\
        \(B\) & \(8\) & \(2\) \\
        \(C\) & \(6\) & \(5\) \\
        \(D\) & \(4\) & \(5\) \\
        \(E\) & \(7\) & \(6\) \\
    \end{tabular}
\end{center}

In the above example, it is obvious that basket \(E\) is strictly than \(C\)
which in turn is strictly greater than \(D\). We can say that there should be
a \textit{strict preference} for \(E > C > D\). It is not obvious however how
\(E\) should be compared to \(A\) or \(B\). \par
For this problem, we introduce some special notation for comparison of various
baskets.
\begin{itemize}
    \item The statement \(A \succ B\) indicates a preference for \(A\) over
        \(B\). In the above case, this means that the agent would happily
        sacrifice \(4\) units of other goods for \(6\) units of food.
    \item The statement \(A \sim B\) means that there is no strong preference
        between \(A\) and \(B\).
    \item \(A \succsim B\) suggests that \(A\) is as good or better than \(B\).
        In this case \(B \nsucc A\). The symbol behaves similarly to \(\geq\).
\end{itemize}
These assumptions have the following assumptions
\begin{itemize}
    \item Completeness; a consumer can always assign an operator to compare two
        baskets. They can pick and choose between all available baskets.
    \item Transitivity; \(A \succ B\) and \(B \succ C\) means \(A \succ C\).
    \item More is better (ceteris paribus)
    \item Convexity; consumers prefer a combination of goods over an equivalent
        value of a single good.
\end{itemize}

\subsubsection*{Indifference Curves}

An indifference curve is a curve drawn on a plot of two different goods that
describes the set of good baskets that the consumer is indifferent between.

\begin{econplot}{\(J\)}{\(F\)}
    \labelledpoint{2.67, 2.64}{above right}{\(B\)};
    \labelledpoint{1.5, 4}{above right}{\(A\)};
    \labelledpoint{3.5, 3.2}{right}{\(\:C\)};
    \addplot[black, dashed, name path = n] coordinates {(0, 4)(5, 4)};
    \addplot[black, dashed, name path = m] coordinates {(1.5, 0)(1.5, 6)};
    \path[name path = bot] (0, 0) -- (1.5, 0);
    \addplot[fill=red, opacity=0.2] fill between [
        of = n and bot,
        soft clip = {domain=0:1.5}
    ];
    \path[name path = top] (1.5, 6) -- (5, 6);
    \addplot[fill=blue, opacity=0.2] fill between [
        of = n and top,
        soft clip = {domain=1.5:5}
    ];
    \addplot[ultra thick, black, <->] {2.618^(-1*(x - 2.22)) + 2};
\end{econplot}

In the above plot, the black line indicates the possible baskets of the goods
\(J\) and \(F\) that a hypothetical consumer is indifferent between. Because
\(A\) and \(B\) lie on this line, we know \(A \sim B\). The blue shaded region
indicates baskets of goods \(\succ A\), as they have more of both goods than
\(A\) does. The red shaded region is the baskets \(\nsucc A\). If we considered
the equivalent regions for \(B\), we see that \(C\) can never lie on the
indifference curve as \(C \succ B\). \par
For the same reason that \(C \succ B\), we
know that two indifference curves can never intersect as all indifference curves
must be at different value levels. Thus, we can view a set of indifference
curves as a kind of welfare topographic map for the consumer, known as an
\textit{indifference map}.

\subsubsection*{Marginal Rate of Substitution}

An indifference curve shows us that to gain more of one good at the same level
of satisfaction, the consumer must give up some of the other good. To generalise
this concept we use the marginal rate of substitution, \(MRS\). So for a curve
of \(F\) and \(J\),
\[MRS_{FJ} = -\dd{F}{J}\]
Note that this is in terms of the amount of the vertical axis good \(F\) a
consumer is willing to sacrifice for a unit of horizontal good \(J\). As we move
to higher \(J\) values, the consumer is willing to give up less \(F\) for more
\(J\). \par
This system works well for most goods, however for some goods which are
\textit{perfect complements} like left and right shoes, a consumer always wants
an exactly equal quantity of the two goods.

\end{flushleft}
\end{document}
