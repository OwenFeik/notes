\documentclass[12pt]{report}
\renewcommand{\familydefault}{\sfdefault}

\usepackage{commath}
\usepackage{pgfplots}

\newenvironment{simpleplot}[2]{
    \begin{center}
        \begin{tikzpicture}
            \begin{axis}[
                xmin = 0,
                xmax = 6,
                ymin = 0,
                ymax = 6,
                width = 12 cm,
                height = 9 cm,
                xlabel = #1,
                xlabel near ticks,
                xtick style = {draw = none},
                xticklabels = \empty,
                axis x line = bottom,
                ylabel = #2,
                ylabel near ticks,
                ytick style = {draw = none},
                yticklabels = \empty,
                axis y line = left,
                samples = 500
            ]
}{
            \end{axis}
        \end{tikzpicture}
    \end {center}
}

\begin{document}
\begin{flushleft}
    
\section*{What is Macroeconomics?}
Macroeconomics is the study of \textit{aggregate economic fluctuations}. Key
points of study are features like output, details of the labour market like 
unemployment and inflation. \par
Through studying it, we can examine how governments should respond to 
recessions why countries are rich or poor and how policy can interact with 
economic structures. \par
A major event leading to the development of macroeconomics as a field was the
greate depression, which lead John Maynard Keynes to develop Keynesian theory,
a fundamental part of macroeconomics, and covered in some detail in this 
course.

\bigskip
Major topics of this subject include
\begin{itemize}
    \item Measurement and evaluation of economies
    \item Short or long run economic cycles
    \item Long term growth and development
    \item Effects and behaviour of the global economy
\end{itemize}

\section*{Measurement}
Understanding the state of an economy relies upon accurate measurement of it,
and this understanding is essential to effectively regulating and improving an
economy. Theories with respect to measurement of economies are developed 
according to the study of data.

\subsection*{GDP}
GDP or Gross Domestic Product is a measure of the output of an economy over 
time. The precise definition used in this course is that GDP is the 
\textit{market value} of the \textit{final goods} produced in a country 
during a given time period. GDP is a useful measure of market activity, which
can be used to document change in an economy, compare between economies, and
evaluate the performance of an economy with respect to assorted variables. GDP
is a kind of summary of the performance of an economy. \par
Richard Stone, one of the core forces behind the development of the concept of
GDP was awarded the 1984 Nobel Prize in economics for his efforts.

\subsubsection*{Market Value}
Market value is a way to describe the output of an economy across a wide range
of different goods. The most common form of market value is the financial value
of goods. While this measure is generally quite good, it ignores household 
production as well as grey or black market activities. \par
Because government production has no market price, it is usually valued at
cost of production. This covers public goods such as education, defence and 
public health services.

\subsubsection*{Final Goods and Services}
GDP calculations consider only end products, ignoring constituent goods. For 
example while the loaf of bread sold by a bakery is counted, the flour 
purchased by the bakery and the wheat used to produce that flower is not. This
is done to avoid double counting the contributions of earlier stages. \par
In addition, resold goods are ignored. Second-hand houses or cars are ignored.
Financial assets, such as shares are not a good or service, so they are also
ignored.

\bigskip
GDP considers production in a region; goods produced by international 
businesses in Australia are counted toward Australian GDP while goods produced
internationally by Australian businesses are not.

\subsubsection*{Measurement of GDP}
There are three methods of calculating GDP. These are
\begin{itemize}
    \item Income
    \item Expenditure
    \item Production
\end{itemize}
They should theoretically be equivalent, but practically vary by small margins
due to differences in collection methodology and other real world 
inconveniences.

\subsubsection*{Income}
Every time a good is purchased, money is transferred from a consumer to a
producer. This money is transferred to the firm, which distributes it between
the workers at the firm and the owners of the firm.
\[Y = wL + rK\]
Here, \(Y\) is the GDP, \(w\) is the labour income, \(L\) is the hours worked,
\(r\) is the rental rate and \(K\) is the total capital. Rental rate describes
the amount a production owner earns per unit capital invested into a business.
\par
This method allows examination of the division between capital owners and 
labour providers (workers). Through this examination, one can observe that over
time the share of wealth earned by workers has declined.

\subsubsection*{Expenditure}
The expenditure method examines the amount spent by each group from among four
sectors in the local economy. Each of these is a subtly different form of 
spending.
\begin{itemize}
    \item Houshold sector, which accounts for everyday purchases.
    \item Business sector, which covers investment by businesses.
    \item Government sector, which includes government expenditure less transer
        payments like pension or unemployment benefits.
    \item Overseas sector, which totals net exports (exports less imports).
\end{itemize}
The equation for this sum is
\[Y = C + I + G + X - M\]
Where once again \(Y\) denotes GDP, \(X\) denotes exports, \(M\) is imports and
the other symbols are the remaining three sectors. \par
This measure has the benefit of allowing the examination of consumption versus
investment for instance.

\subsubsection*{Production}
The production method uses a \textit{value added approach}, examining the 
amount of value each producer in an economy. For example, considering a \$3 
loaf of bread.
\begin{itemize}
    \item The bread may have been produced from \$2 worth of flour. Therefore
        the baker add \$1 in value.
    \item The flour may have been milled from \$1 worth of wheat. Therefore the
        miller added \$1 in value.
    \item The farmer, in producing the wheat created \$1 worth of value.
    \item Thus, the total value added throughout production of the loaf of 
        bread was \$3.
\end{itemize}
This approach allows easy comparison between different industries in an 
economy, by considering how much value was added by each industry.

\subsubsection*{Equivalence}
The three methods addressed are in essence equivalent. Value added must be
equivalent to expenditure because someone had to pay for the goods. Because 
someone was payed for the goods, the money must have been distributed to the 
firm, and therefore received as income. \par
An essential caveat for this equivalence to be true is with respect to the
treatment of unsold goods; if a car is produced by not sold, it must be valued
as though it had been purchased in order for the different methods to agree on
GDP; otherwise the value added approach would consider the value added in the
production of the car, while the expenditure approach would not see the 
purchase of the car and would therefore ignore its value, causing a 
discrepancy.

\subsubsection*{Circular Flow of Income}
Another way to understand these different methods is through considering a 
cycle of income between consumers and producers.
\begin{itemize}
    \item Households supply labour and capital to firms, who in exchange 
        provide labour income and payments for capital.
    \item Firms supply finished goods to households, who in turn provide
        capital to firms by purchasing these goods.
\end{itemize}

\subsubsection*{Real GDP}
Thus far, we have examined nominal GDP. While useful, this measure has some 
flaws.
\begin{itemize}
    \item It isn't very effective for considering changes over time.
    \item It doesn't offer very good isolation of variables; changes in GDP
        could be due to change in quantity or price.
\end{itemize}
Real GDP attempts to remedy this by calculating change in quantity while 
disregarding changes in price. The most simple method to do this is to use a
base year for the value of goods, and calculate the value of outputs in later
years using the values of goods set in this base year. This method has the 
benefit of being fairly simple and intuitive, and effectively captures changes
in economic activity over time, however, it can struggle to function well as
more drastic changes occur in an economy. For example
\begin{itemize}
    \item Changes in product preference altering the composition of an economy,
        causing valuations to be non-representative of output.
    \item Proportional changes in pricing between goods causing poor modelling
        using dated price data.
    \item Introduction of new goods which cannot be price using old data.
\end{itemize}
We can calculate GDP in this way by summing the quantity of each good produced
in a year multiplied by the value of that good in the starting year. So for
a year \(t\) and goods \(i\):
\[\mathrm{Real}\:\mathrm{GDP} 
    = \frac{\sum_i p_{i,0}q_{i,t}}{\sum_i p_{i,0}q_{i,0}}\]
Here \(\sum_i\) indicates iteration across all goods \(i\), for each of which
a \(p_{i,0}\) exists, which is the price of the good in year \(0\) and a 
\(q_{i,t}\) exists, which is the quantity produced in year \(t\). We divide 
through by the GDP of year \(0\) to create an \textit{index value}, a ratio
to the base year i.e. \(1 +\) the percentage change in market value, when 
valuing at year \(0\) pricing. Another way to consider this is as the 
cumulative product of these ratios across the years \(\set{0 \ldots t}\).
If \(t = 3\) for instance,
\[\mathrm{Real}\:\mathrm{GDP} 
= \frac{\sum_i p_{i,0}q_{i,3}}{\sum_i p_{i,0}q_{i,0}} = 
\frac{\sum_i p_{i,0}q_{i,1}}{\sum_i p_{i,0}q_{i,0}} \cdot 
\frac{\sum_i p_{i,0}q_{i,2}}{\sum_i p_{i,0}q_{i,1}} \cdot
\frac{\sum_i p_{i,0}q_{i,3}}{\sum_i p_{i,0}q_{i,2}}\]
The modern approach to calculating real GDP uses \textit{chain weighted prices}
to compensate for the drift in accuracy caused by outdated pricing data. Rather
than using \(p_{i,0}\) for the entire calculation, this approach iteratively
calculates the real GDP based on the previous years GDP according to
\[\frac{\sum_i p_{i,t - 1}q_{i,t}}{\sum_i p_{i,t - 1}q_{i,t - 1}}\]
So for the example of \(t = 3\), the full calculation would look like
\[\mathrm{Real}\:\mathrm{GDP} = 
\frac{\sum_i p_{i,0}q_{i,1}}{\sum_i p_{i,0}q_{i,0}} \cdot 
\frac{\sum_i p_{i,1}q_{i,2}}{\sum_i p_{i,1}q_{i,1}} \cdot
\frac{\sum_i p_{i,2}q_{i,3}}{\sum_i p_{i,2}q_{i,2}}\]
This approach prevents outdated price data from skewing the GDP figure, while
still maintaining the measure as relative to changes in production levels.

\subsection*{Inflation}

Inflation is important to consider in macroeconomics, because it has some 
important implications for the economy. High inflation imposes serious costs
upon an economy. It provides information about the performance of an economy.
It also has impacts on the behaviour of price indexes, such as in GDP. Several
measures for measuring inflation exist, one of which is the Consumer Price
Index or CPI.

\subsubsection*{Measurement of Inflation}

The CPI measures the cost of purchasing a specific basket of goods and services
relative to a base year. This entails collecting data on prices of goods over 
time, collecting data on household expenditure, to figure what a reasonable 
basket to measure is, and do this regularly; in Australia the CPI is measured
quarterly. The measurement is performed as follows, similarly to the GDP 
calculation.
\[P_t = \sum_i p_{i,t}q_{i,0}\]
So the total price is given by the sum across the prices of the goods in year
\(t\) in the quantities defined in the base year, \(0\). These goods \(i\) are
drawn from a basket of \(I\) items with associated quantities defined in the 
base year. This can be converted to an index, known as the CPI by dividing by 
the price in the base year.
\[\frac{P_t}{P_0} (\times 100)\]
This values is often multiplied by \(100\) as a matter of preference. The CPI
calculated this way can then be used to calculate the inflation rate through
\[100 \times \frac{\mathrm{CPI}_t - 
\mathrm{CPI}_{t - 4}}{\mathrm{CPI_{t - 4}}}\]
Where \(t\) is the number of quarters since the base year (or base quarter).
This calculation yields the annual inflation rate. The quarterly inflation rate
can be found by simply using the previous quarter instead of the past year.
\[100 \times \frac{\mathrm{CPI}_t - 
\mathrm{CPI}_{t - 1}}{\mathrm{CPI_{t - 1}}}\]
In general the quarterly rate is more relevant for short term considerations,
as it is more volatile and vulnerable to random fluctuations. The approach of 
the CPI can be used to examine inflation in a specific good by simply looking 
at change in price of that good. An example is the education is growing more
inflated while clothing is largely declining. \par
Inflation data is very useful for indexing certain government expenditures, 
like welfare payments. It can be used to compare the value of investments; 
whether its better to receive \(\$x\) than \(\$y\) in the future.
Deflation is the term for negative inflation. \par
Inflation measurement suffers from bias as composition of goods changes over
time, as cheaper goods are substituted for more expensive goods. This implies
an exaggerated CPI. It can also be difficult to compare goods which advance
significantly technically, such as computers.

\subsubsection*{Costs of Inflation}

In general a low rate of inflation (\(\leq 3\%\)) is fairly harmless, but 
higher rates of inflation (\(\geq 10\%\)) can be quite damaging, causing issues
including
\begin{itemize}
    \item Noise in the price system, where the relative prices of goods change
        too rapidly for individuals are firms to properly evaluate different 
        purchases.
    \item Brack creep in nominal tax systems such as in Australia, where 
        inflation can cause incomes to rise to higher rates of taxation, 
        resulting in lower take-home incomes for workers.
    \item Cash assets will depreciate rapidly under high inflation rates.
    \item It can be difficult to plan savings for purposes such as retirement.
    \item Menu costs; costs of updating pricing over time 
    (such as in a restaurant menu).
    \item People might hold too little cash, creating inconveniences.
\end{itemize}

\subsubsection*{Prices and Real and Nominal GDP}
Nominal and real GDP can be linked with a price index.
\[\mathrm{Nominal\:GDP} = \mathrm{Price\:Level}\times\mathrm{Real\:GDP}\]
\[P_tY_t = P_t\times Y_t\]
This equation also holds for growth rates.
\[\mathrm{growth}(P_tY_t) = \mathrm{growth}(P_t) + \mathrm{growth}(Y_t)\]
Here, price level is not the CPI but a slightly different measure known as the
\textit{GDP deflator}.

\subsection*{Interest Rates}

In general, investment in a financial asset requires a turn on investment, an
appreciation. This rate of return is generally described by an interest rate.
This rate affects the decision to invest versus consume or save. \par
A nominal interest rate \(i\) implies that an investment of \(\$1\) will yield
a return of \(\$(1 + i)\). However, this nominal interest rate doesn't quite
provide sufficient information. The real point of interest is not the quantity
of wealth one possesses, but what one could do with that wealth, i.e. its
\textit{purchasing power}. \par
The \textit{real interest rate} reflects the increase in consumption an 
investment returns. When receiving income, one has two choices; to save and
consume later, or to consume immediately. If saving, it can be assumed that
the savings are invested. There is a price level at time of receipt of \(P_0\)
and a price at later time of \(P_1\). \(\$1\) at time \(0\) purchases 
\(\frac{1}{P_0}\) units of consumption bundle, while \(\$(1 + i)\) purchases
\(\frac{1 + i}{P_1}\) units of consumption at a later date. Therefore, the real
rate of interest is given by
\[1 + r = \frac{\frac{1 + i}{P_1}}{\frac{1}{P_0}} = \frac{1 + i}{1 + \pi}\]
Where \(\frac{P_1}{P_0} = 1 + \pi\) and \(\pi\) is the inflation rate.
Therefore, only when the nominal interest rate is greater than the inflation 
rate is the investment good. This equation is known as the Fisher equation,
and can be simplified to
\[r \approx i - \pi\]
Which is a good approximation, only as long as \(i\) and \(\pi\) are quite low,
which will generally be the case in this course. An issue arises when using 
this as a decision making tool when we consider that we don't have perfect 
information about the future. For this, we introduce an expected real interest
rate:
\[r^e = i - \pi^e\]
Where \(\pi^e\) is the expected rate of inflation.

\subsubsection*{Saving}

Consumption and savings are two sides of the same coin; they represent the two
ways one can use their capital. The relevance of this to macroeconomic theory 
is that consumption makes up the majority of GDP in a modern economy. 
Investment too contributes to GDP, often being a good indicator of movements in
productivity over time. In general, savings and investment contribute to future
productivity. \par
Saving is specifically described in macroeconomics as holding capital without 
using it. Investment is described distinctly to this. An individuals wealth is
the sum of an individual's assets, both financial such as shares or savings and
real, such as housing or resaleable goods. Saving increases an individuals 
wealth. If they are consuming more than their income, they are 
\textit{dis-saving}. Wealth has a significant impact on economic decisions. 

\bigskip
A distinction can be made between stock variables and flow variables. A stock
variable describes a level, like a volume in a bath tub while a flow variable
describes how much of something occurs per unit time (usually a quarter), 
parallel to the flow rate of water into a bath tub. Wealth is a stock variable.
Rate of saving is a flow variable. GDP is also a flow variable.

\bigskip
A variety of incentives exist for saving.
\begin{itemize}
    \item Lifecycle saving; people tend to borrow money when their income is
    low and to save money when their income is high. This results in a flatter
    consumption curve, maintaining a consistent quality of life despite 
    changing income.
    \item Precautionary saving; people save for unexpected events.
    \item Bequest saving; people save for the next generation.
\end{itemize}
Several factors are indicative of saving in a group or time.
\begin{itemize}
    \item Real interest rates; opportunity cost of saving. A lower real 
    interest rate incentivises present consumption.
    \item Demographics. Age structure, etc are very important to savings 
    levels.
    \item Feelings about future events. People who foresee instability may have
    more precautionary savings.
\end{itemize}

It is not only households that can save, firms and even the government can save
for future events.

\subsubsection*{Savings and Investment}

In a closed economy, there is no trade. This is a commonly used model to 
understand behaviours of an economy in a simpler setting. In this case,
\[Y = C + I + G\]
GDP is equal to consumption plus investement plus government expenditure. \(C\)
and \(G\) are both considered consumption expenditure. This leaves \(I\) as the
savings in the system, which can then be written as
\[Y - C - G = I = S\]
This essentially states that the lvel of production minus the level of
consumption is equal to the level of savings, which in a closed economy must be
equal to the level of investment. Savings can also be analysed in terms of 
public and private savings.
\[S = Y - C - G - T + T = (Y - C - T) + (T - G)\]
Here, \(T\) stands for total taxation. \(Y - C - T\) is the total private 
saving in the economy and \(T - G\) is the public saving in the economy, 
equivalent to the government surplus.

\subsubsection*{Investment}

\textit{Capital stock} is the stock of durable goods, such as machinery that
exist at a point in time and can be used as part of the production process.
\textit{Investment} describes new expenditure on durable goods that increase
the capital stock. This includes inventory investment such as the production of
goods to be sold in future periods. Notably, purchasing financial goods such
as shares is not investment in a macroeconomic sense.
\[K_{t + 1} = (1 - \delta)K_t + I_t\]
The above equation describes the relationship between capital and investment 
for a future period. The total capital for the next period is given by the 
capital of the present period multiplied by \(1\) minus the depreciation rate
\(\delta\), plus the capital investment undertaken in the period. \par
A variety of factors inform the decisions of a firm to invest. A framework for
this requires several standard assumptions. The first is that capital is costly
to acquire; that firms must pay an interest rate plus a depreciation cost for
using capital. However, a trade off exists as increased capital stock increases
output according to 
\[y = F(k)\]
It is assumed that an increase in \(k\) entails an increase in \(y\) i.e. 
\(F^\prime(k) > 0\). It is also assumed that dimishing returns apply to this
function, i.e. \(F^{\prime\prime}(k) < 0\). Finally, it is assumed that output
is sold at a fixed price. To maximise profit in this instance, the following
equation can be used
\[\Pi(k) = pF(k) - (r + \delta)k\]
Here, \(pF(k) = y\), \(r\) describes the interest payed on capital and \(\Pi\)
describes the total profit for this level of \(k\). To maximise this, we set
the derivative equal to \(0\) i.e.
\[pF^\prime(k) - (r + \delta) = 0\]
Here, the first term is the marginal revenue of capital while the second term
is the marginal cost of capital. Examining this, we can identify the key
determinants of investment decisions.
\begin{itemize}
    \item The marginal product of capital; a higher marginal product encourages
        greater capital investment.
    \item The output price as compared to the capital price; a higher output
        price encourages higher capital investment.
    \item The real interest rate; a higher interest rate encourages lower 
        capital investment.
    \item The rate of depreciation; a higher rate of depreciation also 
        encourages lower capital investment.
\end{itemize}

\begin{simpleplot}{\(k\)}{\(pF^\prime(k)\)}
    \addplot[ultra thick, red, ->] coordinates {(0, 2)(5, 2)}
    node [above, pos = 0.8] {\((r + \delta)\)};
    \addplot[ultra thick, blue, ->] {-2 * (0.5 * x)^(1/2) + 4}
    node [above right, pos = 0.2] {\(pF^\prime(k)\)};
    \addplot[thick, black, dashed] coordinates {(2, 0)(2, 2.5)}
    node [above] {\(k^*\)};
    \node[circle, fill, inner sep = 2pt] at (axis cs: 2, 2) {};
\end{simpleplot}

In the above plot, a function for marginal revenue of capital is plotted 
against another for marginal cost of capital. Their intersection is the optimal
level of capital, \(k^*\). A profit-maximising firm will invest an optimal
quantity \(i^*\) to increase their level of capital to \(k^*\) at each unit
time.

\subsubsection*{Market for Loanable Funds}
In all economies, there are some agents that wish to save and others that wish
to invest. Interest rate is the price that adjusts to equate this market. With
higher interest rates, individual saving is more incentivised, and firm 
investment is disincentivised.

\begin{simpleplot}{\(S(r) / I(r)\)}{\(r\)}
    \addplot[ultra thick, red, ->] {x}
    node [above left, pos = 0.9] {\(S_0(r)\)};
    \addplot[ultra thick, blue, ->] {-1 * x + 5}
    node [below left, pos = 0.7] {\(I(r)\)};
    \addplot[ultra thick, red, ->] {x - 1}
    node [below right, pos = 0.95] {\(S_1(r)\)};
    \addplot[thick, black, dashed] coordinates {(2.5, 0)(2.5, 3.5)}
    node [above] {\(t_0^*\)};
    \addplot[thick, black, dashed] coordinates {(3, 0)(3, 3.5)}
    node [above] {\(t_1^*\)};
\end{simpleplot}

The interest rate is determined by the relative attractiveness of saving \(S\)
and investment \(I\). For example, in the above plot we see that a shock 
incentivising savings, such as an increase in uncertainty about the future,
has the effect of reducing the real interest rate. The real interest rate in
many modern economies, including Australia, is around \(0-1\%\)

\end{flushleft}
\end{document}
