\documentclass[12pt]{report}
\renewcommand{\familydefault}{\sfdefault}

\begin{document}
\begin{flushleft}
    
\section*{What is Macroeconomics?}
Macroeconomics is the study of \textit{aggregate economic fluctuations}. Key
points of study are features like output, details of the labour market like 
unemploment and inflation. \par
Through studying it, we can examine how governments should respond to 
recessions why countries are rich or poor and how policy can interact with 
economic structures. \par
A major event leading to the development of macroeconomics as a field was the
greate depression, which lead John Maynard Keynes to develop Keynesian theory,
a fundamental part of macroeconomics, and covered in some detail in this 
course.

\bigskip
Major topics of this subject include
\begin{itemize}
    \item Measurement and evaluation of economies
    \item Short or long run economic cycles
    \item Long term growth and development
    \item Effects and behaviour of the global economy
\end{itemize}

\section*{Measurement}
Understanding the state of an economy relies upon accurate measurement of it,
and this understanding is essential to effectively regulating and improving an
economy. Theories with respect to measurement of economies are developed 
according to the study of data.

\end{flushleft}
\end{document}
