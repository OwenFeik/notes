\documentclass[12pt]{report}
\renewcommand{\familydefault}{\sfdefault}
\begin{document}
\title{ECON10004 Notes}
\chapter*{Introduction}

\section*{What is economics?}
Economics is a study of choice and of human decision making in the face of scarce resources.
Humans have virtually unlimited resources, but lack the meaning to satisfy these.
Thus, economics is the study of why people/``economic agents'' choose what they do.
A firm, for instance, must decide:
\begin{itemize}
\item What to produce?
\item How to produce it?
\item Who to produce it for?
\end{itemize}
As a social science, economics aims to understand the choices of individuals and how these inform the behaviour of larger groups.

\section*{Microeconomic Theory}
Microeconomics centres around creating models, creating ``rational agents'' built on axioms.
These axioms are as follows: Individual actors are autonomous, individuals have freedom and that individuals matter.
This theory is then compared with real world data and is validated or disproven.
\par
The theory is thus useful for attempting to understand how individual agents can cooperate
with (or counteract) each other within an economic system. 

\chapter*{Prisoner's Dilemma}
The prisoner's dilemma; that of two prisoners who are offered a light charge if neither confesses,
a serious charge if both confess and absolution for the confessor if just one confesses.
This model can (and has) been used to understand a variety of situations from board meetings, 
to economics to diplomacy. It highlights that the individual interest is not always aligned with that of the group.
\par
Under the assumption that the only goal of each individual is personal benefit, it makes sense for both to
choose to confess; if the other chooses not to confess, the confessor gains major benefit, if they choose confess,
 the result is indifferent. Thus, this is a model. However, it does not perfectly represent reality. Many agents are
 ``conditional cooperators''; they want to split only when the other person also splits. Thus their action depends
 explicitly on what they believe the other party will do; just like for most people.
\par
\section*{Lesson 1}
This dilemma highlights that individual incentives can be exploited; the optimal for the prisoners here occurs when neither confesses,
however it is optimal from a theoretical standpoint for both to confess. Thus, the system efficiently manages
incentive to exploit the prisoners. This situation can be used as a model for the design of systems; those where
individual incentives are at odds with a designers intent are likely to be unstable and lead to inefficiences.

\subsubsection{Example}
A local government aiming to reduce carbon emissions by offering a 40\% rebate on vehicles capable 
of using LNG, a lower emission fuel. However, their system was flawed in that it had no usage requirement;
consumers simply purchased \$1000 dollar secondary fuel tanks and where rebated \$20,000+ without actually
using the lower emission alternative.
\par
Thus, this program was a financial and ecological disaster.

\begin{quotation}
``Economics is a highly sophisticated field of thought that is superb at explaining to policymakers precisely 
why the choices they made in the past were wrong. About the future, not so much...''
\par
``However, careful economic analysis does have one important benefit, which is that it can help kill ideas that are completely
logically inconsistent or wildly at variance with the data. This insight covers at least 90 percent of proposed economic policies.''
\par
-- Ben Bernanke
\end{quotation}

\section*{Lesson 2}
In addition, this problem can help us to understand that people will adapt to imperfect systems:
over time, people will come to understand what action they should take in a situation and thus they
will create outcomes that may align with the expected outcome of theory.

\section*{Lesson 3}
In economic systems, we can never know the true, underlying preferences and decision 
making process of an individual; they have private information. This can make it difficult
to design effective centralised systems.
\par
Thus, it can be more efficient to pursue decentralised systems such as markets,
where people act in perfect self interest, thus revealing their underlying incentives, which 
inform their behaviour.

\end{document}
