\documentclass[12pt]{report}
\renewcommand{\familydefault}{\sfdefault}
\begin{document}
\title{ECON10004 Notes}
\chapter*{Introduction}

\section*{What is economics?}
Economics is a study of choice and of human decision making in the face of scarce resources.
Humans have virtually unlimited resources, but lack the meaning to satisfy these.
Thus, economics is the study of why people/``economic agents'' choose what they do.
A firm, for instance, must decide:
\begin{itemize}
    \item What to produce?
    \item How to produce it?
    \item Who to produce it for?
\end{itemize}
As a social science, economics aims to understand the choices of individuals and how these inform the behaviour of larger groups.
Economics is, fundamentally, about human wellbeing. Wellbeing, from the perspective of an economist is the consumption
of goods and services. These goods and services are produced using a variety of resources, including:
\begin{itemize}
    \item Natural reosurces 
    \item Physical capital; machines, infrastructure
    \item Human capital; labour, intellectual labour, etc
    \item Time
\end{itemize}
Because these resources are finite, humanity needs to make decisions on the allocation of these resources to optimise wellbeing.

\section*{Decision Theory}
An economic agent is always making choices; it is trying to select the best option based on a 
cost-benefit analysis. 
\[\mathrm{Net\:benefit}\:NB = \mathrm{Total\:benefit}\:TB - \mathrm{Total\:costs}\:TC\]
For example, a firm's goal is to maximise profits; they will minimise costs and maximise revenue to achieve this goal.

\subsection*{How do we measure costs?}
While people are generally quite good at evaluating benefits, they sometimes struggle when
trying to evaluate costs. Often, people will, when considering a purchase, simply consider the 
explicit cost; the price on the price tag. However, this ignores the opportunity cost; any
purchase is to the exclusion of other purchases.

\subsubsection*{Example}
Consider 2 individuals sharing a home, with two tasks that need to be completed, say 
cooking and laundry. If person \(A\) is good at laundry but poor at cooking, person \(B\)
with the opposite skill set, it is obviously advantageous for them to trade; to each
attend to their favoured task. However, consider the scenario when person \(A\) is better
at both.
\par
We can model this as a "production possibilities frontier"; modelling the amount of each
good they can produce if they perform a mixture of the two tasks for a certain time. This will
form a linear relationship tracking from the \(100:0\) time spent state through the \(50:50\) to the \(0:100\) state. Depending on the 
exact relationship between their capabilities, trade may nonetheless be advantageous for both 
parties; the yield for \(A\) may be substantially greater when say, cooking, per unit time,
compared to laundry, and thus, having \(B\) perform some of the laundry, even if less efficient,
may increase the total consumption of both \(A\) and \(B\), thus increasing both parties overall
wellbeing. 
\par
To find this solution, \(A\) needed to consider the opportunity cost of performing laundry,
which in fact proved to be greater than instead primarily cooking and trading with \(B\) to have the laundry
performed.
\par
In the language of economics, \(A\) has an \textit{absolute advantage} in both cooking and doing laundry,
however \(B\) has a \textit{comparative advantage} in laundry.

\subsubsection*{Definitions}
\begin{quotation}    
Absolute advantage is the ability to produce a good using fewer inputs than another producer.
\end{quotation}
\begin{quotation}
Comparative advantage is the ability to produce a good at a lower opportunity cost than another producer.    
\end{quotation}

\section*{Sunk Costs}
When making a decision, one should ignore any costs that have been incurred prior to a decision.
Because these resources have already be used, the perceived loss of them has no real bearing on 
the future outcome of a decision.

\section*{Opportunity Cost}
Total opportunity costs are composed of two seperate parts:
\begin{itemize}
    \item Direct Opportunity Costs: This is the cost of resources that will be used in the chosen alternative.
    \item Indirect Opportunity Costs: This is the benefits minus the explicit costs of the next best action.
\end{itemize}

\subsubsection*{Example}
Say you have three options;
\begin{itemize}
    \item Option A, worth \$3000, cost of \$2500
    \item Option B, worth \$500, cost \$200
    \item Option C, worth \$100, with a \$20 downpayment already payed
\end{itemize}
Thus, you would select Option A. As B is the next best option
this is your indirect opportunity costs. Because B is worth \$300, the total opportunity cost is equal
to \$2500 + \$300 = \$2800. Thus Option A is still worthwhile.

\section*{Marginal Analysis}
Rather than focussing on total benefits and total costs, we should really focus on Marginal Benefits
and costs. Consider whether consuming / producing \(n\) amount of units will be profitable or costing.
i.e. as long as the Marginal Benefit of and action is greater than the Marginal Cost of that action,
we should take that action as a rational agent.

\subsubsection*{Example}
Say we are hiring workers for a store. These workers are payed \$20/hr. The first worker hired is expected
to increased sales by \$60/hr, with additional workers diminishing in value by \$15/hr until they reach \$0/hr.
Thus, the first worker increases net benefit by \$40/hr, the second by \$25/hr etc. So for the first worker,
the marginal benefit will be \$60/hr, while the marginal cost is a constant, \$20/hr. So looking at the marginal 
benfit vs marginal costs for each worker, workers 1, 2 and 3 will have \(MB > MC\) while worker 4 will have \(MB < MC\), so 
we should hire 3 workers.

\subsection*{Calculating Benefit}
Suppose \(NB(x) = TB(x) - TC(x)\). To maximise \(NB(x)\) using calculus, we can differentiate \(NB(x)\) with respect
to \(x\) and set this to \(0\).
\[\frac{dNB(x)}{dx} = \frac{dTB{x}}{dx} - \frac{dTC(x)}{dx} = 0\]
\[\frac{dNB(x)}{dx} = MB(x) - MC(x) = 0\]
Thus the optimal \(x\) is:
\[MB(x) = MC(x)\]


\section*{Microeconomic Theory}
Microeconomics centres around creating models, creating ``rational agents'' built on axioms.
These axioms are as follows: Individual actors are autonomous, individuals have freedom and that individuals matter.
This theory is then compared with real world data and is validated or disproven.
\par
The theory is thus useful for attempting to understand how individual agents can cooperate
with (or counteract) each other within an economic system.

\chapter*{Prisoner's Dilemma}
The prisoner's dilemma; that of two prisoners who are offered a light charge if neither confesses,
a serious charge if both confess and absolution for the confessor if just one confesses.
This model can (and has) been used to understand a variety of situations from board meetings, 
to economics to diplomacy. It highlights that the individual interest is not always aligned with that of the group.
\par
Under the assumption that the only goal of each individual is personal benefit, it makes sense for both to
choose to confess; if the other chooses not to confess, the confessor gains major benefit, if they choose confess,
 the result is indifferent. Thus, this is a model. However, it does not perfectly represent reality. Many agents are
 ``conditional cooperators''; they want to split only when the other person also splits. Thus their action depends
 explicitly on what they believe the other party will do; just like for most people.
\par
\section*{Lesson 1}
This dilemma highlights that individual incentives can be exploited; the optimal for the prisoners here occurs when neither confesses,
however it is optimal from a theoretical standpoint for both to confess. Thus, the system efficiently manages
incentive to exploit the prisoners. This situation can be used as a model for the design of systems; those where
individual incentives are at odds with a designers intent are likely to be unstable and lead to inefficiences.

\subsubsection{Example}
A local government aiming to reduce carbon emissions by offering a 40\% rebate on vehicles capable 
of using LNG, a lower emission fuel. However, their system was flawed in that it had no usage requirement;
consumers simply purchased \$1000 dollar secondary fuel tanks and where rebated \$20,000+ without actually
using the lower emission alternative.
\par
Thus, this program was a financial and ecological disaster.

\begin{quotation}
``Economics is a highly sophisticated field of thought that is superb at explaining to policymakers precisely 
why the choices they made in the past were wrong. About the future, not so much...''
\par
``However, careful economic analysis does have one important benefit, which is that it can help kill ideas that are completely
logically inconsistent or wildly at variance with the data. This insight covers at least 90 percent of proposed economic policies.''
\par
-- Ben Bernanke
\end{quotation}

\section*{Lesson 2}
In addition, this problem can help us to understand that people will adapt to imperfect systems:
over time, people will come to understand what action they should take in a situation and thus they
will create outcomes that may align with the expected outcome of theory.

\section*{Lesson 3}
In economic systems, we can never know the true, underlying preferences and decision 
making process of an individual; they have private information. This can make it difficult
to design effective centralised systems.
\par
Thus, it can be more efficient to pursue decentralised systems such as markets,
where people act in perfect self interest, thus revealing their underlying incentives, which 
inform their behaviour.

\end{document}
