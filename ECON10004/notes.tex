\documentclass[12pt]{report}
\renewcommand{\familydefault}{\sfdefault}
\usepackage{commath}
\begin{document}
\title{ECON10004 Notes}
\chapter*{Introduction}

\section*{What is economics?}
Economics is a study of choice and of human decision making in the face of scarce resources.
Humans have virtually unlimited resources, but lack the meaning to satisfy these.
Thus, economics is the study of why people/``economic agents'' choose what they do.
A firm, for instance, must decide:
\begin{itemize}
    \item What to produce?
    \item How to produce it?
    \item Who to produce it for?
\end{itemize}
As a social science, economics aims to understand the choices of individuals and how these inform the behaviour of larger groups.
Economics is, fundamentally, about human wellbeing. Wellbeing, from the perspective of an economist is the consumption
of goods and services. These goods and services are produced using a variety of resources, including:
\begin{itemize}
    \item Natural reosurces 
    \item Physical capital; machines, infrastructure
    \item Human capital; labour, intellectual labour, etc
    \item Time
\end{itemize}
Because these resources are finite, humanity needs to make decisions on the allocation of these resources to optimise wellbeing.

\section*{Decision Theory}
An economic agent is always making choices; it is trying to select the best option based on a 
cost-benefit analysis. 
\[\mathrm{Net\:benefit}\:NB = \mathrm{Total\:benefit}\:TB - \mathrm{Total\:costs}\:TC\]
For example, a firm's goal is to maximise profits; they will minimise costs and maximise revenue to achieve this goal.

\subsection*{How do we measure costs?}
While people are generally quite good at evaluating benefits, they sometimes struggle when
trying to evaluate costs. Often, people will, when considering a purchase, simply consider the 
explicit cost; the price on the price tag. However, this ignores the opportunity cost; any
purchase is to the exclusion of other purchases.

\subsubsection*{Example}
Consider 2 individuals sharing a home, with two tasks that need to be completed, say 
cooking and laundry. If person \(A\) is good at laundry but poor at cooking, person \(B\)
with the opposite skill set, it is obviously advantageous for them to trade; to each
attend to their favoured task. However, consider the scenario when person \(A\) is better
at both.
\par
We can model this as a "production possibilities frontier"; modelling the amount of each
good they can produce if they perform a mixture of the two tasks for a certain time. This will
form a linear relationship tracking from the \(100:0\) time spent state through the \(50:50\) to the \(0:100\) state. Depending on the 
exact relationship between their capabilities, trade may nonetheless be advantageous for both 
parties; the yield for \(A\) may be substantially greater when say, cooking, per unit time,
compared to laundry, and thus, having \(B\) perform some of the laundry, even if less efficient,
may increase the total consumption of both \(A\) and \(B\), thus increasing both parties overall
wellbeing. 
\par
To find this solution, \(A\) needed to consider the opportunity cost of performing laundry,
which in fact proved to be greater than instead primarily cooking and trading with \(B\) to have the laundry
performed.
\par
In the language of economics, \(A\) has an \textit{absolute advantage} in both cooking and doing laundry,
however \(B\) has a \textit{comparative advantage} in laundry.

\subsubsection*{Definitions}
\begin{quotation}    
Absolute advantage is the ability to produce a good using fewer inputs than another producer.
\end{quotation}
\begin{quotation}
Comparative advantage is the ability to produce a good at a lower opportunity cost than another producer.    
\end{quotation}

\section*{Sunk Costs}
When making a decision, one should ignore any costs that have been incurred prior to a decision.
Because these resources have already be used, the perceived loss of them has no real bearing on 
the future outcome of a decision.

\section*{Opportunity Cost}
Total opportunity costs are composed of two seperate parts:
\begin{itemize}
    \item Direct Opportunity Costs: This is the cost of resources that will be used in the chosen alternative.
    \item Indirect Opportunity Costs: This is the benefits minus the explicit costs of the next best action.
\end{itemize}

\subsubsection*{Example}
Say you have three options;
\begin{itemize}
    \item Option A, worth \$3000, cost of \$2500
    \item Option B, worth \$500, cost \$200
    \item Option C, worth \$100, with a \$20 downpayment already payed
\end{itemize}
Thus, you would select Option A. As B is the next best option
this is your indirect opportunity costs. Because B is worth \$300, the total opportunity cost is equal
to \$2500 + \$300 = \$2800. Thus Option A is still worthwhile.

\section*{Marginal Analysis}
Rather than focussing on total benefits and total costs, we should really focus on Marginal Benefits
and costs. Consider whether consuming / producing \(n\) amount of units will be profitable or costing.
i.e. as long as the Marginal Benefit of and action is greater than the Marginal Cost of that action,
we should take that action as a rational agent.

\subsubsection*{Example}
Say we are hiring workers for a store. These workers are payed \$20/hr. The first worker hired is expected
to increased sales by \$60/hr, with additional workers diminishing in value by \$15/hr until they reach \$0/hr.
Thus, the first worker increases net benefit by \$40/hr, the second by \$25/hr etc. So for the first worker,
the marginal benefit will be \$60/hr, while the marginal cost is a constant, \$20/hr. So looking at the marginal 
benfit vs marginal costs for each worker, workers 1, 2 and 3 will have \(MB > MC\) while worker 4 will have \(MB < MC\), so 
we should hire 3 workers.

\subsection*{Calculating Benefit}
Suppose \(NB(x) = TB(x) - TC(x)\). To maximise \(NB(x)\) using calculus, we can differentiate \(NB(x)\) with respect
to \(x\) and set this to \(0\).
\[\frac{dNB(x)}{dx} = \frac{dTB{x}}{dx} - \frac{dTC(x)}{dx} = 0\]
\[\frac{dNB(x)}{dx} = MB(x) - MC(x) = 0\]
Thus the optimal \(x\) is:
\[MB(x) = MC(x)\]


\section*{Microeconomic Theory}
Microeconomics centres around creating models, creating ``rational agents'' built on axioms.
These axioms are as follows: Individual actors are autonomous, individuals have freedom and that individuals matter.
This theory is then compared with real world data and is validated or disproven.
\par
The theory is thus useful for attempting to understand how individual agents can cooperate
with (or counteract) each other within an economic system.

\chapter*{Prisoner's Dilemma}
The prisoner's dilemma; that of two prisoners who are offered a light charge if neither confesses,
a serious charge if both confess and absolution for the confessor if just one confesses.
This model can (and has) been used to understand a variety of situations from board meetings, 
to economics to diplomacy. It highlights that the individual interest is not always aligned with that of the group.
\par
Under the assumption that the only goal of each individual is personal benefit, it makes sense for both to
choose to confess; if the other chooses not to confess, the confessor gains major benefit, if they choose confess,
 the result is indifferent. Thus, this is a model. However, it does not perfectly represent reality. Many agents are
 ``conditional cooperators''; they want to split only when the other person also splits. Thus their action depends
 explicitly on what they believe the other party will do; just like for most people.
\par
\section*{Lesson 1}
This dilemma highlights that individual incentives can be exploited; the optimal for the prisoners here occurs when neither confesses,
however it is optimal from a theoretical standpoint for both to confess. Thus, the system efficiently manages
incentive to exploit the prisoners. This situation can be used as a model for the design of systems; those where
individual incentives are at odds with a designers intent are likely to be unstable and lead to inefficiences.

\subsubsection{Example}
A local government aiming to reduce carbon emissions by offering a 40\% rebate on vehicles capable 
of using LNG, a lower emission fuel. However, their system was flawed in that it had no usage requirement;
consumers simply purchased \$1000 dollar secondary fuel tanks and where rebated \$20,000+ without actually
using the lower emission alternative.
\par
Thus, this program was a financial and ecological disaster.

\begin{quotation}
``Economics is a highly sophisticated field of thought that is superb at explaining to policymakers precisely 
why the choices they made in the past were wrong. About the future, not so much...''
\par
``However, careful economic analysis does have one important benefit, which is that it can help kill ideas that are completely
logically inconsistent or wildly at variance with the data. This insight covers at least 90 percent of proposed economic policies.''
\par
-- Ben Bernanke
\end{quotation}

\section*{Lesson 2}
In addition, this problem can help us to understand that people will adapt to imperfect systems:
over time, people will come to understand what action they should take in a situation and thus they
will create outcomes that may align with the expected outcome of theory.

\section*{Lesson 3}
In economic systems, we can never know the true, underlying preferences and decision 
making process of an individual; they have private information. This can make it difficult
to design effective centralised systems.
\par
Thus, it can be more efficient to pursue decentralised systems such as markets,
where people act in perfect self interest, thus revealing their underlying incentives, which 
inform their behaviour.

\chapter*{Markets}

\section*{What is a market?}
A market is a mechanism through which buyers and sellers trade a particular good or service with 
near-identical characteristics. These characteristics can include:
\begin{itemize}
    \item Type
    \item Delivery Location
    \item Grade and Quality
    \item Time
\end{itemize}
For example, oil varies considerably in chemical compoisition (e.g. sulphur content) and density;
usually the ``Brent'' crude oil price is discussed when the price of oil is referred to; it represents
an independent market with a very specific set of characteristics.

\bigskip 
In a market, buyers supply demand and sellers offer supply. Both agents must consider marginal 
cost and benefit when evaluating a transaction. As they are rational economic agents, they
will only transact when \(MB \geq MC\) for both buyer and seller.

\bigskip
In an indealised perfectly competitive market, there are many buyers and sellers trading exactly
identical goods. These factors inform competition because buyers are aware that they can choose from
a variety of sellers with no differentiation of products. In this setting, an individual does not 
have the ability to manipulate the market through their individual actions; the individual seller
is a ``price-taker'', accepting the market price. Although real markets tend not to be \textit{perfectly} 
competitive, they are often highly competitive.

\subsection*{Demand}
The demand curve will map the number of units for which there is demand at a given price.
In general, as price decreases, demand will increase. This is the thesis of the ``Law of Demand''. 
Other factors can, however, also inform demand includings:
\begin{itemize}
    \item Price
    \item Tastes
    \item Price and quality of alternative goods
    \item Income
    \item Future price expectations
    \item Number of buyers
\end{itemize}
How can changes in income alter the characteristics of a demand curve?
\begin{itemize}
    \item For a \textit{normal good}, demand is usually proportional to income.
    \item For an \textit{inferior good}, demand is inversely related to income.
    \item When income changes, we see a \textit{shift} in the demand curve.
\end{itemize}
How about the prices of other goods?
\begin{itemize}
    \item For \textit{substitutes} such as butter and margerine, the price of one good is proportional to demand for the other.
    \item \textit{Complements} have a negative correlation between the price of one good and a complement good.
\end{itemize}
When these prices change, we see a shift in the demand curve.

\subsection*{Supply}
Supply is the complement to demand, representing the quantity of goods sellers have 
on offer for buyers. For a supply curve, price and quantity in general be positively correlated.

\bigskip
Supply can be affected by factors such as technology advances improving manufacturing
efficiency, shifting the supply curve to greater supply. Inversely, supply might be 
negatively impacted by rising prices for input goods (such as commodities).

\subsection*{Market Equilibrium}
Market equilibrium describes a situation when supply and demand are in balance; when the 
quantity supplied is equal to the quantity demanded. This can be observed graphically as
the intersection of the demand and supply curves. 

\bigskip
This is an equilibrium because neither buyers
nor sellers have an incentive to change their behaviour; if price where to increase, the
additional goods would not be sold; if it were to decrease, suppliers would not find it
worthwhile to produce the demanded goods.

\subsubsection*{Example}
If \(Q_D\) is quantity demanded and \(Q_S\) is quantity supplied, given by these equations:
\[Q_D = 120 - 20P\]
\[Q_S = 20P\]
We can solve the system for equilibrium by setting \(Q_D = Q_S\):
\[120 - 20P = 20P \Rightarrow 120 = 40P \Rightarrow P = 3\]
\[Q_D = 120 - 60 = 60\]
Thus, market equilibrium for this system is found at:
\[P^* = 3\]
\[Q^* = 60\]
What happens when the market is not at equilibrium? Well, when \(P \neq P^*\),
competition will tend to drive \(P\) toward \(P^*\). To identify this equilibrium,
we can observe what would happen if \(P\) where to change; if a point is at equilibrium,
it will return to that state after a disturbance.

\bigskip
For example, if the price is too high, demand will be lower. Thus, manufacturers will
have excess stock and will be incentivised to lower their prices to move their goods,
and as rational agents, will do so.

\subsection*{Comparative Statics}
Comparative statics refer to comparing static snapshots of supply/demand curves
and the effects of market shocks, such as e.g. COVID19 on the supply/demand of hand sanitiser.
COVID19 is an example of a positive shock increasing the demand curve; shifting it to the right and 
thus increasing demand at every price point, and causing an increase in \(P^*\). This is referred to as an \textit{exogenous} effect;
i.e. it comes from a source external to the system.

\bigskip
Positive supply shocks increase supply at ever price point; an example of an \textit{exogenous} effect
causing this might be a technological advance making it more affordable to supply goods. The 
reverse might happen if some disaster occured such as storms wiping out banana crops in Queensland.

\bigskip
In the case of both a positive supply shock and a positive demand shock of equivalent magnitude,
the intersection will tend to simply increase in quantity. The increasing supply places negative pressure on the price while the increasing
demand places positive pressure, thus maintaining \(P^*\). If we don't know the relative magnitudes,
we can still say that there will unambiguosly be an increase in demand, but we can't say for sure
what will happen to the price.

\bigskip
An example of a long term shift could be lobster. Historically, they were seen as a disgusting low
grade food which was extremely plentiful. However, over time they became more respected and seen
 as a restaurant delicacy, and they were overharvested leading to a reduction in supply.
 Thus, the demand has increasing radically increasing price and the supply has dropped significantly,
 also driving up the price. This is how a meal that was \$4 in 1870 might cost \$30 or more today.
Because of the competing forces on quantity, it is difficult to be sure of how \(Q*\) has changed.

\section*{Elasticity}

Elasticity measures the responsiveness of quantity demanded to it's determinants. This allows us
to analyse in detail how markets will respond to a given shock, i.e. magnitude in addition to direction.

\subsection*{Demand}

Price elasticity of demand measures the responsiveness of quantity demanded to a change in price. If
\(\Delta P\) denotes change in price and \(\Delta Q_D\) denotes change in quantity demanded then 
\(\frac{\Delta P}{P}\) yields percentage change in price and likewise \(\frac{\Delta Q_D}{Q_D}\) yields
percentage change in \(Q_D\).

\bigskip
Elasticity for \(Q_D\) is then given by:
\[\epsilon_D = \abs{\frac{\Delta P / P}{\Delta Q_D / Q_D}}\]
By convention, it is a positive number. We can take a derivative to determine
\textit{point-price elasticity}. Knowing that the rate of change of \(Q_D\) with
respect to \(P\) is given by \(\frac{dQ_D}{dP}\):
\[\epsilon_D = \abs{\frac{dQ_D}{dP}\frac{P}{Q_D}}\]
This is the elasticity at a particular price, rather than for the entire demand curve.

\subsubsection*{Example}
Consider a demand curve of \(Q_D = 120 - 20P\). Find the point-price elasticity at \(Q_D = 20\):
\[\frac{dQ_D}{dP} = -20\]
\[\epsilon_D = \abs{\frac{dQ_D}{dP}\frac{P}{Q_D}} = \abs{-20(\frac{5}{20})} = 5)\]
and at \(Q_D = 100\):
\[\epsilon_D = \abs{\frac{dQ_D}{dP}\frac{P}{Q_D}} = \abs{-20(\frac{1}{100})} = \frac{1}{5})\]

\subsection*{Common Elasticities}
\begin{itemize}
    \item Perfectly inelastic: \(\epsilon_D = 0\)
    \item Inelastic: \(1 > \epsilon_D > 0\)
    \item Unit elastic: \(\epsilon_D = 1\)
    \item Elastic: \(\infty > \epsilon_D > 1\)
    \item Perfectly elastic: \(\epsilon_D = \infty\)
\end{itemize}

\subsection*{Factors of Elasticity}
A variety of factors can affect elasticity, such as degree of necessity; if a product is necessary,
it's price will affect demand little. Availability of substitutes is another factor; if a product
can be easily replaced by another product, it's likely to be highly elastic.

\bigskip
Elasticity can be used to consider change in revenue cause by a change in price. Understanding that 
a shift upward in price causes a shift downward in demand allows us to see that an increase in 
price will lead to more revenue only if the effect of that change is more significant than the 
effect of the lost demand. Understanding that elasticity denotes how responsive a quantity is 
to the change of it's inputs allows us to derive the folowing:
\[MR(P) = (1 - \epsilon_D)Q_D\]
Thus, the total revenue is maximised at \(\epsilon_D = 1\). For example, increasing price when \(\epsilon_D = 0.8\)
is likely to yield an increase in revenue while doing the same at \(\epsilon_D = 1.8\) is likely to
reduce revenue. 

\bigskip
Elasticity can be considered not only for price, but also for other factors such as income. Income 
elasticity is denoted \(\epsilon_\gamma\). In general, normal goods have \(\epsilon_\gamma > 0\) while
inferior goods have \(\epsilon_\gamma < 0\); demand for normal goods increases with income, while inferior goods
obey the opposite. Necessary goods have \(1 > \epsilon > 0\), while luxury goods
have \(\epsilon_\gamma > 1\), which makes sense; only when income is high will most people consider buying
luxury goods, so elasticity is high.

\bigskip
Cross-price elasticity considers the demand for one good with respect to the price of another good. For
substitutes, \(\epsilon_{AB} > 0\) while for complements, \(\epsilon_{AB} < 0\); if the xbox gets 
cheaper, playstation demand increases; if the xbox price increases, demand for xbox games decreases.

\section*{Behaviours of Perfectly Competitive Markets}
When a price is fixed, the area between the demand curve and the price line is the total surplus gained
by consumers in the market created. In a perfectly competitive market, this is the area above \(P^*\) under
the demand curve. For a producer, the area under \(P^*\) above the supply curve. For a society, the sum of
these areas is described as the total surplus; the total benefit derived from this market for society. This
concept is linked to welfare.

\bigskip
Consider an example of market behaviours in the shift of a market after the legalization of marijuana.
While illegal, marijuana has a relatively high marginal cost because it has to be imported or produced 
in secret, and distributed as such. We would also expect demand to be reasonably low, due to the potential
for arrest and criminal offences. Thus we would expect a relatively high \(P^*\) and a relatively low \(Q^*\).
With the introduction of legal marijuana, we see a seperate but closely related market emerge; one with lower
marginal cost (in general) than the illegal market, and which, being legal, has substantially higher demand.
We would then expect an increase in \(Q^*\), but would be unable to tell for sure the effect on \(P^*\), as
although demand has increased, the marginal cost of supply has probably also been lowered. The introduction
of this legal market does not necessarily mean the end of the illegal market; a group may exist for which
illegal marijuana is fine as long as it is cheaper than the legal variety. Thus, the market will endure a 
massive demand shock resulting in a new equilibrium with significantly lower \(P^*\) and \(Q^*\).

\bigskip
Some groups thus have an incentive to oppose legalisation; pharmaceutical companies who produce substitutes
will oppose the policy as it will decrease demand for their product; alcohol groups for similar reasons.

\bigskip
In an perfectly competitive market, the equilibrium quantity is always going to be the most efficient
outcome in terms of maximising total surplus. Thus, even a market where all actors are perfectly selfish
maximises total surplus for society.

\subsection*{Government Intervention}
\subsubsection*{Indirect Interventions}
Indirect interventions describe factors like taxes or subsidies. Taxation is usually used by governments to
raise revenue. It introduces a \textit{tax wedge} between price paid by buyer and the price received by sellers.
If the tax is imposed on the seller, sellers are only willing to sell if the price they receive is \(\$t\) higher
than their cost, so the supply curve shifts to the left; \(MC = MC + t\). \(t\) can be a constant or percentage;
it doesn't make much difference conceptually. Thus, taxation will increase \(P^*\) and decrease \(Q^*\). Because of 
the reduction of demand, the seller is unable to pass the entire costs of the taxation to the buyer.

\bigskip
If the taxation is on the buyers side, we instead have the demand curve shifting to the left \(MB = MB - t\). Thus,
we experience a decrease in \(Q^*\) and therefore in \(P^*\). Thus resulting in the same outcome as the seller tax 
in terms of effect on total surplus. Tax is given in all cases by \(t = P_D - P_S\) where \(P_D\) is demand price and 
\(P_S\) is the seller price.

\bigskip
A subsidy is essentially a negative tax; \(s = P_S - P_D\). It incentivises trade in the market. If applied to a
seller, it will drive the supply curve down, lower price and increasing quantity. If applied to a buyer, it will 
drive the demand curve up, increasing demand and price.

\end{document}
