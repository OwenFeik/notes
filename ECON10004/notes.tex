\documentclass[12pt]{report}
\renewcommand{\familydefault}{\sfdefault}
\usepackage{commath}
\usepackage{pgfplots}
\usepackage{calc}

\pgfplotsset{compat = 1.16}
\usepgfplotslibrary{fillbetween}

% Create simple plots with consistent styling
\newenvironment{simpleplot}[1][1]{
    \begin{center}
        \begin{tikzpicture}
            \begin{axis}[
                xmax = 6,
                ymax = 6,
                width = 12 * #1 cm,
                height = 9 * #1 cm,
                xlabel = Quantity,
                xlabel near ticks,
                xtick style = {draw = none},
                xticklabels = \empty,
                axis x line = bottom,
                ylabel = Price,
                ylabel near ticks,
                ytick style = {draw = none},
                yticklabels = \empty,
                axis y line = left
            ]
                \path[name path = axis] (rel axis cs: 0, 0) -- (rel axis cs: 6, 0);
                \path[name path = mid] (-6, 0) -- (6, 0);
}{
            \end{axis}
        \end{tikzpicture}
    \end {center}
}

\begin{document}
\title{ECON10004 Notes}
\chapter*{Introduction}

\section*{What is economics?}
Economics is a study of choice and of human decision making in the face of scarce resources.
Humans have virtually unlimited resources, but lack the meaning to satisfy these.
Thus, economics is the study of why people/``economic agents'' choose what they do.
A firm, for instance, must decide:
\begin{itemize}
    \item What to produce?
    \item How to produce it?
    \item Who to produce it for?
\end{itemize}
As a social science, economics aims to understand the choices of individuals and how these inform the behaviour of larger groups.
Economics is, fundamentally, about human wellbeing. Wellbeing, from the perspective of an economist is the consumption
of goods and services. These goods and services are produced using a variety of resources, including:
\begin{itemize}
    \item Natural reosurces 
    \item Physical capital; machines, infrastructure
    \item Human capital; labour, intellectual labour, etc
    \item Time
\end{itemize}
Because these resources are finite, humanity needs to make decisions on the allocation of these resources to optimise wellbeing.

\section*{Decision Theory}
An economic agent is always making choices; it is trying to select the best option based on a 
cost-benefit analysis. 
\[\mathrm{Net\:benefit}\:NB = \mathrm{Total\:benefit}\:TB - \mathrm{Total\:costs}\:TC\]
For example, a firm's goal is to maximise profits; they will minimise costs and maximise revenue to achieve this goal.

\subsection*{How do we measure costs?}
While people are generally quite good at evaluating benefits, they sometimes struggle when
trying to evaluate costs. Often, people will, when considering a purchase, simply consider the 
explicit cost; the price on the price tag. However, this ignores the opportunity cost; any
purchase is to the exclusion of other purchases.

\subsubsection*{Example}
Consider 2 individuals sharing a home, with two tasks that need to be completed, say 
cooking and laundry. If person \(A\) is good at laundry but poor at cooking, person \(B\)
with the opposite skill set, it is obviously advantageous for them to trade; to each
attend to their favoured task. However, consider the scenario when person \(A\) is better
at both.
\par
We can model this as a "production possibilities frontier"; modelling the amount of each
good they can produce if they perform a mixture of the two tasks for a certain time. This will
form a linear relationship tracking from the \(100:0\) time spent state through the \(50:50\) to the \(0:100\) state. Depending on the 
exact relationship between their capabilities, trade may nonetheless be advantageous for both 
parties; the yield for \(A\) may be substantially greater when say, cooking, per unit time,
compared to laundry, and thus, having \(B\) perform some of the laundry, even if less efficient,
may increase the total consumption of both \(A\) and \(B\), thus increasing both parties overall
wellbeing. 
\par
To find this solution, \(A\) needed to consider the opportunity cost of performing laundry,
which in fact proved to be greater than instead primarily cooking and trading with \(B\) to have the laundry
performed.
\par
In the language of economics, \(A\) has an \textit{absolute advantage} in both cooking and doing laundry,
however \(B\) has a \textit{comparative advantage} in laundry.

\subsubsection*{Definitions}
\begin{quotation}    
Absolute advantage is the ability to produce a good using fewer inputs than another producer.
\end{quotation}
\begin{quotation}
Comparative advantage is the ability to produce a good at a lower opportunity cost than another producer.    
\end{quotation}

\section*{Sunk Costs}
When making a decision, one should ignore any costs that have been incurred prior to a decision.
Because these resources have already be used, the perceived loss of them has no real bearing on 
the future outcome of a decision.

\section*{Opportunity Cost}
Total opportunity costs are composed of two seperate parts:
\begin{itemize}
    \item Direct Opportunity Costs: This is the cost of resources that will be used in the chosen alternative.
    \item Indirect Opportunity Costs: This is the benefits minus the explicit costs of the next best action.
\end{itemize}

\subsubsection*{Example}
Say you have three options;
\begin{itemize}
    \item Option A, worth \$3000, cost of \$2500
    \item Option B, worth \$500, cost \$200
    \item Option C, worth \$100, with a \$20 downpayment already payed
\end{itemize}
Thus, you would select Option A. As B is the next best option
this is your indirect opportunity costs. Because B is worth \$300, the total opportunity cost is equal
to \$2500 + \$300 = \$2800. Thus Option A is still worthwhile.

\section*{Marginal Analysis}
Rather than focussing on total benefits and total costs, we should really focus on Marginal Benefits
and costs. Consider whether consuming / producing \(n\) amount of units will be profitable or costing.
i.e. as long as the Marginal Benefit of and action is greater than the Marginal Cost of that action,
we should take that action as a rational agent.

\subsubsection*{Example}
Say we are hiring workers for a store. These workers are payed \$20/hr. The first worker hired is expected
to increased sales by \$60/hr, with additional workers diminishing in value by \$15/hr until they reach \$0/hr.
Thus, the first worker increases net benefit by \$40/hr, the second by \$25/hr etc. So for the first worker,
the marginal benefit will be \$60/hr, while the marginal cost is a constant, \$20/hr. So looking at the marginal 
benfit vs marginal costs for each worker, workers 1, 2 and 3 will have \(MB > MC\) while worker 4 will have \(MB < MC\), so 
we should hire 3 workers.

\subsection*{Calculating Benefit}
Suppose \(NB(x) = TB(x) - TC(x)\). To maximise \(NB(x)\) using calculus, we can differentiate \(NB(x)\) with respect
to \(x\) and set this to \(0\).
\[\frac{dNB(x)}{dx} = \frac{dTB{x}}{dx} - \frac{dTC(x)}{dx} = 0\]
\[\frac{dNB(x)}{dx} = MB(x) - MC(x) = 0\]
Thus the optimal \(x\) is:
\[MB(x) = MC(x)\]


\section*{Microeconomic Theory}
Microeconomics centres around creating models, creating ``rational agents'' built on axioms.
These axioms are as follows: Individual actors are autonomous, individuals have freedom and that individuals matter.
This theory is then compared with real world data and is validated or disproven.
\par
The theory is thus useful for attempting to understand how individual agents can cooperate
with (or counteract) each other within an economic system.

\chapter*{Prisoner's Dilemma}
The prisoner's dilemma; that of two prisoners who are offered a light charge if neither confesses,
a serious charge if both confess and absolution for the confessor if just one confesses.
This model can (and has) been used to understand a variety of situations from board meetings, 
to economics to diplomacy. It highlights that the individual interest is not always aligned with that of the group.
\par
Under the assumption that the only goal of each individual is personal benefit, it makes sense for both to
choose to confess; if the other chooses not to confess, the confessor gains major benefit, if they choose confess,
 the result is indifferent. Thus, this is a model. However, it does not perfectly represent reality. Many agents are
 ``conditional cooperators''; they want to split only when the other person also splits. Thus their action depends
 explicitly on what they believe the other party will do; just like for most people.
\par
\section*{Lesson 1}
This dilemma highlights that individual incentives can be exploited; the optimal for the prisoners here occurs when neither confesses,
however it is optimal from a theoretical standpoint for both to confess. Thus, the system efficiently manages
incentive to exploit the prisoners. This situation can be used as a model for the design of systems; those where
individual incentives are at odds with a designers intent are likely to be unstable and lead to inefficiences.

\subsubsection{Example}
A local government aiming to reduce carbon emissions by offering a 40\% rebate on vehicles capable 
of using LNG, a lower emission fuel. However, their system was flawed in that it had no usage requirement;
consumers simply purchased \$1000 dollar secondary fuel tanks and where rebated \$20,000+ without actually
using the lower emission alternative.
\par
Thus, this program was a financial and ecological disaster.

\begin{quotation}
``Economics is a highly sophisticated field of thought that is superb at explaining to policymakers precisely 
why the choices they made in the past were wrong. About the future, not so much...''
\par
``However, careful economic analysis does have one important benefit, which is that it can help kill ideas that are completely
logically inconsistent or wildly at variance with the data. This insight covers at least 90 percent of proposed economic policies.''
\par
-- Ben Bernanke
\end{quotation}

\section*{Lesson 2}
In addition, this problem can help us to understand that people will adapt to imperfect systems:
over time, people will come to understand what action they should take in a situation and thus they
will create outcomes that may align with the expected outcome of theory.

\section*{Lesson 3}
In economic systems, we can never know the true, underlying preferences and decision 
making process of an individual; they have private information. This can make it difficult
to design effective centralised systems.
\par
Thus, it can be more efficient to pursue decentralised systems such as markets,
where people act in perfect self interest, thus revealing their underlying incentives, which 
inform their behaviour.

\chapter*{Markets}

\section*{What is a market?}
A market is a mechanism through which buyers and sellers trade a particular good or service with 
near-identical characteristics. These characteristics can include:
\begin{itemize}
    \item Type
    \item Delivery Location
    \item Grade and Quality
    \item Time
\end{itemize}
For example, oil varies considerably in chemical compoisition (e.g. sulphur content) and density;
usually the ``Brent'' crude oil price is discussed when the price of oil is referred to; it represents
an independent market with a very specific set of characteristics.

\bigskip 
In a market, buyers supply demand and sellers offer supply. Both agents must consider marginal 
cost and benefit when evaluating a transaction. As they are rational economic agents, they
will only transact when \(MB \geq MC\) for both buyer and seller.

\bigskip
In an indealised perfectly competitive market, there are many buyers and sellers trading exactly
identical goods. These factors inform competition because buyers are aware that they can choose from
a variety of sellers with no differentiation of products. In this setting, an individual does not 
have the ability to manipulate the market through their individual actions; the individual seller
is a ``price-taker'', accepting the market price. Although real markets tend not to be \textit{perfectly} 
competitive, they are often highly competitive.

\subsection*{Demand}
The demand curve will map the number of units for which there is demand at a given price.
In general, as price decreases, demand will increase. This is the thesis of the ``Law of Demand''. 
Other factors can, however, also inform demand includings:
\begin{itemize}
    \item Price
    \item Tastes
    \item Price and quality of alternative goods
    \item Income
    \item Future price expectations
    \item Number of buyers
\end{itemize}
How can changes in income alter the characteristics of a demand curve?
\begin{itemize}
    \item For a \textit{normal good}, demand is usually proportional to income.
    \item For an \textit{inferior good}, demand is inversely related to income.
    \item When income changes, we see a \textit{shift} in the demand curve.
\end{itemize}
How about the prices of other goods?
\begin{itemize}
    \item For \textit{substitutes} such as butter and margerine, the price of one good is proportional to demand for the other.
    \item \textit{Complements} have a negative correlation between the price of one good and a complement good.
\end{itemize}
When these prices change, we see a shift in the demand curve.

\subsection*{Supply}
Supply is the complement to demand, representing the quantity of goods sellers have 
on offer for buyers. For a supply curve, price and quantity in general be positively correlated.

\bigskip
Supply can be affected by factors such as technology advances improving manufacturing
efficiency, shifting the supply curve to greater supply. Inversely, supply might be 
negatively impacted by rising prices for input goods (such as commodities).

\subsection*{Market Equilibrium}
Market equilibrium describes a situation when supply and demand are in balance; when the 
quantity supplied is equal to the quantity demanded. This can be observed graphically as
the intersection of the demand and supply curves. 

\bigskip
This is an equilibrium because neither buyers
nor sellers have an incentive to change their behaviour; if price where to increase, the
additional goods would not be sold; if it were to decrease, suppliers would not find it
worthwhile to produce the demanded goods.

\subsubsection*{Example}
If \(Q_D\) is quantity demanded and \(Q_S\) is quantity supplied, given by these equations:
\[Q_D = 120 - 20P\]
\[Q_S = 20P\]
We can solve the system for equilibrium by setting \(Q_D = Q_S\):
\[120 - 20P = 20P \Rightarrow 120 = 40P \Rightarrow P = 3\]
\[Q_D = 120 - 60 = 60\]
Thus, market equilibrium for this system is found at:
\[P^* = 3\]
\[Q^* = 60\]
What happens when the market is not at equilibrium? Well, when \(P \neq P^*\),
competition will tend to drive \(P\) toward \(P^*\). To identify this equilibrium,
we can observe what would happen if \(P\) where to change; if a point is at equilibrium,
it will return to that state after a disturbance.

\bigskip
For example, if the price is too high, demand will be lower. Thus, manufacturers will
have excess stock and will be incentivised to lower their prices to move their goods,
and as rational agents, will do so.

\subsection*{Comparative Statics}
Comparative statics refer to comparing static snapshots of supply/demand curves
and the effects of market shocks, such as e.g. COVID19 on the supply/demand of hand sanitiser.
COVID19 is an example of a positive shock increasing the demand curve; shifting it to the right and 
thus increasing demand at every price point, and causing an increase in \(P^*\). This is referred to as an \textit{exogenous} effect;
i.e. it comes from a source external to the system.

\bigskip
Positive supply shocks increase supply at ever price point; an example of an \textit{exogenous} effect
causing this might be a technological advance making it more affordable to supply goods. The 
reverse might happen if some disaster occured such as storms wiping out banana crops in Queensland.

\bigskip
In the case of both a positive supply shock and a positive demand shock of equivalent magnitude,
the intersection will tend to simply increase in quantity. The increasing supply places negative pressure on the price while the increasing
demand places positive pressure, thus maintaining \(P^*\). If we don't know the relative magnitudes,
we can still say that there will unambiguosly be an increase in demand, but we can't say for sure
what will happen to the price.

\bigskip
An example of a long term shift could be lobster. Historically, they were seen as a disgusting low
grade food which was extremely plentiful. However, over time they became more respected and seen
 as a restaurant delicacy, and they were overharvested leading to a reduction in supply.
 Thus, the demand has increasing radically increasing price and the supply has dropped significantly,
 also driving up the price. This is how a meal that was \$4 in 1870 might cost \$30 or more today.
Because of the competing forces on quantity, it is difficult to be sure of how \(Q*\) has changed.

\section*{Elasticity}

Elasticity measures the responsiveness of quantity demanded to it's determinants. This allows us
to analyse in detail how markets will respond to a given shock, i.e. magnitude in addition to direction.

\subsection*{Demand}

Price elasticity of demand measures the responsiveness of quantity demanded to a change in price. If
\(\Delta P\) denotes change in price and \(\Delta Q_D\) denotes change in quantity demanded then 
\(\frac{\Delta P}{P}\) yields percentage change in price and likewise \(\frac{\Delta Q_D}{Q_D}\) yields
percentage change in \(Q_D\).

\bigskip
Elasticity for \(Q_D\) is then given by:
\[\epsilon_D = \abs{\frac{\Delta P / P}{\Delta Q_D / Q_D}}\]
By convention, it is a positive number. We can take a derivative to determine
\textit{point-price elasticity}. Knowing that the rate of change of \(Q_D\) with
respect to \(P\) is given by \(\frac{dQ_D}{dP}\):
\[\epsilon_D = \abs{\frac{dQ_D}{dP}\frac{P}{Q_D}}\]
This is the elasticity at a particular price, rather than for the entire demand curve.

\subsubsection*{Example}
Consider a demand curve of \(Q_D = 120 - 20P\). Find the point-price elasticity at \(Q_D = 20\):
\[\frac{dQ_D}{dP} = -20\]
\[\epsilon_D = \abs{\frac{dQ_D}{dP}\frac{P}{Q_D}} = \abs{-20(\frac{5}{20})} = 5)\]
and at \(Q_D = 100\):
\[\epsilon_D = \abs{\frac{dQ_D}{dP}\frac{P}{Q_D}} = \abs{-20(\frac{1}{100})} = \frac{1}{5})\]

\subsection*{Common Elasticities}
\begin{itemize}
    \item Perfectly inelastic: \(\epsilon_D = 0\)
    \item Inelastic: \(1 > \epsilon_D > 0\)
    \item Unit elastic: \(\epsilon_D = 1\)
    \item Elastic: \(\infty > \epsilon_D > 1\)
    \item Perfectly elastic: \(\epsilon_D = \infty\)
\end{itemize}

\subsection*{Factors of Elasticity}
A variety of factors can affect elasticity, such as degree of necessity; if a product is necessary,
it's price will affect demand little. Availability of substitutes is another factor; if a product
can be easily replaced by another product, it's likely to be highly elastic.

\bigskip
Elasticity can be used to consider change in revenue cause by a change in price. Understanding that 
a shift upward in price causes a shift downward in demand allows us to see that an increase in 
price will lead to more revenue only if the effect of that change is more significant than the 
effect of the lost demand. Understanding that elasticity denotes how responsive a quantity is 
to the change of it's inputs allows us to derive the folowing:
\[MR(P) = (1 - \epsilon_D)Q_D\]
Thus, the total revenue is maximised at \(\epsilon_D = 1\). For example, increasing price when \(\epsilon_D = 0.8\)
is likely to yield an increase in revenue while doing the same at \(\epsilon_D = 1.8\) is likely to
reduce revenue. 

\bigskip
Elasticity can be considered not only for price, but also for other factors such as income. Income 
elasticity is denoted \(\epsilon_\gamma\). In general, normal goods have \(\epsilon_\gamma > 0\) while
inferior goods have \(\epsilon_\gamma < 0\); demand for normal goods increases with income, while inferior goods
obey the opposite. Necessary goods have \(1 > \epsilon > 0\), while luxury goods
have \(\epsilon_\gamma > 1\), which makes sense; only when income is high will most people consider buying
luxury goods, so elasticity is high.

\bigskip
Cross-price elasticity considers the demand for one good with respect to the price of another good. For
substitutes, \(\epsilon_{AB} > 0\) while for complements, \(\epsilon_{AB} < 0\); if the xbox gets 
cheaper, playstation demand increases; if the xbox price increases, demand for xbox games decreases.

\section*{Behaviours of Perfectly Competitive Markets}
When a price is fixed, the area between the demand curve and the price line is the total surplus gained
by consumers in the market created. In a perfectly competitive market, this is the area above \(P^*\) under
the demand curve. For a producer, the area under \(P^*\) above the supply curve. For a society, the sum of
these areas is described as the total surplus; the total benefit derived from this market for society. This
concept is linked to welfare.

\bigskip
Consider an example of market behaviours in the shift of a market after the legalization of marijuana.
While illegal, marijuana has a relatively high marginal cost because it has to be imported or produced 
in secret, and distributed as such. We would also expect demand to be reasonably low, due to the potential
for arrest and criminal offences. Thus we would expect a relatively high \(P^*\) and a relatively low \(Q^*\).
With the introduction of legal marijuana, we see a seperate but closely related market emerge; one with lower
marginal cost (in general) than the illegal market, and which, being legal, has substantially higher demand.
We would then expect an increase in \(Q^*\), but would be unable to tell for sure the effect on \(P^*\), as
although demand has increased, the marginal cost of supply has probably also been lowered. The introduction
of this legal market does not necessarily mean the end of the illegal market; a group may exist for which
illegal marijuana is fine as long as it is cheaper than the legal variety. Thus, the market will endure a 
massive demand shock resulting in a new equilibrium with significantly lower \(P^*\) and \(Q^*\).

\bigskip
Some groups thus have an incentive to oppose legalisation; pharmaceutical companies who produce substitutes
will oppose the policy as it will decrease demand for their product; alcohol groups for similar reasons.

\bigskip
In an perfectly competitive market, the equilibrium quantity is always going to be the most efficient
outcome in terms of maximising total surplus. Thus, even a market where all actors are perfectly selfish
maximises total surplus for society.

\subsection*{Government Intervention}
\subsubsection*{Indirect Interventions}
Indirect interventions describe factors like taxes or subsidies. Taxation is usually used by governments to
raise revenue. It introduces a \textit{tax wedge} between price paid by buyer and the price received by sellers.
If the tax is imposed on the seller, sellers are only willing to sell if the price they receive is \(\$t\) higher
than their cost, so the supply curve shifts to the left; \(MC = MC + t\). \(t\) can be a constant or percentage;
it doesn't make much difference conceptually. Thus, taxation will increase \(P^*\) and decrease \(Q^*\). Because of 
the reduction of demand, the seller is unable to pass the entire costs of the taxation to the buyer.

\bigskip
If the taxation is on the buyers side, we instead have the demand curve shifting to the left \(MB = MB - t\). Thus,
we experience a decrease in \(Q^*\) and therefore in \(P^*\). Thus resulting in the same outcome as the seller tax 
in terms of effect on total surplus. Tax is given in all cases by \(t = P_D - P_S\) where \(P_D\) is demand price and 
\(P_S\) is the seller price.

\bigskip
A subsidy is essentially a negative tax; \(s = P_S - P_D\). It incentivises trade in the market. If applied to a
seller, it will drive the supply curve down, lower price and increasing quantity. If applied to a buyer, it will 
drive the demand curve up, increasing demand and price.

\subsubsection*{Interventions and Welfare}
The effect of taxation thus is to introduce a \textit{deadweight loss}, a decrease in the 
total market surplus. This is because the perfect efficiency of the equilibrium is disrupted.
In the case of a tax, this happens because the increased price of goods causes people to
not purchase goods they would otherwise buy, resulting in a loss of utility. In the case of
a subsidy, consumers purchase goods they would not otherwise purchase, resulting in the sale of
goods for below their production price, the difference in costs making up the deadweight loss.

\begin{simpleplot}
    \addplot[name path = q, blue, ultra thick] {-x}
    node [pos = 0.2, above right] {\(Q_D\)};
    \addplot[name path = p, red, ultra thick] {x}
    node [pos = 0.6, below right] {\(Q_S\)};
    \addplot[name path = pt, red, ultra thick] {x + 2}
    node [pos = 0.6, above left] {\(S_t\)};
    \addplot[black, ultra thick, dashed] (-1, x)
    node [pos = 0.15, left] {\(Q_{**}\)};
    \addplot[black, ultra thick, dashed] (0, x)
    node [pos = 0.2, right] {\(Q_*\)};

    \path[name path = deq] (-6, 1) -- (6, 1);
    \path[name path = peq] (-6, -1) -- (6, -1);

    \addplot[fill = blue, opacity = 0.2] fill between [
        of = deq and mid,
        soft clip = {domain = -6:-1}
    ];
    \addplot[fill = red, opacity = 0.2] fill between [
        of = mid and peq,
        soft clip = {domain = -6:-1}
    ];
    \addplot[fill = red, opacity = 0.3] fill between [
        of = p and peq,
        soft clip = {domain = -6:-1}
    ];
    \addplot[fill = blue, opacity = 0.3] fill between [
        of = q and deq,
        soft clip = {domain = -6:-1}
    ];
    \addplot[fill = gray, opacity = 0.3] fill between [
        of = q and p,
        soft clip = {domain = -1:0}
    ];
\end{simpleplot}

This is illustrated in the above chart. The line \(S_t\) shows the supply
curve under a tax on suppliers, with \(Q_{**}\) indicating the new equilibrium
quantity traded. The lighter coloured block here shows taxation revenue for the
government, which contributes to total surplus but is paid for by consumers and
supplies. The blue section of this block is payed for by consumers, while the 
red section is payed by suppliers. The inefficiency introduced by taxation is
apparent in the grey shaded triangle, representing deadweight loss of the 
system due to the tax.

\subsubsection*{Tax Burden}
The effect of a tax will tend to be heavier on the side of the market which
has the lower elasticity. This is intuitive when we consider that those
for who price has little relevance to quantity will continue to interact in
much the same way after a tax, while those with higher elasticity will 
be more likely to reduce quantity due to a tax.

\bigskip
The impact of a subsidy can be understood in much the same way. When a subsidy
is applied, the supply curve is lowered. Thus \(Q_{**}\) will be larger than 
\(Q_*\) and the equilibrium price will be lower. An additional surplus will be 
introduced for suppliers and consumers, payed for by the government. The side 
with more elasticity will derive more benefit from the subsidy than the side 
with lower elasticity. Although the subsidy will create a surplus for consumers
and producers, it will cost more than this surplus and thus net a deadweight 
loss.

\subsubsection*{Price Controls}
Direct market controls can take the form of a price floor or price ceiling.
For each of these, they are \textit{binding} when they are set so as to affect
the equilibrium price, so when they are above the equilibrium price in the case
of floors and below in the case of ceilings. If they are not binding, price
controls have no effect.

\bigskip
In the case of a binding price floor, an excess supply will be created due
to the lower equilibrium quantity. This will lower the equilibrium quantity 
force and result in a loss of consumer surplus, due to reduced competition, 
and a change to supplier surplus, with a loss of volume but an increase of 
margin. It will introduce a deadweight loss, as the increase in supplier 
surplus will be less than the less of consumer surplus.

\bigskip
For a price ceiling, the inverse will occur, with an excess demand created.
Once again, the equilibrium quantity will decrease and this will result in
a loss of supplier surplus and a change to consumer surplus, suffering a
loss of volume but increase in value per transaction. A deadweight loss is
introduced in the lost volume.

\subsubsection*{Quotas}
A quota imposes a maximum quantity traded for a good or service. To be 
binding, a quota must be below the equilibrium quantity. This will have the
effect of setting the quantity to the quantity specified by the quota,
thus increasing price to the relevant point on the supply curve. This
results in a decrease in consumer surplus and a change in consumer surplus,
having essentially the same effect as a price floor. Once again, it creates
a deadweight loss.

\subsection*{Trade}
Countries trade because they have different resource endowments; some countries
have a comparative or absolute advantage in certain industries, which allows 
them to produce these goods more efficiently, providing opportunities to 
increase total surplus through trade.

\bigskip
Despite the various benefits of trade, its impacts are not equally distributed;
it can lead to inequalities, domination of smaller economies by larger ones 
causing national security issues, etc.

\bigskip
Comparative advantage arises when an entity can produce goods at a lower 
opportunity cost than their competition, while absolute advantage arises when a
good can be produced with fewer inputs than used by their competition. In 
general, countries will produce more of the goods in which they have a 
comparative advantage, and export the excess, while they produce less of goods
in which they suffer a comparative disadvantage.

\bigskip
Countries isolated from international trade are said to be in \textit{Autarky}.
This means that only domestic trade is possible in their economies. By
considering a country in Autarky and comparing this to the resultant
market after introduction of trade, we can understand the effect of trade
on a market. The world market will usually be essentially pre-determined, 
making an individual country a price-taker, because it is too small to 
significantly effect the global market.

\bigskip
In the case that \(P_W\), the world price is greater than \(P^*\), the
local equilibrium price, sellers who were selling at \(P^*\) will be able
to export and so the local price \(P^*\) must rise to match \(P_W\).

\begin{simpleplot}
    \addplot[name path = qd, blue, ultra thick] {-x}
    node [pos = 0.2, above right] {\(Q_D\)};
    \addplot[name path = qs, red, ultra thick] {x}
    node [pos = 0.8, below right] {\(Q_S\)};
    \addplot[name path = pw, blue, ultra thick] (x, 1)
    node [pos = 0.8, below] {\(P_W\)};

    \addplot[black, ultra thick, dashed] (-1, x)
    node [pos = 0.1, left] {\(Q_D(P_W)\)};
    \addplot[black, ultra thick, dashed] (0, x)
    node [pos = 0.1, left] {\(Q_*\)};
    \addplot[black, ultra thick, dashed] (1, x)
    node [pos = 0.1, right] {\(Q_S(P_W)\)};

    \addplot[fill = red, opacity = 0.2] fill between [
        of = pw and qs,
        soft clip = {domain = -6:1}
    ];
    \addplot[fill = red, opacity = 0.2] fill between [
        of = qs and mid,
        soft clip = {domain = -6:0}
    ];
    \addplot[fill = blue, opacity = 0.4] fill between [
        of = qd and pw,
        soft clip = {domain = -6:-1}
    ];
\end{simpleplot}

For local suppliers, this is excellent, as can be seem on the above graph.
Their surplus has increased by the area of the light red section, with the 
section under \(Q_D\) to the left of \(Q_D(P_W)\) coming directly out of 
local consumer surplus, and remainder constituting the surplus acquired through
international trade.

\bigskip
In this case, the total added surplus for the local market is equal to the area
of the small triangle in the centre. The gains made by the suppliers outweigh
the losses of the consumers.

\bigskip
Intuitively, in the case that \(P_W < P^*\), consumers will be bettered by
imports of goods for which the country is at a comparative disadvantage,
while producers will be harmed by the necessary reduction of \(P^*\) to \(P^W\).
Once again, the total surplus rises, though in this case at the cost of producers
rather than consumers.

\subsubsection*{Trade and Interventions}
An import tariff is a tax applied to goods imported from overseas. This
reduces the comparative advantage of overseas goods, moving the market price
closer to the Autarky price in the case of \(P_W < P^*\). This will result
in an increase of local supply of the good, by artificially making 
local goods more competitive. This will be good for suppliers, but bad
for producers. This tariff inevitably introduces a deadweight loss into
the system, in addition to taxation revenue for the government.

\bigskip
Quotas have the effect of fixing the maximum quantity of good that can be
imported. If a binding quota is applied to a good, it has the effect of setting
the price of a good for a quantity up to the maximum of the quota, at which 
stage the normal local market behaviour resumes (in the case \(P_W < P^*\)).
Thus, it will result in a lower price, but one which is less significant
than is there was no quota. This results in a reduction in consumer surplus
and an increase in producer surplus when compared with the equilibrium.
There is a deadweight loss when compared with the unrestricted market.
A quota introduces a benefit for those who have permission to fill the quota;
because this is limited, not all everyone can benefit from it. A quota
has essentially the same impact as a tariff, in that the losses are in the
imports which don't occur due to reduced foreign comparative advantage.

\subsection*{Market Failure}
We have seen thus far the perfectly competitive markets generate perfectly
efficient outcomes, while government interventions tend to result in 
deadweight losses. However, the reality is that markets operating in an
unfettered fashion often result in outcomes that are socially and or 
economically undesirable. An assortment of causes can cause market failures:

\begin{itemize}
    \item Externalities
    \item Asymmetric information
    \item Imperfect competition
    \item Public goods
\end{itemize}

\subsubsection*{Externalities}

A \textit{negative externality} causes costs of others which aren't borne
by the agent. This could be something like a consumer deciding
to smoke in public; this can cause cancer for the people around them for
which the smoker needn't pay. A \textit{positive externality} occurs when
a decision-maker's action causes benefits for others, which the decision-maker
doesn't receive. An example could be vaccinations; if someone gets vaccinated,
they protect not only themselves but those they come into contact with are
saved from a disease. Because the costs of these actions are not borne
by the decision-maker, they are often not considered when making decisions.

\bigskip
When considering a market, we look at the \textit{private} benefits and 
costs of the individual. Thus, we can interpret demand as \textit{private
marginal benefit} (\(PMB\)) and supply as \textit{private marginal cost}
(\(PMC\)). Thus market equilibrium occurs at \(PMB = PMC\) and \(Q^*\)
maximises private net benefits. To consider a society's broader perspective,
we can add \textit{social} marginal benefits and marginal costs, to arrive
at a socially optimal outcome \(Q^{**}\). If \(PMB = SMB\) and \(PMC = SMC\),
the market equilibrium is socially efficient. This balance can be disrupted
by externalities. The following types of externalities can occur:

\begin{itemize}
    \item Positive externality in production (\(SMC < PMC\)), such as between
    a beekeeper and orchardist. The more honey the beekeeper produces, the
    more efficient the orchardists farm. Thus, the beekeeper is effecting a
    net good for the orchardist, the benefits of which they don't see.
    \item Negative externality in production (\(SMC > PMC\)), such as in the
    case of industrial pollution. A dye factory may pollute a river system,
    reducing the efficiency of the downstream fishery causing social marginal
    cost to be higher than the private marginal cost of the polluter.
    \item Positive externality in consumption (\(SMB > PMB\)), for example
    vaccination. When one gets a vaccination, they contribute to heard 
    immunity, a societal good.
    \item Negative externality in consumption (\(SMB < PMB\)), like smoking
    causing unpleasantness for those around them or incurring health care 
    costs through a socialised health care system.
\end{itemize}

\subsubsection*{Example}

\begin{figure}[h]
    \centering
    \hspace{-1.5cm}
    \begin{minipage}{0.4\textwidth}
        \begin{simpleplot}[0.6]
            \addplot[name path = q, blue, ultra thick] {-x}
            node [pos = 0.8, below left] {\(MB\)};
            \addplot[name path = pmc, red, ultra thick] {x}
            node [pos = 0.8, below right] {\(PMC\)};
            \addplot[name path = smc, violet, ultra thick] {x + 2}
            node [pos = 0.77, above left] {\(SMC\)};
            \addplot[black, ultra thick, dashed] (0, x)
            node [pos = 0.15, left] {\(Q_*\)};
            \addplot[name path = p, black, ultra thick, dashed] (x, 0)
            node [pos = 0.8, below] {\(P_*\)};

            \addplot[fill = red, opacity = 0.3] fill between [
                of = pmc and p,
                soft clip = {domain = -2:0}
            ];
            \addplot[fill = red, opacity = 0.3] fill between [
                of = smc and pmc,
                soft clip = {domain = -6:-2}
            ];
            \addplot[fill = blue, opacity = 0.3] fill between [
                of = q and p,
                soft clip = {domain = -6:-0}
            ];
            \addplot[fill = violet, opacity = 0.3] fill between [
                of = p and smc,
                soft clip = {domain = -6:-2}
            ];
            \addplot[fill = gray, opacity = 0.3] fill between [
                of = smc and q,
                soft clip = {domain = -1:0}
            ];
        \end{simpleplot}
    \end{minipage}
    \hspace{1cm}
    \begin{minipage}{0.4\textwidth}
        \begin{simpleplot}[0.6]
            \addplot[name path = q, blue, ultra thick] {-x}
            node [pos = 0.8, below left] {\(MB\)};
            \addplot[name path = pmc, red, ultra thick] {x}
            node [pos = 0.8, below right] {\(PMC\)};
            \addplot[name path = smc, violet, ultra thick] {x + 2}
            node [pos = 0.7, above left] {\(SMC\)};
            \addplot[black, ultra thick, dashed] (-1, x)
            node [pos = 0.15, left] {\(Q_{**}\)};
            \addplot[name path = p, black, ultra thick, dashed] (x, 1)
            node [pos = 0.7, below] {\(P_{**}\)};

            \addplot[fill = red, opacity = 0.3] fill between [
                of = smc and pmc,
                soft clip = {domain = -6:-1}
            ];
            \addplot[fill = blue, opacity = 0.3] fill between [
                of = q and p,
                soft clip = {domain = -6:-1}
            ];
            \addplot[fill = violet, opacity = 0.3] fill between [
                of = p and smc,
                soft clip = {domain = -6:-1}
            ];
        \end{simpleplot}
    \end{minipage}
\end{figure}

In the above plots, the purple area represents social producer surplus;
the surplus generated when considering social marginal cost, adding this to
the red area yields the surplus considering private marginal cost of 
suppliers and the blue area is the marginal benefit of consumers (and 
society, as \(PMB = SMB\) for this example). In this case, social loss 
caused by a negative externality is given by the sum of the areas where
\(PMC < SMC\). In the first case, that of market equilibrium (\(Q_*\)),
there is a significant social cost, visible as the grey shaded region.
This deadweight loss eats into the total surplus of the society.
In the second case, \(Q_{**}\) or social equilibrium, the price is 
dependent on the social marginal cost of production, and thus the total
social surplus is conserved, with no deadweight societal loss.
Thus, it is clear that the perfectly efficient market doesn't necessarily
result in the maximum social surplus. This is an example of how a negative
production externality might leave to market failure.

\bigskip
For a different example, consider a postitive externality in consumption.
In this case, there will be a deadweight loss at the market equilibrium
because individual agents will lack the incentive to perform an action
that will increase social surplus but not private surplus.

\subsubsection*{Accounting for Externalities}
To coerce the market into behaving in a socially optimal fashion, a
government can encourage people to internalise an externality; get
agents to consider the social costs and benefits of their actions.
If more force is required, a government can directly intervene in a
market, through taxes, subsidies, quotas, etc. Here, a government
is rectifying an inefficiency introduced by a free market, rather
than distorting an efficient market as we have previously observed.

\bigskip
An example of such an intervention to rectify the case of a negative
production externality is a \textit{Pigouvian tax}. This is a tax on
a producer equal to the difference between the social marginal cost
and the private marginal cost, thus forcing the market into the
social equilibrium state, though in this case with a slice of total
surplus benefitting government rather than producers or suppliers.
To be efficient, a Pigouvian must be \textit{exactly} equal to the
externality it aims to resolve.

\subsubsection*{Asymmetric Information}
Asymmetric information occurs when one agent knows more than another.
Because in an ideal market, both sides have perfect information, when
this is not the case issues can arise. When there are goods of varying
qualities and some sections of a market know more about these than others, 
market failures can arise. Goods are categorised into three categories
for the purposes of information.
\begin{itemize}
    \item \textit{Search goods} have characteristics which are easy to
    evaluate pre-purchase, such as commodities, groceries or toys.
    \item \textit{Experience goods} have important characteristics which
    are not easily observable at time of purchase, but which teh consumer
    learns about over time. These include used cards, haircuts and travel.
    \item \textit{Credence goods} are goods where quality is both difficult
    to determine at time of purchase, and continue to be mysterious, such as
    vitamins, plumbing or education.
\end{itemize}
For experience goods, the seller generally has more information about the
goods than the buyer. This creates asymmetry. Sellers with lower-quality 
goods may be more eager to sell goods than those with higher-quality goods,
creating a situation known as \textit{adverse selection}. This is where an
offer from an informed party (usually the seller) reveals negative information
about the product on offer.

\subsubsection*{Example}
A classical example of adverse selection is the market for lemons, or used
cars. In a situation where there are three grades of car:
\begin{itemize}
    \item High, 10\% chance of failure
    \item Medium, 50\% chance of failure
    \item Low, 90\% change of failure
\end{itemize}
And buyers value a working car at \$10000 and a non-working car at \$0, while
sellers value cars at \$1000 less than the buyers. We then have these prices
for cars in a market with perfect information:
\begin{center}
    \begin{tabular}{|c|c|c|}
        \hline
        Quality & Buyers & Sellers \\
        \hline
        \hline
        High & \$9000 & \$8000 \\
        Medium & \$5000 & \$4000 \\
        Low & \$1000 & \$0 \\
        \hline
    \end{tabular}
\end{center}
Thus, we will essentially have three markets transacting at the buyer price
for each car type. A total surplus of \$1000 per car will exist. Even if
neither side knows what grade a car is in, we still have symmetric information.
The value of a car becomes:
\[\frac{\$9000 + \$5000 + \$1000}{3} = \$5000\]
\[\frac{\$8000 + \$4000 + \$0}{3} = \$4000\]
And so, assuming that both sides are risk-neutral; that is they are happy
to transact based on expected value, the three smaller markets combine into
one larger market for all qualities of cars, essentially at random. 
Symmetry is retained, as is the surplus of \$1000 per car.

\bigskip
However, let us now consider an environment where sellers know the grade of
their cars, but not buyers, one with asymmetric information. Buyers have the
same valuation of a random car of \$5000, and so this would be the market
price. However, sellers with high quality cars will not transact at \$5000,
because they value their cars at \$8000. Thus, the market can contain only
medium and low quality cars, changing the value to a buyer to
\((\$5000 + \$1000) \div 2 = \$3000\). Disaster! As this is below the seller
value of a medium quality car, they will also refuse to transact, and the
market will be left selling only low quality cars, at a market price of \$1000.
The total surplus will be only a third of what it would be if all cars where
transacted. Total welfare drops and only the lowest quality goods are traded.
This is adverse selection.

\subsubsection*{Accounting for Adverse Selection}
A variety of mechanisms exist for government intervention to protect against
adverse selection. These include:
\begin{itemize}
    \item Legal systems (lemon laws in the US protect purchasers of new cars)
    \item Quality inspections and certifications
    \item Reputation systems, such as those used on eBay or Amazon
    \item Signals such as warranties and service agreements
\end{itemize}

\bigskip
Adverse selection is also prevalent in insurance; when purchasing insurance,
the insuree generally has more information than the insurer. The insurer will
be incentivised to price their insurance at the average cost for all people,
which will disincentivise low risk individuals from purchasing insurance, thus
driving up costs as the average risk of a customer increases. This 
disincentivises slightly less row risk people, and the death spiral has begun.
Governments often resolve this issue in car insurance by mandating its 
purchase, thus forcing even the lowest risk drivers to purchase it. In the case
of health insurance, universal health care can be an alternative that doesn't
suffer this issue.

\subsubsection*{Moral Hazard}
\textit{Moral hazard} arises when one agent engages in risky behaviour because
the other agent bears the consequences of this action. This is an issue in
markets like insurance. For example, after obtaining health insurance one might
(perhaps unwisely) take up unhealthy behaviours because the insurer will foot
the bill for any medical issues. Moral hazard is another example of an issue
with asymmetric information; the seller cannot monitor the buyers actions.

\bigskip
One way to resolve this is by transferring some responsibility to the buyer.
If a car dealer offers a full warranty, and a driver enjoys driving recklessly,
increasing the risk of the car breaking down, then the driver will do so as
it costs them nothing for the duration of the warranty, and the cost to the 
dealer is increased. However, if the dealer instead offers a partial insurance,
meaning that they will pay some portion of the costs associated with a broken
car, but the buyer will also pay some, the driver is now disincentivised from
driving recklessly and is more likely to drive safely. Thus the risk for the 
dealer is reduced and the incentive for bad behaviour from the buyer is 
removed. These systems are seen in practice in places such as:
\begin{itemize}
    \item Insurance deductables, where insurees pay some fees upon a claim
    \item CEO payment schemes, where salary is perfomance dependent
    \item Grades, where students are incentivised to study
\end{itemize}

\section*{Firms}
In the examinations of markets to date, we have generally considered small,
independent buyers and sellers, a construct which aligns poorly with the
reality of massive corporations. To remedy this, we now examine firms as a
concept.

\bigskip
The role of a firm is to produce; to utilise inputs to produce outputs to
sell onward to consumers. Firms offer a means of coordinating work. They have
numerous benefits. They eliminate the need to negotiate each task a worker
undertakes. They can internalise externalities in different layers of a 
production process, reducing losses. There are many varieties of firms, from
single proprietor entities to larger corporations.

\bigskip
The most basic fact of a firm is, speaking economically, its 
\textit{technology}, the method through which it transforms inputs into 
outputs. These inputs might include labour, capital or raw materials.

\subsubsection*{Example}
Consider a bookbinding firm. The books are bound by a machine, an item of 
capital, which can bind 36 books per minute with a crew of 6 people. With less
workers, more books can be output. The following table describes the quantity
produced for each number of workers:

\begin{center}    
    \begin{tabular}{c|c|c}
        Workers & Output & \(F(1, L)\) \\
        \hline
        \(1\) & \(10\) & \(10\) \\
        \(2\) & \(18\) & \(18\) \\
        \(3\) & \(24\) & \(24\) \\
        \(4\) & \(30\) & \(30\) \\
        \(5\) & \(34\) & \(34\) \\
        \(6\) & \(36\) & \(36\) \\
        \(7\) & \(34\) & \(36\) \\
        \(8\) & \(32\) & \(36\) \\
        \(9\) & \(30\) & \(36\) \\
        \(10\) & \(26\) & \(36\) \\  
    \end{tabular}
\end{center}

This is a \textit{production function}, which describes the highest output
(denoted \(Q\)) a firm can produce for every specified combination of inputs.
While firms can use a wide variety of inputs, here we will consider labour 
(\(L\)) and capital (\(K\)). Thus, the production function, shown for \(K = 1\)
in the third column is:
\[Q = F(K, L)\]
As we add machines, the maximum productivity for higher numbers of workers will
increase. While in this example the production function is discrete, in general
we consider the production function as outputting an average, and so represent 
the production function as a continuous multi-variable function.

\bigskip
If we plot capital against labour, we can draw curves of \(F(K, L) = n\) which
show how a given quantity can be produced with different combinations of 
capital and labour. When describing productivity, we use the constructs of 
total and marginal product. Total product (\(TP\)) describes the total quantity
of outputs given some inputs. Marginal product (\(MP\)) describes the increase
in production from an additional unit of input. In the example of \(F(K, L)\),
we can find the productivity of labour by holding capital as a constant and 
taking the limit of \(L\), i.e. take the partial derivative of \(F\) with 
respect to \(L\).
\[MP_L = \frac{\partial F(K, L)}{\partial L}\]
We could do the reverse to find the marginal product of capital. In general,
we assume that the law of diminishing returns applies to both quantities.

\bigskip
The profit of a firm is given by subtracting its total costs, what it pays for
its input, from its total revenue, what it is paid for its outputs. The goal of
a firm is to maximise this value. Let us assume a firm can sell each unit of 
output for \(p\), must pay \(w\) for each unit of labour and \(r\) for each 
unit of capital. Then:
\[pF(K, L) - wL - rK\]
This is the equation which the firm seeks to optimise. While for a simple 
equation this is fairly simple, it will be more complex in a more realistic
scenario. We can simplify the problem by considering that:
\[\pi(Q) = TR(Q) - TC(Q)\]
Where \(\pi(Q)\) is the total profit for a given quantity. Here, \(TR(Q)\) is
simply \(pQ\). However, it is a little more difficult to find a function 
\(TC(Q)\). For this purpose we can look to our production functions. We can
plug in the costs of each unit of production and capital to our production 
function and define \(TC(Q)\) to be the lowest cost combination of different
inputs to the production function to produce \(Q\) units. 

\bigskip
The slope of this function will be governed by the relative costs of capital
and labour. If the two are roughly in balance the function will track 
\(x = y\), if labour is more expensive it will favour capital, etc.

\bigskip
To maximise \(\pi(Q)\), we can differentiate it.
\[\frac{d\pi(Q)}{dQ} = \frac{dTR(Q)}{dQ} - \frac{dTC(Q)}{dQ} = MR(Q) - MC(Q)\]
If we set this equal to 0 to maximise the function, we find that
\[MR(Q^*) = MC(Q^*)\]

\subsubsection*{Time and Costs}
While conceptually we can simply try and maximise this function, in reality it
can be difficult to vary inputs like capital. Thus, we define two \textit{time 
horizons}. In the short term, only one variable can be changed, usually labour,
while in the long term, all variables can be changed. The function examined
previously was for the long term, as it was assumed that all inputs could be
adjusted.

\bigskip
In the short run, we have costs which do not vary with the level of output, as
the input cannot be varied. These are fixed costs. In addition, we have 
variable costs, which are dependent on quantity.
\[SRTC(Q) = FC + VC(Q)\]
For our example of the bookbinder, we can consider a situation where the 
machines are rented on a yearly contract. Say that \(2\) machines have been 
rented for a \(1\) year period. Thus \(F(2, L)\) will give the variable costs.
Here, \(Q\) is capped at \(72\), because the maximum production for a single
machine is \(36\). In this case, the short run function will only coincide
with the long run function in the case that \(2\) units of capital is optimal.
The long run function will always be less than or equal to the short run 
function, because it can always take the value of the short run function, but
can additionally make other optimisations.

\bigskip
Firms generally operate in markets which are quite changeable over both the 
short and long runs. They are influenced by factors such as the behaviour of
their competitors, the present market conditions and externalities. To maintain
viability, they need to solve problems often revolving around these issues:
\begin{itemize}
    \item What technology should the firm use?
    \item How much should the firm produce?
    \item When should the firm exit the market?
\end{itemize}
To answer these, the most useful tool is usually an accurate picture of the 
costs involved. We can use not only long-run total costs and short-run total
costs, but a variety of other, more specialised costs.

\begin{itemize}
    \item Long-run total costs (\(LRTC(Q)\)) are the total cost of producing a
    quantity \(Q\) units with the optimal technology and inputs.
    \item The long-run average total cost (\(LRATC(Q)\)) is the cost per unit 
    for long-run costs. \(LRATC(Q) = LRTC(Q) \div Q\)
    \item Fixed costs (\(FC\)) are short run costs which cannot be changed and
    are not dependent on quantity, but can be recovered by shutting down the
    firm. Most often capital costs.
    \item Variable costs (\(VC(Q)\)) are the short run costs which vary with 
    \(Q\). These include things like labour and material costs.
    \item Short-run total costs (\(SRTC(Q)\)) describe the total cost of 
    producing \(Q\) using the best possible combination of variable inputs.
    \(SRTC(Q) = FC + VC(Q)\).
    \item Short-run marginal costs (\(SRMC(Q)\)) describe the cost to increase
    \(Q\) by \(1\) unit. This is given by the rate of change of the short-run
    total costs, or equivalently, the rate of change of the variable costs.
    \item Average fixed costs (\(AFC(Q)\)) are the average fixed cost per unit
    at a production level of \(Q\) units. \(AFC(Q) = FC \div Q\)
    \item Average variable costs (\(AVC(Q)\)) are the average variable costs
    per unit. \(AVC(Q) = VC(Q) \div Q\)
    \item Short-run average total cost (\(SRATC(Q)\)) describes the average 
    cost per unit in the short run for \(Q\) units. \(SRATC = SRTC(Q) \div Q
    = AFC(Q) + AVC(Q)\)
    \item Sunk costs (\(SC\)) are costs that have already been paid and 
    \textit{cannot} be recovered by shutting down. As these costs are already
    lost, they have not influence on future decision making. 
\end{itemize}

\subsubsection*{Choice of Technology}
Let us consider first the behaviour of short-run average costs. This is a per
unit measure, made up of the sum of the average fixed costs and average 
variable costs. Average fixed costs is a constant divided by \(Q\), and thus
must be a declining curve. Variable costs are usually increasing functions of
\(Q\), because of diminishing returns usually at an increasing rate. Thus, the
shape of a short-run average cost curve is usually U-shaped, with high costs
due to fixed costs at low \(Q\) values, and high costs due to inefficiencies
in variable costs at high \(Q\) values. The lowest average cost is somewhere
in between.

\bigskip
When a firm is considering which technology to use, the trade off is often 
between an option with a higher capital cost and a lower marginal cost and an
option with a lower capital cost but higher marginal cost. Equivalently, it
might be said that the first option has a constant marginal product while the
second has a diminishing marginal product.

\bigskip
The case of an option with a lower capital cost and lower marginal product is
defined by an increasing average variable cost and decreasing average fixed
cost, resulting in the U-shape. In this case, the marginal cost will increase.
This is because the slope of the average cost is increasing. This marginal
cost function will eventually intersect with the short-run average total cost
function at the minimum of the function. This is intuitive when we consider
that the average total cost can only increase when the marginal cost of 
creating another unit is greater than the average cost of producing a unit.

\bigskip
For the case of a higher capital cost and a constant marginal product, the
short run marginal cost and average variable cost will be constant over all
quantities. In this case the short run average total costs will decline with
increasing quantity.

\bigskip
In the long run, the firm can choose between either of these, and thus will 
choose the method with the lowest cost for a given quantity. Therefore, the
long run average total cost for a given quantity is always less than or equal
to the short run average total cost. In the example, for a low quantity the
option with lower capital costs will be more efficient while for a high 
quantity the option with higher capital costs will be more efficient. The
long run average total cost will be the minimum of the two.

\bigskip
In a general example, the restricting factor for short run average total cost
is capital, and the ability to change this in the long term results in 
dramatically more flexibility. This means that while it might be inefficient
in the short term to dramatically increase production, the long term capacity
is significantly higher. Low quantities of production will tend to be 
inefficient because they struggle to justify their capital costs and extremely
high production quantities will struggle to garner enough capital to produce
efficiently. Thus the \textit{efficient scale} will lie between the two.

\end{document}
