\documentclass[12pt]{report}

\usepackage{import}
\import{../}{preamble.tex}

\newenvironment{amatrix}[1]{%
  \left[\begin{array}{@{}*{#1}{c}|c@{}}
}{%
  \end{array}\right]
}

\begin{document}
\begin{flushleft}

\section*{Linear Equations}

\subsection*{Systems of Linear Equations}
Systems of equations and row operations.

For example, consider a network of flows; nodes with a given
inflow and outflow. To compute flows in this network we can use a
system linear equations.

Data fitting using a polynomial. Sometimes, we want to find a
function of a certain form which fits to a set of data points. To
find the relevant coefficients, we can use a system of linear
equations.

In general, we will take the variables in a linear equation to
be \(x_1, x_2, \ldots, x_n\) and the coefficients to be
\(a_1, \ldots, a_n\).

A finite collection of linear equations of a given set of variables
is called a \textit{system of linear equations} or a \textit{linear system}.
\[x_1 + 5x_2 + 6x_3 = 100\]
\[x_2 - x_3 = -1\]
\[-x_1 + x_3 = 11\]
Here, despite missing \(x_1\) and \(x_2\) respectively, the second and third
equations are still part of the same system as they implicitly have a term
with a \(0\) coefficient. \par
The organisation of the above system, with all variables on the right and
constants at left is the standard form of presenting a system.

A \textit{homogenous} linear system is one where all of the constants at right
are \(0\). These systems are easier to solve, and by solving a homogenous
version of a non-homogenous system, we can find a solution to the non-homogenous
variant.

A solution to a linear system is a set of values for variables that cause all
equations in the system to be true.

\subsubsection*{Solving by Elimination}

\[(1):\: 2x - y = 3\]
\[(2):\: x + y = 0\]

\[(2) \Rightarrow y = -x\]
\[(2) \& (1) \Rightarrow 2x - (-x) = 3 \Rightarrow 3x = 3\]
\[x = 1 \Rightarrow y = -1\]

A key to the applicability of this method is that we can divide by the
coefficients, which will not always be a valid assumption. This method can
be implemented algorithmically and will always either yield a solution or tell
you there is none.

\subsubsection*{Matrices}

Really, the variables in a linear system aren't really important; it is simply
the coefficients which define their relations. A matrix, a rectangular array of
numbers, can be used to store these values. A \(p \times q\) matrix has \(p\)
rows and \(q\) columns.

A \textit{augmented matrix} for a linear system is the matrix formed from the
coefficients in the equations and the constant terms, separated by a vertical
line. For example

\[
    \begin{array}{r}
        2x - y = 3 \Rightarrow 2x + -1y = 3 \\
        x + y = 0 \Rightarrow 1x + 1y = 0
    \end{array}
    =
    \begin{amatrix}{2}
        2 & -1 & 3 \\ 1 & 1 & 0
    \end{amatrix}
\]

With coefficients at left and constants at right. The number of rows should be
equal to the number of equations. Each column corresponds to a given variable
in the equations.

We can perform some \textit{elementary row operations} to such a matrix without
changing its solutions. These are
\begin{itemize}
    \item Interchanging two rows
    \item Multiplying a row by a non-zero constant
    \item Adding a multiple of a row to another row
\end{itemize}

\end{flushleft}
\end{document}
