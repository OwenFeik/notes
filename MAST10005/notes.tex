\documentclass[12pt]{report}
\renewcommand{\familydefault}{\sfdefault}
\usepackage{bbold}
\begin{document}
\title{MAST10005 Notes}
\chapter*{Sets}
\begin{flushleft}

\section*{Common Sets}
\begin{itemize}
\item \textbb{N} is the set of positive integers: \(\{ 1, 2, 3, 4, ... \}\)
\item \textbb{Z} is the set of all integers: \(\{ ..., -1, 0, 1, 2, 3, ... \}\)
\item \textbb{Q} is the set of rational numbers, those that can be written as fractions
\item \textbb{R} is the set of real numbers.
\item \textbb{C} is the set of complex numbers.        
\end{itemize}


\[\mathbb{N} \subset \mathbb{Z} \subset \mathbb{Q} \subset \mathbb{R} \subset \mathbb{C}\]
Some numbers cannot be written as fractions, and these we describe as irrational. 
These include \(\pi\), \(e\), \(\sqrt{2}\), etc. These numbers are not in \textbb{Q}, but they are real numbers, i.e. in \textbb{R}. 
It can be difficult to prove numbers are irrational. An even larger set of numbers exists, that of the complex numbers \textbb{C}. ``Most'' real numbers are inrrational.


\section*{Denoting Membership}
If \(A\) is a set, saying \(x \in A\) means \(x\) is in the set \(A\). 
We can also use \(\notin\) to denote non-membership. 
Sets can be expressed as 2-dimensional regions, with points within them as elements.

\begin{itemize}
\item \(n \in \mathbb{Z}\) means \(n\) is an integer.
\item \(x \in \mathbb{Q}\) means \(x\) can be expressed as a fraction.
\end{itemize}

Note that \(n\) is used in the first example while \(x\) is used in the second; conventionally, \(n\) is used to represent integers.


\section*{Desciptive Notation}
Descriptive notation is a way of defining a set by stating a property which all of its elements possess. 
For example: 
\[A = \{x \in \mathbb{R} | x^2 + 1 > 37\}\]
Which can be read as the set of all real numbers \(x\) such that \(x^2 + 1 > 37\). 
The statement must be a ``predicate''; it must be true or false for all values. 
The vertical bar is read as ``such that''. A colon is sometimes used instead.

\subsubsection*{Examples}

\begin{itemize}
\item Express the set of real numbers whose natural (base \(e\)) logarithm is positive with descriptive notation: \({x \in \mathbb{R} | log(x) > 0}\). 
Note that \(log(x)\) denotes \(log_{e}(x)\). There is however an issue with this answer; 
because \(log(x)\) is sometimes undefined, ``\(log(x) > 0\)'' is not a predicate. The set could instead be expressed as:
\[\{x \in \mathbb{R}_{\scriptscriptstyle{>0}} | log(x) > 0\}\;\;\;\mathbb{R}_{\scriptscriptstyle{>0}} = \{x \in \mathbb{R} | x > 0\}\]
\item Express the set of integers whose cube is even in descriptive notation: 
\[\{n \in \mathbb{Z} | n^3 \:\mathrm{is\:even}\}\]
\item Describe the set \(\{n \in \mathbb{N} | sin(n) > 0\}\) in words: 
\begin{quotation}
The set of all natural numbers n such that sin(n) is greater than 0.
\end{quotation}
\end{itemize}


\section*{Abbreviated Notation}
Set notation is often abbreviated. For example \(\{x \in \mathbb{R} | sin(x) = 0\}\) could be expressed as:
\[\{k\pi | k \in \mathbb{Z}\}\]
This will be all integer multiples of pi; i.e. all values for which sin(x) = 0. 
This could also be expressed as: 
\[\{x \in \mathbb{R} | x = k\pi \:\mathrm{for\:some}\:k \in \mathbb{Z}\}\]

\subsubsection*{Examples}

\begin{itemize}
\item Express the set of odd integers in abbreviated set notation: 
\[\{2k + 1 | k \in \mathbb{Z}\} \:\mathrm{or}\: \{2k - 1 | k \in \mathbb{Z}\}\]
\item Express the set \(\{x \in \mathbb{R} | cos(x) = 0\}\) in abbreviated set notation: 
\[\{\frac{\pi}{2} + k\pi | k \in \mathbb{Z}\}\]
\end{itemize}
\end{flushleft}
\end{document}
