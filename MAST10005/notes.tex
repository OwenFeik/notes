\documentclass[12pt]{report}
\renewcommand{\familydefault}{\sfdefault}
\usepackage{bbold}
\usepackage{amssymb}
\begin{document}
\title{MAST10005 Notes}
\chapter*{Sets}
\begin{flushleft}

\section*{Common Sets}
\begin{itemize}
\item \textbb{N} is the set of positive integers: \(\{ 1, 2, 3, 4, ... \}\)
\item \textbb{Z} is the set of all integers: \(\{ ..., -1, 0, 1, 2, 3, ... \}\)
\item \textbb{Q} is the set of rational numbers, those that can be written as fractions
\item \textbb{R} is the set of real numbers.
\item \textbb{C} is the set of complex numbers.        
\end{itemize}


\[\mathbb{N} \subseteq \mathbb{Z} \subseteq \mathbb{Q} \subseteq \mathbb{R} \subseteq \mathbb{C}\]
Some numbers cannot be written as fractions, and these we describe as irrational. 
These include \(\pi\), \(e\), \(\sqrt{2}\), etc. These numbers are not in \textbb{Q}, but they are real numbers, i.e. in \textbb{R}. 
It can be difficult to prove numbers are irrational. An even larger set of numbers exists, that of the complex numbers \textbb{C}. ``Most'' real numbers are inrrational.


\section*{Denoting Membership}
If \(A\) is a set, saying \(x \in A\) means \(x\) is in the set \(A\). 
We can also use \(\notin\) to denote non-membership. 
Sets can be expressed as 2-dimensional regions, with points within them as elements.

\begin{itemize}
\item \(n \in \mathbb{Z}\) means \(n\) is an integer.
\item \(x \in \mathbb{Q}\) means \(x\) can be expressed as a fraction.
\end{itemize}

Note that \(n\) is used in the first example while \(x\) is used in the second; conventionally, \(n\) is used to represent integers.


\section*{Desciptive Notation}
Descriptive notation is a way of defining a set by stating a property which all of its elements possess. 
For example: 
\[A = \{x \in \mathbb{R} | x^2 + 1 > 37\}\]
Which can be read as the set of all real numbers \(x\) such that \(x^2 + 1 > 37\). 
The statement must be a ``predicate''; it must be true or false for all values. 
The vertical bar is read as ``such that''. A colon is sometimes used instead.

\subsubsection*{Examples}

\begin{itemize}
\item Express the set of real numbers whose natural (base \(e\)) logarithm is positive with descriptive notation: \({x \in \mathbb{R} | log(x) > 0}\). 
Note that \(log(x)\) denotes \(log_{e}(x)\). There is however an issue with this answer; 
because \(log(x)\) is sometimes undefined, ``\(log(x) > 0\)'' is not a predicate. The set could instead be expressed as:
\[\{x \in \mathbb{R}_{\scriptscriptstyle{>0}} | log(x) > 0\}\;\;\;\mathbb{R}_{\scriptscriptstyle{>0}} = \{x \in \mathbb{R} | x > 0\}\]
\item Express the set of integers whose cube is even in descriptive notation: 
\[\{n \in \mathbb{Z} | n^3 \:\mathrm{is\:even}\}\]
\item Describe the set \(\{n \in \mathbb{N} | sin(n) > 0\}\) in words: 
\begin{quotation}
The set of all natural numbers n such that sin(n) is greater than 0.
\end{quotation}
\end{itemize}


\section*{Abbreviated Notation}
Set notation is often abbreviated. For example \(\{x \in \mathbb{R} | sin(x) = 0\}\) could be expressed as:
\[\{k\pi | k \in \mathbb{Z}\}\]
This will be all integer multiples of pi; i.e. all values for which sin(x) = 0. 
This could also be expressed as: 
\[\{x \in \mathbb{R} | x = k\pi \:\mathrm{for\:some}\:k \in \mathbb{Z}\}\]
This can be thought of as a kind of ``generating'' notation, whereas descriptive notation
could be seen to excise a set from a larger set through a condition.

\subsubsection*{Examples}

\begin{itemize}
\item Express the set of odd integers in abbreviated set notation: 
\[\{2k + 1 | k \in \mathbb{Z}\} \:\mathrm{or}\: \{2k - 1 | k \in \mathbb{Z}\}\]
\item Express the set \(\{x \in \mathbb{R} | cos(x) = 0\}\) in abbreviated set notation: 
\[\{\frac{\pi}{2} + k\pi | k \in \mathbb{Z}\}\]
\end{itemize}


\section*{Intervals}
\((a, b)\) means the set of real numbers between \(a\) and \(b\) (exclusive). Therefore:
\[(a, b) = \{ x \in \mathbb{R} | a < x \:\mathrm{and}\: x < b\} = \{ x \in \mathbb{R} | a < x < b\} \]
This can be illustrated on a number line with an open circle at \(a\) and \(b\). This is an open interval.
To denote a closed interval (inclusive of \(a\) and \(b\)), we use a square brackets (\([a, b]\)) and a filled circle.

\subsubsection*{Examples}
Give definitions of \([a, b)\) and \((a, b]\) with diagrams:
\begin{itemize}
\item \([a, b)\) represents all numbers from \(a\) (inclusive) to \(a\) (exclusive). On a number line, it would be drawn as a filled dot at \(a\) and an open dot at \(b\).
\item \((a, b]\) represents all numbers greater than \(a\) and less than or equal to \(b\). It could be illustrated with a number line with an empty dot at \(a\) and a filled dot at \(b\).
\end{itemize}


\section*{Unbounded Intervals}
For any \(a \in \mathbb{R}\), 
\[(-\infty, a) = \{x \in \mathbb{R} | x < a\}\]
Note that \(-\infty \:\mathrm{and}\: \infty\) are not elements of \textbb{R}.

\section*{Small Sets}
Sets with finitely many elements can be described with a list of elements.
\[\{x \in \mathbb{R} | x^2 - 1 = 0\} = \{-1, 1\}\]
\[\{x \in \mathbb{Z} | x^2 - 1 < 0\} = \{0\}\]
\[\{x \in \mathbb{Z} | x^3 - x = 0\} = \{-1, 0, 1\}\]

\subsubsection*{Example}
Write the set of prime numbers less than 20 in descriptive and list of elements form:
\[\{2, 3, 5, 7, 11, 13, 17, 19\}\]

We can use a kind of "list of elements" notation to denote some infinite sets with ellipses.
e.g.: \[(2k + 1)\pi | k \in \mathbb{Z} = \{..., -3\pi, -\pi, \pi, 3\pi, ...\}\]

The smallest set is taken to be the empty set, written as \(\emptyset\). It can also be written as 
\(\{\}\) in list of elements form.

\subsubsection*{Examples}
\[\{x \in \mathbb{R} | x^2 + 1 = 0\} = \emptyset\]
\[\{x \in \mathbb{R} | cos(x) > 1\} = \emptyset\]

\section*{Subsets}
If \(A\) and \(B\) are sets then \(A \subseteq B\) means that every element of \(A\) is also an element
of \(B\). It is read as \(A\) is a subset of \(B\).
\[x \in A \Rightarrow x \in B\]

The empty set is taken to be a subset of every set.

\section*{Proofs}
A proof begins with a set of true assumptions, from which mathematical reasoning is used
to prove a conclusion.
To prove \(A \subseteq B\), we need to show that if something meets the criteria of \(A\),
it must also meet those of \(B\).

\subsubsection*{Example}
Prove \(A \subseteq B\) where:
\[A = \{n \in \mathbb{N} | sin(n) > 0\}\]
\[B = \{n \in \mathbb{N} | sin^2(n) \leq sin(n)\}\]
\par
Let \(n \in A\), thus \(sin(n) > 0\).
\par
Since \(sin(n) \leq 1\) and if \(x \leq y\) and \(a > 0\) then \(ax \leq ay\), we have \(sin(n)sin(n) \leq sin(n)\).
Therefore \(n \in B\) so \(A \subseteq B\).

\bigskip

To prove a statement is false, we must find a counterexample - an example for which a
statement does not hold. Thus, to prove \(A\) is not a subset of \(B\), we must find any
\(x\) in \(A\) but not in \(B\).

\subsubsection*{Example}
Prove that \(A \nsubseteq B\) where \(A = \{3n + 1 | n \in \mathbb{Z}\}\) and \(B = \{6m + 1 | m \in \mathbb{Z}\}\).
\par
\[A = \{..., -2, 1, 4, 7, ...\}\]
\[B = \{..., 1, 7, ...\}\]
\par
We guess \(4 \notin B\), \(4 \in A\).
To prove this:
\begin{itemize}
\item Claim \(4 \in A\); this is true since \(4 = 3*1 + 1\). 
\item Claim \(4 \notin B\). 
\item Suppose for a contradiction that \(4 \in B\).
\item This would mean \(4 = 6m + 1\) for some \(m \in \mathbb{Z}\).
\item But then, \(3 = 6m\)
\end{itemize}
\[\therefore m = \frac{3}{6} = \frac{1}{2}\]
This is a contradiction! Thus we conclude \(4 \notin B\).


\section*{Proof by Contradiction}

How do we prove that \(x^2 + x + 1 = 0\) has no real solutions?
Proof by contradiction. Assume that there is a solution, \(x \in \mathbb{R}\) and show
that this leads to an absurd conclusion.
\[x^2 + x + 1 = 0 \Rightarrow (x + \frac{1}{2})^2 + \frac{3}{4} = 0 \Rightarrow (x + \frac{1}{2})^2 = \frac{-3}{4}\]

Note that \(p \Rightarrow q\) means if p is true, so is q. \(p \Leftrightarrow q\) means that if either
is true, both are true; \(p\) is true \textit{if and only if} \(q\) is true.
\section*{Union}
The elements of \(A \cup B\) are all the elements that are in either \(A\) or \(B\). 

\subsubsection*{Examples}
Express the following sets using a union of intervals.
\begin{itemize}
\item \(\{x \in \mathbb{R} | x^2 > 1\}\): \((-\infty, -1) \cup (1, \infty)\)
\item \(\{x \in (-2\pi, 2\pi] | sin(x) \leq 0\}\): 
\[\{x \in (-2\pi, 0] | sin(x) \leq 0\} \cup \{x \in (0, 2\pi] | sin(x) \leq 0\} = [-\pi, 0] \cup [\pi, 2\pi]\]
\item \(\{x \in [-2, 2] | x \notin \mathbb{Z}\}\): \((-2, -1) \cup (-1, 0) \cup (0, 1) \cup (1, 2)\)
\end{itemize}

\[(2, 8) \cup [3, 10] = (2, 10]\]
Thus, \((2, 8) \cup [3, 10]\) is an interval. However is \((0, \sqrt{2}) \cup [\frac{\pi}{2}, 3)\) an interval?
\par
It is not, because \(\sqrt{2} < \frac{\pi}{2}\), thus two pieces are necessary.

\section*{Intersection}
\(A \cap B\) is the set of all elements that are in both \(A\) and \(B\).

\subsubsection*{Examples}
\begin{itemize}
    \item Express \((2, 8) \cap [3, 10]\) as an interval: \([3, 8)\)
    \item Express \((0, \sqrt{2}) \cap [\frac{\pi}{2}, 3)\) in the simplest ay possible: \(\emptyset\)
    \item Express \(\mathbb{Z} \cap [-\pi, \pi]\) in list of elements form: \(\{-3, -2, -1, 0, 1, 2, 3\}\)
    \item Express \(\mathbb{Z} \cap \{x \in \mathbb{R} | x^2 - 5 < 0\}\) in list of elements form: \(\{-2, -1, 0, 1, 2\}\)
    \item Express the set of reals with positive sine and negative cosine with an intersection: \(\{x \in \mathbb{R} | \mathrm{sin}(x) > 0\} \cap \{x \in \mathbb{R} | \mathrm{cos}(x) < 0\}\)
\end{itemize}

When interpreting questions, a question involving ``and'' is likely to be an intersection question.

\section*{Complement}
For \(A\) and \(B\) the \textit{relative complement} of \(A\) and \(B\) is the set of elements
that are in \(A\) but not in \(B\). This is written as:
\[A \backslash B = \{x \in A | x \notin B\}\]

\subsubsection*{Examples}
\begin{itemize}
    \item Find \((0, 2) \backslash (1, 3)\): \((0, 1]\)
    \item Find \((1, 3) \backslash (0, 2)\): \([2, 3)\)
    \item Is it generally true that \(A \backslash B = B \backslash A\). No. See previous questions for example. 
\end{itemize}
Express each of the following as a complemet:
\begin{itemize}
    \item \(\{x \in \mathbb{R} | x^2 > 1\}\): \(\mathbb{R} \backslash \{x \in \mathbb{R} | x^2 < 1\}\)
    \item \(\{x \in [-2, 2] | x \notin \mathbb{Z}\}\): \([-2, 2] \backslash \mathbb{Z}\)
    \item \((-\infty, 0) \cup (0, \infty)\): \(\mathbb{R} \backslash \{0\}\)
\end{itemize}

\section*{Cartesian Product}
For sets \(A\) and \(B\):
\[A \times B = \{(x, y) | x \in A \:\mathrm{and}\: y \in B\}\]
\(a, b\) only equals \(a, b\), not \(b, a\).
\par
e.g.:
\[\{0, 1, 5\} \times \{e, \pi\} = \{(0, e), (0, \pi), (1, e), (1, \pi), (5, e), (5, \pi)\}\]
We can also take \(A^2\) to be \(A \times A\). This is most often seen in \(\mathbb{R}^2\), which is used to represent
the cartesian plane. \(\mathbb{Z}^2 \subseteq \mathbb{R}^2\).


\chapter*{Inequalities}
When we add a constant to an inequality, the inequality is order preserved, it still faces the same direction.
Subtraction, obviously, obeys the same principle.
\[x > y \Rightarrow x + a > y + a\]
When we multiply by a positive constant, \(a > 0\), is order preserving, while
multiplying by a negative constant, \(a < 0\), is order reversing.
As division is simply multiplication by reciprocal, it follows the same principles as
multiplication; i.e. only order reversing if the divisor \(< 0\).

\subsubsection*{Examples}
Express the set \(A = \{x \in \mathbb{R} | -2 -\frac{1}{2}x > -4\}\) as an interval.
\begin{itemize}
    \item \(2 -\frac{1}{2}x > -2\)
    \item \(-\frac{1}{2} > -2\)
    \item \(x < 4\)
    \item \(x \in (-\infty, 4)\)
\end{itemize}
Express the set \(A = \{x \in \mathbb{R} | 1 - x < 3x + 2\}\) as an interval.
\begin{itemize}
    \item \(-1 -x < 3x\)
    \item \(-1 < 4x\)
    \item \(-\frac{1}{4} < x\)
    \item \(x > -\frac{1}{4}\)
    \item \(A = (-\frac{1}{4}, \infty)\)
\end{itemize}

\subsection*{Transitivity}
\[x < y \:\mathrm{and}\: y < z \Rightarrow x < z\] 
\subsubsection*{Example}
Prove that \(x < y \:\mathrm{and}\: a < b \Rightarrow x + a < y + b\).
\begin{itemize}
    \item \(0 < y - x\)
    \item \(a - b < 0\)
    \item \(a - b < y - x\)
    \item \(a < y - x + b\)
    \item \(a + x < y + b\)
\end{itemize}

\subsection*{Reversing and Non-Reversing Functions}
A function is order preserving if it is strictly increasing for the interval containing
both sides of an inequailty. It is order-reversing if it is strictly decreasing for the same. 
i.e. a function is order preserving if:
\[a < b \Rightarrow f(a) < f(b) \:\mathrm{for\:all}\:a, b \in I\]
and order reversing if:
\[a < b \Rightarrow f(a) > f(b) \:\mathrm{for\:all}\:a, b \in I\]
For example, the logarithm and exponential function are order preserving. A function need
not be strictly decreasing across \textbb{R} to be applied to an inequality; if both sides
of an inequality are known to lie within an interval that is strictly decreasing or increasing,
the function may still be applied.

\chapter*{Complex Numbers}
Complex numbers extend the real numbers by adding a new number with the property
\[i^2 = -1\]
The usual rules of algebra nonetheless still apply. Though one should note that:
\[\sqrt{-1}\sqrt{-1} \neq \sqrt{1}\]
This value can be used to solve equations such as:
\[x^2 + 1 = 0 \Rightarrow x^2 = -1\ \Rightarrow x = \pm i\]
\(i\) can also be used to solve equations such as \(x^2 + 4 = 0\):
\[x^2 + 4 = 0 \Rightarrow x^2 = \sqrt{-2} \Rightarrow x = i\sqrt{2}\]
Solutions of the form \(yi | y \in \mathbb{R}\) are useful for solving many polynomials. For 
some others, such as \(x^2 -2x + 5 = 0\) we need solutions of the form \(x + iy\):
\[(x - 1)^2 + 4 = 0\ \Rightarrow (x - 1)^2 = -4 \Rightarrow x = 1 \pm 2i\]
Thus, we define complex numbers, typically denoted as \(z\) as a quantity consisting of a real number
added to a real multiple of \(i\):
\[z = x + iy\]
Thus \textbb{C} is defined as:
\[\mathbb{C} = \{x + iy | x, y \in \mathbb{R}\}\]

\subsubsection*{Examples}
\begin{itemize}
    \item Using \(i\) write down two square roots of \(-25\): \(5i\), \(-5i\)
    \item Simplify \(i^7\): \(i^0 = 1\), \(i^1 = i\), \(i^2 = -1\), \(i^3 = ii^2 = -i\), \(i^4 = (i^2)^2 = 1\), \(i^5 = i\), \(i^6 = -1\), \(i^7 = i(i^2)^3 = -1^3 = -i\)
\end{itemize}
For a complex number, the real part is \(x\) while \(y\) is known as the imaginary part.

\subsubsection*{Example}
For \(z = 2 - 3i\), write down:
\begin{itemize}
    \item \(\mathrm{Re}(z) = 2\)
    \item \(\mathrm{Im}(z) = -3\)
    \item \(\mathrm{Re}(z) - \mathrm{Im}(z) = 5\)
\end{itemize}


\section*{The Complex Plane}
A complex number \(z\) can be seen as a point on a plane, where \(\mathrm{Re}(z)\) denotes
the \(x\) or horizontal component while \(\mathrm{Im}(z)\) denotes the \(y\) or vertical component.
This is sometimes known as the Argand plane.

\bigskip
Addition and subtraction work as expected with complex numbers.
\[z_1 + z_2 = (a + ib) + (c + id) = (a + c) + i(b + d)\]
This can be interpreted as a vector operation, in the same way that adding two vectors
entails summing their components, complex numbers seen as points are effectively vectors
which add in essentially the same way. Multiplication by a real number is essentially stretching
the complex number, the same as multiplying a vector by a scalar.

\bigskip
Multiplication is a little more complex.
\[(a + ib)(c + id) = a(c +id) + ib(c + id) =\]
\[ac + aid + ibc + ibid = ac + adi + bci - bd = (ac -bd) + i(ad + bc)\]

The complex conjugate of a number \(z = a + ib\) is denoted \(\bar{z}\) and is defined:
\[\bar{z} = a - ib\]
The real part remains the same, while the imaginary part has its sign reversed. In the plane,
this operation represents reflection in the \(x\) or real axis.

\subsubsection*{Example}
By completing the square find the solutions of \(z^2 - 6z + 10 = 0\):
\[z^2 - 6z + 10 = (z - 3)^2 + 1\]
\[(z - 3)^2 = -1\]
\[z - 3 = \pm i\]
\[z = 3 \pm i\]


\end{flushleft}
\end{document}
