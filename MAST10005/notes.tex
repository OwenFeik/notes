\documentclass[12pt]{report}
\renewcommand{\familydefault}{\sfdefault}
\usepackage{bbold}
\usepackage{amssymb}
\usepackage{commath}

\newcommand{\N}{\mathbb{N}}
\newcommand{\Z}{\mathbb{Z}}
\newcommand{\Q}{\mathbb{Q}}
\newcommand{\R}{\mathbb{R}}
\newcommand{\C}{\mathbb{C}}

\begin{document}
\title{MAST10005 Notes}
\chapter*{Sets}
\begin{flushleft}

\section*{Common Sets}
\begin{itemize}
\item \textbb{N} is the set of positive integers: \(\{ 1, 2, 3, 4, ... \}\)
\item \textbb{Z} is the set of all integers: \(\{ ..., -1, 0, 1, 2, 3, ... \}\)
\item \textbb{Q} is the set of rational numbers, those that can be written as fractions
\item \textbb{R} is the set of real numbers.
\item \textbb{C} is the set of complex numbers.        
\end{itemize}


\[\N \subseteq \Z \subseteq \Q \subseteq \R \subseteq \C\]
Some numbers cannot be written as fractions, and these we describe as irrational. 
These include \(\pi\), \(e\), \(\sqrt{2}\), etc. These numbers are not in \textbb{Q}, but they are real numbers, i.e. in \textbb{R}. 
It can be difficult to prove numbers are irrational. An even larger set of numbers exists, that of the complex numbers \textbb{C}. ``Most'' real numbers are inrrational.


\section*{Denoting Membership}
If \(A\) is a set, saying \(x \in A\) means \(x\) is in the set \(A\). 
We can also use \(\notin\) to denote non-membership. 
Sets can be expressed as 2-dimensional regions, with points within them as elements.

\begin{itemize}
\item \(n \in \Z\) means \(n\) is an integer.
\item \(x \in \Q\) means \(x\) can be expressed as a fraction.
\end{itemize}

Note that \(n\) is used in the first example while \(x\) is used in the second; conventionally, \(n\) is used to represent integers.


\section*{Desciptive Notation}
Descriptive notation is a way of defining a set by stating a property which all of its elements possess. 
For example: 
\[A = \{x \in \R | x^2 + 1 > 37\}\]
Which can be read as the set of all real numbers \(x\) such that \(x^2 + 1 > 37\). 
The statement must be a ``predicate''; it must be true or false for all values. 
The vertical bar is read as ``such that''. A colon is sometimes used instead.

\subsubsection*{Examples}

\begin{itemize}
\item Express the set of real numbers whose natural (base \(e\)) logarithm is positive with descriptive notation: \({x \in \R | log(x) > 0}\). 
Note that \(log(x)\) denotes \(log_{e}(x)\). There is however an issue with this answer; 
because \(log(x)\) is sometimes undefined, ``\(log(x) > 0\)'' is not a predicate. The set could instead be expressed as:
\[\{x \in \R_{\scriptscriptstyle{>0}} | log(x) > 0\}\;\;\;\R_{\scriptscriptstyle{>0}} = \{x \in \R | x > 0\}\]
\item Express the set of integers whose cube is even in descriptive notation: 
\[\{n \in \Z | n^3 \:\mathrm{is\:even}\}\]
\item Describe the set \(\{n \in \N | sin(n) > 0\}\) in words: 
\begin{quotation}
The set of all natural numbers n such that sin(n) is greater than 0.
\end{quotation}
\end{itemize}


\section*{Abbreviated Notation}
Set notation is often abbreviated. For example \(\{x \in \R | sin(x) = 0\}\) could be expressed as:
\[\{k\pi | k \in \Z\}\]
This will be all integer multiples of pi; i.e. all values for which sin(x) = 0. 
This could also be expressed as: 
\[\{x \in \R | x = k\pi \:\mathrm{for\:some}\:k \in \Z\}\]
This can be thought of as a kind of ``generating'' notation, whereas descriptive notation
could be seen to excise a set from a larger set through a condition.

\subsubsection*{Examples}

\begin{itemize}
\item Express the set of odd integers in abbreviated set notation: 
\[\{2k + 1 | k \in \Z\} \:\mathrm{or}\: \{2k - 1 | k \in \Z\}\]
\item Express the set \(\{x \in \R | cos(x) = 0\}\) in abbreviated set notation: 
\[\{\frac{\pi}{2} + k\pi | k \in \Z\}\]
\end{itemize}


\section*{Intervals}
\((a, b)\) means the set of real numbers between \(a\) and \(b\) (exclusive). Therefore:
\[(a, b) = \{ x \in \R | a < x \:\mathrm{and}\: x < b\} = \{ x \in \R | a < x < b\} \]
This can be illustrated on a number line with an open circle at \(a\) and \(b\). This is an open interval.
To denote a closed interval (inclusive of \(a\) and \(b\)), we use a square brackets (\([a, b]\)) and a filled circle.

\subsubsection*{Examples}
Give definitions of \([a, b)\) and \((a, b]\) with diagrams:
\begin{itemize}
\item \([a, b)\) represents all numbers from \(a\) (inclusive) to \(a\) (exclusive). On a number line, it would be drawn as a filled dot at \(a\) and an open dot at \(b\).
\item \((a, b]\) represents all numbers greater than \(a\) and less than or equal to \(b\). It could be illustrated with a number line with an empty dot at \(a\) and a filled dot at \(b\).
\end{itemize}


\section*{Unbounded Intervals}
For any \(a \in \R\), 
\[(-\infty, a) = \{x \in \R | x < a\}\]
Note that \(-\infty \:\mathrm{and}\: \infty\) are not elements of \textbb{R}.

\section*{Small Sets}
Sets with finitely many elements can be described with a list of elements.
\[\{x \in \R | x^2 - 1 = 0\} = \{-1, 1\}\]
\[\{x \in \Z | x^2 - 1 < 0\} = \{0\}\]
\[\{x \in \Z | x^3 - x = 0\} = \{-1, 0, 1\}\]

\subsubsection*{Example}
Write the set of prime numbers less than 20 in descriptive and list of elements form:
\[\{2, 3, 5, 7, 11, 13, 17, 19\}\]

We can use a kind of "list of elements" notation to denote some infinite sets with ellipses.
e.g.: \[(2k + 1)\pi | k \in \Z = \{..., -3\pi, -\pi, \pi, 3\pi, ...\}\]

The smallest set is taken to be the empty set, written as \(\emptyset\). It can also be written as 
\(\{\}\) in list of elements form.

\subsubsection*{Examples}
\[\{x \in \R | x^2 + 1 = 0\} = \emptyset\]
\[\{x \in \R | cos(x) > 1\} = \emptyset\]

\section*{Subsets}
If \(A\) and \(B\) are sets then \(A \subseteq B\) means that every element of \(A\) is also an element
of \(B\). It is read as \(A\) is a subset of \(B\).
\[x \in A \Rightarrow x \in B\]

The empty set is taken to be a subset of every set.

\section*{Proofs}
A proof begins with a set of true assumptions, from which mathematical reasoning is used
to prove a conclusion.
To prove \(A \subseteq B\), we need to show that if something meets the criteria of \(A\),
it must also meet those of \(B\).

\subsubsection*{Example}
Prove \(A \subseteq B\) where:
\[A = \{n \in \N | sin(n) > 0\}\]
\[B = \{n \in \N | sin^2(n) \leq sin(n)\}\]
\par
Let \(n \in A\), thus \(sin(n) > 0\).
\par
Since \(sin(n) \leq 1\) and if \(x \leq y\) and \(a > 0\) then \(ax \leq ay\), we have \(sin(n)sin(n) \leq sin(n)\).
Therefore \(n \in B\) so \(A \subseteq B\).

\bigskip

To prove a statement is false, we must find a counterexample - an example for which a
statement does not hold. Thus, to prove \(A\) is not a subset of \(B\), we must find any
\(x\) in \(A\) but not in \(B\).

\subsubsection*{Example}
Prove that \(A \nsubseteq B\) where \(A = \{3n + 1 | n \in \Z\}\) and \(B = \{6m + 1 | m \in \Z\}\).
\par
\[A = \{..., -2, 1, 4, 7, ...\}\]
\[B = \{..., 1, 7, ...\}\]
\par
We guess \(4 \notin B\), \(4 \in A\).
To prove this:
\begin{itemize}
\item Claim \(4 \in A\); this is true since \(4 = 3*1 + 1\). 
\item Claim \(4 \notin B\). 
\item Suppose for a contradiction that \(4 \in B\).
\item This would mean \(4 = 6m + 1\) for some \(m \in \Z\).
\item But then, \(3 = 6m\)
\end{itemize}
\[\therefore m = \frac{3}{6} = \frac{1}{2}\]
This is a contradiction! Thus we conclude \(4 \notin B\).


\section*{Proof by Contradiction}

How do we prove that \(x^2 + x + 1 = 0\) has no real solutions?
Proof by contradiction. Assume that there is a solution, \(x \in \R\) and show
that this leads to an absurd conclusion.
\[x^2 + x + 1 = 0 \Rightarrow (x + \frac{1}{2})^2 + \frac{3}{4} = 0 \Rightarrow (x + \frac{1}{2})^2 = \frac{-3}{4}\]

Note that \(p \Rightarrow q\) means if p is true, so is q. \(p \Leftrightarrow q\) means that if either
is true, both are true; \(p\) is true \textit{if and only if} \(q\) is true.
\section*{Union}
The elements of \(A \cup B\) are all the elements that are in either \(A\) or \(B\). 

\subsubsection*{Examples}
Express the following sets using a union of intervals.
\begin{itemize}
\item \(\{x \in \R | x^2 > 1\}\): \((-\infty, -1) \cup (1, \infty)\)
\item \(\{x \in (-2\pi, 2\pi] | sin(x) \leq 0\}\): 
\[\{x \in (-2\pi, 0] | sin(x) \leq 0\} \cup \{x \in (0, 2\pi] | sin(x) \leq 0\} = [-\pi, 0] \cup [\pi, 2\pi]\]
\item \(\{x \in [-2, 2] | x \notin \Z\}\): \((-2, -1) \cup (-1, 0) \cup (0, 1) \cup (1, 2)\)
\end{itemize}

\[(2, 8) \cup [3, 10] = (2, 10]\]
Thus, \((2, 8) \cup [3, 10]\) is an interval. However is \((0, \sqrt{2}) \cup [\frac{\pi}{2}, 3)\) an interval?
\par
It is not, because \(\sqrt{2} < \frac{\pi}{2}\), thus two pieces are necessary.

\section*{Intersection}
\(A \cap B\) is the set of all elements that are in both \(A\) and \(B\).

\subsubsection*{Examples}
\begin{itemize}
    \item Express \((2, 8) \cap [3, 10]\) as an interval: \([3, 8)\)
    \item Express \((0, \sqrt{2}) \cap [\frac{\pi}{2}, 3)\) in the simplest ay possible: \(\emptyset\)
    \item Express \(\Z \cap [-\pi, \pi]\) in list of elements form: \(\{-3, -2, -1, 0, 1, 2, 3\}\)
    \item Express \(\Z \cap \{x \in \R | x^2 - 5 < 0\}\) in list of elements form: \(\{-2, -1, 0, 1, 2\}\)
    \item Express the set of reals with positive sine and negative cosine with an intersection: \(\{x \in \R | \mathrm{sin}(x) > 0\} \cap \{x \in \R | \mathrm{cos}(x) < 0\}\)
\end{itemize}

When interpreting questions, a question involving ``and'' is likely to be an intersection question.

\section*{Complement}
For \(A\) and \(B\) the \textit{relative complement} of \(A\) and \(B\) is the set of elements
that are in \(A\) but not in \(B\). This is written as:
\[A \backslash B = \{x \in A | x \notin B\}\]

\subsubsection*{Examples}
\begin{itemize}
    \item Find \((0, 2) \backslash (1, 3)\): \((0, 1]\)
    \item Find \((1, 3) \backslash (0, 2)\): \([2, 3)\)
    \item Is it generally true that \(A \backslash B = B \backslash A\). No. See previous questions for example. 
\end{itemize}
Express each of the following as a complemet:
\begin{itemize}
    \item \(\{x \in \R | x^2 > 1\}\): \(\R \backslash \{x \in \R | x^2 < 1\}\)
    \item \(\{x \in [-2, 2] | x \notin \Z\}\): \([-2, 2] \backslash \Z\)
    \item \((-\infty, 0) \cup (0, \infty)\): \(\R \backslash \{0\}\)
\end{itemize}

\section*{Cartesian Product}
For sets \(A\) and \(B\):
\[A \times B = \{(x, y) | x \in A \:\mathrm{and}\: y \in B\}\]
\(a, b\) only equals \(a, b\), not \(b, a\).
\par
e.g.:
\[\{0, 1, 5\} \times \{e, \pi\} = \{(0, e), (0, \pi), (1, e), (1, \pi), (5, e), (5, \pi)\}\]
We can also take \(A^2\) to be \(A \times A\). This is most often seen in \(\R^2\), which is used to represent
the cartesian plane. \(\Z^2 \subseteq \R^2\).


\chapter*{Inequalities}
When we add a constant to an inequality, the inequality is order preserved, it still faces the same direction.
Subtraction, obviously, obeys the same principle.
\[x > y \Rightarrow x + a > y + a\]
When we multiply by a positive constant, \(a > 0\), is order preserving, while
multiplying by a negative constant, \(a < 0\), is order reversing.
As division is simply multiplication by reciprocal, it follows the same principles as
multiplication; i.e. only order reversing if the divisor \(< 0\).

\subsubsection*{Examples}
Express the set \(A = \{x \in \R | -2 -\frac{1}{2}x > -4\}\) as an interval.
\begin{itemize}
    \item \(2 -\frac{1}{2}x > -2\)
    \item \(-\frac{1}{2} > -2\)
    \item \(x < 4\)
    \item \(x \in (-\infty, 4)\)
\end{itemize}
Express the set \(A = \{x \in \R | 1 - x < 3x + 2\}\) as an interval.
\begin{itemize}
    \item \(-1 -x < 3x\)
    \item \(-1 < 4x\)
    \item \(-\frac{1}{4} < x\)
    \item \(x > -\frac{1}{4}\)
    \item \(A = (-\frac{1}{4}, \infty)\)
\end{itemize}

\subsection*{Transitivity}
\[x < y \:\mathrm{and}\: y < z \Rightarrow x < z\] 
\subsubsection*{Example}
Prove that \(x < y \:\mathrm{and}\: a < b \Rightarrow x + a < y + b\).
\begin{itemize}
    \item \(0 < y - x\)
    \item \(a - b < 0\)
    \item \(a - b < y - x\)
    \item \(a < y - x + b\)
    \item \(a + x < y + b\)
\end{itemize}

\subsection*{Reversing and Non-Reversing Functions}
A function is order preserving if it is strictly increasing for the interval containing
both sides of an inequailty. It is order-reversing if it is strictly decreasing for the same. 
i.e. a function is order preserving if:
\[a < b \Rightarrow f(a) < f(b) \:\mathrm{for\:all}\:a, b \in I\]
and order reversing if:
\[a < b \Rightarrow f(a) > f(b) \:\mathrm{for\:all}\:a, b \in I\]
For example, the logarithm and exponential function are order preserving. A function need
not be strictly decreasing across \textbb{R} to be applied to an inequality; if both sides
of an inequality are known to lie within an interval that is strictly decreasing or increasing,
the function may still be applied.

\chapter*{Complex Numbers}
Complex numbers extend the real numbers by adding a new number with the property
\[i^2 = -1\]
The usual rules of algebra nonetheless still apply. Though one should note that:
\[\sqrt{-1}\sqrt{-1} \neq \sqrt{1}\]
This value can be used to solve equations such as:
\[x^2 + 1 = 0 \Rightarrow x^2 = -1\ \Rightarrow x = \pm i\]
\(i\) can also be used to solve equations such as \(x^2 + 4 = 0\):
\[x^2 + 4 = 0 \Rightarrow x^2 = \sqrt{-2} \Rightarrow x = i\sqrt{2}\]
Solutions of the form \(yi | y \in \R\) are useful for solving many polynomials. For 
some others, such as \(x^2 -2x + 5 = 0\) we need solutions of the form \(x + iy\):
\[(x - 1)^2 + 4 = 0\ \Rightarrow (x - 1)^2 = -4 \Rightarrow x = 1 \pm 2i\]
Thus, we define complex numbers, typically denoted as \(z\) as a quantity consisting of a real number
added to a real multiple of \(i\):
\[z = x + iy\]
Thus \textbb{C} is defined as:
\[\C = \{x + iy | x, y \in \R\}\]

\subsubsection*{Examples}
\begin{itemize}
    \item Using \(i\) write down two square roots of \(-25\): \(5i\), \(-5i\)
    \item Simplify \(i^7\): \(i^0 = 1\), \(i^1 = i\), \(i^2 = -1\), \(i^3 = ii^2 = -i\), \(i^4 = (i^2)^2 = 1\), \(i^5 = i\), \(i^6 = -1\), \(i^7 = i(i^2)^3 = -1^3 = -i\)
\end{itemize}
For a complex number, the real part is \(x\) while \(y\) is known as the imaginary part.

\subsubsection*{Example}
For \(z = 2 - 3i\), write down:
\begin{itemize}
    \item \(\mathrm{Re}(z) = 2\)
    \item \(\mathrm{Im}(z) = -3\)
    \item \(\mathrm{Re}(z) - \mathrm{Im}(z) = 5\)
\end{itemize}


\section*{The Complex Plane}
A complex number \(z\) can be seen as a point on a plane, where \(\mathrm{Re}(z)\) denotes
the \(x\) or horizontal component while \(\mathrm{Im}(z)\) denotes the \(y\) or vertical component.
This is sometimes known as the Argand plane.

\bigskip
Addition and subtraction work as expected with complex numbers.
\[z_1 + z_2 = (a + ib) + (c + id) = (a + c) + i(b + d)\]
This can be interpreted as a vector operation, in the same way that adding two vectors
entails summing their components, complex numbers seen as points are effectively vectors
which add in essentially the same way. Multiplication by a real number is essentially stretching
the complex number, the same as multiplying a vector by a scalar.

\bigskip
Multiplication is a little more complex.
\[(a + ib)(c + id) = a(c +id) + ib(c + id) =\]
\[ac + aid + ibc + ibid = ac + adi + bci - bd = (ac -bd) + i(ad + bc)\]

The complex conjugate of a number \(z = a + ib\) is denoted \(\bar{z}\) and is defined:
\[\bar{z} = a - ib\]
The real part remains the same, while the imaginary part has its sign reversed. In the plane,
this operation represents reflection in the \(x\) or real axis.

\subsubsection*{Example}
By completing the square find the solutions of \(z^2 - 6z + 10 = 0\):
\[z^2 - 6z + 10 = (z - 3)^2 + 1\]
\[(z - 3)^2 = -1\]
\[z - 3 = \pm i\]
\[z = 3 \pm i\]

\subsubsection*{Properties of Conjugates}
Let \(z = x + iy\) and \(w = a + ib\) be complex numbers. Then:
\[z + \bar{z} = 2x = 2\mathrm{Re}(z)\:\mathrm{(real)}\]
\[z - \bar{z} = 2yi = 2 \mathrm{Im}(z)i\]
\[z\bar{z} = x^2 + y^2\:\mathrm{(real)}\]
\[\bar{z + w} = \bar{z} + \bar{w}\]
\[\bar{zw} = \bar{z}\bar{w}\]
\(\sqrt{z\bar{z}}\) can be understood as the length of a complex number by Pythagoras' theorem.

\subsubsection*{Division}
\[\frac{1}{i} = -i\]
In general:
\[\frac{\bar{z}}{x^2 + y^2} = \frac{1}{z}\]
We can express the division of two complex numbers as follows:
\[\frac{a + ib}{c + id} = \frac{a + ib}{c + id}\times\frac{c - id}{c - id} = \frac{(a + ib)(c - id)}{c^2 + d^2}\]
\[= \frac{1}{c^2 + d^2}(ac + bd + i(-ad + bc))\]
In essence, we rationalise the denominator, much as with surds.

\subsubsection*{Examples}
\[\frac{1 + 2i}{-1 + 3i} = \frac{1 + 2i}{-1 + 3i} \times \frac{-1 - 3i}{-1 - 3i} = \frac{(1 + 2i)(-1 - 3i)}{1 + 9}\]
\[=\frac{1 - i}{2}\]
Calculate: \((1 + \sqrt{3}i)^6\)
\[(1 + \sqrt{3}i)^6 = ((1 + \sqrt{3}i)^2)^2 \times (1 + \sqrt{3}i)^2\]
\[(1 + \sqrt{3}i)^2 = 1 + 2\sqrt{3}i + (\sqrt{3}i)^2 = 1 + 2\sqrt{3}i - 3 = -2 + 2\sqrt{3}i\]
\[((1 + \sqrt{3}i)^2)^2 = (-2 + 2\sqrt{3}i)^2 = (-2)^2 + 2 \times (-2) \times 2\sqrt{3}i + (2\sqrt{3}i)^2\]
\[= 4 - 8\sqrt{3}i - 12 = -8 - 8 \sqrt{3}i\]
\[(1 + \sqrt{3}i)^6 = (-8 -8\sqrt{3}i)(-2 + 2\sqrt{3}i)\]
\[= 16 - 16\sqrt{3}i + 16\sqrt{3}i - 8 \times 2 \times (\sqrt{3})^2i^2\]
\[= 16 + 8 \times 2 \times 3\]
\[= 64\]

\subsubsection*{Polar Form}
To solve this problem more easily, we can instead consider complex numbers in polar form.
This takes the length of the line and it's angle from the origin instead. In this case we 
take a distance \(r\) and an angle \(\theta\).
\[r = |z| = \sqrt{x^2 + y^2}\]
To find \(\theta\) we can draw \(z\) in the plane and use triangles to find \(\theta\). \(\theta \in (-\pi, \pi]\), and
is called the principal argument of \(z\) and can be denoted \(\mathrm{Arg}(z)\).

\subsubsection*{Examples}
\begin{itemize}
    \item \(1 + \sqrt{3}i\): \(|z| = 2, \theta = \frac{\pi}{3}\)
    \item \(-3 - 3i\): \(|z| = \sqrt{18} = 3\sqrt{2}, \theta = \frac{-3\pi}{4}\)
    \item \(3 + 4i\): \(|z| = 5, \theta = \arctan(\frac{4}{3})\)
    \item \(-1\): \(|z| = 1, \theta = \pi\)
\end{itemize}

The trigonemtric polar form of a complex number is:
\[z = r(\cos(\theta) + i\sin(\theta)))\]

\subsection*{The Complex Exponential}

For any \(\theta\), the complex exponential is given by:

\[e^{i\theta} = 1 + i\theta + \frac{1}{2!}(i\theta)^2 + \frac{1}{3!}(i\theta)^3 \: ...\]
\[e^{i\theta} = \cos(\theta) + i\sin\theta\]
\[\abs{e^{i\theta}} = 1\]

\subsubsection*{Exponential Polar Form}

Knowing this, we can write a given complex number \(z\) in exponential polar form:
\[z = re^{i\theta}\]
This happens to be an extremely useful way of expressing complex numbers. An example:

\bigskip
Express \(z = -2 + 2\sqrt{3}i\) in exponential polar form.
\[z = 4(-\frac{1}{2} + \frac{\sqrt{3}}{2}i)\]
\[4e^{i\frac{2\pi}{3}}\]
The complex exponential indicates a point on the unit circle, with the scalar r
modifying this vector to point to the actual location of \(z\).

\bigskip
The conjugate of a number in exponential form is given by:
\[e^{-i\theta} = \overline{e^{i\theta}}\]
It takes the form of inversion across the x-axis.

\subsubsection*{Properties of the Complex Exponential}
\[e^{i0} = 1\]
\[e^{i\theta_1}e^{i\theta_2} = e^{i\theta_1+i\theta_2} = e^{i(\theta_1+\theta_2)}\]
\[\frac{e^{i\theta_1}}{e^{i\theta_2}} = e^{i\theta_1 - i\theta_2} = e^{i(\theta_1-\theta_2)}\]
\[e^{i\theta_1} = e^{i\theta_2} \Leftrightarrow \theta_2 = \theta_1 + 2k\pi\]

Multiplication by \(e^{i\theta}\) is equivalent to rotation by\(\theta\). For two complex
numbers, \(z_1\) and \(z_2\), the product is given by:
\[z_1z_2 = r_1r_2e^{i(\theta_1+\theta_2)}\]
Thus, the modulus of the result is simply the product of the two inputs modulus. The
angle is increased by \(\theta\). It is intuitive that multiplication by \(i^2\) constitutes
inversion of the vector; reflection in the y-axis. The resultant angle will not necessarily be
the principle argument; one can simply add or subtract \(2\pi\) to find the principle argument.

\bigskip
Division is simply given by:
\[\frac{z_1}{z_2} = \frac{r_1}{r_2}e^{i(\theta_1-\theta_2)}\]

Exponentiation works through:
\[z^n = r^ne^{in\theta}\]

\subsection*{Subsets of the Complex Plane}
We can represent certain shapes in the complex plane through sets. For example:
\(\{z \in \C | \mathrm{Re}(z) = 3\}\) describes the line \(x = 3\). A
circle can be drawn as \(\{z \in \C:  \abs{z - p} = r\}\), where \(p\) is
the center point and \(r\) is the radius.

\subsection*{Roots}
Suppose we are trying to find all of the \(n\)-th roots of a complex number \(w\).
This means we need to find all \(z\)s that satisfy \(z^n = w\). Writing \(z = re^{i\theta}\) gives us:
\[(re^{i\theta})^n = se^{i\alpha} \Rightarrow se^{i\alpha}\]
\[\Rightarrow \abs{z} = \abs{s} \Rightarrow r^n = s \Rightarrow r = s^\frac{1}{n} = \sqrt[^n]{s}\]
We also have:
\[e^{in\theta} = e^{i\alpha} \Rightarrow n\theta = \alpha + 2k\pi \Rightarrow \theta = \frac{1}{n}(\alpha + 2k\pi)\]
Curiously, for power \(n\), there will be \(n\) solutions. Using the \(\alpha + 2k\pi\) equation, we can find an 
infinite number of solutions. However, we can stop at \(n - 1\), because \(n = k\) will yield the same result as \(k = 0\), 
as it will simply add \(2pi\) to the angle. Thus our solutions will be of the form:
\[\{\sqrt[^n]{s}e^{i\frac{1}{n}(\alpha + 2k\pi)} | k \in [0, n) \cap \N\}\]

\subsubsection*{Example}
Find the set of cube roots of \(-8\).
\begin{itemize}
    \item Solve to find \(z\) such that \(z^3 = -8\)
    \item Thus \(s = 8\), \(\alpha = \pi\)
    \item \(z = re^{i\theta}, r^3e^{i3\theta} = 8e^{i\pi}\)
    \item \(\therefore r^3 = 8 \Rightarrow r = 2\)
    \item \(3\theta = \pi + 2k\pi, k \in {0, 1, 2}\)
    \item \(\therefore \theta = \frac{\pi}{3}, \pi, \frac{5\pi}{3}\)
    \item \(z \in \{2e^{i\frac{pi}{3}}, 2e^{i\pi}, 2e^{i\frac{5\pi}{3}}\}\)
    \item Note we found the expected root; \(2e^{i\pi} = -2\)
\end{itemize}

These solutions represent 3 evenly spaced points on the circle with center \((0, 0)\) and radius 2.

\subsection*{Trigonometric Functions}
The complex exponential can be used to express \(\cos\) and \(\sin\) in a useful form for many calculations.
\[e^{i\theta} = \cos(\theta) + i\sin(\theta)\]
\[\overline{e^{i\theta}} = \cos(\theta) - i\sin(\theta)\]
\[e^{i\theta}+\overline{e^{i\theta}} = 2\cos(\theta)\]
\[\therefore \cos(\theta) = \frac{1}{2}(e^{i\theta}{e^{-i\theta}})\]
This is obvious when we consider that \(z + \bar{z} = 2\mathrm{Re}(z)\). \(\sin\) found through the same logic is:
\[\sin(\theta) = \frac{1}{2i}(e^{i\theta}-e^{-i\theta})\]

Using the binomial formula, we can expand statements such as \(\sin^4(\theta)\):
\[\sin^4(\theta) = (\frac{1}{2i})^4(e^{i\theta}-e^{-i\theta})^4\]
\[\frac{1}{2^4}((e^{i\theta})^4 - 4(e^{i\theta})^3e^{-i\theta} + 6(e^{i\theta})^2(e^{-i\theta})^2 - 4e^{i\theta}(e^{-i\theta})^3 + (e^{-i\theta})^4)\]
\[\frac{1}{2^4}(e^{i4\theta} - 4e^{i2\theta} + 6 - 4e^{-i2\theta} + e^{-i4\theta})\]
\[6 + 2\cos(4\theta) - 4 \times 2\cos(2\theta)\]
\[\sin^4(\theta) = \frac{1}{8}(3 + cos(4\theta) - 4cos(2\theta))\]

Using this approach, we can convert expressions of the form \(\sin^m(\theta)\cos^n(\theta)\) to an equivalent
form that is often more useable.

\subsection*{Square Roots}
For \(w = re^{i\theta} \in \C \backslash \{0\}\), the root finding formula gives two solutions of \(z^2 = w\):
\[z_1 = \sqrt{r}e^{i(\frac{\theta}{2})}, z_2 = \sqrt{r}e^{i(\frac{\theta}{2} + \pi)} = -z_1\]
\(z_1\) is called the \textit{principal square root} of \(w\), thus the square roots are \(\pm \sqrt{w}\).
For a negative real number, \(\sqrt{w}\) will be equal to \(\sqrt{\:\abs{w}}i\).

\section*{Polynomials}
Real polynomials are those with coefficients exclusively in \(\R\), while complex polynomials
extend this bank of coefficients to include all of \(\C\). All real polynomials are complex 
polynomials, but the inverse is not necessarily true.

\subsection*{Factorising Polynomials}
To solve differential equations, it is important to be able to integrate polynomials of the form:
\[\frac{p(x)}{q(x)}\]
We can use the quadratic formula to solve complex polynomials just as we would for real numbers.
\[\frac{-b \pm \sqrt{b^2 - 4ac}}{2a}\]
For example, factorising \(p(z) = z^2 -3iz - 2\):
\[\frac{3i \pm \sqrt{-9 -4 \times (1) \times (-2)}}{2}\]
\[= \frac{3}{2}i \pm \frac{1}{2}\sqrt{-1} = \frac{3}{2}i \pm \frac{1}{2}i\]
\[\therefore p(z) = (z - 2i)(z - i)\]
Any complex polynomial of degree \(n\) can be factorised into \(n\) linear factors over \(\C\).
\[p(z) = a(z - \alpha_1)(z - \alpha_2) ... (z - \alpha_n)\]
Thus, there is an \(x\) intercept at each value of \(\alpha\). For a polynomial of degree \(n\),
there at \(\leq n\) roots in \(\C\).

\bigskip
If \(p(z)\) is a real polynomial; the non-real roots of \(p(z)\) occur in complex conjugate pairs
\(z\) and \(\bar{z}\). For example:
\[p(z) = z^3 - 3iz^2 - 2z = z(z^2 - 3iz - 2) = z(z - 2i)(z - i) = (z - 0)(z - 2i)(z - i)\]
Because in this example, our polynomial is non-real, the roots are not in conjugate pairs. How about:
\begin{center}
    \[\{z \in \C | z^4 + z^2 -12 = 0\}\]
    \[\mathrm{set}\:w = z^2 \Rightarrow p(z) = w^2 + w - 12 = 0\]
    Now, we need to factor \(p(z)\) in terms of \(w\), and solve \(z^2 = w\).
    \[w^2 + w - 12 = (w - 3)(w + 4)\]
    \[z^2 = 3 \Rightarrow z = \sqrt{3}, -\sqrt{3}\]
    \[z^2 = -4 \Rightarrow z = 2i, -2i\]
    \[\{\sqrt{3}, -\sqrt{3}, 2i, -2i\}\]
\end{center}

\bigskip
In this example, because our roots are real, our solutions come in complex conjugate pairs.

\bigskip
If the coefficients of a polynomial \(p(x)\) are all real, then \(p(x)\) can be written as a 
product of linear and irreducible quadratic factors with exclusively real coefficients. Considering the previous example:
\[p(z) = (z - \sqrt{3})(z + \sqrt{3})(z - 2i)(z + 2i)\]
\[= (z - \sqrt{3})(z + \sqrt{3})(z^2 + 2iz - 2iz - 4i^2)\]
\[= (z - \sqrt{3})(z + \sqrt{3})(z^2 + 4)\]

\subsubsection*{Geometric Progression}
For any \(z \in \C \backslash \{1\}\):
\[z^n + z^{n - 1} + ... + z + 1 = \frac{z^{n + 1} - 1}{z - 1}\]
Using this, let us factor \(p(z) = z^3 + z^2 + z + 1\) over \(\R\).
\begin{center}
    \[(z - 1)(z^3 + z^2 + z + 1) = z^4 - 1\]
    The roots of the RHS are the 4th roots of \(w = 1\) which are:
    \[1, -1, i, -i\]
    We can also discover these by the general method:
    \[z^4 = 1 \Rightarrow r^4e^{i4\theta} = 1e^{i0}\]
    \[\Rightarrow r = 1,\: 4\theta = 2k\pi,\: k = 0, 1, 2, 3 \:\:\: \theta = \frac{1}{2}k\pi,\: k = 0, 1, 2, 3\]
    \[= 0, \frac{\pi}{2}, \pi, \frac{3\pi}{2} = 1, i, -1, -i\]
    Because these are the roots of \(z^4 - 1\), and we know that \(z - 1\) has
    the root of \(1\), we know that the roots of our polynomial are:
    \[z^3 + z^2 + z + 1 = (z + 1)(z - i)(z + i)\]
\end{center}

\section*{Functions}
Functions are a way of expressing quantities that depend on one another.
For example, a function could express the temperature of a beverage placed on a 
counter with respect to the time passed since the object was placed there.

\bigskip
More technically, a function accepts an input and maps it to an output. These inputs
are drawn from a nonempty set \(A\) called the \textit{domain} of \(f\). The outputs
are drawn from a second nonempty set, \(B\), known as the codomain of \(f\). Finally 
there is a rule that associates each element of \(A\) to a unique element of \(B\).
To express a function \(f\) with domain \(A\) and codomain \(B\) we write:
\[f: A \rightarrow B\]
This can also be read as \(f\) maps \(A\) into \(B\). For example
\[f: \R \rightarrow \R, f(x) = \frac{x + 1}{x^2 + 4}\]
is a valid function. However,
\[h: \C \rightarrow \C, h(z) = \frac{x + 1}{x^2 + 4}\]
is not, as the rule \(h(z)\) is undefined for some \(z \in \C\).
To make it a valid function, we could change the domain to
\(\C \backslash \{2i, -2i\}\).
When a function has a domain and codomain as subsets of \(\R\), we can
visualise it as a graph. Formally, a graph of \(f: A \rightarrow B\) is the set defined by:
\[\{(a, f(a)) | a \in A\} \subseteq A \times B\]
i.e. all points that can be generated by the form (input, output). The codomain
of a function does not necessarily need to be minimal or maximal. For \(f(x) = \sin(x)\),
the range of the function is \([-1, 1]\), but any codomain inclusive of this range (such 
as \(\R\)) can be chosen.

\bigskip
The image of a set under a function is the outputs of each element of that set when passed through
the function. So, for a function \(f: A \rightarrow B\) and a set \(S \subseteq A\), the image of
\(S\) under \(f\) is:
\[f(S) = \{b \in B | b = f(s), s \in S\} = \{f(s) | s \in S\}\]
The range of a function is simply the image of the domain under the function. This range
is the smallest valid codomain for the function, and thus \(\mathrm{range} \subseteq \mathrm{codomain}\)
in all cases.

\subsubsection*{Description of Functions}
\textit{Injective} functions are one-to-one functions. This means that for a given input,
there will be a unique output, shared by no other inputs; for two different inputs, there will
be two different outputs. To prove injectivity, we can show \(f(x) = f(y) \Rightarrow x = y\).

\bigskip
\textit{Monotone} functions are either increasing or decreasing across their entire domain.
These functions are always injective.

\bigskip
\textit{Surjective} functions map at least one input value to each legal output value. So
for each element of the codomain, there is at least one element of the domain which maps to it.
This can be thought of as the function having a minimal codomain; only possible outputs are
included in the codomain.

\bigskip
\textit{Bijective} functions are both injective and surjective. There is a precise
correspondence between each element of \(A\) and each element of \(B\); each element in \(A\) has 
a partner in \(B\) with no excess elements in \(B\). For example a simple bijective function is
\[f: (0, \infty) \rightarrow (0, \infty)\:\:\:f(x) = x\]
A slightly more complex example could be:
\[g: (0, \infty) \rightarrow (0, \infty)\:\:\:g(x) = \frac{1}{x}\]

\subsubsection*{Function Equality}
For functions to be equal, all of their attributes must be the same. This implies that
the domain, codomain and function are the same. However, the syntax of the 
function could be different; \(f(x) = sin(2x)\) is equivalent to \(g(x) = 2sin(x)cos(x)\). 

\subsubsection*{Composition}
Composition of functions entails taking the output of one function and feeding it into
another. i.e. taking \(f(g(x))\). This is denoted by:
\[(f \circ g)(x) = f(g(x))\]
For a function composition to be valid the codomain of \(f\) must be a subset of the
domain of \(g\). The function \(f \circ g\) will accept an argument from the domain
of \(f\) and output from the codomain of \(g\). If \(f: A \rightarrow B\) and 
\(g: C \rightarrow D\):
\[(f \circ g): A \rightarrow D\]
Composition is not commutative: \(g \circ f \neq f \circ g\). It is however
associative: \(h \circ (f \circ g) = (h \circ f) \circ g\).

\bigskip
When composing functions, we may know the domain and codomain of each of the parts.
For example, taking the following functions, we know:
\[f: (0, \infty) \rightarrow \R \:\:\: f(x) = \log(x)\]
\[g: \R \rightarrow (-\infty, 1] \:\:\: g(x) = 1 - x^2\]
But what of the composition?
\[(f \circ g)(x) = \log(1 - x^2)\]
Let us consider this problem more generally. For two functions:
\[f: \mathrm{dom}(f) \rightarrow \mathrm{range}(f)\]
\[g: \mathrm{dom}(g) \rightarrow \mathrm{range}(g)\]
For composition to be possible, \(\mathrm{range}(g)\) and \(\mathrm{dom}(f)\) must
intersect. Otherwise, the resultant function would have an empty domain. Thus, the implied
domain of the resultant function is:
\[\mathrm{dom}(f \circ g) = \{x \in \mathrm{dom}(g) | g(x) \in \mathrm{dom}(f)\}\]
And the implied range is given by:
\[\mathrm{range}(f \circ g) = f(\mathrm{range}(g) \cap \mathrm{dom}(f))\]
So for our original functions:
\[\mathrm{dom}(f \circ g) = \{x \in R | g(x) \in (0, \infty)\} = (-1, 1)\]
\[\mathrm{range}(f \circ g) = f((-\infty, 1] \cap (0, \infty)) = f((0, 1]) = (-\infty, 0]\]

\subsubsection*{Inverses}
If \(f: A \rightarrow B\) the inverse of \(f\) will be \(g: B \rightarrow A\)
such that \(g \circ f(x) = x\) and \(f \circ g(y) = y\) i.e. \(f\) undoes \(g\)
and \(g\) undoes \(f\). Inverse functions are always unique. A function must be bijective
to have an inverse function. To find an inverse function, one can observe the operations
carried out by the function and then execute the inverses of these operations in the 
reverse order.

\bigskip
Inverse functions are used implicitly to solve equations such as \(e^{x^3 + 1} = e^3\):
it is our knowledge that an inverse function to \(e^x\) exists (\(\log(x)\)) that
allows us to state with confidence that \(x^3 + 1 = 3\) and solve the equation.

For example, we can ask does \(f: \C \rightarrow \C\:\:\:f(z) = \bar{z}\) 
have an inverse? Is it bijective?

\bigskip
Yes it does. Since \(\bar{\bar{z}} = z\), \(f\) is its own inverse.

\bigskip
Sometimes, we will want to find an inverse of a function which is not bijective.
To create a bijective function, we can restrict domain or codomain to ensure these 
characteristics. To create an injective function, we can find a range which is
strictly increasing or decreasing and restrict the domain to this range. To create
a surjective function, we can simply remove all elements of the codomain which the
input doesn't map to i.e. we can restrict the codomain of the function to its range.

\bigskip
As an example, the \(\sin\) function is not injective. Thus, we take the domain
\([\frac{-\pi}{2}, \frac{\pi}{2}]\) to yield the inverse function arcsine. Note that the
range restrictions mean that while \(\theta = \arcsin(y) \Rightarrow \sin(\theta) = y\),
\(\sin(\theta) = x \Rightarrow \theta = \arcsin(x)\) only if \(\theta \in [\frac{-\pi}{2}, \frac{\pi}{2}]\)

\bigskip
Graphically, an inverse function can be understood stood as reflection across the line
\(x = y\).

\subsubsection*{Inverse Trigonometric Functions}
Alongside \(\arcsin\) we additionally have \(\arccos\) and \(\arctan\).
While \(\arctan\) has the same domain of \([\frac{-\pi}{2}, \frac{\pi}{2}]\) as
\(\arcsin\), \(\arccos\) takes the domain \([0, \pi]\) as it is strictly positive
in \([\frac{-\pi}{2}, \frac{\pi}{2}]\).

\[\mathrm{cosec}(x) = \frac{1}{\sin(x)}\]
\[\mathrm{secant}(x) = \frac{1}{\cos(x)}\]
\[\mathrm{cot}(x) = \frac{\cos(x)}{\sin(x)}\]
It turns out that \(\mathrm{cot}(x)\) is not strictly the inverse function of \(\tan(x)\).
Because \(\tan{x} = 0\) for multiples of \(\pi\), these values would be excluded from the function
of \(\frac{1}{\tan(x)}\), which they are in \(\mathrm{cot}(x)\), however, the values for which
\(\tan(x)\) is undefined, (\(\{k\pi + \frac{\pi}{2} | k \in \Z\}\)) are defined 
for \(\mathrm{cot}(x)\) thus yielding a different domain to that of \(\frac{1}{\tan(x)}\).

\section*{Vectors}
Because not all quantities are measurable by a scalar value, we sometimes need to reach
for a vector, which by being a representation of multiple values can offer both magnitude
and direction. Formally, a vector is a member of the set \(\R^2\) where
\[\R^2 = \R \times \R = \set{(a_1, a_2) | a_1 \in \R \:\mathrm{and}\: a_2 \in \R}\]
We can consider higher dimensions through \(n\)-dimensional space with \(\R^n\):
\[\R^n = \set{(a_1, a_2, ..., a_n) | a_1, a_2, ..., a_n \in \R}\]
Vectors have definitions of arithmetical operations \textit{component-wise}.
\begin{itemize}
    \item Addition: \(\vec{a} + \vec{b} = (a_1, a_2) + (b_1, b_2) = (a_1 + b_1, a_2 + b_2)\)
    \item Subtraction: \(\vec{a} - \vec{b} = (a_1, a_2) - (b_1, b_2) = (a_1 - b_1, a_2 - b_2)\)
    \item Multiplication \textit{by scalar}: \(n \times \vec{a} = (n \times a_1, n \times a_2)\)
\end{itemize}
Vectors cannot be multiplied or divided by other vectors. It is important to understand
what the elements of a vector stand for; in general vectors with differing numbers of
elements may not be added.

\bigskip
For two points \(A = (a_1, a_2)\) and \(B = (b_1, b_2)\) the vector displacement \(\vec{AB}\) is
defined as the vector subtraction of \(A\) from \(B\) i.e.
\[\vec{AB} = (b_1 - a_1, b_2 - a_2)\]
If we take the origin as \(O\), \(\vec{AB} = \vec{OB} - \vec{OA}\).

\bigskip
Two vectors are parallel if through multiplication by some constant one can
be transformed into the other. This is obvious when we consider that this is essentially
the same as saying they point in the same direction, as a consequence of the
fact that scalar multiplication doesn't change direction.

\bigskip
The length or \textit{norm} of a vector can be determined through Pythagoras's Theorem, and is
denoted \(||\vec{v}||\):
\[||\vec{v}|| = \sqrt{v_1^2 + v_2^2}\]
In higher dimensions, this extends simply through:
\[||\vec{v}|| = \sqrt{v_1^2 + v_2^2 + ... + v_n^2}\]
Which essentially corresponds to the recursive application of Pythagoras in each
dimension.

\bigskip
Some important properties of vector norms:
\begin{itemize}
    \item For any \(\vec{u}\), \(||\vec{u}|| = 0\) precisely if 
    \(\vec{u} = 0\).
    \item \(||\lambda\vec{u}|| = |\lambda|||\vec{u}||\) for any 
    \(\lambda\in\R\) and and vector \(\vec{u}\).
    \item \(||\vec{u} + \vec{v}|| \leq ||\vec{u}|| + ||\vec{v}||\) for any
    \(\vec{u}, \vec{v}\).
\end{itemize}

Vectors of length \(1\) are called \textit{unit vectors}. They are denoted by
the ``hat'': \(\hat{u}\). These are special, because they describe a direction.
To find a unit vector for a given vector \(\vec{u}\):
\[\hat{u} = \frac{1}{||\vec{u}||}\vec{u} = \frac{\vec{u}}{||\vec{u}||}\]
So for a piecewise example:
\[\vec{u} = (1, 2, -1)\]
\[||\vec{u}|| = \sqrt{1^2 + 2^2 + (-1)^2} = \sqrt{6}\]
\[\hat{u} = (\frac{1}{\sqrt{6}}, \frac{2}{\sqrt{6}}, \frac{-1}{\sqrt{6}})\]

\bigskip
There are standard unit vectors for the axis vectors up to \(\R^3\):
\begin{itemize}
    \item \(\hat{i} = (1, 0, 0)\)
    \item \(\hat{j} = (0, 1, 0)\)
    \item \(\hat{k} = (0, 0, 1)\)
\end{itemize}
These can be used to write any vector in \(\R^3\).

\subsubsection*{The Dot Product}
For two vectors \(\vec{u}\) and \(\vec{v}\) we take the scalar or dot product 
to be
\[\vec{u}\cdot\vec{v} = u_1v_1 + u_2v_2 + ... + u_nv_n\]
This operation is commutative and distributive. Order does not matter, and the
operation applied to a set of parenthesis will affect all operations within the
parenthesis. In addition, scalar multiplication by either of the vectors before 
the operation is the same as scalar multiplication by the resultant vector. 
One other useful property is that:
\[\vec{u}\cdot\vec{u} = ||\vec{u}||^2\]
For \(\vec{u}\) and \(\vec{v}\), the \textit{angle between} them is \(\theta\),
which will always be \(\leq\pi\). This angle is closely related to the dot 
product \(\vec{u}\cdot\vec{v}\). Using the Law of Cosines 
(\(c^2 = a^2 + b^2 - 2ab\cos(C)\)), we can find that:
\[||\vec{v} - \vec{u}||^2 = ||\vec{u}||^2 + ||\vec{v}||^2 - 2||\vec{u}||
||\vec{v}||\cos(\theta)\]
Rearranging this, we can solve for \(\cos(\theta)\)
\[\cos(\theta) = \frac{\vec{u}\cdot\vec{v}}{||\vec{u}||||\vec{v}||}\]
\[\theta = \arccos\left(\
\frac{\vec{u}\cdot\vec{v}}{||\vec{u}||||\vec{v}||}\right)\]
Some information can be drawn from the dot product with using trigonemtric
functions. Depending on the sign of the dot product we can determine the 
following:
\begin{itemize}
    \item \(\theta\) is acute if \(\vec{u}\cdot\vec{v} > 0\)
    \item \(\theta\) is obtuse if \(\vec{u}\cdot\vec{v} < 0\)
    \item \(\theta = \frac{\pi}{2}\) if \(\vec{u}\cdot\vec{v} = 0\)
\end{itemize}

\end{flushleft}
\end{document}
