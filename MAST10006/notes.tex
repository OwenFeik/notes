\documentclass[12pt]{report}
\renewcommand{\familydefault}{\sfdefault}

\usepackage{amsmath}
\usepackage{amssymb}
\usepackage{bbold}
\usepackage{commath}
\usepackage{pgfplots}

\newenvironment{plot}[1][
    xmin = -5,
    xmax = 5,
    ymin = -5,
    ymax = 5
]{
    \begin{center}
        \begin{tikzpicture}
            \begin{axis}[
                samples = 1000,
                #1,
                width = 12cm,
                height = 9cm,
                xlabel = \(x\),
                ylabel = \(y\),
                axis lines = middle,
                restrict y to domain = -10:10
            ]
}{
            \end{axis}
        \end{tikzpicture}
    \end {center}
}

\newcommand{\N}{\mathbb{N}}
\newcommand{\Z}{\mathbb{Z}}
\newcommand{\Q}{\mathbb{Q}}
\newcommand{\R}{\mathbb{R}}
\newcommand{\C}{\mathbb{C}}

\newcommand{\dx}{\:\mathrm{d}x}
\newcommand{\dy}{\:\mathrm{d}y}

\newcommand{\mand}{\:\mathrm{and}\:}

\newcommand{\limit}{\lim\limits}

\pgfplotsset{compat = 1.16}

\begin{document}
\begin{flushleft}

\section*{Notation}
In this subject, a variety of notation is used.
\begin{itemize}
    \item \(\mid\) read ``such that'' used largely in set definitions: 
        \(\set{x \in \R \mid x \geq 2}\)
    \item \(\forall\) read ``for all''.
    \item \(\exists\) read ``there exists''.
    \item \(\equiv\) read ``is equivalent to' used to signify function
        equality, etc.
    \item \(\ll\) read ``much less than''.
    \item \(\log\) denotes the natural logarithm, \(\ln\).
    \item Inverse trigonometric functions are written \(\arcsin\) rather than
        \(\sin^{-1}\).
\end{itemize}

Some common sets have specific symbols associated with them.
\begin{itemize}
    \item Natural numbers, \(\N = \set{1, 2, 3, \ldots}\). Exclusive of \(0\)
        in this course.
    \item Integers, \(\Z = \set{\ldots, -2, -1, 0, 1, 2, \ldots}\)
    \item Rational numbers, 
        \(\Q = \set{\frac{m}{n} \mid m, n \in \Z, n \neq 0}\)
    \item Real numbers, \(\R\)
    \item Complex numbers, \(\C = \set{x + iy \mid x, y \in \R, i^2 = -1}\)
    \item \(xy\) plane, \(\R^2 = \set{(x, y) \mid x, y \in \R}\)
    \item Three dimensional space, 
        \(\R^3 = \set{(x, y, z) \mid x, y, z \in \R}\)
\end{itemize}

\section*{Limits, Continuity, Sequences, Series}
\subsection*{Limits}
Limits form the fundamental concept behind the definition of a derivative;
the instantaneous rate of change of a function at a point is defined in terms
of a limit. \par
A limit is defined as follows. For a function \(f(x)\), the limit of \(f(x)\) 
as \(x\) approaches \(a\) is \(L\), written as
\[\lim_{x\rightarrow a}f(x) = L\]
If \(f(x)\) gets continually closer to \(L\) but \(x \neq a\). If a limit 
exists, it must be a unique finite real number.

\subsubsection*{Examples}
If \(f(x)\) is defined as follows 
\[
    f(x) =
    \begin{cases}
        2x, x \neq 1 \\
        4, x = 1 \\
    \end{cases}
\]
Evaluate the following
\[\lim_{x\rightarrow 1} f(x)\]
As \(f(x)\) gets arbritrarily close to \(2\) whenever \(x\) is close to but not
equal to \(1\), therefore the limit as \(x\rightarrow1\) is \(2\).

\bigskip\bigskip
Evaluate the following limit.
\[\lim_{x\rightarrow0}f(x) = \frac{1}{x^2}\]
As \(f(x)\) is unbounded as \(x\rightarrow0\). Therefore, \(f(x)\) cannot be 
made arbritrarily close to any one number and the limit does not exist.

\bigskip\bigskip
Evaluate the following limit.
\[
    \lim_{x\rightarrow0}
    \begin{cases}
        1, x < 0 \\
        2, x \geq 0 \\
    \end{cases}
\]
As \(x\) approaches \(0\) from the right, it grows arbritrarily close to \(2\),
while as it approaches from the left it grows arbritrarily close to \(1\). As 
it doesn't grow arbritrarily close to a single value, no limit exists here.

\subsubsection*{Additional Limt Notation}
In the previous examples, we noticed that a common way to examine limits is by
considering the values the function approaches from each side. Because this is
such a common construct, notation exists for the left and right limits
independently. The right and left limits of the function \(f\) approaching 
\(0\), examined in the final example above are written as
\[\lim_{x\rightarrow0^+}f(x) = 2\]
\[\lim_{x\rightarrow0^-}f(x) = 1\]
In general, for a limit to exist the following statement must be true
\[\lim_{x\rightarrow a} = L \Leftrightarrow \lim_{x\rightarrow a^-} = L
\mand \lim_{x\rightarrow a^+} = L\]
i.e. for the limit to be \(L\), both the left and right limits must be \(L\).

\subsubsection*{Limit Laws}
For two real valued functions \(f\) and \(g\), and \(c \in \R\) a constant. If
the limits \(\limit_{x\rightarrow a} f(x)\) and 
\(\limit_{x\rightarrow a} g(x)\) exist, then the following limit laws 
apply.

\begin{center}    
    \begin{tabular}{||c||}
        \(\limit_{x\rightarrow a}\left[f(x) + g(x)\right]
        = \limit_{x\rightarrow a} f(x) 
        + \limit_{x\rightarrow a} g(x)\) \\ [12pt]
        \(\limit_{x \rightarrow a}\left[cf(x)\right] 
        = c\limit_{x \rightarrow a}f(x)\) \\ [12pt]
        \(\limit_{x \rightarrow a}\left[f(x)g(x)\right] 
        = \limit_{x \rightarrow a}f(x) \cdot
        \limit_{x \rightarrow a}g(x)\) \\ [12pt]
        \(\limit_{x \rightarrow a}\left[\frac{f(x)}{g(x)}\right] 
        = \frac{\limit_{x \rightarrow a}f(x)}{
        \limit_{x \rightarrow a}g(x)} 
        \:\left(\limit_{x \rightarrow a}g(x) \neq 0\right)\) \\ [12pt]
        \(\limit_{x \rightarrow a} c = c\) \\ [12pt]
        \(\limit_{x \rightarrow a} x = a\) \\ [12pt]
    \end{tabular}
\end{center}

\subsubsection*{Example}
Using the limit laws, evaluate the limit
\[\lim_{x\rightarrow2}\frac{x^3 + 2x^2 - 1}{5 - 3x}\]
\[\lim_{x\rightarrow2}\frac{x^3 + 2x^2 - 1}{5 - 3x} 
= \frac{\limit_{x\rightarrow2} x^3 + 2x^2 - 1}{
\limit_{x\rightarrow2} 5 - 3x} = \frac{8 + 8 - 1}{5 - 6} 
= \frac{15}{-1} = -15\]
One could use more of the limit laws to break this down further; for instance
breaking the added terms into individual limits, or breaking the \(x^n\) terms
down using the product law.

\subsubsection*{Limits and Infinity}
We can talk about a limit as \(x\rightarrow\infty\), referring to what happens
to the limit as \(x\) is made arbritrarily large. So the limit
\[\lim_{x\rightarrow\infty} f(x) = L\]
Is stating that as \(x\) is made arbritrarily larger, \(f(x)\) becomes 
arbritrarily closer to \(L\). \(L\) must be finite for \(f(x)\) to take it's
value.

\begin{plot}
    \addplot[blue, thick] {2.71828^(-1*x)}
    node[above, pos = 0.9] {\(e^{-x}\)};
\end{plot}

For example, in the above plot we can see that as \(x\) becomes arbritrarily 
close to infinity, \(e^{-x}\) becomes arbritrarily close to \(0\). Therefore,
\[\lim_{x\rightarrow\infty} e^{-x} = 0\]
While this process of examining a graph and noting its convergence is practical
for certain examples, it is a little laborious in the long term, so we use the 
constructs of standard limits to solve many problems.

\subsubsection*{Standard Limits}
The following limits can be used without proof in this subject, with their 
truth taken as gospel.

\begin{center}    
    \begin{tabular}{||c||}
        \(\limit_{x\rightarrow\infty}\frac{1}{x^p} = 0 \:\:\:(p > 0)\) \\[12pt]
        \(\limit_{x\rightarrow\infty}r^x = 0 \:\:\:(0 \leq r < 1)\) \\[12pt]
    \end{tabular}
\end{center}

For example, in the case of \(e^{-x}\), we can use the second limit from above
by taking \(r = \frac{1}{e}\).

\subsubsection*{Terminology}
If a limit exists, we can state that \(f(x)\) \textit{converges} as \(x\) 
approaches \(a\). Inversely, we can state \(f(x)\) \textit{diverges} as \(x\)
approaches \(a\). \par
For example, as \(\sin(x)\) oscillates between \(-1\) and \(1\), it cannot 
approach a single number and therefore diverges as \(x\rightarrow\infty\). \par
It is important to note that \(\infty\) isn't a number; in general it denotes
``any arbritrarily large number''. A limit cannot be equal to infinity. Because
infinity is not a number, certain cases become indeterminate.
\[\limit_{x\rightarrow\infty} \frac{3x^2 - 2x + 3}{x^2 + 4x + 4}\]
Here, we can see that both the numerator and the denominator approach infinity
as \(x\rightarrow\infty\). We cannot therefore divide their individual limits 
to find the overall limit and must instead alter the form to find the limit. 
\par
In this case, we can do this by dividing numerator and denominator by 
\(\frac{1}{x^2}\).
\[\frac{\frac{1}{x^2}}{\frac{1}{x^2}} \cdot \frac{3x^2 - 2x + 3}{x^2 + 4x + 4}
= \frac{3 - \frac{2}{x} + \frac{3}{x^2}}{1 - \frac{4}{x} + \frac{4}{x^2}}\]
\[\frac{\limit_{x\rightarrow\infty} 3 - \frac{2}{x} + \frac{3}{x^2}}{
\limit_{x\rightarrow\infty} 1 - \frac{4}{x} + \frac{4}{x^2}} = 
\frac{3}{1} = 3\]
By modifying the fraction and then applying limit laws, we can solve this 
initially indeterminate limit.

\subsubsection*{The Sandwhich Theorem}
The Sandwhich Theorem states that if \(g(x) \leq f(x) \leq h(x)\) when 
\(x \approx a\) but \(x \neq a\) and
\[\limit_{x\rightarrow a}g(x) = \limit_{x\rightarrow a}h(x) = L\]
Then \(\limit_{x\rightarrow a}f(x) = L\). In essence, it states that if a 
function lies between two other functions who each converge to \(L\) at \(a\),
then that function must also converge to \(a\). This theorem can also be used
to solve functions of the indeterminate form \(\infty - \infty\). For example,
\(\limit_{x\rightarrow\infty} f(x) = \sqrt{x^2 + 1} - x\) is of this form. We
can simplify to some degree, but we need the Sandwhich Theorem to finish the 
problem.
\[\limit_{x\rightarrow\infty} \left(\sqrt{x^2 + 1} - x\right) \cdot 
\frac{\sqrt{x^2 + 1} + x}{\sqrt{x^2 + 1 + x}} 
= \limit_{x\rightarrow\infty} \frac{x^2 + 1 - x^2}{\sqrt{x^2 + 1} + x} 
= \limit_{x\rightarrow\infty} \frac{1}{\sqrt{x^2 + 1} + x}\]
Looking at this function, it looks like both \(\sqrt{x^2 + 1}\) and \(x\) 
become arbritrarily large as \(x\rightarrow\infty\). Thus, we would expect this
function to converge to \(0\) as it approaches \(\infty\). To prove this with 
the Sandwhich Theorem, we must find a lower bound and upper bound that each 
converge to \(0\). \par
A lower bound for this function is easy; it never drops below \(0\), so 
\(g(x)\equiv0\) will do nicely. To find an upper bound, we can simply make the 
denominator smaller, so perhaps \(h(x) = \frac{1}{x}\) is a good fit. As we
know that both of these functions converge to \(0\) as \(x\rightarrow\infty\)
through the limit laws and standard limits, we can confidently state per the
Sandwhich Theorem that \(f\rightarrow0\) as \(x\rightarrow\infty\).

\subsubsection*{Example}
A function which lends itself to use of the Sandwhich Theorem is 
\(x^2\sin(\frac{1}{x})\)

\begin{plot}[
    xmax = 0.5,
    xmin = -0.5,
    ymax = 0.2,
    ymin = -0.2
]
    \addplot[blue, thick] {x^2*sin(deg(1/x))} 
    node[left, pos = 0.48, xshift = -0.2cm] 
    {\(x^2\sin\left(\frac{1}{x}\right)\)};
    \addplot[red, thick] {x^2} 
    node[left, pos = 0.52, xshift = -0.2cm] {\(x^2\)};
    \addplot[red, thick] {-1*x^2} 
    node[left, pos = 0.52, xshift = -0.2cm] {\(-x^2\)};
\end{plot}

As shown on the above plot, the function never strays beyond the bounds of
two parabolas. We can therefore evaluate its limit through the limits of the
two bounding functions.
\[\limit_{x\rightarrow0}x^2 = \limit_{x\rightarrow0} -x^2 = 0
\Rightarrow \limit_{x\rightarrow0} x^2\sin\left(\frac{1}{x}\right) = 0\]

\subsection*{Continuity}
Continuity is a property of a function which essentially describes the 
``smoothness'' of the function. For a function \(f\) to be continuous at a 
point \(x\), the limit
\[\limit_{x\rightarrow a}f(x) = f(a)\]
Must be true; i.e. the value of \(f\) at \(x\) must be the value that \(f\) 
approaches as it becomes arbritrarily close to \(x\). As a simple example, let
us check if \(f\) is continuous at \(x = 1\).
\[f(x) = 
    \begin{cases}
        2x, x \neq 1 \\
        4, x = 1 \\
    \end{cases}
\]
\[\limit_{x\rightarrow1} f(x) = \limit_{x\rightarrow1} 2x = 2\]
\[f(1) = 4 \neq 2\]
\[\therefore f \mathrm{\:is\:not\:continuous\:at\:}1\]

\subsubsection*{Continuity Theorems}
If \(f\) and \(g\) are real valued functions and \(c\) is a constant, then
assuming \(f\) and \(g\) are continuous at \(x = a\), the following functions
are additionally continuous at \(x = a\).
\begin{itemize}
    \item \(f + g\)
    \item \(cf\)
    \item \(fg\)
    \item \(\frac{f}{g} \mathrm{\:if\:} g(a) \neq 0\)
\end{itemize}
If \(f\) is continuous at \(x = a\) and \(g\) is continuous at \(x = f(a)\),
then \(g \circ f\) is continuous at \(x = a\). Thus if two continuous functions
are composed, the resultant function will in addition be continuous.

\bigskip
All of the following function types are continuous across their domain.
\begin{itemize}
    \item Polynomials
    \item Trigonometric functions
    \item Exponential functions
    \item Logarithmic functions
    \item \(n\)th root functions
    \item Hyperbolic functions
\end{itemize}

\subsubsection*{Example}
For which values of \(x\) is \(f\) continuous?
\[f(x) = \frac{\sin(x^2 + 1)}{\log(x)}\]
We know that \(\sin\) is continuous across \(\R\), as is \(x^2 + 1\). 
\(\log(x)\) is defined for \(\R^+\). Using function composition, we
know that \(\sin(x^2 + 1)\) is continuous on \(\R\), and using division we can
see that \(f\) will be continuous for all values in the domain of \(\log(x)\) 
where \(\log(x) \neq 0\), i.e. \(\R^+ \backslash \set{1}\).

\bigskip
If \(f\) is continuous at \(b\) and \(\limit_{x\rightarrow a} g(x) = b\) then
\[\limit_{x\rightarrow a} f\left[g(x)\right] 
= f\left[\limit_{x\rightarrow a}g(x)\right] = f(b)\]

\subsubsection*{Example}
\[\limit_{x\rightarrow\infty} \sin\left(e^{-x}\right)
= \sin\left(\limit_{x\rightarrow\infty} e^{-x}\right) = \sin(0) = 0\]
We can only do this because \(\sin\) is continuous on \(\R\).

\subsubsection*{Derivatives}
The derivative is defined using a limit.
\[f^\prime(a) = \limit_{h\rightarrow0}\frac{f(a + h) - f(a)}{h}\]
If this limit exists, the function is differentiable at \(a\). Geometrically,
this implies that a tangent line can be drawn at \(a\) on the graph with 
gradient yielded by the above limit.

\begin{plot}
    \addplot[thick, red] {(x-2)^2};
    \addplot[thick, blue] {2(x-2)};
\end{plot}

If a function is differentiable at \(x = a\), the the function is continuous
at that point.

\subsubsection*{L'H\^{o}pital's Rule}
Given \(f\) and \(g\) are differentiable functions near some value \(x = a\)
and \(g^\prime(x) \neq 0\) near \(a\) but \(\neq a\). If the limit
\[\limit_{x\rightarrow a} \frac{f(x)}{g(x)}\]
Has an indeterminate form of
\[\frac{0}{0} \:\mathrm{or}\: \frac{\infty}{\infty}\]
Then L'H\^{o}pital's Rule is applicable.
\[\limit_{x\rightarrow a}\frac{f(x)}{g(x)} 
= \limit_{x\rightarrow a}\frac{f^\prime(x)}{g^\prime(x)}\]
For example, we can solve limits like the one below much more easily.
\[\limit_{x\rightarrow0}\frac{\sin(x)}{x} 
= \limit_{x\rightarrow0}\frac{\cos(x)}{1} = \frac{\cos(0)}{1} = 1\]
L'H\^{o}pital's Rule can be applied repeatedly to the same function as long
as its conditions still hold. Sometimes the form of a function must be modified
to have a quotient before the rule can be applied to it.

\subsubsection*{Rigour}
Thus far, we have used the phrase \textit{arbritrarily close} without a proper
definition of what that means. A more formal definition of this exists, where
we pick an arbritray positive number \(\epsilon\). When we do this, there is
another positive number \(\delta\) for which \(\abs{f(x) - L} < \epsilon\) when
\(0 < \abs{x - a} < \delta\).

\end{flushleft}
\end{document}
