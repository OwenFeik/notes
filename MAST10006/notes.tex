\documentclass[12pt]{report}
\renewcommand{\familydefault}{\sfdefault}

\usepackage{bbold}
\usepackage{commath}

\newcommand{\N}{\mathbb{N}}
\newcommand{\Z}{\mathbb{Z}}
\newcommand{\Q}{\mathbb{Q}}
\newcommand{\R}{\mathbb{R}}
\newcommand{\C}{\mathbb{C}}

\newcommand{\dx}{\:\mathrm{d}x}
\newcommand{\dy}{\:\mathrm{d}y}

\begin{document}
\begin{flushleft}

\section*{Notation}
In this subject, a variety of notation is used.
\begin{itemize}
    \item \(\mid\) read ``such that'' used largely in set definitions: 
        \(\set{x \in \R \mid x \geq 2}\)
    \item \(\forall\) read ``for all''.
    \item \(\exists\) read ``there exists''.
    \item \(\equiv\) read ``is equivalent to' used to signify function
        equality, etc.
    \item \(\ll\) read ``much less than''.
    \item \(\log\) denotes the natural logarithm, \(\ln\).
    \item Inverse trigonometric functions are written \(\arcsin\) rather than
        \(\sin^{-1}\).
\end{itemize}

Some common sets have specific symbols associated with them.
\begin{itemize}
    \item Natural numbers, \(\N = \set{1, 2, 3, \ldots}\). Exclusive of \(0\)
        in this course.
    \item Integers, \(\Z = \set{\ldots, -2, -1, 0, 1, 2, \ldots}\)
    \item Rational numbers, 
        \(\Q = \set{\frac{m}{n} \mid m, n \in \Z, n \neq 0}\)
    \item Real numbers, \(\R\)
    \item Complex numbers, \(\C = \set{x + iy \mid x, y \in \R, i^2 = -1}\)
    \item \(xy\) plane, \(\R^2 = \set{(x, y) \mid x, y \in \R}\)
    \item Three dimensional space, 
        \(\R^3 = \set{(x, y, z) \mid x, y, z \in \R}\)
\end{itemize}

\end{flushleft}
\end{document}
