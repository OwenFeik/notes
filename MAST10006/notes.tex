\documentclass[12pt]{report}
\renewcommand{\familydefault}{\sfdefault}

\usepackage{amsmath}
\usepackage{bbold}
\usepackage{commath}

\newcommand{\N}{\mathbb{N}}
\newcommand{\Z}{\mathbb{Z}}
\newcommand{\Q}{\mathbb{Q}}
\newcommand{\R}{\mathbb{R}}
\newcommand{\C}{\mathbb{C}}

\newcommand{\dx}{\:\mathrm{d}x}
\newcommand{\dy}{\:\mathrm{d}y}

\newcommand{\mand}{\:\mathrm{and}\:}

\begin{document}
\begin{flushleft}

\section*{Notation}
In this subject, a variety of notation is used.
\begin{itemize}
    \item \(\mid\) read ``such that'' used largely in set definitions: 
        \(\set{x \in \R \mid x \geq 2}\)
    \item \(\forall\) read ``for all''.
    \item \(\exists\) read ``there exists''.
    \item \(\equiv\) read ``is equivalent to' used to signify function
        equality, etc.
    \item \(\ll\) read ``much less than''.
    \item \(\log\) denotes the natural logarithm, \(\ln\).
    \item Inverse trigonometric functions are written \(\arcsin\) rather than
        \(\sin^{-1}\).
\end{itemize}

Some common sets have specific symbols associated with them.
\begin{itemize}
    \item Natural numbers, \(\N = \set{1, 2, 3, \ldots}\). Exclusive of \(0\)
        in this course.
    \item Integers, \(\Z = \set{\ldots, -2, -1, 0, 1, 2, \ldots}\)
    \item Rational numbers, 
        \(\Q = \set{\frac{m}{n} \mid m, n \in \Z, n \neq 0}\)
    \item Real numbers, \(\R\)
    \item Complex numbers, \(\C = \set{x + iy \mid x, y \in \R, i^2 = -1}\)
    \item \(xy\) plane, \(\R^2 = \set{(x, y) \mid x, y \in \R}\)
    \item Three dimensional space, 
        \(\R^3 = \set{(x, y, z) \mid x, y, z \in \R}\)
\end{itemize}

\section*{Limits, Continuity, Sequences, Series}
\subsection*{Limits}
Limits form the fundamental concept behind the definition of a derivative;
the instantaneous rate of change of a function at a point is defined in terms
of a limit. \par
A limit is defined as follows. For a function \(f(x)\), the limit of \(f(x)\) 
as \(x\) approaches \(a\) is \(L\), written as
\[\lim_{x\rightarrow a}f(x) = L\]
If \(f(x)\) gets continually closer to \(L\) but \(x \neq a\). If a limit 
exists, it must be a unique finite real number.

\subsubsection*{Examples}
If \(f(x)\) is defined as follows 
\[
    f(x) =
    \begin{cases}
        2x, x \neq 1 \\
        4, x = 1 \\
    \end{cases}
\]
Evaluate the following
\[\lim_{x\rightarrow 1} f(x)\]
As \(f(x)\) gets arbritrarily close to \(2\) whenever \(x\) is close to but not
equal to \(1\), therefore the limit as \(x\rightarrow1\) is \(2\).

\bigskip\bigskip
Evaluate the following limit.
\[\lim_{x\rightarrow0}f(x) = \frac{1}{x^2}\]
As \(f(x)\) is unbounded as \(x\rightarrow0\). Therefore, \(f(x)\) cannot be 
made arbritrarily close to any one number and the limit does not exist.

\bigskip\bigskip
Evaluate the following limit.
\[
    \lim_{x\rightarrow0}
    \begin{cases}
        1, x < 0 \\
        2, x \geq 0 \\
    \end{cases}
\]
As \(x\) approaches \(0\) from the right, it grows arbritrarily close to \(2\),
while as it approaches from the left it grows arbritrarily close to \(1\). As 
it doesn't grow arbritrarily close to a single value, no limit exists here.

\subsubsection*{Additional Limt Notation}
In the previous examples, we noticed that a common way to examine limits is by
considering the values the function approaches from each side. Because this is
such a common construct, notation exists for the left and right limits
independently. The right and left limits of the function \(f\) approaching 
\(0\), examined in the final example above are written as
\[\lim_{x\rightarrow0^+}f(x) = 2\]
\[\lim_{x\rightarrow0^-}f(x) = 1\]
In general, for a limit to exist the following statement must be true
\[\lim_{x\rightarrow a} = L \Leftrightarrow \lim_{x\rightarrow a^-} = L
\mand \lim_{x\rightarrow a^+} = L\]
i.e. for the limit to be \(L\), both the left and right limits must be \(L\).

\subsubsection*{Limit Laws}
For two real valued functions \(f\) and \(g\), and \(c \in \R\) a constant. If
the limits \(\lim\limits_{x\rightarrow a} f(x)\) and 
\(\lim\limits_{x\rightarrow a} g(x)\) exist, then the following limit laws 
apply.

\begin{center}    
    \begin{tabular}{||c||}
        \(\lim\limits_{x\rightarrow a}\left[f(x) + g(x)\right]
        = \lim\limits_{x\rightarrow a} f(x) 
        + \lim\limits_{x\rightarrow a} g(x)\) \\ [12pt]
        \(\lim\limits_{x \rightarrow a}\left[cf(x)\right] 
        = c\lim\limits_{x \rightarrow a}f(x)\) \\ [12pt]
        \(\lim\limits_{x \rightarrow a}\left[f(x)g(x)\right] 
        = \lim\limits_{x \rightarrow a}f(x) \cdot
        \lim\limits_{x \rightarrow a}g(x)\) \\ [12pt]
        \(\lim\limits_{x \rightarrow a}\left[\frac{f(x)}{g(x)}\right] 
        = \frac{\lim\limits_{x \rightarrow a}f(x)}{
        \lim\limits_{x \rightarrow a}g(x)} 
        \:\left(\lim\limits_{x \rightarrow a}g(x) \neq 0\right)\) \\ [12pt]
        \(\lim\limits_{x \rightarrow a} c = c\) \\ [12pt]
        \(\lim\limits_{x \rightarrow a} x = a\) \\ [12pt]
    \end{tabular}
\end{center}

\subsubsection*{Example}
Using the limit laws, evaluate the limit
\[\lim_{x\rightarrow2}\frac{x^3 + 2x^2 - 1}{5 - 3x}\]
\[\lim_{x\rightarrow2}\frac{x^3 + 2x^2 - 1}{5 - 3x} 
= \frac{\lim\limits_{x\rightarrow2} x^3 + 2x^2 - 1}{
\lim\limits_{x\rightarrow2} 5 - 3x} = \frac{8 + 8 - 1}{5 - 6} 
= \frac{15}{-1} = -15\]
One could use more of the limit laws to break this down further; for instance
breaking the added terms into individual limits, or breaking the \(x^n\) terms
down using the product law.

\end{flushleft}
\end{document}
