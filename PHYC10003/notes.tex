\documentclass[12pt]{report}
\renewcommand{\familydefault}{\sfdefault} % San Serif Font
\usepackage{gensymb}
\begin{document}
\title{PHYC10003 Notes}
\begin{flushleft}
\chapter*{Motion}

\section*{Galiliean Mechanics}
Galilieo's Principle can describe an event using two variables (x, t) and (x', t')
In this case, x' = x + vt and t' = t. This treats the laws of mechanics as the 
same in any inertial frame; a frame moving at a constant speed. While these laws
function at low velocities, they start to break down as v approaches c.

\section*{Fundamental Quantities}
There are seven fundamental quantities, of which three are relevant; length, time
and mass. These seven can be used to define all other physical quantities.

\section*{Dimensional Analysis}
Dimensonal analysis is the process of taking a formula, breaking it down into input
units, and solving this equation to ascertain whether the formula is accurate. e.g.
\bigskip
\begin{center}
Fundamental quantities: length (\(L\)) and time (\(T\)).
\end{center}
\[v = v_0 + \frac{1}{2}at^2\]
\[\frac{L}{T} = \frac{L}{T} + \frac{L}{T^2}T^2\]
\[\frac{L}{T} = \frac{L}{T} + L \Rightarrow \mathrm{incorrect}\]

\section*{Kinematics}
\begin{itemize}
    \item Position is defined as a vector indicating the distance from a reference point, the origin. It is directional.
    \item Displacement is the change in \(x\) or position. i.e. the \(\Delta x\).
    \item Velocity is the ratio of displacement to time. 
    \item Acceleration is the rate of change of velocity.
\end{itemize}

\subsection*{Velocity}

On a graph of \(x\) vs \(t\), average velocity is the slope of the line that connects two points.

\medskip

Average speed is total distance divided by time. Thus, it is always positive.

\medskip

Instantaneous velocity is obtained at a single moment in time, it is given by the slope of a curve
at that point in time, i.e. 
\[v_t = \frac{dx}{dt}\]

\subsection*{Acceleration}
Acceleration is a vector; it has direction and magnitude. Acceleration can be expressed
in \(ms^{-2}\) or in units of \(g\): \(9.8ms^{-2}\).
\[a_{\mathrm{avg}} = \frac{v_2 - v_1}{t_2 - t_1} = \frac{\Delta v}{\Delta t}\]
\[a = \frac{dv}{dt}\]
For many cases acceleration is constant, such as for objects falling due to the influence of gravity.
In these instances, the equations of constant acceleration can be used. Where \(s = x - x_0\).  These include:
\[v = v_0 + at\] 
\[s = v_0 + \frac{1}{2}at^2\]
\[s = vt - \frac{1}{2}at^2\]


\section*{Vectors}

A vector is a mathematical object with magnitude and direction. Position, velocity and acceleration
are examples of vector quantity. Scalar quantities, such as speed, don't have a direction.

\subsubsection*{Operations}

The vector sum, or resultant vector is the net displacement (or velocity, acceleration, etc)
of two or more vectors.
\[\vec{s} = \vec{a} + \vec{b}\]
Vector addition is commutative. \(\vec{a} + \vec{b} = \vec{b} + \vec{a}\) is the commutative law.
Vector addition is also associative. Any order of addition will yield the same result. 
\par
A negative sign reverse vector direction.
\[\vec{b}+(-\vec{b}) = 0\]
We use this to define vector subtraction.
\[\vec{d} = \vec{a} - \vec{b} = \vec{a} + (-\vec{b})\]
These rules hold for all vectors, irrespective of what quantity they depict. Obviously, only
vectors of the same type, with the same units can be added. This can be checked with dimensional 
analysis.

\subsubsection*{Components}

Rather than adding them graphically, one can add vectors by breaking them down into their
components. Components in two dimensions can be found with:
\[a_x = a \cos(\theta) \:\mathrm{and}\: a_y = a \sin(\theta)\]
\[a = \sqrt{a^2_x + a^2_y}\]
In \(3\) dimensions we need more components so we use \(a, \theta, \phi\) or \(a_x, a_y, a_z\). 
Unit vectors. A unit vector has the following properties:
\begin{itemize}
    \item Has magnitude \(1\)
    \item Has a particular direction
    \item Lacks dimension and unit
    \item is labeled with a hat: \(\hat{i}\)
\end{itemize}
\[\vec{a} = a_x\hat{i} + a_y\hat{j}\ (+ a_z\hat{k})\]
\(a_x\) and \(a_y\) alone are scalar components. Vectors can be added with these components:
\(r_x = a_x + b_x\) etc. To subtract two vectors, we substract components: \(r_x = a_x - b_x\) etc.

\subsubsection*{Rotations}

Because vectors are independent of their coordinate system, we can rotate the system while
maintaining the vector.
\[a = \sqrt{a^2_x + a^2_y} = \sqrt{a'^2_x + a'^2_y}\]

\subsubsection*{Scalar Multiplication}

To multiply a vector by a scalar, we simply multiply each component by the scalar. The direction
is unchanged unless the scalar is negative, in which case it is reversed.
\[3a = 3a_x\hat{i} + 3a_y\hat{j}\]

\subsubsection*{Dot Product}

The dot product or scalar product of two vectors results in a scalar where a and b are magnitudes and
\(\phi\) is the angle between the directions of the tow vectors.
\[\vec{a} \cdot \vec{b} = ab\cos(\phi)\]
Dot product is the projection of one vector onto another. When \(\theta = 90\degree\) the dot product
is \(0\). \(\hat{i}, \hat{j}, \hat{k}\) are described as \textit{orthonormal}:
\[\hat{i} \cdot \hat{i} = 1\]
\[\hat{i} \cdot \hat{j} = 0\]
The dot product is commutative. \(a \cdot b = b \cdot a\).

\subsubsection*{Vector Product}

The vector product is another ay of multiply two vectors, also known as the vector product.
\[c = ab\:\sin(\phi)\]
If \(a\) and \(b\) are parallel or antiparallel the vector product is \(0\). It is at a maximum
when they are perpendicular. The vector product is not commutative. The direction of \(c\)
is perpendicular to both \(a\) and \(b\). So for example:
\[\hat{i} \times \hat{j} = \hat{k}\]
\[\hat{i} \times \hat{i} = 0\]

\subsubsection*{Position Vector}
\[\vec{r} = x\hat{i} + y\hat{j} +z\hat{k}\]
Change is position vector is displacement:
\[\Delta\vec{r} = \Delta x \hat{i} + \Delta y \hat{j} + \Delta z \hat{k}\]

\subsubsection*{Instantaneous Values}
\[\vec{v} = \frac{d\vec{r}}{dt}\]
\[\vec{a} = \frac{d\vec{v}}{dt}\]
In general, the instantaneous value of a changing quantity is given by the relavant
derivative, just as with scalars.

\section*{Projectile Motion}
\[\vec{v_0} = v_{0x}\hat{i} + v_{0y}\hat{j}\]
\[v_{0x} = v_0\cos{\theta} \:\mathrm{and}\: v_{0y} = v_0\sin{\theta}\]
A projectile has a duration of flight equal to twice the time taken from the highest point to 
the ground. Thus, when comparing projectiles, the one with the lower highest point will
have a shorter period of flight. The range is given by:
\[R = \frac{v^2_0}{g}\sin 2 \theta_0\]

\section*{Uniform Circular Motion}
In uniform circular motion, velocity and acceleration each have constant magnitude but changing direction.
This acceleration is called centripetal acceleration. The velocity is always at tangent to the circular
path, while the accceleration is always inward toward the center of the circle.
\[a = \frac{v^2}{R} \Rightarrow F = m\frac{v^2}{R}\]
\[T = \frac{2\pi R }{v}\]
The centripetal acceleration can come from many sources, such as gravity in a space station, friction for 
a car drive in a circle, or tension for a ball whirled on a string.

\bigskip
Consider a bicycle going around a vertical loop. At the top of the loop, the forces acting on the cyclists
are the normal force (downward, opposing the centripetal force) and gravity, also downward.
\[-F_N - mg = m(-\frac{v^2}{R})\]
For the cyclist to not fall, normal force must be at minimum \(0\); centripetal force must cancel out with gravity.
Thus, for a loop of radius \(2.7\mathrm{m}\):
\[v = \sqrt{gR} = \sqrt{9.8 \times 2.7} = 5.1\mathrm{m/s}\]

\section*{Relative Motion}
If reference frames are moving relative to each other, they may each observe differing velocities of 
an event, because the frames velocity will be added to the velocity of what they observe. Each 
observer will, however, observe the same acceleration of the event, assume both reference frames
are inertial. If \(P\) is the event being observed, \(A\) is our primary reference frame, stationery
with respect to \(P\) and \(B\) is our moving reference frame:
\[\vec{r}_{PA} = \vec{r}_{PB} + \vec{r}_{BA}\]

\section*{Force}
A force is a "push or pull" on an object, which causes acceleration. Newton's laws of 
motion are applicable to objects which aren't moving at near \(c\) or at atomic scale.

\bigskip
Newton's first law states that: if no net force acts on a body, the body's velocity cannot change.
These laws also only apply in inertial frames; over long enough distances, even the surface of the 
earth is non-inertial.

\bigskip
Mass is inversely proportional to acceleration due to force. Thus we arrive at Newton's second law:
\[\vec{F} = m\vec{a}\]
Acceleration along an axis is affected only by forces along the same axis. Thus, complex
problems can be solved through decomposition.

\bigskip
Weight is mass under gravitation force. Given by:
\[W = mg\]

\subsubsection*{Normal Force}
The normal force is the pushback of a surface against a force (such as weight) exerted on it.
This force is always opposite to the force applied to the surface, and is always equal to the force
applied to the surface, so that the forces on the object are in balance and thus the object remains 
stationery. This force is described by Newton's third law of motion.
\[\vec{F}_{BC} = -\vec{F}_{CB}\]
These forces are described as a third law force pair; a concept which arises any time 
two objects interact.

\subsubsection*{Friction}
This occurs when an object is attempting to slide over another. It opposes the direction
of motion of the moving object. If the friction is significant enough to stop the objects 
movement, then it will be equivalent to the force moving the object. Friction exists because
surfaces are slightly rough, thus causing them to stick together a little.

\bigskip
Friction is essential for tasks such as picking things up for for propulsion, but overcoming
it is also important for applications like efficiency in engines or motion through means such as 
roller skates. In general, anything intended to remain in motion must overcome friction.

\bigskip
Two types of frictional force exist; static frictional force acts as a kind of ``normal'' frictional
force, preventing the motion of objects. It can increase from 0 to some maximum force. Thus, the maximum
static frictional force must be overcome to begin motion for an object. Kinetic frictional force is the 
opposing force that acts on an object in motion, and is constant and usually weaker than static force. To
maintain constant velocity, a constant force equal to the kinetic frictional force must be applied.

\bigskip
The magnitude of static friction is given by:
\[f_{s, \mathrm{max}} = \mu_sF_N\]
Where \(\mu_s\) is the \textit{coefficient of static friction}. If the applied force
exceeds \(f_{s, \mathrm{max}}\), sliding begins. The coefficient depends on the two surfaces
involved. For example, rubber on dry concrete has a coefficient of static friction of \(\mu_s = 1.0\) and
of kinetic friction \(\mu_k = 0.8\). Kinetic frictional force in general is given by:
\[f_k = \mu_kF_N\]
In general, \(\mu_s > \mu_k\). The force required to overcome frictional is known as \(F_C\), and is 
related linearly to mass. To determine \(\mu_s\) we can use a setup with an object on a slope. If we slowly
increase the angle, at the moment when the block just begins to move:
\[Mg\sin(\theta) = \mu_sN\]
\[Mg\cos(\theta) = N\]
\[\Rightarrow \tan(\theta) = \mu_s\]
Where \(\theta\) is the angle between the surface and the flat. If we further increase the angle, the block
will accelerate down the slope with acceleration:
\[a = (g\sin(\theta) - \mu_kg\cos(\theta))\]
For a force applied at an angle, we need to decompose into \(x\) and \(y\) components to 
address frictional forces.

\subsubsection*{Drag Force}
A fluid is anything that can flow, i.e. a gas or liquid. Where there is a relative velocity between fluid
and object, there is a resultant drag force. This force will oppose the relative motion, and will point
in the direction of flow, relative to the body.

\bigskip
Drag force for a moving object with constant velocity is given by:
\[D = \frac{1}{2}C\rho Av^2\]
Where:
\begin{itemize}
    \item \(v\) is relative velocity.
    \item \(\rho\) is the air density.
    \item \(C\) is an experimentally determined drag coefficient.
    \item \(A\) is the cross-sectional area of the body.
\end{itemize}
In practice, \(C\) will not be constant for all values of \(v\).
For a falling object:
\[D - F_g = ma\]
Once the drag force equals \(F_g\), the object will reach terminal velocity at:
\[v_t = \sqrt{\frac{2F_g}{C\rho A}}\]
If \(A\) is increased, \(v_t\) decreases; the principle behind parachutes.

\subsubsection*{Tension}
Tension occurs when a cord or rope is attached to an object and pulled to apply a force to
the object. The cord will apply force to the object equal to it's tension force.
If a system has multiple attachment points to a tensioned rope, each point pulling upward
constitutes a force of \(T\) upward on the system; for example, with \(2\) attachment points, There
is a mechanical advantage factor of \(2\).

\subsubsection*{Example}
A car is moving with an acceleration of \(1.2ms^{-2}\). In this car, a pendulum hangs. Find
the angle at which this pendulum hangs relative to the normal.
\[a = 1.2ms^{-2}\]
\[x: ma = F_T\sin(\theta)\]
\[y: 0 = F_T\cos(\theta)\ - mg\]
\[\frac{F_T\sin(\theta)}{F_T\cos(\theta)} = \frac{ma}{mg}\]
\[\tan(\theta) = 0.122\]
\[\theta = 7\degree\]

\subsection*{Perception of Forces}
We percieve forces due to our awareness of our inertia and our sensing of the normal
force of the ground against us. Thus, we are in freefall \((F_N = 0)\), we feel weightless
because out acceleration is simply \(g\) and there is no normal force. When we are in a \
lift accelerating upward, we feel heavier because we are experiencing a greater normal force
due to our greater acceleration. The \textit{otic labyrinth} is an organ inside the ear, which
uses hairs connected to an elastically suspended membrane. As the hairs move, cells detect their
force and report this to the brain.

\bigskip
When a person is in a high acceleration vehicle, the perceive the angle to be tilting upward 
even when it isn't. This is because the resultant force vector of the normal force due to gravity
and the normal force from the accceleration points upward relative to the flat plane.

\section*{Energy}
Energy is required for any sort of motion. It is a scalar quantity. The transfer of 
energy is known as work.

\subsection*{Kinetic Energy}
The faster an object moves, the higher its kinetic energy (\(E_k\)). For a stationery object,
\(E_k = 0\). For an object moving at non-relativistic speeds:
\[E_k = \frac{1}{2}mv^2\]
Kinetic energy is measured in joules (\(\mathrm{J}\)).
\[1 \:\mathrm{joule} = 1\mathrm{J} = 1\mathrm{kg}\mathrm{m}^2\mathrm{s}^{-2}\]
If an objects kinetic energy changes, then work has been done on that object by a force.
If energy is transferred to an object, positive work has been done, if energy is transferred
from an object, negative work. For example, in a descending lift, gravity is doing positive work,
by increasing the kinetic energy of the elevator, while the tension in the cable is doing negative
work. Work can be determined from velocity through:
\[\frac{1}{2}mv^2 - \frac{1}{2}mv_0^2 = Fd\]
Where \(d\) denotes distance. To calculate work done with vector quantities:
\[W = Fd\cos(\phi)\]
\[W = \vec{F}\cdot\vec{d}\]
Work is also measured in joules. For two or more forces, the net work is the sum
of work done. When working with vectors, we can either calculate work for each vector
independently and sum these, or take a vector sum and calculate work done once.
\[\Delta E_k = E_{k,f} - E_{k,i} = W\]
i.e. the change in kinetic energy is equal to the net work done.

\subsection*{Hooke's Law}
When compressed or stretched, springs will attempt to return to the origin, exerting
force given by:
\[\vec{F}_s = -k\vec{d}\]
Where \(k\) is the springs experimentally determined \textit{spring constant}.
\(k\) is a measure of the stiffness of the spring. \(F_s\) is a variable force,
and exhibits a linear relationship between \(F\) and \(d\).

\bigskip
We can find the work done by integrating:
\[W_s = \int^{x_f}_{x_i}-F_x\:dx\]
Plugging in \(kx\) for \(F_x\):
\[W_s = \frac{1}{2}kx^2_i - \frac{1}{2}kx^2_f\]
If the spring ends up closer to the origin (\(x = 0\)), \(W_s\) will be positive.
If it is further away, \(W_s < 0\). For an initial posisition of \(x = 0\):
\[W_s = \frac{1}{2}kx^2\]

\subsection*{Power}
Power is the time rate at which a force does work. It is measured in joules per 
second or watts (\(W\)). If a force does \(W\) work in time \(\Delta t\) the average
power due to the force is:
\[P_\mathrm{avg} = \frac{W}{\Delta t}\]
and the instantaneous power at a given moment is:
\[P = \frac{dW}{dt}\]
\[P = \vec{F}\cdot\vec{v}\]

\end{flushleft}
\end{document}
