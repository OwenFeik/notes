\documentclass[12pt]{report}
\renewcommand{\familydefault}{\sfdefault} % San Serif Font
\begin{document}
\title{PHYC10003 Notes}
\begin{flushleft}
\chapter*{Motion}

\section*{Galiliean Mechanics}
Galilieo's Principle can describe an event using two variables (x, t) and (x', t')
In this case, x' = x + vt and t' = t. This treats the laws of mechanics as the 
same in any inertial frame; a frame moving at a constant speed. While these laws
function at low velocities, they start to break down as v approaches c.

\section*{Fundamental Quantities}
There are seven fundamental quantities, of which three are relevant; length, time
and mass. These seven can be used to define all other physical quantities.

\section*{Dimensional Analysis}
Dimensonal analysis is the process of taking a formula, breaking it down into input
units, and solving this equation to ascertain whether the formula is accurate. e.g.
\bigskip
\begin{center}
Fundamental quantities: length (\(L\)) and time (\(T\)).
\end{center}
\[v = v_0 + \frac{1}{2}at^2\]
\[\frac{L}{T} = \frac{L}{T} + \frac{L}{T^2}T^2\]
\[\frac{L}{T} = \frac{L}{T} + L \Rightarrow \mathrm{incorrect}\]

\section*{Kinematics}
\begin{itemize}
    \item Position is defined as a vector indicating the distance from a reference point, the origin. It is directional.
    \item Displacement is the change in \(x\) or position. i.e. the \(\Delta x\).
    \item Velocity is the ratio of displacement to time. 
    \item Acceleration is the rate of change of velocity.
\end{itemize}

\subsection*{Velocity}

On a graph of \(x\) vs \(t\), average velocity is the slope of the line that connects two points.

\medskip

Average speed is total distance divided by time. Thus, it is always positive.

\medskip

Instantaneous velocity is obtained at a single moment in time, it is given by the slope of a curve
at that point in time, i.e. 
\[v_t = \frac{dx}{dt}\]

\subsection*{Acceleration}
Acceleration is a vector; it has direction and magnitude. Acceleration can be expressed
in \(ms^{-2}\) or in units of \(g\): \(9.8ms^{-2}\).
\[a_{\mathrm{avg}} = \frac{v_2 - v_1}{t_2 - t_1} = \frac{\Delta v}{\Delta t}\]
\[a = \frac{dv}{dt}\]
For many cases acceleration is constant, such as for objects falling due to the influence of gravity.
In these instances, the equations of constant acceleration can be used. Where \(s = x - x_0\).  These include:
\[v = v_0 + at\] 
\[s = v_0 + \frac{1}{2}at^2\]
\[s = vt - \frac{1}{2}at^2\]

\end{flushleft}
\end{document}
