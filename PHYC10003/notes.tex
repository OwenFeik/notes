\documentclass[12pt]{report}
\renewcommand{\familydefault}{\sfdefault} % San Serif Font
\begin{document}
\title{PHYC10003 Notes}
\begin{flushleft}
\chapter*{Motion}
\section*{Galiliean Mechanics}
Galilieo's Principle can describe an event using two variables (x, t) and (x', t')
In this case, x' = x + vt and t' = t. This treats the laws of mechanics as the 
same in any inertial frame; a frame moving at a constant speed. While these laws
function at low velocities, they start to break down as v approaches c.

\section*{Fundamental Quantities}
There are seven fundamental quantities, of which three are relevant; length, time
and mass. These seven can be used to define all other physical quantities.

\section*{Dimensional Analysis}
Dimensonal analysis is the process of taking a formula, breaking it down into input
units, and solving this equation to ascertain whether the formula is accurate. e.g.
\bigskip
\begin{center}
Fundamental quantities: length (\(L\)) and time (\(T\)).
\end{center}
\[v = v_0 + \frac{1}{2}at^2\]
\[\frac{L}{T} = \frac{L}{T} + \frac{L}{T^2}T^2\]
\[\frac{L}{T} = \frac{L}{T} + L \Rightarrow \mathrm{incorrect}\]

\end{flushleft}
\end{document}