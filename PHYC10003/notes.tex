\documentclass[12pt]{report}
\renewcommand{\familydefault}{\sfdefault} % San Serif Font
\usepackage{gensymb}
\usepackage{amsmath}
\usepackage{commath}
\usepackage{bbold}

\begin{document}
\begin{flushleft}
\title{PHYC10003 Notes}
\chapter*{Motion}

\section*{Galiliean Mechanics}
Galilieo's Principle can describe an event using two variables (\(x\), \(t\))
and (\(x\prime\), \(t\prime\)). In this case, \(x\prime = x + vt\) and 
\(t\prime = t\). This treats the laws of mechanics as the same in any inertial
frame; a frame moving at a constant speed. While these laws function at low 
velocities, they start to break down as v approaches c.

\section*{Fundamental Quantities}
There are seven fundamental quantities, of which three are relevant; length, 
time and mass. These seven can be used to define all other physical quantities.

\section*{Dimensional Analysis}
Dimensonal analysis is the process of taking a formula, breaking it down into 
input units, and solving this equation to ascertain whether the formula is 
accurate. e.g.

\bigskip
\begin{center}
    Fundamental quantities: length (\(L\)) and time (\(T\)).
\end{center}
\[v = v_0 + \frac{1}{2}at^2\]
\[\frac{L}{T} = \frac{L}{T} + \frac{L}{T^2}T^2\]
\[\frac{L}{T} = \frac{L}{T} + L \Rightarrow \mathrm{incorrect}\]

\section*{Kinematics}
\begin{itemize}
    \item Position is defined as a vector indicating the distance from a 
    reference point, the origin. It is directional.
    \item Displacement is the change in \(x\) or position. i.e. the 
    \(\Delta x\).
    \item Velocity is the ratio of displacement to time. 
    \item Acceleration is the rate of change of velocity.
\end{itemize}

\subsection*{Velocity}

On a graph of \(x\) vs \(t\), average velocity is the slope of the line that 
connects two points.

\medskip

Average speed is total distance divided by time. Thus, it is always positive.

\medskip

Instantaneous velocity is obtained at a single moment in time, it is given by 
the slope of a curve at that point in time, i.e. 
\[v_t = \frac{dx}{dt}\]

\subsection*{Acceleration}
Acceleration is a vector; it has direction and magnitude. Acceleration can be 
expressed in \(ms^{-2}\) or in units of \(g\): \(9.8ms^{-2}\).
\[a_{\mathrm{avg}} = \frac{v_2 - v_1}{t_2 - t_1} = \frac{\Delta v}{\Delta t}\]
\[a = \frac{dv}{dt}\]
For many cases acceleration is constant, such as for objects falling due to the
influence of gravity. In these instances, the equations of constant 
acceleration can be used. Where \(s = x - x_0\).  These include:
\[v = v_0 + at\] 
\[s = v_0 + \frac{1}{2}at^2\]
\[s = vt - \frac{1}{2}at^2\]


\section*{Vectors}

A vector is a mathematical object with magnitude and direction. Position, 
velocity and acceleration are examples of vector quantity. Scalar quantities, 
such as speed, don't have a direction.

\subsubsection*{Operations}

The vector sum, or resultant vector is the net displacement (or velocity, 
acceleration, etc) of two or more vectors.
\[\vec{s} = \vec{a} + \vec{b}\]
Vector addition is commutative. \(\vec{a} + \vec{b} = \vec{b} + \vec{a}\) is
the commutative law. Vector addition is also associative. Any order of addition
will yield the same result. 
\par
A negative sign reverse vector direction.
\[\vec{b}+(-\vec{b}) = 0\]
We use this to define vector subtraction.
\[\vec{d} = \vec{a} - \vec{b} = \vec{a} + (-\vec{b})\]
These rules hold for all vectors, irrespective of what quantity they depict.
Obviously, only vectors of the same type, with the same units can be added.
This can be checked with dimensional analysis.

\subsubsection*{Components}

Rather than adding them graphically, one can add vectors by breaking them down
into their components. Components in two dimensions can be found with:
\[a_x = a \cos(\theta) \:\mathrm{and}\: a_y = a \sin(\theta)\]
\[a = \sqrt{a^2_x + a^2_y}\]
In \(3\) dimensions we need more components so we use \(a, \theta, \phi\) or
\(a_x, a_y, a_z\). Unit vectors. A unit vector has the following properties:
\begin{itemize}
    \item Has magnitude \(1\)
    \item Has a particular direction
    \item Lacks dimension and unit
    \item is labeled with a hat: \(\hat{i}\)
\end{itemize}
\[\vec{a} = a_x\hat{i} + a_y\hat{j}\ (+ a_z\hat{k})\]
\(a_x\) and \(a_y\) alone are scalar components. Vectors can be added with
these components: \(r_x = a_x + b_x\) etc. To subtract two vectors, we 
substract components: \(r_x = a_x - b_x\) etc.

\subsubsection*{Rotations}

Because vectors are independent of their coordinate system, we can rotate the
system while maintaining the vector.
\[a = \sqrt{a^2_x + a^2_y} = \sqrt{a'^2_x + a'^2_y}\]

\subsubsection*{Scalar Multiplication}

To multiply a vector by a scalar, we simply multiply each component by the 
scalar. The direction is unchanged unless the scalar is negative, in which
case it is reversed.
\[3a = 3a_x\hat{i} + 3a_y\hat{j}\]

\subsubsection*{Dot Product}

The dot product or scalar product of two vectors results in a scalar where a 
and b are magnitudes and \(\phi\) is the angle between the directions of the
tow vectors.
\[\vec{a} \cdot \vec{b} = ab\cos(\phi)\]
Dot product is the projection of one vector onto another. When 
\(\theta = 90\degree\) the dot product is \(0\). \(\hat{i}, \hat{j}, \hat{k}\)
are described as \textit{orthonormal}:
\[\hat{i} \cdot \hat{i} = 1\]
\[\hat{i} \cdot \hat{j} = 0\]
The dot product is commutative. \(a \cdot b = b \cdot a\).

\subsubsection*{Vector Product}

The vector product is another ay of multiply two vectors, also known as the 
vector product.
\[c = ab\:\sin(\phi)\]
If \(a\) and \(b\) are parallel or antiparallel the vector product is \(0\).
It is at a maximum when they are perpendicular. The vector product is not 
commutative. The direction of \(c\) is perpendicular to both \(a\) and \(b\).
So for example:
\[\hat{i} \times \hat{j} = \hat{k}\]
\[\hat{i} \times \hat{i} = 0\]

\subsubsection*{Position Vector}
\[\vec{r} = x\hat{i} + y\hat{j} +z\hat{k}\]
Change is position vector is displacement:
\[\Delta\vec{r} = \Delta x \hat{i} + \Delta y \hat{j} + \Delta z \hat{k}\]

\subsubsection*{Instantaneous Values}
\[\vec{v} = \frac{d\vec{r}}{dt}\]
\[\vec{a} = \frac{d\vec{v}}{dt}\]
In general, the instantaneous value of a changing quantity is given by the 
relevant derivative, just as with scalars.

\section*{Projectile Motion}
\[\vec{v_0} = v_{0x}\hat{i} + v_{0y}\hat{j}\]
\[v_{0x} = v_0\cos{\theta} \:\mathrm{and}\: v_{0y} = v_0\sin{\theta}\]
A projectile has a duration of flight equal to twice the time taken from the 
highest point to the ground. Thus, when comparing projectiles, the one with the
lower highest point will have a shorter period of flight. The range is given 
by:
\[R = \frac{v^2_0}{g}\sin 2 \theta_0\]

\section*{Uniform Circular Motion}
In uniform circular motion, velocity and acceleration each have constant 
magnitude but changing direction. This acceleration is called centripetal 
acceleration. The velocity is always at tangent to the circular path, while 
the acceleration is always inward toward the centre of the circle.
\[a = \frac{v^2}{R} \Rightarrow F = m\frac{v^2}{R}\]
\[T = \frac{2\pi R }{v}\]
The centripetal acceleration can come from many sources, such as gravity in a 
space station, friction for a car drive in a circle, or tension for a ball 
whirled on a string.

\bigskip
Consider a bicycle going around a vertical loop. At the top of the loop, the 
forces acting on the cyclists are the normal force (downward, opposing the 
centripetal force) and gravity, also downward.
\[-F_N - mg = m(-\frac{v^2}{R})\]
For the cyclist to not fall, normal force must be at minimum \(0\); centripetal
force must cancel out with gravity. Thus, for a loop of radius 
\(2.7\mathrm{m}\):
\[v = \sqrt{gR} = \sqrt{9.8 \times 2.7} = 5.1\mathrm{m/s}\]

\section*{Relative Motion}
If reference frames are moving relative to each other, they may each observe 
differing velocities of an event, because the frames velocity will be added to
the velocity of what they observe. Each observer will, however, observe the 
same acceleration of the event, assume both reference frames are inertial. If
\(P\) is the event being observed, \(A\) is our primary reference frame, 
stationary with respect to \(P\) and \(B\) is our moving reference frame:
\[\vec{r}_{PA} = \vec{r}_{PB} + \vec{r}_{BA}\]

\section*{Force}
A force is a "push or pull" on an object, which causes acceleration. Newton's
laws of motion are applicable to objects which aren't moving at near \(c\) or 
at atomic scale.

\bigskip
Newton's first law states that: if no net force acts on a body, the body's 
velocity cannot change. These laws also only apply in inertial frames; over 
long enough distances, even the surface of the earth is non-inertial.

\bigskip
Mass is inversely proportional to acceleration due to force. Thus we arrive at
Newton's second law:
\[\vec{F} = m\vec{a}\]
Acceleration along an axis is affected only by forces along the same axis. 
Thus, complex problems can be solved through decomposition.

\bigskip
Weight is mass under gravitation force. Given by:
\[W = mg\]

\subsubsection*{Normal Force}
The normal force is the pushback of a surface against a force (such as weight)
exerted on it. This force is always opposite to the force applied to the 
surface, and is always equal to the force applied to the surface, so that the
forces on the object are in balance and thus the object remains stationary. 
This force is described by Newton's third law of motion.
\[\vec{F}_{BC} = -\vec{F}_{CB}\]
These forces are described as a third law force pair; a concept which arises 
any time two objects interact.

\subsubsection*{Friction}
This occurs when an object is attempting to slide over another. It opposes the
direction of motion of the moving object. If the friction is significant enough
to stop the objects movement, then it will be equivalent to the force moving 
the object. Friction exists because surfaces are slightly rough, thus causing
them to stick together a little.

\bigskip
Friction is essential for tasks such as picking things up for for propulsion,
but overcoming it is also important for applications like efficiency in engines
or motion through means such as roller skates. In general, anything intended to
remain in motion must overcome friction.

\bigskip
Two types of frictional force exist; static frictional force acts as a kind of
``normal'' frictional force, preventing the motion of objects. It can increase
from 0 to some maximum force. Thus, the maximum static frictional force must be
overcome to begin motion for an object. Kinetic frictional force is the 
opposing force that acts on an object in motion, and is constant and usually 
weaker than static force. To maintain constant velocity, a constant force equal
to the kinetic frictional force must be applied.

\bigskip
The magnitude of static friction is given by:
\[f_{s, \mathrm{max}} = \mu_sF_N\]
Where \(\mu_s\) is the \textit{coefficient of static friction}. If the applied
force exceeds \(f_{s, \mathrm{max}}\), sliding begins. The coefficient depends
on the two surfaces involved. For example, rubber on dry concrete has a 
coefficient of static friction of \(\mu_s = 1.0\) and of kinetic friction 
\(\mu_k = 0.8\). Kinetic frictional force in general is given by:
\[f_k = \mu_kF_N\]
In general, \(\mu_s > \mu_k\). The force required to overcome frictional is 
known as \(F_C\), and is related linearly to mass. To determine \(\mu_s\) we
can use a setup with an object on a slope. If we slowly increase the angle, at
the moment when the block just begins to move:
\[Mg\sin(\theta) = \mu_sN\]
\[Mg\cos(\theta) = N\]
\[\Rightarrow \tan(\theta) = \mu_s\]
Where \(\theta\) is the angle between the surface and the flat. If we further
increase the angle, the block will accelerate down the slope with acceleration:
\[a = (g\sin(\theta) - \mu_kg\cos(\theta))\]
For a force applied at an angle, we need to decompose into \(x\) and \(y\) 
components to address frictional forces.

\subsubsection*{Drag Force}
A fluid is anything that can flow, i.e. a gas or liquid. Where there is a 
relative velocity between fluid and object, there is a resultant drag force.
This force will oppose the relative motion, and will point in the direction of
flow, relative to the body.

\bigskip
Drag force for a moving object with constant velocity is given by:
\[D = \frac{1}{2}C\rho Av^2\]
Where:
\begin{itemize}
    \item \(v\) is relative velocity.
    \item \(\rho\) is the air density.
    \item \(C\) is an experimentally determined drag coefficient.
    \item \(A\) is the cross-sectional area of the body.
\end{itemize}
In practice, \(C\) will not be constant for all values of \(v\).
For a falling object:
\[D - F_g = ma\]
Once the drag force equals \(F_g\), the object will reach terminal velocity at:
\[v_t = \sqrt{\frac{2F_g}{C\rho A}}\]
If \(A\) is increased, \(v_t\) decreases; the principle behind parachutes.

\subsubsection*{Tension}
Tension occurs when a cord or rope is attached to an object and pulled to apply
a force to the object. The cord will apply force to the object equal to it's 
tension force. If a system has multiple attachment points to a tensioned rope,
each point pulling upward constitutes a force of \(T\) upward on the system; 
for example, with \(2\) attachment points, There is a mechanical advantage 
factor of \(2\).

\subsubsection*{Example}
A car is moving with an acceleration of \(1.2ms^{-2}\). In this car, a pendulum
hangs. Find the angle at which this pendulum hangs relative to the normal.
\[a = 1.2ms^{-2}\]
\[x: ma = F_T\sin(\theta)\]
\[y: 0 = F_T\cos(\theta)\ - mg\]
\[\frac{F_T\sin(\theta)}{F_T\cos(\theta)} = \frac{ma}{mg}\]
\[\tan(\theta) = 0.122\]
\[\theta = 7\degree\]

\subsection*{Perception of Forces}
We percieve forces due to our awareness of our inertia and our sensing of the
normal force of the ground against us. Thus, we are in freefall \((F_N = 0)\),
we feel weightless because out acceleration is simply \(g\) and there is no 
normal force. When we are in a lift accelerating upward, we feel heavier 
because we are experiencing a greater normal force due to our greater 
acceleration. The \textit{otic labyrinth} is an organ inside the ear, 
which uses hairs connected to an elastically suspended membrane. As the hairs 
move, cells detect their force and report this to the brain.

\bigskip
When a person is in a high acceleration vehicle, the perceive the angle to be
tilting upward even when it isn't. This is because the resultant force vector 
of the normal force due to gravity and the normal force from the acceleration
points upward relative to the flat plane.

\section*{Energy}
Energy is required for any sort of motion. It is a scalar quantity. The 
transfer of energy is known as work.

\subsection*{Kinetic Energy}
The faster an object moves, the higher its kinetic energy (\(E_k\)). For a 
stationary object, \(E_k = 0\). For an object moving at non-relativistic 
speeds:
\[E_k = \frac{1}{2}mv^2\]
Kinetic energy is measured in joules (\(\mathrm{J}\)).
\[1 \:\mathrm{joule} = 1\mathrm{J} = 1\mathrm{kg}\mathrm{m}^2\mathrm{s}^{-2}\]
If an objects kinetic energy changes, then work has been done on that object by
a force. If energy is transferred to an object, positive work has been done, if
energy is transferred from an object, negative work. For example, in a 
descending lift, gravity is doing positive work, by increasing the kinetic 
energy of the elevator, while the tension in the cable is doing negative work.
Work can be determined from velocity through:
\[\frac{1}{2}mv^2 - \frac{1}{2}mv_0^2 = Fd\]
Where \(d\) denotes distance. To calculate work done with vector quantities:
\[W = Fd\cos(\phi)\]
\[W = \vec{F}\cdot\vec{d}\]
Work is also measured in joules. For two or more forces, the net work is the
sum of work done. When working with vectors, we can either calculate work for 
each vector independently and sum these, or take a vector sum and calculate 
work done once.
\[\Delta E_k = E_{k,f} - E_{k,i} = W\]
i.e. the change in kinetic energy is equal to the net work done.

\subsection*{Hooke's Law}
When compressed or stretched, springs will attempt to return to the origin, 
exerting force given by:
\[\vec{F}_s = -k\vec{d}\]
Where \(k\) is the springs experimentally determined \textit{spring constant}.
\(k\) is a measure of the stiffness of the spring. \(F_s\) is a variable force,
and exhibits a linear relationship between \(F\) and \(d\).

\bigskip
We can find the work done by integrating:
\[W_s = \int^{x_f}_{x_i}-F_x\:dx\]
Plugging in \(kx\) for \(F_x\):
\[W_s = \frac{1}{2}kx^2_i - \frac{1}{2}kx^2_f\]
If the spring ends up closer to the origin (\(x = 0\)), \(W_s\) will be 
positive. If it is further away, \(W_s < 0\). For an initial posisition of 
\(x = 0\):
\[W_s = \frac{1}{2}kx^2\]

\subsection*{Power}
Power is the time rate at which a force does work. It is measured in joules per
second or watts (\(W\)). If a force does \(W\) work in time \(\Delta t\) the 
average power due to the force is:
\[P_\mathrm{avg} = \frac{W}{\Delta t}\]
and the instantaneous power at a given moment is:
\[P = \frac{dW}{dt}\]
\[P = \vec{F}\cdot\vec{v}\]
Using our understanding of conservation of energy, we can understand average 
and instantaneous power in terms of energy.
\[P_\mathrm{avg} = \frac{\Delta E}{\Delta t}\]
\[P = \frac{dE}{dt}\]

\subsection*{Potential Energy}
Potential energy, denoted by \(U\) or \(E_u\) is energy associated with the 
configuration of a system of objects that exert forces on one another. As an
example, we can consider the earth and a bungee jumper as a system. In this 
system, gravitational potential energy accounts for the increase in kinetic 
energy during the fall, while elastic potential energy, held in the bonds
between the particles in the rubber, accounts for the deceleration at the 
bottom of the fall. Elastic potential energy can be calculated with:
\[U(x) = \frac{1}{2}kx^2\]

\bigskip
When thinking about energy, it is important to understand that a system 
consists of two or more objects. A force acts between a particle and the rest
of the system. When this force changes the configuration, the force does work
denoted \(W_1\). If the configuration change is reversed, the force reverses 
the energy transfer, doing work \(W_2\).

\bigskip
For \textit{conservative} forces \(W_1 = -W_2\) is always true. Gravitational 
force, spring force and magnetic force are conservative. For these forces we 
can speak of potential energy.
\textit{Nonconservative} forces include kinetic friction force and drag force.
For kinetic friction, this is because some energy is released as heat, which 
cannot be recovered into kinetic energy by friction.

\bigskip
For conservative forces, when an ideal object is moved around a closed path
back to the beginning, no work is done. This implies that moving between two 
points costs the same amount of work done irrespective of the paths taken. This
means that for calculations involving conservative forces, complex paths can be
simplified. For an object being raised of lowered:
\[\Delta U = -W\]
This also applies to an elastic block-spring system: when a spring is 
compressed, elastic potential energy is stored, which is released as the spring
returns to origin. In general, work can be calculated as:
\[W = \int^{x_f}_{x_i}F(x)dx\]
\[\Rightarrow \Delta U = -\int^{x_f}_{x_i}F(x)dx\]
In one dimension (\(x\)) force and \(E_u\) are related by work through:
\[F(x) = -\frac{dU(x)}{dx}\]
This allows us to find force from a graph of potential energy through finding
the slope.

\subsection*{Mechanical Energy}
The mechanical energy of a system is the sum of its potential and kinetic 
energy.
\[E_\mathrm{mec} = E_k + E_u\]
Work done by conservative forces increases \(E_k\) and decreases \(E_u\), so:
\[\Delta E_k = - \Delta E_u\]
If we take two instants of time \(1\) and \(2\):
\[E_{k,2} + E_{u,2} = E_{k,1} + E_{u,1}\]
i.e. the system has constant mechanical energy. This is known as the 
conservation of mechanical energy:
\[\Delta E_\mathrm{mec} = \Delta E_k + \Delta E_u = 0\]
Work on a \textit{system} can be considered to be equal to 
\(\Delta E_\mathrm{mec}\). Although energy must be conserved within the system,
the total energy of the system can be increased through external forces, for 
example by raising it, engendering an increase in gravitational potential 
energy.

\bigskip
However, most systems aren't ideal and thus suffer from friction, causing a 
release of thermal energy. Thus, work done is given in reality by:
\[\Delta E_\mathrm{th} = f_kd\]
\[W = \Delta E_\mathrm{mec} + \Delta E_\mathrm{th}\]
Where \(E_\mathrm{th}\) is thermal energy and \(f_k\) is the frictional
coefficient of the surface involved.

\subsection*{Equilibrium}
Points where \(E_k = 0\) will be turning points in the motion of an object. 
At these points, objects have maximum potential energy and reverse in 
direction. A particle is said to be in \textit{neutral equilibrium} if it is 
stationary and has only potential energy, with net force zero. An example of 
this is a marble on a flat surface.

\bigskip
\textit{Unstable equilibrium} occurs when a particle is stationary with only 
potential and net force \(0\). However, if it is displaced slightly it will 
continue to move in that direction under a new force. An example of this is a 
marble balanced on a curved surface.

\bigskip
\textit{Stable equilibrium} occurs when a particle is under the same 
conditions, but if displaced, will return to its original position. For 
example, consider a marble at the base of the ball. This is often observed in
oscillating systems like pendulums.

\subsection*{Conservation of Energy}
Energy transferred between systems can always be accounted for. Thus, the total
energy \(E\) of a system, is constant. This includes mechanical, thermal and 
all other internal energy.
\[W = \Delta E = \Delta E_\mathrm{mec} + 
\Delta E_\mathrm{th} + \Delta E_\mathrm{int}\]
In an isolated system, there is no external energy transfer. The total energy
of this system cannot change.
\[\Delta E_\mathrm{mec} + \Delta E_\mathrm{th} + \Delta E_\mathrm{int} = 0\]
Or, for two instants of time we can observe that:
\[E_{mec,2} = E_{mec,1} - \Delta E_\mathrm{th} - \Delta E_\mathrm{int}\]
The total energy of an isolated system is a function of its state; the
intervening events are irrelevant.

\subsection*{Centre of Mass}
For complex rotating objects, it can be extremely complex to consider their 
motion. However, for a specific point, the centre of mass, the motion is as of
a particle, and thus is simple to analyse. For example, a cricket bat tossed 
spinning will have a very complex pattern of movement, however if we simply 
consider its centre of mass, the particle will transcribe a parabola, just as a
ball would, with the rest of the bat rotating about this centre point.

\bigskip
For a system of particles, the centre of mass is the point that moves as 
though all of the system's mass were concentrated there. For two particles 
\(d\) units apart, where the origin is taken to be at the position of particle
\(1\):
\[x_\mathrm{com} = \frac{m_2}{m_1 + m_2}d\]
For two particles, with an arbitrary choice of origin:
\[x_\mathrm{com} = \frac{m_1x_1 + m_2x_2}{m_1 + m_2}\]
This generalises easily to any number of particles. If \(M\) is taken to be the
sum of the masses of all of the particles:
\[x_\mathrm{com} = \frac{m_1x_1 + m_2x_2 + m_3x_3 + ... + m_nx_n}{M}\]
\[= \frac{1}{M}\sum^n_{i = 1}m_ix_i\]
For three dimensions we can use:
\[y_\mathrm{com} = \frac{1}{M}\sum^n_{i = 1}m_iy_i\ \:\mathrm{and}\: 
z_\mathrm{com} = \frac{1}{M}\sum^n_{i = 1}m_iz_i\]
This can be written in terms of vectors, where \(\vec{r}_i\) is the position 
vector of the \(i\)'th particle:
\[\vec{r}_\mathrm{com} = \frac{1}{M}\sum^n_{i = 1}m_i\vec{r}_i\]
For solid bodies, we assume an infinite sum of infinitely small particles. We
can solve this through integration.
\[x_\mathrm{com} = \frac{1}{M}\int x dm\]
Which, of course, generalises to each dimension. If we assume that our object 
is of uniform density (\(\rho)\)), then:
\[\rho = \frac{dm}{dV} = \frac{M}{V}\]
We can substitute this into our centre of mass formula to find that:
\[x_\mathrm{com} = \frac{1}{V}\int x dV\]
If the object has symmetry, such as in a sphere or cube, the centre of mass
lies at this point. So it lies at the centre of a sphere or cube, or for a 
sphere, it lies along the centre line of the circle and halfway down, etc. Note
that the centre of mass need not be on the object; for a torus the centre of 
mass is in the centre of the whole. Because of this property, it is often 
possible to simply deduce the centre of mass of an object.

\bigskip
For systems with multiple objects, we can take centre mass of each object and
calculate the centre mass of the resultant system. This approach can allow us
to determine the result of removing a section of an object; the effect will be
the inverse of adding a second copy of the object. For example, consider 
removing a smaller circle for a larger circle. One could find the centre mass
of the smaller circle and the larger circle, then invert this result across the
axes to find the resultant centre of mass of the larger circle.

\bigskip
For a system as a whole, the motion of the centre of mass will continue 
unaffected by internal forces. This assumes a closed system.
\[\vec{F}_{net} = M\vec{a}_\mathrm{com}\]

\subsection*{Linear Momentum}
Linear momentum is given by:
\[\vec{p} = m\vec{v}\]
Momentum points in the same direction as velocity, and can only be changed by a
net external force for a system. The rate of change of momentum a particle or 
system with respect to time is equal to the net force on that object and is in
the direction of that force. 
\[\vec{F}_\mathrm{net} = \frac{d\vec{p}}{dt}\]
This is essentially Newton's second law (\(f = ma\)). Without external force, 
linear momentum cannot change. This is known as the law of conservation of 
linear momentum.

\subsection*{Collisions and Impulse}
During a collision, momentum of a particle can change, as a net external force
is acting on the particle. This is described as the application of an impulse 
(\(J\)) to the system or particle.
\[\vec{J} = \int^{t_f}_{t_i}\vec{F}(t)dt\]
The applied impulse is equal to the change in momentum during the collision, a
face which is 
known as the linear momentum-impulse theorem.
\[\Delta\vec{p} = \vec{J}\]
Because momentum is conserved, we know that for a closed system:
\[(m_1v_1)_i + (m_2v_2)_i = (m_1v_1)_f + (m_2v_2)f\]

\subsection*{Elastic and Inelastic Collisions}
When a tennis ball is placed atop a basketball and dropped, the tennis
ball will rebound with a velocity in excess of what it's velocity would
suggest, while the basketball will bounce very little.

\bigskip
Across subsequent bounces, the momentum of the ball will diminish, because
it is not ideally elastic and thus loses energy as heat or sound with each 
bounce. The coefficient of restitution \(e\) is given by:
\[e = \frac{\mathrm{relative\:velocity\:after\:collision}}{\mathrm
{relative\:velocity\:before\:collision}}\]
When a ball is bouncing in parabolic arcs, the square root of the ratio of the
height of one bounce as compared to the next is a direct measure of \(e\).
\begin{itemize}
    \item \(e = 1\) is perfectly elastic
    \item \(e = 0\) is perfectly inelastic
\end{itemize}
A basketball would have a reasonably high coefficient of restitution, while
a material like mud or gelatin would have a very low coefficient of
restitution.

\bigskip
Collisions come in several varieties.

\begin{itemize}
    \item In \textit{elastic collisions} total kinetic energy is conserved.
    A useful approximation for common situations, such as on a pool table. 
    In real situations, some energy is always transferred; as sound or friction
    in our billiards example.
    \item \textit{Inelastic collisions} transfer some energy.
    \item \textit{Completely inelastic collisions} result in the two objects
    stuck together, such as in the case of a bullet lodging in another object. 
\end{itemize}
For two moving particles undergoing an inelastic collision, the resultant
transfer of energy will be described by
\[m_1v_{1i} + m_2v_{2i} = m_1v_{if} + m_2v_{2f}\]
For a completely inelastic collision with \(m_2\) at rest,
\[m_1v_{1i} = (m_1 + m_2)v_f\]
In the case of a completely inelastic collision, the velocity of the centre
of mass will remain unchanged.
\[\vec{v}_{\mathrm{com}} = \frac{\vec{P}}{m_1 + m_2} = \frac{\vec{p}_{1i} + 
\vec{p}_{2i}}{m_1 + m_2}\]
For elastic collisions, the total kinetic energy is always conserved. In the
case of a stationary \(m_2\) for momentum:
\[m_1v_{1i} = m_1v_{1f} + m_2v_{2f}\]
and for kinetic energy:
\[\frac{1}{2}m_1v^2_{1i} = \frac{1}{2}m_1v^2_{1f} + \frac{1}{2}m_2v^2_{2f}\]
For the individual objects:

\begin{minipage}[h]{0.4\linewidth}
    \[v_{1f} = \frac{m_1 - m_2}{m_1 + m_2}v_{1i}\]    
\end{minipage}
\begin{minipage}[h]{0.4\linewidth}
    \[v_{2f} = \frac{2m_1}{m_1 + m_2}v_{1i}\]    
\end{minipage}

\bigskip
This makes perfect sense when we consider with an example. For a pool ball, 
\(m_1 \approx m_2\), all of the velocity of \(m_1\) will be transferred to
\(m_2\), which will continue with that speed, exactly the situation observed
in pool. As another example, consider a ball colliding with a building. Here,
\(m_1 \ll m_2\) and so \(v_{1f}\) will be roughly equal to \(-v_{1i}\), i.e.
the ball has bounced back with roughly its initial velocity, as we would 
expect. What this tells us is that when two masses collide, their final 
velocities are determined by the ratio of their masses.

\bigskip
For higher dimensions, these formulas can be used by breaking the problem
down into components for each dimension.

\subsubsection*{Rockets}
Rockets and their exhaust products form an isolated system. Thus momentum
within the system must be conserved. However, because the rocket is ejecting
mass out, its mass is not a constant. Thus it will be accelerating with time.
\[M_v = -dMU + (M + dM)(v + dv)\]
This can be simplified using relative speed, given by
\[U = v + dv - v_\mathrm{rel}\]
This leads us to the first rocket equation:
\[Rv_\mathrm{rel} = Ma\]
Where:
\begin{itemize}
    \item \(R\) is the mass rate of fuel consumption (\(\mathrm{kgs}^{-1}\))
    \item \(v_\mathrm{rel}\) is the relative velocity of the ejected fuel
    \item \(M\) is the (time dependent) mass of the rocket
    \item \(a\) is the acceleration of the rocket
\end{itemize}
Thus, \(Rv_\mathrm{rel}\) is the \textit{thrust}, \(T\). Which is the force 
exerted by the rocket engine. This eventually gives us the second rocket 
equation:
\[v_f - v_i = v_\mathrm{rel}\ln\frac{M_i}{M_f}\]
Which in words states that the change in velocity of the rocket will be equal
to the relative velocity multiplied by the logarithm of the ration between the
initial and final mass of the rocket. This logarithmic relationship is why
it is so expensive to launch spacecraft; the fuel must make up a significant
proportion of the mass to achieve sufficient velocity. This is why rockets
often drop stages to increase this ratio.

\subsection*{Rotational Motion}
Rotational motion occurs with rigid bodies, which rotate as units, locked 
together without changing in shape or other properties. In this subject 
rotation around a fixed axis, usually an axis of symmetry. Thus the rotation
of a star, where layers of gas rotate independently is excluded. In addition 
a rolling bowling ball is excluded as both rotation and translation are 
ocurring.

\bigskip
Rotation of bodies around an axis of rotation is measured by angular position 
from the \(x\)-axis. Rotation is measured in radians, which are dimensionless.
\[1 \:\mathrm{revolution} = 360\degree = 2\pi \:\mathrm{rad}\]
Radians do not reset to \(0\) after passing \(2\pi\), they continue to 
increase. If we know \(\theta(t)\), the rotation of the object with respect to
time, we know all we need to about its kinematics. We can define angular 
displacement as the size of the angle between the initial position 
(the \(x\)-axis) and the current angle:
\[\Delta\theta = \theta_2 - \theta_1\]
It is worth noting that ``clocks are negative''; a counterclockwise rotation in
the context of angular rotation is positive.

\bigskip
Rotations are \textit{non-commutative}, that is the order in which rotations
are performed is significant. Rotation in the \(y\) followed by in the \(x\)
is not equivalent to the inverse.

\bigskip
Average angular velocity is given by the average angular displacement during
a time interval.
\[\omega_{\mathrm{avg}} = \frac{\theta_2 - \theta_1}{t_2 - t_1}
 = \frac{\Delta\theta}{\Delta t}\]
Instantaneous angular velocity can be found by taking the limit of this:
\[\omega = \lim_{\Delta t \rightarrow 0} \frac{\Delta\theta}{\Delta t}
 = \frac{d\theta}{dt}\]
The magnitude of angular velocity is the angular speed. Average angular
acceleration is given by the change in angular velocity.
\[\alpha_{\mathrm{avg}} = \frac{\omega_2 - \omega_1}{t_2 - t_1}
 = \frac{\Delta\omega}{\Delta t}\]
We can of course take the limit of this for instantaneous angular acceleration.
For a rigid body, these statements will be true for all points; all points will
have the same angular velocity. Angular velocity and acceleration can be 
written as vectors. The \textit{right hand rule} is used to determine the 
direction of these vectors. If one wraps the fingers of their right hand around
the axis of rotation, with the fingers pointing in the direction of rotation,
the extended thumb will point in the direction of the resultant vector.

\subsubsection*{Equations}
The system of equations of constant acceleration can be converted simply into
rotation by simply considering angle where one would consider position; our
first derivative of position is angular velocity, our second angular 
acceleration.

\begin{center}
    \begin{tabular}{c|c}
        \(\mathrm{Position}\) & \(\mathrm{Angle}\) \\
        \hline
        \(v = v_0 + at\) & 
        \(\omega = \omega_0 + at\) \\[7pt]
        \(\Delta x = v_0t + \frac{1}{2}at^2\) & 
        \(\Delta\theta = \omega_0t + \frac{1}{2}\alpha t^2\) \\[7pt]
        \(\Delta x = vt - \frac{1}{2}at^2\) &
        \(\Delta\theta = \omega t - \frac{1}{2}\alpha t^2\) \\[7pt]
        \(\Delta x = \frac{1}{2}(v_0 + v)t\) &
        \(\Delta\theta = \frac{1}{2}(\omega_0 + \omega)t\) \\[7pt]
        \(v^2 = v_0^2 + 2a\Delta x\) & 
        \(\omega^2 = \omega_0^2 + 2\alpha\Delta\theta\) \\[7pt]
    \end{tabular}        
\end{center}

To relate angular variables with linear variables, we need to use the angle and
the radius. For a \(\theta\) in radians and \(r\) in metres:
\begin{itemize}
    \item For position, \(s = \theta r\)
    \item For speed, \(v = \omega r\)
    \item Tangential acceleration is given by \(a_t = \alpha r\)
    \item Radial acceleration, \(a_r = \frac{v^2}{r} = \omega^2 r\)
    \item For period, \(T = \frac{2\pi}{\omega}\)
\end{itemize}
Note that tangential acceleration is the acceleration at a tangent to the
rotation while radial acceleration is the centripetal acceleration back in
toward the axis of rotation.

\subsubsection*{Energy}
For a rotation system, the rotation kinetic energy is given by summing the
individual kinetic energies of all particles in the system.
\[E_k = \sum^n_{i=0} \frac{1}{2}m_iv_i^2\]
For a rigid body, all particles have the same rotational velocity. We can use
this fact to express the kinetic energy as follows:
\[E_k = \frac{1}{2}\left(\sum^n_{i=0}m_ir_i^2\right)\omega^2\]
The term in brackets here is known as the \textit{rotational inertia} or 
\textit{moment of inertia} of the body in the axis of rotation, denoted \(I\).
This is a property of the body being considered. The formula for kinetic energy
can then be rewritten:
\[E_k = \frac{1}{2}I\omega^2\]
Relating this to linear quantities, \(I\) relates to mass while \(\omega\), as 
elsewhere relates to velocity. \(I\) in effect tells us how difficult it is to
change the speed of rotation of a body. The moment of inertia can be considered
in a general sense by taking the limit across a solid body:
\[I = \int r^2dm\]
Rather than calculate the moment of inertia in this course, it is more common
to use one of a few common shapes which have intuitive moments.
\begin{itemize}
    \item A thin hoop around an axis with radius \(r\) has a moment of inertia
    given by \(I = mr^2\) where \(m\) is the mass of the hoop. This is the 
    maximum motion of inertia for a given mass.
    \item An annulus, which is a hoop with thickness, the moment of inertia is
    given by \(I = \frac{1}{2}m(r_1^2 + r_2^2)\), where \(r_1\) is the inner
    radius and \(r_2\) is the outer radius. The limit of this case is 
    \(r_1 = r_2\), the thin hoop.
    \item A cylinder has moment given by \(\frac{1}{2}mr^2\), the other limit
    of the annulus, where \(r_1 = 0\). 
    \item A sphere of uniform density has \(I = \frac{2}{5}mr^2\).
    \item A hollow sphere has \(I = \frac{2}{3}mr^2\), because the mass is more
    concentrated further away from the center.
\end{itemize}
These concepts are used often in machinery, where objects with a high moment of
intertia are used to store energy. A classical example is in a flywheel.

\bigskip
If we want to consider different axes of rotation for the same object, as long
as they are parallel and one of them is through the center of mass, we can use
the \textit{parallel-axis theorem} to relate the center of mass' moment of 
inertia to that of another parallel axis.
\[I = I_\mathrm{com} + mh^2\]
Where \(I_\mathrm{com}\) is the moment of inertia of the axis through the 
center of mass, \(m\) is the mass of the body and \(h\) is the distance to
the other axis of rotation.

\subsubsection*{Force}
The force necessary to rotate an object depends on the angle of the force and
where on the object it is applied. When force is applied to an object, only 
force applied tangentially will cause a rotation. The magnitude of the rotation
will be dependent on the radial distance at which the force is applied. The
construct used to understand this \textit{torque}.
\[\tau = rF\sin(\phi)\]
Here, \(r\) is the radial distance, \(F\) is the magnitude of the force applied
and \(\phi\) is the angle between the force and a line from the rotation axis.
The perpendicular between the force and the axis if one extends out the force 
as a line is known as the \textit{moment arm} of the force. Torque is expressed
in \(\mathrm{Nm}\), the same as Joules, but as it is not an energy not 
\textit{in} Joules. A torque is defined as positive if it would cause a 
counterclockwise rotation and negative if it would cause a clockwise rotation.
For several torques, we can sum them to find the net or resultant torque.

\bigskip
With torque, we now have the tools to rewrite \(F = ma\) with our rotational
variables. Torque will take the role of force, with \(I\) replacing mass and
\(\alpha\) acceleration.
\[\tau = I\alpha\]

\bigskip
We can understand rotational work as the change in rotational kinetic energy.
\[\Delta E_k = E_{k, f} - E_{k, i} = 
\frac{1}{2}I\omega^2_f - \frac{1}{2}I\omega^2_i = W\]
For rotation in a fixed axis, work is the area under a torque vs angle graph.
\[W = \int^{\theta_f}_{\theta_i}\tau d\theta\]
When we have a constant torque, this is simply \(W = \tau(\theta_f - 
\theta_i)\), the rotational equivalent of \(W = Fd\). Work can be related to
power through.
\[P = \frac{dW}{dt} = \tau\omega\]

\subsubsection*{Analogy Summary}
\begin{center}
    \begin{tabular}{c|c}
        \(\mathrm{Translation}\) & \(\mathrm{Rotation}\) \\
        \hline
        \(x\) & \(\theta\) \\[7pt]
        \(v = \frac{dx}{dt}\) & \(\omega = \frac{d\theta}{dt}\) \\[7pt]
        \(a = \frac{dv}{dt}\) & \(\alpha = \frac{d\omega}{dt}\) \\[7pt]
        \(m\) & \(I = mr^2\) \\[7pt]
        \(F_\mathrm{net} = ma\) & \(\tau_\mathrm{net} = I\alpha\) \\[7pt]
        \(W = \int Fdx\) & \(W = \int\tau d\theta\) \\[7pt]
        \(E_K = \frac{1}{2}mv^2\) & \(E_K = \frac{1}{2}I\omega^2\) \\[7pt]
        \(P = Fv\) & \(P = \tau\omega\) \\[7pt]
        \(W = \Delta E_K\) & \(W = \Delta E_K\) \\[7pt]        
    \end{tabular}        
\end{center}

\subsubsection*{Rolling Objects}
For a rolling object, both translational and rotational motion is occurring.
For a wheel for instance, the centre of mass is always moving in a straight
line, with the rest of the object rotating. For a wheel rolling in this way,
one period is the time taken for a full rotation to occur. The displacement
between the starting position of a rotation and the end position will be given
by \(s = 2\pi R\), the circumfrence of the wheel. For an arc of size \(\theta\)
the distance covered while the wheel turns this angle will be \(s = \theta R\).
We can take the derivative to find the velocity of the centre of the mass:
\[v_\mathrm{com} = \omega R\]
The translational velocity of the wheel is directly related to its speed of
rotation. This is only true for a wheel with no slippage. For a turning wheel,
a point on one side of the wheel as a velocity exactly opposite to the velocity
of a point on the other side. For a wheel which is purely translating, these
points will have an equal velocity \(= v_\mathrm{com}\). To find the velocity
of these points for a rolling wheel, we can add the vectors, finding that
the point at the top has a velocity of twice the centre of mass, while the
point at the bottom has no velocity.

\bigskip
For this system, the kinetic energy must be the sum of the translational and
rotational kinetic energy.
\[E_K = \frac{1}{2}I_\mathrm{com}\omega^2 + \frac{1}{2}mv^2_\mathrm{com}\]
For a wheel to accelerate, its angular speed must change. This can be seen
simply from the derivation of the velocity formula, and yields:
\[a_\mathrm{com} = \alpha R\]
If a slip occurs, an object is not rolling smoothly. In this case, the object
has a frictional force exerting a torque on the object. For all rolling 
objects, we can write the moment of inertia in the form
\[I = cmr^2\]
This can be seen in the earlier examples of a sphere (\(c = \frac{2}{5}\)) or
the hoop (\(c = 1\)). We can accomodate any other rolling object by adjusting
to a different value of \(c\). For a point particle \(c = 0\), because there is
no accumulation of rotational motion with a very small radius. For an object
rolling down a ramp of height \(h\):
\[mgh = \frac{1}{2}I_\mathrm{com}\omega^2 + \frac{1}{2}mv^2_\mathrm{com}\]
Here, we can replace \(\omega\) with \(\frac{v_\mathrm{com}}{r}\) to simplify.
\[\frac{1}{2}cmr^2\left(\frac{v_\mathrm{com}}{r}\right)^2+
\frac{1}{2}mv^2_\mathrm{com} = mgh\]
\[\frac{1}{2}m(1 + c)v^2_\mathrm{com} = mgh\]
Rearranging for \(v_\mathrm{com}\) we find
\[v_\mathrm{com} = \sqrt{\frac{2gh}{1 + c}}\]
This tells us that the final velocity of the object is independent of its mass
and of its radius, and depends only on the geometry of the object, represented
by \(c\). This makes sense; the large the value of \(c\), the more rotational
energy the object gains.

\bigskip
Using equations of constant acceleration, we can find the acceleration of the
object, and after some rearranging, this yields the following expression, where
\(\theta\) is the angle of incline down which the object rolls.
\[a_\mathrm{com} = \frac{g\sin(\theta)}{1 + c}\]
This tells us that a point particle (\(c = 0\)), will have the greatest 
acceleration of any shape. This makes sense as for a point particle, no energy
is gained as rotational kinetic energy, and so all gain in kinetic energy is as
downward velocity. A sphere (\(c = \frac{2}{5}\)) will have a greater 
acceleration than a hoop (\(c = 1\)). The relationship between these 
accelerations can be analysed through
\[a_\mathrm{com} = \frac{a_\mathrm{particle}}{1 + c} \:\mathrm{for}\: c > 0\]

\subsubsection*{Torque}
Previously, we defined torque for a rotating body and a fixed axis. We can now
redefined this for any particle moving along a path. As this path need not be 
circle, torque will be defined here as a vector. The direction of torque is 
defined with the right hand rule, if \(a\) is the first vector and \(b\) is the
second, the right hand rule tells us the direction of torque by taking \(a\) as
the index finger and \(b\) as the middle finger, and taking the direction of 
the extended thumb.

\bigskip
If we take \(\vec{r}\) to be a vector from the origin to the particles current
position and \(\vec{F}\) to be the force on the particle, the torque is then 
given by
\[\vec{\tau} = \vec{r}\times\vec{F}\]
And the magnitude of the torque is given by
\[\tau = rF\sin(\phi)\]
Thus, the torque will be a maximum when the force and \(\vec{r}\) are at 
\(\phi = \frac{\pi}{2}\) from each other, with fits with our understanding
of reality. If \(\vec{F}\) is in the same direction as the position vector,
no torque will be exerted. This can be expressed in two equivalent ways.
\[\tau = rF_\perp = r_\perp F\]
The direction of torque will be perpendicular to the plane defined by 
\(\vec{r}\) and \(\vec{F}\). The cross product is calculated as follows.
\[\vec{a} = 2\hat{i} + 3\hat{j}\]
\[\vec{b} = 2\hat{i} + 3\hat{j} + 2\hat{k}\]
\[\vec{a} \times \vec{b} = 4(\hat{i} \times \hat{i}) + 6(\hat{i} \times 
\hat{j}) + 4(\hat{i} \times \hat{k}) + 6(\hat{j} \times \hat{i}) + 
9(\hat{j} \times \hat{j}) + 6(\hat{j} \times \hat{k})\]
Because the cross product of two vectors which are proportional to each other
is \(0\), \(\hat{i}\times\hat{i} = 0\), so most of these terms disappear. In
addition, if we reverse the order of vectors in a cross product, we reverse the
direction of the resultant vector. So \(\hat{i}\times\hat{j} = 
-\hat{j}\times\hat{i} = \hat{k}\). Thus, each of these pairs produces another
unit vector: \(\hat{i}\times\hat{j} = \hat{k}\), \(\hat{j}\times\hat{k} = 
\hat{i}\) and \(\hat{k}\times\hat{i} = \hat{j}\). This fact is known as the 
\textit{cyclic permutation}; if we write these letters in a triangle and read
them clockwise, we can remember these equalities. Applying this knowledge to 
\(\vec{a} \times \vec{b}\):
\[\vec{a} \times \vec{b} = 0 + 6(\hat{i} \times \hat{j}) + 4(\hat{i} \times 
\hat{k}) + 6(\hat{j} \times \hat{i}) + 0 + 6(\hat{j} \times \hat{k})\]
\[= 6(\hat{i} \times \hat{j}) + 4(\hat{i} \times 
\hat{k}) - 6(\hat{i} \times \hat{j}) + 6(\hat{j} \times \hat{k})\]
\[= -4(\hat{k} \times \hat{i}) + 6(\hat{j} \times \hat{k})\]
\[= 6\hat{i} - 4\hat{j}\]
This, however is a little bit long winded, and the value can instead be 
calculated using a \textit{determinant}. To find the determinant of 
\(\vec{r}\times\vec{F}\) we can use the following matrix, where \((x, y, z)\)
is \(\vec{r}\), the position vector and \((F_x, F_y, F_z)\) is \(\vec{F}\), the
force vector.
\[
    \begin{vmatrix}
        \hat{i} & \hat{j} & \hat{k} \\
        x & y & z \\
        F_x & F_y & F_z
    \end{vmatrix}
\]
What this asks us to do, in essence, is to take the top left cell, eliminate
its row and column, and multiply the top left (\(y\)) and bottom right 
(\(F_z\)) cells of the resultant matrix, then subtract the product of the top
right (\(z\)) and bottom left (\(F_y\)) from this, then subtract from this term
the result of doing the same with the middle top cell and add the result of the
top right cell. As an equation, this becomes:
\[\tau = (yF_z - zF_y)\hat{i} - (xF_z - zF_x)\hat{j} + (xF_y - yF_x)\hat{k}\]
\[= (yF_z - zF_y)\hat{i} + (zF_x - xF_z)\hat{j} + (xF_y - yF_x)\hat{k}\]
This structure can of course be used more generally to solve any cross product.
The first vector goes on the second line and the second on the third line. With
the previous example:
\[\vec{a} = 2\hat{i} + 3\hat{j}\]
\[\vec{b} = 2\hat{i} + 3\hat{j} + 2\hat{k}\]
\[\vec{a} \times \vec{b} = 
    \begin{vmatrix}
        \hat{i}\: & \hat{j}\: & \hat{k} \\
        2 & 3 & 0 \\
        2 & 3 & 2
    \end{vmatrix}
\]
\[\vec{a} \times \vec{b} = (6 - 0)\hat{i} + (0 - 4)\hat{j} + (6 - 6)\hat{k}\]
\[= 6\hat{i} - 4\hat{j}\]
A property of determinants is that any pair of rows or columns can be swapped
with the result of multiplying the result by \(-1\). This makes clear that 
\(\vec{a}\times\vec{b} = -\vec{b}\times\vec{a}\).

\subsubsection*{Angular Momentum}
This understanding can be extended to a concept of angular momentum. Angular
momentum is denoted \(\vec{l}\) and can be calculated from position and 
momentum or equivalently position, mass and velocity.
\[\vec{l} = \vec{r} \times \vec{p} = m(\vec{r} \times \vec{v})\]
It bears noting that the particle doesn't need to be rotating around the origin
to have angular momentum around it. Angular momentum is measured in the same 
units as momentum, in \(\mathrm{kgms}^{-1}\). The same properties of torque 
hold here, with the magnitude of \(l\) being given by
\[l = rmv\sin(\phi) = rp_\perp = rmv_\perp = r_\perp p = r_\perp mv\]
It is worth reiterating the angular momentum has meaning only with respect to
the specified origin; if the origin is moved, angular momentum will change. In
addition, as a vector product, angular momentum is always perpendicular to the
plane formed by the position and momentum vectors. If position and momentum
are in the same direction, there is no angular velocity.

\bigskip
Using the tool of angular momentum, we can rewrite Newton's second law as
\[\vec{\tau}_\mathrm{net} = \frac{d\vec{l}}{dt}\]
As angular momentum is a vector, we can find the angular momentum of a system
by summing the angular momentum of particles in the systems. Thus the rate of
change of the total angular momentum in the system is the total net torque on 
all particles in the system. We denote angular momentum for a system as \(L\)
as opposed to the \(l\) for a single particle.

\bigskip
When considering these equations, it is essential that both torque and angular
momentum are measured relative to the same origin. In addition, when 
considering a system, if the centre of mass of that system is accelerating, the
origin taken must be the centre of mass, less unbalanced acceleration of the 
particles disrupt the solution.

\bigskip
The angular momentum of a rigid body can be found through summation. For a body
rotating about the \(z\) axis:
\[L_z = \sum^n_{i = 1}l_{iz} = \sum^n_{i = 1} m_iv_ir_{\perp i} =
\sum^n_{i = 1} m_i(\omega r_{\perp i})r_{\perp i}\]
\[= \omega\left(\sum^n_{i = 1}m_ir^2_{\perp i}\right) = \omega I\]
i.e. the total linear momentum of the body is equal to the sum of the linear 
momentums of its particles, is equal to the moment of inertia of the body
multiplied by its rate of rotation.

\bigskip
As with translational momentum, angular momentum is conserved. This implies 
that if the moment of inertia of an object decreases, the angular velocity
must increase, and that if the angular velocity decreases this must coincide
with an increase in moment of inertia. It also suggests that if an object 
with angular momentum is introduced into a system, it will impart some of
its momentum to the system. For example, a person on a rotation capable chair
is handed a spinning wheel. If the person tilts the wheel such that it is 
spinning in the same axis as the rotation of the chair, the chair must rotate
slightly in the opposite direction to maintain the \(0\) angular velocity
it had before the wheel was introduced. This can be expressed like so
\[I_i\omega_i = I_f\omega_f\]
Because angular momentum is a vector, we can seperate it into components and
consider that the net angular momentum \textit{on an axis} remains constant.
So even if the angular momentum in the \(x\) direction changes, the angular
momentum in the \(y\) and \(z\) directions must remain constant. The 
conservation of linear momentum allows us to determine an equation for the
increase of \(\omega\) in terms of the change in linear momentum.
\[\omega_f = \omega_i\frac{l_i}{l_f}\]
Because angular velocity increases, the rotational kinetic energy of a system
increases when its moment of inertia decreases. This doesn't violate 
conservation of energy as work needed to be done to decrease the moment of 
inertia.

\subsubsection*{Precession}
If a spinning object, such as a bicycle wheel, is suspended relative to an axis
and has an external force such as gravity acting on it, a torque will be 
generated relative to the axis causing it to rotate about that axis. This 
phenomena is known as precession. If the angular momentum is initially outward,
it will continue to point outward throughout the rotation. Its magnitude will
remain constant, but its direction will change constantly due to the torque. In
this situation, the bicycle wheel is a gyroscope. If the wheel were not fixed,
it would topple, but by giving it angular momentum, it is stabilised. This 
fact is widely utilised in navigation technologies. The same phenomena is 
observed in the motion of a spinning top.

\bigskip
For a spinning top at an angle \(\phi\) to the ground, with mass \(m\), 
spinning with angular velocity \(\omega\) and with moment of inertia \(I\),
under gravitational acceleration \(g\), the angular speed of precession, or
\textit{precession rate} \(\Omega\) is given by
\[\Omega = \frac{mgr}{I\omega}\]
Note that this is independent of \(\phi\), the angle with the ground, and,
due to the \(m\) term in \(I\), the mass of the top. This implies that any top
of similar geometry will have the same behavior in terms of precession. This 
formula will not hold for a gyroscope spinning too slowly.

\subsubsection*{Equilibrium}
It is common that we want objects to remain static despite forces acting on 
them. For example a ceiling fan, a sliding ice puck or a rolling bicycle wheel
have a constant linear momentum and a constant angular momentum. These objects
are said to be in \textit{equilibrium}, their linear and angular momentums are
each constant. Objects with \(\vec{P} = \vec{L} = 0\), such as a book on a 
table are said to be in \textit{static equilibrium}. If an object would return
to this equilibrium after a disturbance, it is in the previously visited stable
equilibrium. If a disturbance disrupts the equilibrium, the object is in 
unstable equilibrium.

\bigskip
We can define requirements of equilibrium using Newton's second law, for both
the rotational and linear cases.
\[\vec{F}_\mathrm{net} = 0\]
\[\vec{\tau}_\mathrm{net} = 0\]
Often we will only consider equilibrium in a single plane, for example in the
\(xy\) plane.

\bigskip
The gravitational force on a body is the sum of the forces on the particles
making up the body. In general, the force due to gravity on an object can be
assumed to apply to the \textit{center of gravity} of the body, which is for
most purposes well approximated by the center of mass of the body. However this
is an approximation; in reality the attraction of each particle is slightly 
different than for its neighbours.

\bigskip
We can use equilibrium to gain insight into some situations. Consider as an
example a beam of length \(L\) and mass \(m = 1.8\mathrm{kg}\) placed such 
that each end is supported by a fulcrum. The beam has a block of mass 
\(M = 2.7\mathrm{kg}\) placed at \(\frac{L}{4}\) units along it. We can 
determine the force on each of the fulcrums by considering that the momentum
of the system must be \(0\) for the beam to be stationary. In that case,
the net torque on the beam must be \(0\). If we take the leftmost fulcrum
as our rotation axis we can find the net torque due to gravity through
\[\tau = \frac{1}{4}MgL + \frac{1}{2}mgL\]
And we know that this torque must be equal to the torque from the normal force
exerted by the right fulcrum. Thus
\[F_rL = \frac{1}{4}MgL + \frac{1}{2}mgL\] 
\[F_r = \frac{1}{4}Mg + \frac{1}{2}mg\]
\[F_r = \frac{1}{4}\times2.7\times9.8 + \frac{1}{2}\times1.8\times9.8\]
\[F_r = 15.4\mathrm{N}\]
This accounts for an angular momentum of \(0\), so we can use this figure to
find the force on the left fulcrum by observing that the linear momentum must
also be \(0\). For this to be the case, the total normal force must cancel with
the total downward force from gravity.
\[F_l = (M + m)g - F_r\]
\[F_l = (2.7 + 1.8) \times 9.8 - 15.4\]
\[F_l = 28.7\mathrm{N}\]

\subsubsection*{Elasticity}
Some problems suffer from a wide range of unknowns which make them 
unnapproachable with the techniques studied thus far, and these problems are
described as \textit{indeterminate}. Often this is due to a fault assumption;
that the bodies involved are rigid and do not deform. In reality, bodies do not
behave this way and it is through the concept of \textit{elasticity} that we
are to solve some more complex problems.

\bigskip
A \textit{stress}, which is a deforming force applied per unit area, produces
a \textit{strain} or a unit deformation. Three main types of stress exist.
\begin{itemize}
    \item Tensile stress occurs when an object is stretched or squished in a 
    single dimension.
    \item Shearing stress occurs when an object has opposite forces applied to
    each end, causing it to warp.
    \item Hydraulic stress occurs when an object is compressed from all sides.
\end{itemize}
For a given object, an \textit{elastic range} for which the relatioship between
stress and strain is roughly linear. Beyond this range we reach the 
\textit{yield strength} and enter the range of \textit{permanent deformation}
which is the case until the object eventually ruptures at its \textit{ultimate
strength}. This linear relationship is defined by the \textit{modulus of 
elasticity} of the object related through
\[\mathrm{stress} = \mathrm{modulus}\times\mathrm{strain}\]
For simple tension or compression, stress is given by force divided by area.
The strain is a dimension quantity given by the change in length of the object
divided by its total length. The objects modulus of elasticity is also known
as its \textit{Young's modulus} denoted \(E\).
\[\frac{F}{A} = E\frac{\Delta L}{L}\]
Many materials have very different tensile and compressive strengths, even
though they will use the same modulus for both calculations. Concrete for 
example has an extremely high compressive strength and very low tensile 
strength. The region of elastic deformation is characterised by Hooke's law.

\bigskip
Strain guages are objects used to measure strain as their resistance changes
due to strains placed on them. They are commonly used in large buildings to 
monitor the strains on them.

\bigskip
When a force is exerted on an axis perpendicular to the long axis of an object
we use the objects \textit{shear modulus} \(G\) to calculate the results. When
an object has one end stretched by a distance \(\Delta x\) as compared to its
inital state
\[\frac{F}{A} = F\frac{\Delta x}{L}\]
Where \(L\) is the objects long axis length. For hydraulic compression, the
\textit{bulk modulus} \(B\) is used. Where \(\Delta V\) is the change in volume
of the objects volume \(V\). There pressure \(p\) is found through
\[p = B\frac{\Delta V}{V}\]

\section*{Special Relativity}
\subsubsection*{Galiliean Mechanics}
Classical mechanics are useful for many everyday phenomena, but the reality is
somewhat different in reference frames moving at a significant proportion of
the speed of light, \(c\). The speed of light is determined through the 
calculation
\[c = \frac{1}{\sqrt{\epsilon_0\mu_0}} = 3.00\times10^8\mathrm{ms}^{-1}\]
Here, \(\epsilon_0\) is the \textit{permittivity of free space} and \(\mu_0\)
is the \textit{permeability of free space}. Light in space is in reality a
related electric and magnetic field, perpendicular to each other and self
propagating across space. Maxwell's equations are used to explain this 
phenomena, telling us that electromagnetic waves travel at \(c\) in a vacuum
and that they \textit{can} travel in a vacuum; they don't require a medium.

\bigskip
Einstein postulated that the speed of light is the same in all inertial 
reference frame, a fact fundamental to modern physics. The theory is known
as the \textit{Special Theory of Relativity} because it deals exclusively
with inertial reference frames. He extended this to accelerating reference
frames, however this is not studied in this subject. Although the surface of
the earth is not technically an inertial reference frame, the effects of its
acceleration are very minor and can usually be ignored. 

\bigskip
A reference frame is inertial if it moves with a constant velocity with respect
to an inertial reference frame. For instance, a car is an inertial reference
frame. Observing it from the outside, we might say it is moving at 
\(10\mathrm{ms}^{-1}\). We can also take the perspective from the inside; in
this case the surroundings of the car appear to be moving at 
\(-10\mathrm{ms}^{-1}\). If an event occurs in the car, such as a ball being 
tossed, for an external observer, the velocity of the car is added to the 
velocity of the ball to adapt it to the external reference frame. This is a 
\textit{Galiliean transformation}.

\bigskip
In general, for an event occuring at time \(t\) and position \(x^\prime\) in a
moving reference frame at speed \(v\) relative to an external observer, the 
event appears to occur at
\[x = x^\prime + vt\]
If an object in a moving reference frame gains an additional velocity 
\(u^\prime\), the velocity in the external reference frame will be given
by
\[u = u^\prime + v\]
These can be performed as vector additions if necessary. Acceleration is 
constant between frames. The laws of mechanics are the same in all reference
frames.

\subsubsection*{Relativity}
Einstein's principle of relativity states that the speed of light is constant
in all inertial reference frames. Indeed it states that all laws of physics are
the same in all inertial reference frames. If all laws are the same, and 
Maxwell's equations yield the same speed for \(c\) in all frames, it makes 
sense that \(c\) must be a constant. If this is the case, we must adjust our
understanding of space and time to match.

\bigskip
Consider a cyclist moving at speed \(u\). His velocity is given by
\[u = \frac{\Delta x}{\Delta t}\]
If we instead measured \(\Delta x^\prime\) from a moving car, it would appear
that the cyclist had moved less distance over the same interval \(\Delta t\),
and thus that they had a lower velocity. This makes sense; the relative 
velocity of the cyclist compared to the car is lower than the relative velocity
of the cyclist as compared with a stationary reference frame; say the road.

\bigskip
If, however, we replace the cyclist with a beam of light we find:
\[u = \frac{\Delta x^\prime}{\Delta t} = \frac{\Delta x}{\Delta t} = c\]
A problematic result; despite a smaller distance covered 
(\(\Delta x > \Delta x^\prime\)), we know that the velocity of the light 
is unchanged. This implies that the time passed in the car must be lower
than in the stationary reference frame; we have run up against time dilation.

\bigskip
To implement this, we need to add a new quality to our events; where before 
something happened at a given coordinate \((x, y, z)\) at a given time, we must
now consider that the event occured at a given \textit{spacetime coordinate}
\((x, y, z, t)\), all of which are prone to transformation between reference
frames.

\bigskip
An important fact to note is that \(t\) is not the time an event was observed,
but the time at which it occurred. If an event occurs at \(t = 1\mu\mathrm{s}\)
and light takes an additional two microseconds to reach one observer and 1 
further microsecond to reach another, each observer should nonetheless state
the same value of \(t\). Indeed, the delay could be used to calculate \(x\) in
some circustances.

\subsubsection*{Simultaneity Examples}

\bigskip
Two events can be said to be simultaneous only if they take place at the same
time in the same inertial frame. They might be at different positions, but must
be measured to have the same time. For example, with the knowledge that light
travels at \(300\mathrm{m}\) per \(\mu\mathrm{s}\), consider an event 
\(1200\mathrm{m}\) away observed at \(3\mu\mathrm{s}\) and a second event 
\(600\mathrm{m}\) away observed at \(5\mu\mathrm{s}\) are they 
simultaneous?
\[t_1 = 5\mu\mathrm{s} - \frac{1200\mathrm{m}}{300\mathrm{m}\mu\mathrm{s^{-1}}}
= 1\mu\mathrm{s}\]
\[t_2 = 3\mu\mathrm{s} - \frac{600\mathrm{m}}{300\mathrm{m}\mu\mathrm{s^{-1}}}
= 1\mu\mathrm{s}\]
The two events are simultaneous, because they occurred at the same time \(t\).
Events that are simultaneous in a given frame aren't necessarily simultaneous 
in a different reference frame.

\bigskip
As an example to show this, consider a light emitted from a central point 
\(17.5\mathrm{km}\) from each of two detectors. In the stationary reference
frame of the detectors, the light takes \(58.33\mu\mathrm{s}\) to reach each.
and so the two events of light reaching the two detectors are simultaneous.
For an observer passing this scene at a speed of \(600\mathrm{kms}^{-1}\), the
light appears to travel only \(17465\mathrm{m}\), taking \(58.22\mu\mathrm{s}\)
in one direction but \(17535\mathrm{m}\), taking \(58.45\mu\mathrm{s}\) in the
other. Thus, the two events are not simultaneous in the moving observer's
reference frame. Simultaneity of events is only possible in a single reference
frame.

\bigskip
As another example; consider a moving train with two lights on it. A device on
the train indicates from which light source it received light first. An 
external observer perceives the two lights to have flashed simultaneously. This
means that on board the train, light from the light source in the direction
of motion must have reached the detector first, as the train is moving in the
direction of that light, effecting a higher velocity. Thus, the events are 
simultaneous in the external frame but at different times on board the train.

\subsubsection*{Time Dilation and Length Contraction}
Consider a set up with a light emitting box which releases a pulse of light 
toward a mirror, which reflects the light back to the box which measures the
time taken, \(\Delta t\). This time in the (moving) reference frame of the box
is 
\[\Delta t^\prime = \frac{2h}{c}\] 
For a stationary observer, the light appears to have traversed a large distance
because it also moved in the direction of the box, and thus a longer 
\(\Delta t\) is observed. This \(\Delta t\) is given by
\[\Delta t = \frac{\Delta t^\prime}{\sqrt{\frac{1 - v^2}{c^2}}}\]
To simplify this we define
\[\beta = \frac{v}{c}\]
\[\Delta t = \frac{\Delta t^\prime}{\sqrt{1 - \beta^2}}\]
Analysing this we can see that as \(v\) gets closer to \(0\), so too does 
\(\beta\) and so \(\Delta t \approx \Delta t^\prime\), while when 
\(v \rightarrow c\), \(\beta \rightarrow 1\), and so 
\(\Delta t \rightarrow \infty\).

\bigskip
At this stage it is useful to add a notion of \textit{proper time}, defined as
the time between two events that occur at the same position and denoted 
\(\Delta \tau\). In the above case, time in the frame of the emitting box is 
proper time. Proper time is considered to be in the relevant stationary 
reference frame.

\bigskip
If an object is moving very rapidly, time moves more slowly in its reference
frame as compared to a stationary reference frame. This means, that for the
object to measure the same velocity as an observer in an external frame, the
distances covered must be shorter. For an object in the \(S^\prime\) reference
frame, with beta value \(\beta\), the contracted length of a distance of prpoer
length \(L\) is given by
\[L^\prime = L\sqrt{1 - \beta^2}\]
Proper length is the length between two objects in the frame at which those 
objects are at rest.

\bigskip
If we consider a distance \(d\) in a cartesian coordinate system, it can be
said that
\[d^2 = \Delta x^2 + \Delta y^2\]
For some \(x\) and \(y\). An interesting geometric property of this is that it
is \textit{invariant}; if we rotate our coordinate system, \(x\) and \(y\) 
might move but \(d\) will always stay the same. A similar invariant exists for
relativistic physics:
\[s^2=c^2(\Delta t)^2 - (\Delta x)^2 \:\:\: (- (\Delta y)^2 - (\Delta z)^2)\]
This expression has the same value in all inertial reference frames. \(s\) is
known as the \textit{spacetime interval}. We can use the knowledge of this 
invariant to effect transformations of the spacetime coordinates of events 
between reference frames.
\[\gamma = \frac{1}{\sqrt{1 - \beta^2}}\]
The \textit{Lorentz factor} \(\gamma\) allows us to perform these 
transformations. For a stationary reference frame \(S\) and a second reference
frame \(S^\prime\), moving at speed \(v\) relative to \(S\) in the \(x\) 
direction, we transform the coordinates as follows.

\begin{center}
    \begin{tabular}{c|c}
        \(\mathrm{S^\prime}\) & \(\mathrm{S}\) \\
        \hline
        \(x^\prime = \gamma(x - vt)\) & 
        \(x = \gamma(x^\prime + vt^\prime)\) \\[7pt]
        \(y^\prime = y\) & \(y = y^\prime\) \\[7pt]
        \(z^\prime = z\) & \(z = z^\prime\) \\[7pt]
        \(t^\prime = \gamma(t - \frac{vx}{c^2})\) &
        \(t = \gamma(t^\prime + \frac{vx^\prime}{c^2})\) \\[7pt]
    \end{tabular}
\end{center}

It is very important to make sure the events one is comparing are equivalent
and not e.g. the measurement of an event.

\bigskip
If we have stationary frame \(S\), with an object inside it moving at speed 
\(u\) and a second frame \(S^\prime\), moving relative to \(S\) with speed 
\(v\) then we can find \(u\) and \(u^\prime\), the velocity as measured by
an observer in \(S^\prime\) as
\begin{center}
    \begin{tabular}{c|c}
        \(u^\prime = \frac{u - v}{1 - \frac{uv}{c^2}}\) & 
        \(u = \frac{u^\prime + v}{1 + \frac{u^\prime v}{c^2}}\) \\
    \end{tabular}
\end{center}

\subsubsection*{Relativistic Momentum}
Because relativistic velocity doesn't have simple transformations, we need to
reconsider concepts such as momentum for this new framework. If we take 
velocity for a particle of mass \(m\) to be given by
\[p = m \frac{\Delta x}{\Delta \tau}\]
i.e. mass multiplied by velocity (using proper time). If \(u\) is the velocity
of the particle as measured in a given frame \(s\) then \(p\) in that frame is
given by
\[p = \frac{mu}{\sqrt{1 - \frac{u^2}{c^2}}} = \gamma mu\]
With this definition, momentum is conserved in all intertial reference frames.
It can be seen in this formula that as the velocity of the particle increases
closer to \(c\) it's momentum approaches infinity. This is an intuitive way to
understand why particles can't reach the speed of light; it would require an 
infinite quantity of energy.

\bigskip
If we use our equation for the spacetime invariant from earlier, the equation
for relativistic momentum and apply some witchcraft, we find that the total 
energy of a particle is given by
\[E = \gamma mc^2 = E_0 + K = \mathrm{rest\:energy}+\mathrm{kinetic\:energy}\]
Where \(E_0\), the rest energy is given my \(mc^2\). Thus, the kinetic energy,
\(K\), is given by \((\gamma - 1)E_0\). This kinetic energy is roughly the same
as \(\frac{1}{2}mv^2\) for low velocitys, but approaches infinity as velocity 
approaches \(c\). We can find another equation, known as the relativistic 
energy equation:
\[E^2 = E_0^2 + p^2c^2\]
As an example, let us find the energy of a Muon moving at \(0.96c\), with a 
mass \(207\) times that of an electron.
\[E_0 = mc^2 = 207\times9.1\times10^{-31}\times(3\times10^8)^2 = 
1.7\times10^{-11}\mathrm{J}\]
\[\gamma_p = \frac{1}{\sqrt{1 - \frac{(0.96c)^2}{c^2}}} = 3.57\]
\[E = \gamma_p mc^2 = 3.57\times1.7\times10^{-11} = 6\times10^{-11}\mathrm{J}\]
The energy levels here are almost absurdly miniscule; exponents like 
\(10^{-11}\) are somewhat ridiculous. Thus we instead often use a different
unit for energy levels of particles This value is the energy of an electron
after being accelerated through a potential difference of \(1\mathrm{V}\).
\[1\mathrm{eV} = 1.6\times10^{-19}\mathrm{J}\]
The electron volt (\(\mathrm{eV}\)) is this unit. We also use mega-electron
volts, which are simply greater by a factor of \(10^6\).

\bigskip
Using the relativistic energy equation, we can find a momentum for photons, 
despite their having \(0\) mass:
\[E^2 = E_0^2 + p^2c^2 = 0 + p^2c^2\]
\[\Rightarrow p = \frac{E}{c}\]
Thus the momentum of a photon is given by its energy divided by \(c\).
What happens when two particles collide and fuse together? Let us consider
two equivalent particles, each with mass \(m\) and kinetic energy \(K\).
\[E_i = 2E_0 + 2K = 2mc^2 + 2K\]
When they collide from opposite directions, their momentum cancels and we are
left with a single particle of mass \(M\)
\[E_f = E_0 + 0 = Mc^2\]
We can figure out what this mass is using conservation of energy
\[M = 2m + \frac{2K}{c^2}\]
The final mass is greater than the initial mass, as some of the energy has been
converted into mass during the collision. Energy in an isolated system is 
conserved, although mass is not.
\[E = \sum E_i\]
\[E_i = \gamma_im_ic^2\]

\subsubsection*{Newton's Theory of Gravitation}
Planet's move in elliptical orbits with the sun at one focus. The major axis of
the ellipse describes the line across the widest point of the ellipse through 
the sun, while the minor axis is the perpendicular maximum width of the 
ellipse. An interesting property of orbital motion is that if we draw a line 
between the sun and the planet at some time and then a second line at some 
later time, the area traced out by the two lines will be a constant for a given
\(\Delta t\), irrespective of where the planet is in its orbit. Another 
property is that the square of a planet's orbital period is proportional to the
cube of the semi-major axis (half of the major axis) length.

\bigskip
Most of the planets in our solar system have very close to circular orbits; 
mercury has the least circular orbit. While in orbit, a body is in free fall. 
It has some constant tangential speed, and accelerates towards the center,
yielding the circular motion we now. The strength of gravity in this situation
is inversely proportional to the squre of the orbital radius. The force of
gravity between two masses is thus given by
\[F = \frac{Gm_1m_2}{r^2}\]
Where \(G\) is the gravitational constant, 
\(6.67\cdot10^{-11}\mathrm{Nm^2kg^{-2}}\). This value is extremely small, and
explains why gravity is so small compared to some other forces. The only 
reason for the effectiveness of earths or the suns gravity is the huge masses
involved. While one might think it would be necessary to calculate a force 
vector for each individual particle body in a gravitational interaction, for
spheres this is equivalent to considering all the mass to be concentrated at
a single point in the centre.

\bigskip
Gravitational potential energy can be reconsidered through the following 
formula for large scales where the gravity at the start and end heights is
significantly different, making \(U = mgh\) inaccurate.
\[U = -\frac{GMm}{r}\]
Where \(M\) is the central body (e.g. the earth) and \(m\) is the orbital
body. This defines the work required to move away from a body exerting a 
gravitational force. This can be used, for example to determine the escape
velocity for earth.
\[U_i + K_i = U_f + K_f = 0\]
\[K_i = -U_i = \frac{GMm}{R_E}\]
\[\frac{1}{2}mv^2 = \frac{GMm}{R_E} \Rightarrow v = \sqrt{\frac{2GM}{R_E}} = 
1.12\cdot10^4\mathrm{ms}^{-1}\]
For an object orbiting in a circle, the kinetic energy will be given by
\[K = -\frac{1}{2}U_g\]
If the object is given a thrust to raise this kinetic energy, it's orbit
will become elliptical. As an example, we can consider what speed would
be required for an object to orbit the earth at the surface.
\[F = \frac{GMm}{R^2} = ma = \frac{mv^2}{R}\]
\[v = \sqrt{\frac{GM}{R}} = 7900\mathrm{ms}^{-1}\]

\bigskip
Kepler's Third Law states that, for an object orbiting at radius \(r\), this
radius is related to the orbital period by
\[T^2 = \left(\frac{4\pi^2}{GM}\right)r^3\]
An orbit where the period matches the period of rotation of the earth is 
described as geosynchronous. Using Kepler's law, we can find that the
radius needs to be \(36,900\mathrm{km}\).

\bigskip
Kepler's Second Law, that the area swept out across a time \(t\) by an orbiting
body is a constant can be shown through conservation of angular momentum. 
Angular momentum of an orbiting body is given by
\[\vec{l} = \vec{r}\times\vec{p} = rmv\sin(\beta)\]
Where \(\beta\) is the angle between the radius and the velocity. Here the 
value
\[\frac{1}{2}\frac{L}{m}\]
Is a constant.

\subsubsection*{Oscillations}
Simple Harmonic Motion occurs when a system is cycling through a cycle of 
states, such as a block on a spring, oscillating out and back as the spring
compresses and depresses. A system in such a state has the properties \(T\) or
period, which is the time taken for it to complete a full cycle, \(f\), 
frequency, which is the inverse of period, measured in \(\mathrm{Hz}\) or 
cycles per second and amplitude, the maximum displacement from its equilibrium
position.

\bigskip
We can model oscillating systems using the tool of trigonometric functions.
For a system with amplitude \(A\) and frequency \(f\), we can use angular
velocity \(\omega\) to model the displacement at a time \(t\).
\[x(t) = A\cos(\omega t)\]
Here, \(\omega\) is simply \(2\pi f\), the angular frequency. We can take a
derivative of this to find the velocity at a given time.
\[v(t) = \frac{dx}{dt} = -\omega A\sin(\omega t)\]
Through these equations, we can essentially consider oscillating systems to
be examples of circular motion. If we have different starting conditions for
particles we want to compare, we can use a phase constant \(\phi\):
\[x(t) = A\cos(\omega t + \phi)\]
If we want to compare two different sytems, we can take \(\phi_2 - \phi_1\) to
be the \textit{phase difference}.

\bigskip
We can study the energy in an oscillating system by considering that it must
be the sum of kinetic and potential energies.
\[E = K + U = \frac{1}{2}mv^2 + \frac{1}{2}kx^2\]
At the extremes of the system, when \(x = \pm A\), we know the value of \(K\)
and \(U\).
\[x = \pm A \Rightarrow v = 0 \Rightarrow E = U = \frac{1}{2}kA^2\]
\[x = 0 \Rightarrow U = 0 \Rightarrow E = K = \frac{1}{2}mv^2_\mathrm{max}\]
\[\Rightarrow \frac{1}{2}mv^2_\mathrm{max} = \frac{1}{2}kA^2\]
This lets us derive an equation for maximum velocity in an oscillating system
\[v_\mathrm{max} = A\sqrt{\frac{k}{m}} = \omega A \]
\[\Rightarrow \omega = \sqrt{\frac{k}{m}} = 2\pi f = \frac{2\pi}{T}\]
\[\Rightarrow f = \frac{1}{2\pi}\sqrt{\frac{k}{m}}\]
This tells us that the frequency is independent of the amplitude, and is 
proportional to the ratio between \(k\), the spring constant and \(m\).

\bigskip
We can use our function for position to find equations for energy.
\[x(t) = A\cos(\omega t) \:\:\: v(t) = -\omega A\sin(\omega t)\]
\[U(t) = \frac{1}{2}kA^2\cos^2(\omega t) \:\:\: 
K(t) = \frac{1}{2}kA^2\sin^2(\omega t)\]
\[E(t) = U(t)+ K(t) = \frac{1}{2}kA^2(\cos^2(\omega t) + \sin^2(\omega t))\]
\[E = \frac{1}{2}kA^2\]
Thus, we find what we would expect; the energy in the system is constant, due 
to the trigonometric identity cancelling the terms of the potential and kinetic
energies.

\bigskip
Taking a derivative of the velocity function we can find \(a(t)\)
\[a(t) = -\omega^2A\cos(\omega t) = -\omega^2x(t)\]
And we see that the acceleration is related to the displacement. As the 
position grows further from the origin, its acceleration increases by the 
square of the systems angular velocity.

\subsubsection*{Example}
Consider a \(5\mathrm{kg}\) block hanging at an equilibrium position from a 
spring with constant \(2000\mathrm{Nm}^{-1}\). The block is pulled down 
\(5\mathrm{cm}\) and then given an initial velocity of \(1\mathrm{ms}^{-1}\)
upward. Find the frequency, amplitude and total mechanical energy.
\[f = \frac{1}{2\pi}\sqrt{\frac{k}{m}} = \frac{1}{2\pi}\sqrt{\frac{2000}{5}}
= \frac{20}{2\pi} = \frac{10}{\pi}\mathrm{rads}^{-1}\]
\[E = \frac{1}{2}kx^2 + \frac{1}{2}mv^2 = \frac{1}{2}\times2000\times0.05^2 +
\frac{1}{2}\times5\times1^2 = 5\mathrm{J}\]
\[E = \frac{1}{2}kA^2 = 5 = \frac{1}{2}\times2000\times A^2 \Rightarrow
\sqrt{\frac{10}{2000}} = A = 0.071\mathrm{m}\]

\bigskip
For a pendulum hanging on a string of length \(L\), oscillating slightly
around the rest posisition, the displacement equation has a value \(\omega\)
given by
\[\omega = \sqrt{\frac{g}{L}} = 2\pi f\]
The period of oscillation is independent of both the amplitude and mass.
This expression is only valid for small values of \(\theta\), less than
around \(10\degree\), because it relies on \(\theta\approx\sin(\theta)\) to
calculate \(\tau\) as a linear force through
\[\tau = -Mgl\theta\]

As a general rule, any system with a linear restoring force, such as the 
horizontal component of gravity for the pendulum, or the restoring force of the
spring in the previous example, will undergo simple harmonic motion around its
origin position.

\subsubsection*{Damped Oscillation}
While in an ideal system, oscillations would continue indefinitely, in reality
they can only go on for so long before halting. This is due to damping in
real world systems, due to forces like drag and friction. A damping force
is usually of the form
\[\vec{D} = -b\vec{v}\]
Where \(\vec{D}\) is the damping force and \(b\) is the \textit{damping
constant}. This changes the net force equation to
\[F_\mathrm{net} = F_\mathrm{restorative} + D = -kx - bv = ma\]
This yields a somewhat more complex equation of motion
\[\frac{d^2x}{dt^2} + \frac{b}{m}\frac{dx}{dt} + \frac{k}{m}x = 0\]
\[\Rightarrow x(t) = Ae^{-\frac{bt}{2m}}\cos(\omega t + \phi)\]
With \(\omega\) given by
\[\omega = \sqrt{\frac{k}{m} - \frac{b^2}{4m^2}} 
= \sqrt{\omega_0^2 - \frac{b^2}{4m^2}}\]
In this equation, a negative exponential function is added as a damping factor
to the oscillating term. This makes the oscillations smaller and smaller with
time, gradually bringing the energy lower. We can derive a new equation for
the energy of a sytem with a damping factor.
\[E(t) = \frac{1}{2}k\left(Ae^{-\frac{bt}{2m}}\right)^2 
= \frac{1}{2}kA^2e^{-\frac{bt}{m}}\]
The lost energy is usually as thermal energy. Knowing this allows us to write
\[E(t) = U + K \rightarrow 0 \Rightarrow 
E_\mathrm{total} = E(t) + E_\mathrm{thermal}\]
\[\Rightarrow E_\mathrm{thermal} = E_\mathrm{total} - E(t)\]
We take advantage of damping in structures like damping tanks in the Eureka
Tower, where large tanks of water are used to dissipate oscillation due to wind
by allow the water within to slosh about.

\bigskip
A counterpart to damped oscillation is driven oscillation. If a driving
force, such as a person pushing a swing, is applied at a frequency matching
the natural frequency of an oscillator, resonance occurs dramatically 
increasing amplitude of oscillation.

\subsection*{Waves and Particles}
When a particle moves between two points, both matter and energy are 
transferred. On the other hand, when waves move only energy is transferred.
Many types of waves exist. These include
\begin{itemize}
    \item Mechanical waves, such as sound waves, which propagate physically
    through a medium such as air or water
    \item Electromagnetic which do not require a medium and self propagate
    through vacuum or a medium. Visible light is a small section of the 
    electromagnetic spectrum.
    \item Matter waves which describe atomic particles such as electrons,
    governed by quantum mechanics.
\end{itemize}
Waves occur not only in one dimension, such as along a string, but also in two
dimensions, such as the ripples on a lake, where each ripple corresponds to a
peak and thus is a wavelength apart, or in three dimensions such as when sound
waves are created at a point in the air and scurry away in all directions.

\bigskip
Waves can also be broken into transverse waves, sinusoidal waves and 
longitudinal waves. A transverse wave propagates a single peak through
space, such as a wave on the ocean, while a sinusoidal wave resembles a
sine or cosine pattern. Longitudinal waves have displacement parallel to the
direction of motion, such as in the case of a tugged spring.

\bigskip
For a transverse wave propagating along a string, a peak will move along
the string with a speed given by
\[v = \sqrt{\frac{\tau}{\mu}}\]
Where \(\tau\) is the tension in the string, the restoring force term allowing
oscillation and \(\mu\) is the mass per unit length of the string, the inertial
term enforcing oscillation. For a travelling sinusoidal wave, we have a general
form for the displacement at a given time \(t\) and position \(x\) of
\[D(x, t) = A\sin(kx \pm \omega t + \phi_0)\]
Where \(k\) is the \textit{angular wave number}. For this expression to yield
the correct displacement at each point seperated by one wavelength, we know 
that \(k\) must be given by
\[k = \frac{2\pi}{\lambda}\]
Because we know that the displacement of a point on the wave should be constant
as the wave propagates, we know that
\[kx - \omega t + \phi_0\]
Is a constant. This allows us to find a relationship between position \(x\) and
time \(t\) for a point travelling at wave speed \(v\).
\[\frac{d}{dt}(kx - \omega t + \phi_0)\]
\[k\frac{dx}{dt} - \omega = 0 \Rightarrow v = \frac{\omega}{k}\]
The vertical velocity of a point on the wave is given by
\[\frac{dy}{dt} = -\omega A \cos(kx - \omega t + \phi_0)\]
\[\Rightarrow v_{y,\mathrm{max}} = \omega A\]
The phase difference between two points on a sine wave is given by
\[\delta \phi = 2\pi \frac{\Delta x}{\lambda}\]
Thus, waves are out of phase when \(\Delta x = \frac{\lambda}{2}\) and in
phase when \(\Delta x = \lambda\).

\bigskip
A quick summary of useful formulae:
\begin{center}
    \begin{tabular}{||c|c|c|c|c||}
        \(k = \frac{2\pi}{\lambda}\) &
        \(\omega = \frac{2\pi}{T}\) &
        \(f = \frac{1}{T}\) &
        \(v = \frac{\omega}{k}\) &
        \(v = f\lambda\) \\[7pt]  
    \end{tabular}        
\end{center}

\subsubsection*{Superposition}
For multiple travelling waves in the same medium, the total displacement in the
medium is given by the sum of the displacements of the individual waves.
\[D(x, t) = D_1(x, t) + D_2(x, t) + ...\]
This will hold as long as \(D\) is below the elastic limit of the medium. If
\(D\) exceeds this limit, deformation or breaking can occur. A consequence of
this is that any wave can be generated with a sufficient number of sinusoidal
waves, known as Fourier's Theorem.

\bigskip
An important consequence of superposition is standing waves, where two waves
with equal amplitude, wavelength and frequency interact while travelling in
opposite directions. In this situation, the two have the equations
\[D_1 = A\sin(kx - \omega t) \:\:\: D_2 = A\sin(kx + \omega t)\]
\[D = A\sin(kx - \omega t) + A\sin(kx + \omega t)\]
The resultant wave \(D\) will be a standing wave, which appears to be a 
stationary oscillating wave. The equation \(D\) can be rearranged to find
\[D = 2A\sin(kx)\cos(\omega t)\]
Which tells us that the wave doesn't travel, instead the amplitude is dependent
on \(x\) position and the volume at a time changes. In addition, it tells us 
there are points which are always \(0\), where \(\sin(kx) = 0\), at 
\(kx = m\pi\). If we remember our relationship between \(k\) and \(\lambda\),
we can find that
\[kx = m\pi \Rightarrow x = \set{m\frac{\lambda}{2} | m \in \mathbb{Z}}\]
So the \textit{nodes} of the wave occur at integer multiples of a half 
wavelength. Nodes are points where the standing wave is always \(0\). Their
inverse are \textit{antinodes}, where the amplitude of the standing wave is a
maximum, occuring when \(\sin(kx) = \pm1\). These points are given by
\[x = \left(m + \frac{1}{2}\right)\frac{\lambda}{2}\]

\bigskip
One way which standing waves occur, such as in stringed instruments, is through
reflection of waves. When a transverse wave in a string reaches the end of the
string, the wave is reflected back with inverse displacement. This allows a
standing wave to form along a string, the key principle behind stringed 
instruments. 

\bigskip
Reflection of waves occurs not only at boundaries of strings or
media, but also at discontinuities, places where the properties of the medium
change. Because the velocity of the wave is affected by the mass of a string,
if a string drops in mass the nature of the wave changes. For a wave moving 
from a heavier to a lighter string, a portion of the wave will be reflected
backwards (with lower amplitude in the same vertical direction) at lower speed
while another portion will continue onto the lighter string (with greater 
amplitude than the reflected portion due to the lower mass) and higher speed.

\bigskip
For a smaller string moving into a larger string, the larger string acts 
partially like a hard boundary, reflecting part of the wave with reduced
inverse displacement and equal speed, and partially like a transition, with
a slow, smaller wave with displacement in the same direction continuing along
the thicker string.

\bigskip
This principle can be used to diffuse vibrations by including many changes in
thickness of material to reduce transmission of sound or other mechanical waves
in a structure.

\bigskip
A string of a certain length will have standing waves within it of wavelengths
dependent on its length. For a string with both ends fixed, of length \(L\), 
the \textit{first harmonic} will have a wave length given by \(2L\), the second
wavelength of \(L\) and the third of \(\frac{2}{3}L\). In general, the 
wavelength of the \(n\)th harmonic is given by
\[\lambda = \frac{2L}{n}\]
For example consider a violin string of length \(32\mathrm{cm}\) and mass of
\(4.5\cdot10^{-4}\mathrm{kg}\). What is the tension in the string if its 
fundamental frequency (first harmonic) is \(f_1 = 196\mathrm{Hz}\)?
\[v = \sqrt{\frac{\tau}{\mu}}\]
\[\mu = \frac{m}{l} = \frac{4.5\cdot10^{-4}}{0.32} = 0.0014\]
\[v = f\lambda \Rightarrow v = 196 \times (2 \times 0.32) = 125.44\]
\[v^2\mu = \tau \Rightarrow 125.44^2\times0.0014 = \tau = 22.13\mathrm{N}\]

\subsubsection*{Sound Waves}
Sound waves are longitudinal mechanical waves, which can travel through all 
media irrespective of state, operating by compressing the material ahead of 
them while the material behind them relaxes. Where transverse waves apply a
shearing stress to a material, longitudinal waves apply a compressive stress.
It is for this reason that fluids such as water or are cannot transmit a 
transverse wave while solids can. Because fluids can be compressed, they can
transmit longitudinal waves.

\bigskip
Knowing that sound waves move through a medium through compression, we can find
an equation for the speed of sound in a fluid. Once again, we find that it is 
of the form 
\[\sqrt{\frac{\mathrm{restoring\:force}}{\mathrm{inertial\:term}}}\]
The requirement for oscillation which allows waves to transmit. In this case, 
the restoring force is the materials \textit{Bulk modulus}, visited earlier 
when stresses were examined. Here, it is coupled with density to understand the
behaviour of compressed materials and see how they can transmit waves.
\[v = \sqrt{\frac{B}{\rho}}\]
\(B\) is the Bulk modulus of the relevant fluid, while \(\rho\) is the density.
As an example, we can consider air at \(0\degree\mathrm{C}\). Here, air has a
density of \(1.29\mathrm{kgm}^{-3}\) and sound travels through it at
\(331\mathrm{ms}^{-1}\). What is the Bulk modulus of air at this temperature?
\[331 = \sqrt{\frac{B}{1.29}} \Rightarrow 331^2\times1.29 = B = 141333.7\]
For a solid, the speed of sound is given by a slightly different equation. 
Here, the restoring force is given by Young's modulus (\(E\)), rather than the
Bulk  modulus, and density is still used as the denominator. The resultant 
equation for velocity is
\[v = \sqrt{\frac{E}{\rho}}\]

\bigskip
Sound waves are the most common case where we observe wave interference. 
Constructive interference occurs when waves are offset by a whole number of
wavelengths, which can occur due to a phase difference of \(0\) and a starting
position offset by a whole number of wavelengths or in a situation where the
phase difference compensates the position difference, e.g. \(\phi = \pi\) and
\(\Delta x = \frac{\lambda}{2}\).

\bigskip
Destructive interference occurs when the waves are offset by 
\(\frac{\lambda}{2}\), which can occur for the same reasons. Both of these
interferences can occur in higher dimensions. To understand these situations,
we use the concept of path difference. If \(r\) is the distance from a source
to a point, we can say constructive interference occurs at that point if
\[\Delta r = m\lambda\]
Where \(\Delta r\) is the difference in distances from two sources and \(m\) is
an integer. As long as the path difference is a whole number of wavelengths, 
constructive interference will occur. For the inverse case, when a crest of
one wave meets another we have
\[\Delta r = \left(m + \frac{1}{2}\right)\lambda\]
Standing waves can also occur in situations less contrived than in an 
instrument string, such as in a simple room with a speaker at one end, due to
the interference that occurs. The pressure at certain points will be at a 
maximum, offset by \(\pi\) from the points where pressure is at equilibrium
levels, where the wave amplitude is at a maximum.

\bigskip
For a pipe which is closed at one end and open at the other, resonance can 
still occur. In this case, the first harmonic will occur at 
\[L = \frac{\lambda}{4}\]
An interesting property of this set up is that only odd harmonics can occur,
with the wavelengths of these harmonics given by the equation
\[\lambda_n = \frac{4L}{n}\]
For the \(n\)th harmonic, where \(n\) is always odd. The reason for this 
behaviour is because the air at the closed end of the tube must be fixed, which
implies that the air at the open end must be at maximum displacement. Thus, we
have a cycle from \(0\) to maximum across the length of the pipe, i.e. 
\(\frac{\lambda}{4}\) across the length of the pipe.

\bigskip
This makes it simple to understand that for a closed pipe, the air must be
stationary at each end, which gives us the same result as for a fixed string,
with the first harmonic at \(\lambda = 2L\) and the general form of
\[\lambda_n = \frac{2L}{n}\]
A third case can be considered, where both ends are open. This behaves 
(shockingly enough) as the inverse of the case of both ends closed, with 
antinodes at each end in contrast to the closed tubes nodes. The wavelengths
are determined in the same fashion as for the doubly closed tube.

\subsubsection*{Doppler Effect}
An interesting frequency related phenomena is the Doppler effect, which is the
change in frequency which occurs when a sound source and an observer are moving
relative to each other, such as the change in frequency of the siren of a
passing emergency vehicle.

\bigskip
This effect occurs because the velocity of the source is added to the waves in
front of it, resulting in a higher frequency ahead of the source and a lower
frequency behind it, where the velocity is subtracted from that of the waves.
The magnitude of the change in frequency can be calculated using the relative 
velocities of the waves and source. Here, \(v\) is the velocity of the wave,
\(v_s\) is the velocity of source \(f_o\) is the frequency observed and \(f_s\)
is the frequency emitted at the source. In this situation, the source is 
approaching the observer with velocity \(v\). \(f_o > f_s\).
\[f_o = \left(\frac{v}{v - v_s}\right)f_s\]
For a source moving away from the observer, the sign of the source velocity
is flipped. Here, \(f_o < f_s\).
\[f_o = \left(\frac{v}{v + v_x}\right)f_s\]
The same effect occurs when the observer is moving toward the source. For an
observer approaching the source, the observed frequency is given by
\[f_o = \left(\frac{v + v_o}{v}\right)f_s\]
Here, the sign of \(v_o\) is flipped to find the observed frequency for an
observer retreating from the source. As an example, let us find the speed with
which one would need to move toward an observer to make a \(4800\mathrm{Hz}\)
sound shift up to \(5200\mathrm{Hz}\). The speed of sound can be taken to be
\(343\mathrm{ms}^{-1}\).
\[f_o = \left(\frac{v}{v - v_s}\right)f_s\]
\[\frac{f_o}{f_s} = \left(\frac{v}{v - v_s}\right) \Rightarrow 
\frac{f_s}{f_o} = \left(\frac{v - v_s}{v}\right) \Rightarrow
\frac{f_s}{f_o} = 1 - \frac{v_s}{v} \Rightarrow 
v_s = \left(1 - \frac{f_s}{f_o}\right)v\]
\[v_s = \left(1 - \frac{4800}{5200}\right)343 = 26.4\]

In the case that a source of sound is moving faster than the sound waves which
it emits, these sound waves will tend to build up on top of each other, 
increasing the size of the resultant waves through constructive interference.
This is the cause of a sonic boom.

\end{flushleft}
\end{document}
